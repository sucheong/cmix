% File: output.tex
% Time-stamp: <98/03/04 16:02:37 panic>
% $Id: output.tex,v 1.3 1999/04/21 13:12:12 jpsecher Exp $

\providecommand{\docpart}{\renewenvironment{docpart}{}{}
\end{docpart}
\documentclass[twoside]{cmixdoc}
%\bibliographystyle{apacite}

\makeatletter
\@ifundefined{@title}{\title{\cmix-documentation}}{}
\@ifundefined{@author}{\author{The \cmix{} Team}}%
{\expandafter\def\expandafter\@realauthor\expandafter{\@author}%
\author{The \cmix{} Team\\(\@realauthor)}}
\makeatother

\AtBeginDocument{%
\markboth{\hfill\today\quad\timenow\hfill\llap{\cmix\ documentation}}
{\hfill\today\quad\timenow\hfill}}

\renewcommand{\sectionmark}[1]{\markboth
{\hfill\today\quad\timenow\hfill\llap{\cmix\ documentation}}
{\rlap{\thesection. #1}\hfill\today\quad\timenow\hfill}}

%\newboolean{separate}
%\setboolean{separate}{true}

\renewenvironment{docpart}{\begin{document}}%
                          {\bibliography{cmixII}\printindex
                           \end{document}}
\begin{document}\shortindexingon
}
\begin{docpart}

\begin{center}
  \fbox{\huge\textsl{THIS SECTION IS OUT-OF-DATE}}    
\end{center}

\section{Representing output}
\label{sec:output}
The \verb|Output| class is an abstract representation of a piece of
data and a number of annotations on this data. For instance, a piece
of data could be a program, and the annotations could be results of
various analyses on this program. 

The purpose of this representation is to be able to represent a piece
of data together with all its annotations, and thus postpone the
decision of which annotations will be relevant. For instance, a
hypertext browser can view a program together with information about
what objects each declaration can point to; or a program can be
annotated with colours that represent the binding-times of each
construct in the program. In the case of points-to information, it is
highly desirable for an annotation to be able to refer back to the
main piece of data. This is illustrated in
figure~\ref{fig:pgm-pa-bta}.

\begin{figure}[htbp]
  \begin{center}
    \leavevmode
    \epsfig{file=pgm-pa-bta.eps,width=0.7\textwidth}
    \caption{A program with two sets of annotations.}
    \label{fig:pgm-pa-bta}
  \end{center}
\end{figure}

To be able to select a particular annotation set (and thus ignore
other annotation sets), it must be a requirement that annotations only
refer to the main piece of data and/or other annotations in the same
annotation set.

The selection of relevant annotations and how these are visualised is
carried out by a \emph{filter} or \emph{viewer}, see
section~\ref{sec:filter}.

\subsection{The output classes}
What we want to represent is essentially annotated text, so an
abstract output representation must have the following
characteristics.

\begin{itemize}
\item Data consists of a sequence of smaller pieces of information.
\item A piece of data can be a string of text or a newline.
\item Several pieces of data can be grouped together and their
  (visual) relationship can be expressed.
\item Each piece of data can have a number of annotations attached.
  Each such annotation has a specific type.
\item All annotations associated with a piece of data must have
  different type.
\item Each annotation consists of smaller pieces of data. Such pieces
  can refer to portions of the main data and/or to other annotations
  of the same type.
\end{itemize}

Hence it is natural to represent output as trees, where the leaves
are text string, and the interior nodes nodes express a relationship
between subtrees. It would also be handy to be able to annotate text
strings with attributes such as bold, italics, \etc  

Entities of the \texttt{Output} class can be categorised as follows.

\begin{itemize}
\item A block is a sequence of output trees and It can optionally be
  indented by a given amount. If the output interpreter need to break
  lines inside a block, it can be done in two ways: If the block is
  \emph{consistent}, any shortage of line space will force all
  elements of the block to appear on a separate line. If the block is
  \emph{inconsistent}, line breaks are only inserted to avoid over-full
  lines.
\item A break code can be put inside a block, which means that the
  output interpreter can insert space or break an over-full line here.
  If the whole block fits on the line, each break code is replaced by
  the specified number of spaces.
  
  If the enclosing block does not fit on the line and the block is
  \begin{description}
  \item [consistent,] each break code is replaced by a newline and
    each line is indented to the indentation level plus the block
    offset. [Maybe this is not quite right...]
  \item [inconsistent,] if the element fits on the current line, the
    specified number of spaces is output. If not, a newline is issued
    and the new line is indented to the current indentation level plus
    the block offset plus the break offset.
  \end{description}
\item A text is a character string with an attribute
  (see~\ref{sec:textattribute}).
\item A linked output is a piece of output that is associated to
  a list of anchors (see~\ref{sec:anchors}).
\item An anchored output is a piece of output that has an unique
  anchor attached to it.
\item A hard newline is a piece of output.
\end{itemize}

  \begin{verbatim}
      struct Output {
        enum Mark { Block,Break,HardNL,Text,Anno,Label };
        enum BlockType { Consistent, Inconsistent };

        Output(list<Output*>*, unsigned = 0, BlockType = Consistent); // Block.
        Output(unsigned offset, unsigned spaces); // Break.
        Output(const char*, TextAttribute*); // Text.
        Output(Output*, list<Anchor*>*); // Associate a list of anchors
                                         // with the subtree.
        Output(Anchor*,Output*); // Attach an existing anchor to a tree.
        Output(); // HardNL

        // Lookup
        BlockType type();
        unsigned level();
        list<Output*>* blocks();
        unsigned break_offset();
        const char* text();
        TextAttribute* attribute();
        Output* anno_subtree();
        list<Anchor*>* anno_anchors();
        Anchor* label();
        Output* label_subtree();

        friend Output* oconcat(Output*,Output*);
        friend Output* oconcat(list<Output*>*);
      };
      \end{verbatim}

\subsubsection{Output types}
\label{sec:outputtypes}
Output is collected in an \texttt{OutputContainer} class.  As
described earlier, it should be possible to group output into several
groups, depending on the nature and/or origin of the output. This is
done by demanding that each piece of output appended to an
\texttt{OutputContainer} must be accompanied by an \emph{output type}.
Such a type consists of a symbolic name and a parent identification
such that the types are organized in a tree. The output type tree for
figure~\ref{fig:pgm-pa-bta} could be

\[
\xymatrix{
  & Program \ar[ld] \ar[rd] & & \\
  BT\!A \ar[d] \ar[rd] & & P\!A \\
  Static & Dynamic & \\
}
\]

\noindent
Notice that the output type $BT\!A$ is used solely to group the two
different binding-times under. As stated earlier, the purpose of this
is to be able to select a certain \emph{view} of the program: It
should be possible to instruct the output formatter only to consider
(and thus view) a portion of the output. Such a portion must be a
substree from the root to some depth in the output type tree. This
restriction means that a piece of output only can refer to other
pieces of output of the same type or of ancestor type. Reference
points are called \emph{anchors}.

\begin{verbatim}
struct OType {
    OType(const char*, OType*);
    const OType* parent();
    const list<OType*>& children();
    const char* name();
    unsigned number();
    friend bool operator==(OType&,OType&);
};

\end{verbatim}

\subsubsection{Anchors}
\label{sec:anchors}
An anchors is thus a unique label in some piece of output and a type
that tells us which group of output we should search for this
label. The associated type can also be used to ``trim'' a set of
outputs when a certain view has been decided upon.

\begin{verbatim}
struct Anchor {
    Anchor(OType*);
    OType* isIn();
    unsigned anchor();
};
\end{verbatim}

\subsubsection{Text Attributes}
\label{sec:textattribute}
A text attribute is simply a symbolic name that can be used to make
distinctions between pieces of text, \eg variable names, keywords,
\etc

\begin{verbatim}
struct TextAttribute {
    TextAttribute(const char*, unsigned);
};
\end{verbatim}

\subsubsection{Output containers}
\label{sec:outputcontainer}
An output container holds a set of typed output trees and can export
such a set to a linear \textsc{ascii} format, \eg a file. To create an
output container, one needs to specify (the root of) the output type
tree that the container can work on.

\begin{verbatim}
struct OutputContainer {
    OutputContainer(OType*);
    void add(Output*,OType*);
    friend ostream& operator<<(ostream&,OutputContainer*);
};
\end{verbatim}

\subsection{Filters}
\label{sec:filter}
A filter is something that interprets the contents of an output
container. One could imagine an interactive tool that lets the user
select a type subtree, \ie which annotations she wants to see, and
possibly select the appearance of text by mapping text attributes to
specific fonts; or map subtypes to specific colours.

\end{docpart}
%%% Local Variables: 
%%% mode: latex
%%% TeX-master: "cmixII"
%%% End: 


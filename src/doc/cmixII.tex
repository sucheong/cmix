\documentclass[twoside]{cmixdoc}
\makeindex

\includeonly{%
tables,%        Table of contents, etc
overview,%      Overall view of C-Mix/II
parser,%        Parse the original program(s).
typecheck,%     Type check of original program
corec,%         Definition of the Core C language
c2core,%        Translating from C to Core C
callmode,
pa,%            Pointer analysis
calls,%         Function-call analysis
locals,%        Truly-local-variables analysis
dataflow,%      In-use, Read-only analysis
bta,%           Binding time analysis
init,%          Initializer movement
strctsep,%      Structure separation.
sharing,%       Function-sharing analysis
sanity,%        Binding-time sanity check
partstat,%      Partially static structure splitting
structsort,%    Topological sorting of structure declarations
gegen,%         Generating the generating extension
speclib,%       Specialization library functions
output,%        Description of the intermediate annotation output data
infra,%         Boring infrastructure details
end,%           Bibliography, index, etc
endincludeonly}

\title{\cmix documentation}
\author{Arne John Glenstrup \and Henning Makholm \and Jens Peter Secher}
\begin{document}\shortindexingon

% Edit Mode: -*- LaTeX -*-
% File: tables.tex
% Time-stamp: <98/06/08 12:33:53 panic>

\maketitle
\tableofcontents

%%% Local Variables: 
%%% mode: latex
%%% TeX-master: "cmixII"
%%% End: 


%% Each remaining part of the documentation can be used
%% separately. The next three lines makes sure that the can interact
%% with the full document.
\newenvironment{docpart}{\let\maketitle\relax}{}
\def\title#1{}
\def\author#1{}

% File: overview.tex

\providecommand{\docpart}{\renewenvironment{docpart}{}{}
\end{docpart}
\documentclass[twoside]{cmixdoc}
%\bibliographystyle{apacite}

\makeatletter
\@ifundefined{@title}{\title{\cmix-documentation}}{}
\@ifundefined{@author}{\author{The \cmix{} Team}}%
{\expandafter\def\expandafter\@realauthor\expandafter{\@author}%
\author{The \cmix{} Team\\(\@realauthor)}}
\makeatother

\AtBeginDocument{%
\markboth{\hfill\today\quad\timenow\hfill\llap{\cmix\ documentation}}
{\hfill\today\quad\timenow\hfill}}

\renewcommand{\sectionmark}[1]{\markboth
{\hfill\today\quad\timenow\hfill\llap{\cmix\ documentation}}
{\rlap{\thesection. #1}\hfill\today\quad\timenow\hfill}}

%\newboolean{separate}
%\setboolean{separate}{true}

\renewenvironment{docpart}{\begin{document}}%
                          {\bibliography{cmixII}\printindex
                           \end{document}}
\begin{document}\shortindexingon
}
\title{Introduction to \cmix, a Partial Evaluator for C}
\author{Jens Peter Secher}
\begin{docpart}
\maketitle


\section{Introduction}
\label{sec:Introduction}

This is part of the main documentation accompanying the \cmix system, the
other part being the \emph{\cmix user manual}. As the present document is
intended for implementation maintainance and people who would like to
experiment with extending \cmix, a fairly good knowledge of partial
evaluation is assumed. 

\subsection{\cmix Overview}

\cmix consists of two separate parts. The first part analyses the
subject program and produces another program, a \emph{generating
  extension}, based on the initial division of the input data. The
generating extension reads the static data and produces a
\emph{residual} program. This process utilises the second part of
\cmix, the specialisation library (speclib, for short), which takes
care of memory management and memorisation during specialisation.

\subsubsection{Phases}
\cmix has two internal representations of the subject program and
produces a new program in yet another language together with an
annotated program. This is depicted in figure~\ref{fig:phases}. Each
closed box depicts a transformation/analysis phase --- from the
subject program to the generating extension. After the subject program
is transformed into Core C (cf.{ }
chapter~\vref{sec:TheCoreCLanguage}), both the original C
representation and the Core C representation of the subject program is
kept in memory, and the two representations share the annotations made
by later phases. This enables us to produce an annotated program in
two flavours: a short one that resembles the original subject program;
and an elaborate one that shows the representation that the different
analyses work on.

Each phase can be instructed to produce the annotated subject program
as it appears after that particular phase.  Such annotated subject
program is represented in an abstract format (see
section~\ref{sec:output}). A file in this format can be processed by a
filter (see section~\ref{sec:filter}) and inspected by a
\emph{viewer}, \eg a HTML browser.

\begin{figure}[htbp]
  \begin{center}
\[
\entrymodifiers={=<15ex,8ex>[F]}
\xymatrix{
  *=<15ex,8ex>[o][F]{pgm\mathtt{.c}} \ar[d] \\
  {\txt{C Parser}} \ar[r] & {\txt{User\\directived}} \ar[r] & {\txt{Type\\check}}
   \ar[r] & {\txt{C to \coreC\\translation}} \ar[d] \\
  {\txt{Binding-time\\analysis}} \ar[d] &
  {\txt{Truly-locals\\analysis}} \ar[l] & {\txt{Call\\analysis}}
  \ar[l] & {\txt{Pointer\\analysis}} 
  \ar[l] \\
  {\txt{Structure\\seperation}} \ar[r] & {\txt{Sharing\\analysis}}
  \ar[r] & {\txt{Binding-time\\Sanity\\check}}\ar[r] & {\txt{In-use\\analysis}} \ar[d] \\
   *{} & *{} & {\txt{Generating-\\extension\\generator}} \ar[d] &
   {\txt{Topologically\\sort\\structures}}
  \ar[l] 
  \save "3,1"."5,4"*+[F--]\frm{}\restore
  \save "2,1"."5,4"*++[F.]\frm{} \ar"6,1" \restore \\
  *=<15ex,8ex>[o][F]{pgm\mathtt{.ann}} & *{} & *=<15ex,8ex>[o][F]{pgm\mathtt{-gen.cc}} \\
  }
\]
    \caption{\cmix analyser phases. The phases inside the dashed box
      manipulate the \coreC representation pf the program.}
    \label{fig:phases}
  \end{center}
\end{figure}


\subsection{Capabilities and restrictions}
\label{sec:CapabilitiesAndRestricitions}

Ideally, \cmix should be able to treat all strictly conforming \ansiC
programs. At the time of writing, this is not the case, but if the
subject program basically is well-typed, \cmix should not choke on it.
In particular, you should not use \verb|union|s to typecast data, only
to save memory. \cmix have the following characteristics:

\begin{itemize}
\item Polyvariant specialisation of functions: An unbounded number of
  specialised instances of a particular function can be generated.
\item Functions are allowed to contain dynamic actions yet static return value.
\item Polyvariant specialisation of basic blocks: An unbounded number of
  instances of a particular basic block can be generated.
\end{itemize}

\noindent \cmix has the following limitations:
\begin{itemize}
\item Static union restriction, cf.\ Constraint~\vref{cns:SLFUnion}
\item Bit-fields in structs are accepted, but may produce unexpected
  behaviour
\item Monovariant binding times of variables
\item Monovariant function end-configuration
\item Non-local static side effects under dynamic control are
  suspended by a very conservative approach.
\item Programs that use the \texttt{<setjump.h>} or \texttt{<stdarg.h>}
  headers cannot be specialized: the collide with the basic control-flow
  assumptions.
\end{itemize}



\subsection{Files}
\label{sec:Files}

The \cmix system consists of three parts: The analyser, the
specialization library, and the annotation viewer.

\subsubsection{Analyser}
The analyser is written in C{++} and consists of the files described
in table~\ref{tab:AnalyserFilesI}. These files are found in
\textfnam{src/analyzer}.

\begin{table}
   \begin{center}
     \begin{tabular}{ll}
\hline
Filename                    & Explanation \\ \hline
\textfnam{ALoc.\{cc,h\}}    & Abstract location sets (sets of declarations)\\
\textfnam{GNUmakefile.in}   & File for making the \cmix system \\
\textfnam{Plist.h}          & Abstract data structure for manipulating
lists of pointers\\
\textfnam{Pset.h}           & Abstract data structure for manipulating
sets of pointers \\
\textfnam{analyses.h}       & Interface file for all analyses\\
\textfnam{array.\{cc,h\}}    & Abstract data structure \\
\textfnam{auxilary.\{cc,h\}}  & Small auxiliary functions \\
\textfnam{bta.\{cc,h\}},textfnam{bt\{vars,solve\}.cc}
			   & Binding-time analysis (BTA) \\
\textfnam{btsanity.cc}      & Binding-time sanity checker \\
\textfnam{c2core.\{cc,h\}}  & Translation from C to \coreC \\
\textfnam{check.cc}         & Type check of C program \\
\textfnam{closures.cc}      & Compute sets of ``sure locals'' for each
     function \\
\textfnam{cmix.cc}          & Main \cmix function \\
\textfnam{commonout.\{cc,h\}} & Common declarations for annotation generators \\
\textfnam{corec.\{cc,h\}}   & Abstract \coreC syntax objects \\
\textfnam{cpgm.\{cc,h\}}    & Abstract C syntax objects \\
\textfnam{dataflow.cc}      & In-use analysis (IUA) (dummy at present time) \\
\textfnam{diagnostic.\{cc,h\}} & Error and warning message engine \\
\textfnam{direc.\{l,y\}}    & Parser for user directives \\
\textfnam{directives.\{cc,h\}} & User directives interpreter and repository \\
\textfnam{fileops.\{cc,h\}} & Basic file operations with error handling \\
\textfnam{fixiter.\{cc,h\}} & Generic Least-Recently-Fired fixpoint
iteration \\
\textfnam{gegen.h}          & Local definitions for the gegen phase \\
\textfnam{gg-cascades.cc}   & Gegen helper for statically split arrays \\
\textfnam{gg-code.cc}       & Gegen functions for functions,
     statements and control flow \\
\textfnam{gg-decl.cc}       & Generate code to generate residual declarations \\
\textfnam{gg-expr.cc}       & Generate code for expressions and types \\
\textfnam{gg-memo.cc}       & Generate memoisation code \\
\textfnam{gg-struct.cc}     & Handles struct issues at specialization
     and residual time \\
\textfnam{gegen.cc}         & Generating-extension generator main function \\
\textfnam{generator.cc}     & Translating generator directives to \coreC \\
\textfnam{getopt.\{c,h\}}   & Command-line options parser \\
\textfnam{lex.l},\textfnam{parser.h}
\textfnam{gram.\{cc,y\}}     & C parser \\
\textfnam{init.cc}          & Conversion of initializers to assignments \\
\textfnam{liststack.h}      & Stack of lists (used by C parser for scopes) \\
\textfnam{options.org},
\textfnam{options.perl}     & Automatic command-line options generator \\
\textfnam{outanno.h}        & Wrapper for all output generators \\
\textfnam{outcore.cc}       & Core C output generator \\
\textfnam{outcpgm.cc}       & Core C output generator \\
\textfnam{out\{pa,bta,misc\}.h} & annotation generators \\
\textfnam{output.\{cc,h\}}  & Abstract output definitions \\
\textfnam{pa.\{cc,h\}},\textfnam{paprune.cc}  & Points-to analysis (PA) \\
\textfnam{release.cc}       & The release number (mirrored from
    \textfnam{release-stamp} in the top directiry) \\
\textfnam{renamer.\{cc,h\}} & Name management \\
\textfnam{separate.cc}      & Structure splitting \\
\textfnam{share.cc}         & Function-sharability analysis (FSA) \\
\textfnam{strings.\{cc,h\}} & Common string constants \\
\textfnam{structsort.cc}    & Topologocal sorting of structures \\
\textfnam{symboltable.h}    & Generic hash+backet symboltable \\
\textfnam{syntax\{cc,h\}}   & Intermediate abstract syntax (for the parser) \\
\textfnam{taboos.\{org,perl\}}  & Reserved names (for gegen) \\
\textfnam{tags.\{cc,h\}}    & Common tags for C and \coreC \\
\textfnam{traverse.cc}      & A code template for traversing a \coreC program \\
\textfnam{varstrings.cc}    & Configured locations of CPP and shadow headers \\
\textfnam{bindist.cc}       & Generic replacement for \textfnam{bindist.cc}
     used for binary distributions \\
\textfnam{ygtree.\{cc,h\}}  & Generic balanced (yellow-green) trees
(used by lists, sets, etc.) \\
     \end{tabular}
     \caption{Analyser source files}
     \label{tab:AnalyserFilesI}
   \end{center}
 \end{table}%

\subsubsection{Specialisation library}
The specialisation library is written in C and consists of the files
described in table~\ref{tab:SpeclibFiles}. These files are found in
\textfnam{src/speclib}.

\begin{table}[htb]
   \begin{center}
     \begin{tabular}{ll}
\hline
Filename                 & Explanation \\ \hline
\textfnam{aux.c}         & Auxiliary functions \\
\textfnam{cmix.h}        & General header file \\
\textfnam{code.h}        & Intermediate format for residual code \\
\textfnam{code.c}        & Collect residual code as it is generated \\
\textfnam{floats.c}      & Lift and print floating values \\
\textfnam{mem.c}         & Memory management \\
\textfnam{pending.c}     & Management of the pending list \\
\textfnam{rusage.c}, \\
\textfnam{rusage.h}      & Resource usage statistics \\
\textfnam{unparse.c}     & Printing of a residual program \\
\hline
     \end{tabular}
     \caption{Specilisation-library source files}
     \label{tab:SpeclibFiles}
   \end{center}
 \end{table}


\subsubsection{Annotation viewer}
The annotation viewer is written in C and consists of the files
described in table~\ref{tab:CmixshowFiles}. These files are found in
\textfnam{src/cmixshow}.

\begin{table}[htb]
   \begin{center}
     \begin{tabular}{ll}
\hline
Filename                 & Explanation \\ \hline
\textfnam{GNUmakefile.in}    & Makefile \\
\textfnam{annocut.c}         & Annotation-class selection dialog \\
\textfnam{annofront.h}       & General data structures and function
                                 prototypes \\
\textfnam{annohash.c}        & Primitives for the output tree data
                                structures \\
\textfnam{annolink.c}        & Outputting annotated programs as HTML \\
\textfnam{annomisc.c}        & Auxiliary functions \\
\textfnam{annotext.c}        & Outputting annotated programs as text \\
\textfnam{frames.c}          & Dispatcher for the on-line framed browser \\
\textfnam{http.\{c,h\}}      & Generic HTTP protocol stuff \\
\textfnam{latch.\{c,h\}}     & Buffer for storing long lines of output \\
\textfnam{main.c}            & Main program, option parser, etc. \\
\textfnam{outgram.y},
\textfnam{outlex.l}          & Abstract output-format parser \\
\textfnam{server.\{c,h\}}    & Generic server loop \\
%\textfnam{service.c}         & Not distributed \\ 
\textfnam{webfront.c}        & Fork browser client, start server \\
\hline
     \end{tabular}
     \caption{Annotation viewer source files}
     \label{tab:CmixshowFiles}
   \end{center}
 \end{table}

\end{docpart}
%%% Local Variables:
%%% mode: latex
%%% TeX-master: "cmixII"
%%% End:




% File: parser.tex
% Time-stamp: 

\providecommand{\docpart}{\renewenvironment{docpart}{}{}
\end{docpart}
\documentclass[twoside]{cmixdoc}
%\bibliographystyle{apacite}

\makeatletter
\@ifundefined{@title}{\title{\cmix-documentation}}{}
\@ifundefined{@author}{\author{The \cmix{} Team}}%
{\expandafter\def\expandafter\@realauthor\expandafter{\@author}%
\author{The \cmix{} Team\\(\@realauthor)}}
\makeatother

\AtBeginDocument{%
\markboth{\hfill\today\quad\timenow\hfill\llap{\cmix\ documentation}}
{\hfill\today\quad\timenow\hfill}}

\renewcommand{\sectionmark}[1]{\markboth
{\hfill\today\quad\timenow\hfill\llap{\cmix\ documentation}}
{\rlap{\thesection. #1}\hfill\today\quad\timenow\hfill}}

%\newboolean{separate}
%\setboolean{separate}{true}

\renewenvironment{docpart}{\begin{document}}%
                          {\bibliography{cmixII}\printindex
                           \end{document}}
\begin{document}\shortindexingon
}
\begin{docpart}

\begin{center}
  \fbox{\huge\textsl{THIS SECTION IS OUT-OF-DATE}}    
\end{center}

\section{The C Parser}
\label{sec:Parser}\index{parser}

The input to \cmix is a program fragment in form of a set of C source
files. The purpose of the parser is to transform the set of C source
files into an internal representation of the program fragment such
that all dependencies between the files are resolved. In the rest of
this section we will denote such a program fragment as \emph{the
 subject program}.\index{subject program}

\subsection{Lexing and preprocessing}
Each source is first put through the standard C preprocessor available
on the system (defaults to \texttt{cpp}) and then divided into tokens
by GNU \texttt{flex}. The file \texttt{lex.l} specifies how the input
files are divided into tokens. This is done in a standard way, except
that identifiers are divided into \texttt{typedef}ed names and regular
names. The lexer also picks up \texttt{\#pragma}s that contain user
annotations.

\subsection{The parser}
We use the Roskind Grammar~\cite{Roskind:1996:CGrammar}, a freely
available grammar for the full \ansiC. This grammar works with
standard YACC tools (we use GNU bison).

The parser (\texttt{gram.\{y,cc\}}) parses the tokens (produced by by
the lexer) and creates an internal representation of the subject program.
During parsing, a stack of symbol tables
(\texttt{liststack.h},\texttt{symboltable.h}) keep track of the
variable and function defintions such that the stack represents the
static scope of names. 

For parsing declarations, a
set of ``switchboard'' containers (\texttt{syntax.\{h,cc\}}) are
created and filled with information during parsing. From these
switchboards, the internal representation of the subject program's types is
created. For instance, consider the declaration \texttt{char *const
  (*(*g)()) []}, \ie variable \texttt{g} is a pointer to function
returning pointer to array of const pointer to char.  The type of
\texttt{g} is created by merging the switch boards for the prefix and
the postfix (which themselves are merged from other switchboards).

\subsection{Linkage}
\cmix can read in multiple source files and ``link'' them together
to produce a single module for specialization.

Earlier this was implemented by sharing the declarations
between the \ansiC representations of the source files. This created
problems with structs, however: C uses structural equivalence for
struct types between source files, so it is legal for the members of a
struct to have different names in different files. Sharing the
declarations implied that the types would also be identical in several
files, so there was no way to tell the type checker that, say,
\texttt{f().x} is the first member of the struct that \texttt{f}
returns, when mentioned in one file, but the second member of the
struct when mentioned in another file.

Now we use the following strategy:

An identifier with external linkage has a declaration structure for
each of the source files it is declared in. Each of these declarations
is typed with local types.

The parser also maintains, however, a union--find structure on
declarations that tie all the declarations for each externally linked
identifier together. This union--find structure is maintained such
that the definition, if one has been met, is the representative
element of the declaration set.

When a new declaration is constructed, a type unification routine is
run to make sure its type structurally matches the types of any
previous declarations for it. During this unification process it might
be needed to unify usertype declarations; a parallel union--find
structure for usertype declarations is used to build an equivalence
relation that will later be used by the c2core translation of types.

\subsection{Initializers}
Inializers are not evaluated during parsing.

\subsection{Internal C representation}

\subsection{Parsing types}

\subsubsection{basic types}
The \ansiC specification gives tremendous freedom with regard to type
specifications which is a problem when building a parser. The
following observations are important when creating a faithful \ansiC
parser:

\begin{enumerate}
  \item There are two type qualifiers, \texttt{const} and
    \texttt{volatile}. Each of these can be left unspecified or they can
    be specified a number of times, \eg \texttt{const volatile const
      volatile a}, which should give a warning.
  \item There are five basic types which are \texttt{int},
    \texttt{char}, \texttt{float}, \texttt{double}, \texttt{void}. The
    base type can be left unspecified, in which case it will be
    considered an \texttt{int}.
  \item There are two size specifiers, \texttt{short} and
    \texttt{long}. If none is specified, the size is ``normal''.
  \item There are two sign qualifiers, \texttt{signed} and
    \texttt{unsigned}. If non is specified, the type will be
    considered \texttt{signed}.
\end{enumerate}

Testing that a declaration/definition is in order can be done by
having a ``switchboard'' structure (\texttt{Parse\_Type}) with a field
for each of the above groups. During parsing, such a structure can be
filled in and checks can be made whether two such structures can be
combined; this is done by the \texttt{combine} function.
\texttt{combine} is appropriate for productions ``below''
\texttt{type\_qualifier\_list} and \texttt{base\_type\_specifier} in
the grammar.  The result is a new \texttt{BaseType}.

\subsubsection{User-defined types}
We use the term user-defined types about \texttt{struct}ures,
\texttt{union}s and \texttt{enum}erations.  Appropriate for
productions \texttt{sue\_type\_specifier} in the grammar. User defined
types are put in the global symboltable \texttt{userdecl}.

\subsubsection{Type synonyms}
Type synonyms are types defined by \texttt{typedef}s.
\texttt{typedef}ined names have to be registered as such as soon as
possible [insert example that shows this].

All type synonyms are expanded during parsing, so the internal
representation of the subject program does not contain type synonyms.

\subsection{Storage modifiers}
There are five storage modifiers, \texttt{static}, \texttt{extern},
\texttt{typedef}, \texttt{auto}, \texttt{register}. If non is
specified, \texttt{auto} is assumed.  These modifiers are also checked
by \texttt{combine}, because only one of them is allowed at any time.

\subsection{Declarators}
Declarators are variable identifiers and possibly a bit-field size.

\subsection{Internal representation of C programs}
All data structures are annotated by a line number. This makes it
possible to give meaningful user feed-back.

\subsection{Declarations and Definitions}
A declaration that allocates storage is called a
\emph{definition}. Consider the program
  \begin{verbatim}
     struct S2;
     struct S1 { struct S2 *s2p; };
     struct S2 { struct S1 *s1p; };
  \end{verbatim}
The first line is a declaration and last two are definitions. Consider
the legal program
  \begin{verbatim}
     struct S1 { struct S3 *p; };
     struct S2 { struct S3 *q; };
     struct S3 { struct S2 *r; };
  \end{verbatim}
When \texttt{struct S3} is met for the first time (in the definiation
of \texttt{struct S1}), it is not necessary to know the size of
\texttt{struct S3}, since we just need a pointer to such a user type.
This is also the case in the second line, but here we must ensure that
\texttt{p} and \texttt{q} have the same type. It is thus necessary to
introduce a declaration that serves as a place holder as soon as a new
names is met. When the definition appears, the missing information is
filled into the placeholder. A definition must have a \emph{complete}
type.


User types can be declared in an inner scope. During parsing we need
to maintain several levels of scope (\texttt{usertypes}), \eg in
  \begin{verbatim}
     struct S;
     enum { A } E;
     void f()
     {
        enum { A } E;
        struct S;
        ...
     }
  \end{verbatim}
But it seems that most compilers complains about the initializer not
being of the right type in the following (but not a redeclaration).
  \begin{verbatim}
    struct S { int a; };

    int f1 (struct S b)
    {
       struct S;
       struct S { char d; };
       struct S c = b;
       return 42;
    }
  \end{verbatim}
so we can just make a general lookup in case there is no definition of
the structure. If it is a definition, we make a new structure.

\begin{itemize}
\item In declarations it is the \emph{types} that flow upwards in the
  parse tree. If a declaration is a user type, the actual declaration
  (\ie members, etc) is put in the \texttt{usertypedecls} table as a
  side-effect.
\item Shouldn't \emph{varargs} be a type (instead of an attribute for
  functions)? No, I think we need both a tag in a function decl and a
  special type.
\item Types of variables and funtions are lists of
  \texttt{Type}s.
\item \grule{abstract\_declarator} (using
  \grule{postfixing\_abstract\_declarator},
  \grule{array\_abstract\_declarator},
  \grule{unary\_abstract\_declarator},
  \grule{postfix\_abstract\_declarator}) are concerned with pointers,
  arrays and functions in declarations (\ie the cast \texttt{(char
    \underline{**[2][3]});}). They should result in a list of types
  (\ie array 2 of array 3 of pointer to pointer to char).
  $\Rightarrow$ build a temporary list during parsing and use this
  list later on to construct a real type list. For array sizes, make a
  reverse list.
\item All declarations are put in scope as side-effect --- it is the
  types that propagate up to through the grammar rules. When a scope
  ends, the declarations (objecs,structures) are placed in the right
  spot (\eg a function definition, a statement block, etc.). Wait, no!
  The problem is that the identifier in a declaration appears in the
  middle of the type (\eg \texttt{int *\underline{a}[]}). Hmm, in case
  of \grule{indentifier\_declarator} we could propagate the both the
  identifier and the list of postfixes. $\Rightarrow$
  \texttt{Parse\_IdPostfix}.
\item We do not want to make a full type (in the sense of the user
  class \texttt{Type}) or a full declaration
  (\textsl{xxx}\texttt{Decl}) before it can be fully instantiated (but
  we have to, anyway).  Different temporary forms exist
  (\texttt{Parse\_}\textsl{xxx} in \texttt{syntax.h}) as fragments of
  a real type/declaration.
\item When the base type (\ie ``the end point'' of a type chain) is
  known, only pointers, array, and functions postfix modifiers with
  qualifiers (\texttt{const}/\texttt{volatile}) can occur later on.
  Thus, at this point a real type (\texttt{Type}) can be instantiated
  and parsed on. The above mentioned postfixes can then be converted
  to a list of types, effectively pushing the base type in front of
  them.
\item is it wise to represent ellipsis as a type (at the end of a type
  list) as it is now, or should it be an attribute?
\item Some of the namespaces are only used as place-holders for a
  particular name. The actual declaration is placed under another
  declaration, to which it belongs.
\end{itemize}

\subsection{New Design}
\begin{itemize}
\item Typedefs, variables, prototypes, and enum constants can all be
  put in the class \texttt{VarDecl}. Special cases: Prototypes are
  not allowed to have initializers (and ellipsis is a type); Typedefs
  can only have storage specifier \texttt{Typedef} and cannot have
  initializers; Enum constants cannot have storage specifier.
  Function \emph{definitions} are in the same namespace, which means
  we need to look in both scopes to check for name clashes.
\item What about function type declarations? Where are they handled
  (or: are \emph{all} the ``redundant parenthesis'' really redundant)?
\item The grammar would like us to put the (type of) the identifier in
  scope as soon as possible. In the case of \eg \texttt{int
    a=sizeof(a);} we need to be able to look up the type of \texttt{a}
  when the expression is parsed.
\item Lists of declarations (\texttt{list<Vardecl*>}) are parsed
  around between the rules, until such a list can be put in the right
  place, \eg under a function definition. To create a list from \eg
  ``\texttt{int a,*b,c}'', we also need to carry around the \emph{base
    type} of such a declaration; \texttt{int} in this case. An
  alternative to this is to associate a stack of lists with each
  namespace and put declaration there. This would be a good choice for
  user types, as they ``pop up'' inside regular declarations.
\item Struct/enum/typedef types should point to the actual declaration
  so we don't have to search for anything after the program has been
  parsed. This also goes for function calls; they should point to
  function definitions.
\item Now I got it! We hold the user types in a stack of lists that
  follows the normal scope and is re-enterable (\texttt{liststack.h}).
  This enables us to deal with the user types as mere side-effects. At
  the end of a block the full list of user types can be put where it
  belongs. 
\item Old-style function parameters are inserted into current scope as
  declarations with no type. When the proper declaration turns up the
  declaration is updated. The global variable \texttt{old\_function}
  signifies that we are parsing an old-style function definition,
  which enables the appropriate checks and update.
\item Typedef types point to the declaring \texttt{VarDecl}. This way
  we can give meaningful messages and it should be possible to keep
  typedefs as real typedefs (\ie no weird expansion). This, however,
  demands that great care is taken to follow typedefs when comparisons
  are made: both operands in a comparison could potentially be
  typedefs. We a \texttt{normalize} function on types.
\item Local enum declarations in old-style function [this is realy
  bizarrre!] need to temporarily lower the \texttt{old\_function} flag
  because the enum constants are in the same namespace as the fromal
  parameters.
\item A global variable \texttt{reenter\_scope\_at\_next\_block} is needed
  to tell \texttt{enter\_scope()} to continue with the parameter
  declarations when the statements of a funtion definition is parsed.
  Wait, no, the statement can just start yet another scope (as they
  normally do). Now objects, like userdecls can be put in a liststack
  --- the symboltable only holds the names and pointers into this list.
  No, this makes it \emph{realy} hard to enforce the rules about
  variables only being declared once because the parameters of a
  function \emph{is} in the same scope as the rest of the variables
  declared in the body of the function. We still have to separate
  these, though. This can be done by using the liststack method and
  ``emptying'' the list of declarations when the parameter list is
  finish (but keeping the \emph{names} in scope).
\item Initializers are trouble. It is possible to use \texttt{sizeof}
  constructs in constant expressions in a program (which then makes it
  implementation dependend, of course). This means that we have to
  make an array like \texttt{char a[sizeof(int)]} dynamic because the
  size is unknown. This brings us another problem: if such a structure
  has an initializer, we would have to keep this initializer attached
  to the structure in the CoreC representation (\ie not transform all
  initializer to assignments), because it would be an error if the
  initializer is too big, and if it too small the structure should be
  padded with zeros. Solution: we provide an ``evaluate-sizeof''
  switch which enables the user to force sizeof expression to be
  static. If this switch is not turned on and initializers are used
  for a structure that thus becomes dynamic, we abort specialization.
\item We have to backpatch in some cases. An array with no explicit
  size but initializers have to be backpatched, and unspecified types
  in old-style functoin declarations have to be converted to
  \texttt{int}s. It is also necessary to ``normalize'' initializer
  when they are converted to assignments.
\item It is a good idea to keep the string representation of
  constants, in case they must be lifted or as an aid when giving
  feed-back.
\item Characters are kept as ints (we do not support wide charaters).
  In general there is a problem with literal constants. If we don't
  want to be implementation dependend, we cannot do must partial
  evaluation. This is because constants take the ``least'' type they
  fit into.
\item The lexer/parser can be configured to serve as a general purpose
  ANSI C lexer/parser by setting \texttt{only\_ANSI} in \texttt{lex.l}
  to 1. Otherwise, the lexer/parser will recognize special \cmix words
  (\texttt{pure}, \texttt{residual}, \texttt{spectime} and
  \texttt{cmix\_}\dots) which will give rise to special attributes in
  the internal representation\ref{internal-rep}. In the former case
  these attributes can safely be ignored.
\item Labels have function scope $\Rightarrow$ only one symboltable
  for labels. To resolve forward references from \texttt{goto}
  statements to labels, the following is done. The label symboltable
  contains \emph{backpatch fields}; Backpatch fields
  (\texttt{Indirection}) contain pointers to label statements; A label
  statement contains the label name and the actual statement; the goto
  statements contain pointers to backpatch fields. The indirection
  introduced by the backpatch field makes it easy to do backpathing.
\item Shouldn't prototype result in \texttt{FunDef}initions? If so, a
  truly external function could be recognized as one that has a null
  body.
\item Initializer are added after the declaration. This means we need
  the symboltable to cash the last added item. (Returning the last
  added declaration is not a good idea since it then has to be wrapped
  into something in the parser in order to keep it in scope.)
\item \texttt{return} statements have to know what type the embedding
  function must return to be able to make casts/checks.
  
\end{itemize}

\subsection{Sanity check and internal linking}
\begin{itemize}
\item \cmix reads in all files. 
    
\end{itemize}

\subsection{Run-through}
\subsubsection{Types}
When all files are parsed we run through the program. Every type has
to be complete (except \texttt{typedef}s) and every construction need
to abey the ANSI standard. To avoid processing a construction several
times we need a check flag in every construction.

\subsubsection{Types}
Every internal structure should have a \texttt{check} functions that
either approves of the internal state and returns a type, or complains
about it. This is done recursively. We could, at this point, make
every implicit cast explicit --- when all subexpressions have been
checked, we could call an \texttt{explicit} function that either
returns the subexpression as is (if the types match) or inserts an
explicit cast. The latter is also done when an expression statement
discards its return value.

To do this we need a \texttt{composite} function that returns the
composite type of the subexpressions, \eg in \texttt{1 + 2.0} the type
of composite type of the the two subexpressions would be
\texttt{float} and this could be parsed together with each
subexpression to the explicit function. This way the above expression
would become \texttt{(float) 1 + 2.0}. As a side-effect the expression
would get this type.

What about structures? They do not need to have the same name, only be
structurally equal. And this is a problem when we have (mutually)
recursive structs: To avoid infinite comparison we need a ``mark''
that tells us that we already are in the process of comparing these
two structs (or unions). This mark has to be an integer that we are
sure that a depth-first search is not tricked: the two marks have to
have the exact same value. The mark could be a class variable for
\texttt{StructDecl}s and get counted up every time two unmarked
structs are seen.

There are two kinds of type ``equality'' in C: 1) a \emph{full}
equlity where the types have to be \emph{exactly} the same, \ie same
qualifiers and main type (recursively). 2) a \emph{cast} equality
where one type can be more or equally qualified, but not less
(recursively).  The first kind is needed when \eg structures are
compared. The second one is needed in \eg an assignment \texttt{a =
  b}, where \texttt{a} is allowed to be more qualified than
\texttt{b}. Therefore we need a parameterized equality function for
each type to do both sorts of comparison.

[Maybe all types (and declarations) should have a ``isComplete'',
``isScalar'', \etc flag. This way it can be set/resolved once only.]

In function call expressions, the left-hand side has to be a pointer
to a function (!). This is a problem, since a function
\emph{definition} is considered a function \emph{designator} and not a
pointer to a function. May we should make an explicit conversion.

When temporary variables are generated from arrays, they need to be of
type pointer.

There is a problem with ``normalizing'' all types during translation
to Core C: \syntax{typedef}s annotated ``include'' (\ie they are to be
omitted from the residual, but defined in some included file instead)
should not be ``normalized''. Maybe normalizing should only be done to
disambiguate (\eg ``is this an array type?''). What happens if we try
to normalize an incomplete type?

[In general: we need two error and warning functions: one with line
number and one without. And a debug output functions that work with
streams Or a special debug stream; we want free form expressiveness
--- actually we also want that for warnings. What about errors? here
we want to output something (preferably in free form) and then stop
execution. Could this be done by operators on streams alone? \eg.

\begin{verbatim}
    global.fatal << fatalMsg << e->pos()
                 << e->name << `` is not ...'' << die();
\end{verbatim}
Maybe it could be done with macros?
]

We need to copy types in the \texttt{Cpgm} representation.

\texttt{PostExpr} and \texttt{PreExpr} can be the same type.

Expressions have to be tagged whether or not they are rvalues or
lvalues.

\subsection{Output}
\begin{itemize}
\item Potentially, every declaration needs an anchor. To avoid putting
  excessively many anchors in the output, each declaration annotation
  must have an indicator that controls whether this declaration should
  have an anchor. Anchors are taken from a global pool to make them
  unique. [This causes problems when using multiple files.] Also,
  every object that is connected with a declaration needs a back-link
  to the declaration to facilitate output, \ie annotations need the
  name of a declaration (and its anchor) to produce readable output.
\end{itemize}

\subsection{Implementation level (01.03.1999)}
\label{sec:ParserImplementationLevel}
\index{implementation level!parser}

The parser is implemented as described above, except
\begin{enumerate}
\item If several declarations of the same union exist, we require that
  the members are declared in the same order. (According to \ansiC, all
  possible permutations should be tried out.)
\item 
\end{enumerate}

Files: \texttt{lex.l}, \texttt{gram.y}, \texttt{gram.cc},
\texttt{liststack.h}, \texttt{syntax.\{h,cc\}},
\texttt{cpgm.\{h,cc\}}, \texttt{tags.\{h,cc\}},
\texttt{symboltable.h}, \texttt{}, \texttt{}. 


\end{docpart}
%%% Local Variables: 
%%% mode: latex
%%% TeX-master: "cmixII"
%%% End: 


% File: typecheck.tex
% Time-stamp: 
% $Id: typecheck.tex,v 1.1 1999/03/02 17:34:50 jpsecher Exp $

\providecommand{\docpart}{\renewenvironment{docpart}{}{}
\end{docpart}
\documentclass[twoside]{cmixdoc}
%\bibliographystyle{apacite}

\makeatletter
\@ifundefined{@title}{\title{\cmix-documentation}}{}
\@ifundefined{@author}{\author{The \cmix{} Team}}%
{\expandafter\def\expandafter\@realauthor\expandafter{\@author}%
\author{The \cmix{} Team\\(\@realauthor)}}
\makeatother

\AtBeginDocument{%
\markboth{\hfill\today\quad\timenow\hfill\llap{\cmix\ documentation}}
{\hfill\today\quad\timenow\hfill}}

\renewcommand{\sectionmark}[1]{\markboth
{\hfill\today\quad\timenow\hfill\llap{\cmix\ documentation}}
{\rlap{\thesection. #1}\hfill\today\quad\timenow\hfill}}

%\newboolean{separate}
%\setboolean{separate}{true}

\renewenvironment{docpart}{\begin{document}}%
                          {\bibliography{cmixII}\printindex
                           \end{document}}
\begin{document}\shortindexingon
}
\begin{docpart}

\section{Type check}
\label{sec:TypeCheck}\index{type check}

\end{docpart}
%%% Local Variables: 
%%% mode: latex
%%% TeX-master: "cmixII"
%%% End: 

% File: corec.tex
% Time-stamp: 
% $Id: corec.tex,v 1.5 1999/06/10 15:45:28 makholm Exp $

\providecommand{\docpart}{\renewenvironment{docpart}{}{}
\end{docpart}
\documentclass[twoside]{cmixdoc}
%\bibliographystyle{apacite}

\makeatletter
\@ifundefined{@title}{\title{\cmix-documentation}}{}
\@ifundefined{@author}{\author{The \cmix{} Team}}%
{\expandafter\def\expandafter\@realauthor\expandafter{\@author}%
\author{The \cmix{} Team\\(\@realauthor)}}
\makeatother

\AtBeginDocument{%
\markboth{\hfill\today\quad\timenow\hfill\llap{\cmix\ documentation}}
{\hfill\today\quad\timenow\hfill}}

\renewcommand{\sectionmark}[1]{\markboth
{\hfill\today\quad\timenow\hfill\llap{\cmix\ documentation}}
{\rlap{\thesection. #1}\hfill\today\quad\timenow\hfill}}

%\newboolean{separate}
%\setboolean{separate}{true}

\renewenvironment{docpart}{\begin{document}}%
                          {\bibliography{cmixII}\printindex
                           \end{document}}
\begin{document}\shortindexingon
}
\begin{docpart}

\section{The \coreC language}
\label{sec:TheCoreCLanguage}\index{Core C language}
Input to \cmix is any strictly confoming \ansiC
program~\cite{ANSI:1990:ANSIC}. In most analyses\footnote{The
only exceptions are an initial conventional type check and the
resolution of identifiers in user annotation directives.}
we represent it
internally in the \cmix system in a reduced representation
that we call \coreC. The reason for
this is to make it easier both to implement and to ensure correctness
of various analyses and transformations.

\coreC is sufficiently close to C that one can view it as a
restricted subset of \ansiC. However, several details are
treated more orthogonally than the real language.
The most important differences are:
\begin{itemize}
\item \coreC has no structured control constructs apart from
        function calls. The control statements of \ansiC are
        translated into a ``flat'' flow graph where only
        \syntax{goto} statements and ``\syntax{if ($e$) goto
        $l_1$; else goto $l_2$;}''
        appear.
\item \coreC \emph{expressions} have no side effects. Side effects
        are produced by primitive \emph{statements}.
        (Expressions can still have the ``side effect'' of
        trying to derefence a null pointer and killing the program).
\item \coreC has no lvalue expressions. Thus there is no
        generic ``address-of'' operator, but the constructs
        that in \ansiC produce lvalues (such as naming a variable
        or selecting a member of a struct) have \coreC counterparts
        that evaluate to pointers to the location the lvalue
        references.
\item \coreC has no implicit conversions or coercions. Every
        such thing---ranging from promoting an \syntax{int} to
        \syntax{long} so one can add it to another \syntax{long},
        to the ``pointer decay rule'' for arrays---is represented
        as an explicit construct.
\item \coreC has only two lifetimes for variables: they are either
        global or local to a function. \syntax{static} variables
        are made global, and \syntax{auto} or \syntax{register}
        variables declared in compound statements are collected
        in a single list of the function's local variables.
\end{itemize}

\coreC also has an element of ``non-ortogonality'': where a
construction can easily be replaced with an equivalent
representation in a certain context, we generally
disallow it in this description.

Examples of this strategy are the decisions that pointer-valued
expressions cannot be direct operands to the logical operators
\syntax{!}, \syntax{||}, and \verb.&&., or the rule that a
suitable expression \emph{must} be present in the \syntax{return}
statements in a nonvoid function.

This is intended to simplify the analyses: \eg, one could imagine
an analysis that has to do something special when a pointer value
is ``consumed''; such an analysis would benefit from not having to
check the operand types of the logical operators for pointerness.

Internally \coreC is represented as heavily interlinked heap
allocated C++ objects. However, it is instructive to think of
a \coreC program as a tree structure where cross-links occasionally
appear as payload data in the nodes---even if the cross-links are
physically represented as the same types of pointers that parent
nodes use to link to their children. The grammar for this tree
appears in figure \ref{fig:corecinternal}.

\begin{figure}\begin{center}
\small\[
\begin{array}{|l|r@{~}c@{~}l|l|}
\hline
  \mbox{C++ class} & \multicolumn{3}{l|}{\mbox{grammar}} & \mbox{comments} \\
\hline\hline
\syntax{C\_Pgm} & p &::=& (\vec\sigma, \vec\eta, \vec d_v,
                                \vec d_f ,\vec d_{gf}) & \\
\syntax{C\_UserDef} & \sigma
        &::=& \cons{Struct}(id?,\vec{\epsilon},\vec{\coreXref t}\,) & \\
\mbox{~~(struct/union defn)}
        &&|& \cons{Union}(id?,\vec{\epsilon},\vec{\coreXref t}) & \\
\syntax{C\_EnumDef} & \eta   &::=& \cons{Enum}(id?,\vec\epsilon\,) & \\
\mbox{~~(enum definition)} & \epsilon &::=& (id,q,e?)
                & \mbox{enum constant with possible value} \\
\hline
\syntax{C\_Decl} & d &::=& \cons{Var}(id?,q,t,\delta,i?,\vec d\,) & \\
\mbox{~~(declaration)}&&|& \cons{Fun}(id,t,\vec d_p,\vec d_v,\vec b\,)
                        & \mbox{empty } \vec b \mbox{ means prototype} \\
\syntax{C\_Init} & i &::=& \cons{Simple}(e) & \\
\mbox{~~(initializer)}&&|& \cons{FullyBraced}(\vec\imath\,) & \\
                 &   & | & \cons{SloppyBraced}(\vec\imath\,) & \\
		 &   & | & \cons{StringInit}(\mathit{string}) & \\
\syntax{VariableMode} & \delta
  &::=& \cons{VarIntAuto}     & \mbox{Internal variable} \\
  &&| & \cons{VarIntResidual} & \mbox{Internal, declared residual} \\
  &&| & \cons{VarIntSpectime} & \mbox{Internal, declared spectime} \\
  &&| & \cons{VarVisResidual} & \mbox{Declared visible residual} \\
  &&| & \cons{VarVisSpectime} & \mbox{Declared visible spectime} \\
  &&| & \cons{VarExtDefault} ~|~
        \cons{VarExtResidual} & \mbox{External residual variable} \\
  &&| & \cons{VarExtSpectime} & \mbox{External spectime variable} \\
\hline
\syntax{C\_BasicBlock}& b &::=& (\vec s,j) & \\
\syntax{C\_Jump}& j &::=& \cons{If}(e,\coreXref{b_{\mathrm{then}}},
                                \coreXref{b_{\mathrm{else}}})
        & \syntax{if (}e\syntax{) goto }b_{\mathrm{then}}
          \syntax{; else goto }b_{\mathrm{else}}\syntax{;} \\
\mbox{~~(control stmt)}&&|& \cons{Goto}(\coreXref b) & \\
                &   & | & \cons{Return}(e?) & \\
\syntax{C\_Stmt}    & s &::=& \cons{Assign}(e,e') & \syntax{*}e\syntax{=}e' \\
\mbox{~~(statement)}&   & | & \cons{Call}(e?,\gamma,e_f,\vec e\,)
        & \syntax{*}e\syntax{=(*}e_f\syntax{)(}\avec{e}\syntax{)} \\
                    &   & | & \cons{Alloc}(e,d)
        & \syntax{*}e\syntax{=calloc(}e',\syntax{sizeof (}t_d\syntax{))} \\
                    &   & | & \cons{Free}(,e')
        & \syntax{free(}e'\syntax{)} \\
\syntax{CallMode}&\gamma&::=& \cons{CallAtSpectime}
                        & \mbox{call external function at spectime} \\
                    &   & | & \cons{CallOnceSpectime}
                        & \mbox{ditto but illegal if dyn. control} \\
                    &   & | & \cons{CallAuto}
                        & \mbox{pure external call} \\
                    &   & | & \cons{CallResidual} ~|~ \cons{CallDefault}
                        & \mbox{residual call to external function} \\
\hline
\syntax{C\_Expr} & e &::=& \cons{Var}(t,\coreXref d)
                                & \mbox{the address of the variable} \\
\mbox{~~(expression)}&&| & \cons{EnumConst}(t,\coreXref\epsilon)
                        & \mbox{an enum constant (always an \syntax{int})} \\
                 &   & | & \cons{Const}(t,c)
                                & \mbox{(constants can have any type)} \\
                 &   & | & \cons{Null}(t)
                                & \mbox{null pointer constant} \\
                 &   & | & \cons{Unary}(t,\diamond,e) & \\
                 &   & | & \cons{PtrArith}(t,e_1,\circ,e_2) 
                                & \mbox{ptr}\circ\mbox{num}=\mbox{ptr} \\
                 &   & | & \cons{PtrCmp}(t,e_1,\circ,e_2)
                                & \mbox{ptr}\circ\mbox{ptr}=\mbox{num} \\
                 &   & | & \cons{Binary}(t,e_1,\circ,e_2)
                                & \mbox{num}\circ\mbox{num}=\mbox{num} \\
                 &   & | & \cons{Member}(t,e,m)
                & \mbox{ptr to struct}\rightarrow\mbox{ptr to member} \\
                 &   & | & \cons{Array}(t,e)
                & \mbox{ptr to array}\rightarrow\mbox{ptr to first element} \\
                 &   & | & \cons{DeRef}(t,e)
                                & \mbox{contents of pointed-to address} \\
                 &   & | & \cons{Cast}(t,e) & \\
                 &   & | & \cons{SizeofType}(t,t') & \\
                 &   & | & \cons{SizeofExpr}(t,e) & \\
\syntax{UnOp}&\diamond&::=&\syntax{-}~|~\verb.~.~|~\syntax{!} & \\
\syntax{BinOp}&\circ &::=& \syntax{*}~|~\syntax{/}~|~\syntax{\%}~|~
                                \syntax{+}~|~\syntax{-}~|~\syntax{<<}~|~
                                \syntax{>>}~|~\syntax{<}~|~\syntax{>} & \\
                 &   & | & \syntax{<=}~|~\syntax{=>}~|~\syntax{==}~|~
                                \syntax{!=}~|~\syntax{\&}~|~\syntax{|}~|~
                                \verb.^.~|~\syntax{\&\&}~|~\syntax{||} & \\
\hline
\syntax{C\_Type} & t &::=& \cons{Abstract}(id)
                                & \mbox{Non-arithmetic primitive type} \\
\mbox{~~(type)}  &   & | & \cons{Primitive}(id)
                                & \mbox{Arithmetic primitive type} \\
                 &   & | & \cons{Enum}(\coreXref\eta)
                                & \mbox{Enumerated type} \\
                 &   & | & \cons{Array}(t,e?,q)
                                & \mbox{Array of }q~t\mbox{, length }e \\
                 &   & | & \cons{Pointer}(t,q)
                                & \mbox{Pointer to } q ~ t \\
                 &   & | & \cons{FunPtr}(\cons{Fun}(t,\vec t\,))
                                & \mbox{Pointer to function} \\
                 &   & | & \cons{Fun}(t,\vec t\,)
                        & \mbox{Function of }\vec t\mbox{ returning }t\\
                 &   & | & \cons{StructUnion}(q,\coreXref\sigma,\vec t\,)
                        & \mbox{Struct or union with member types }\vec t \\
\syntax{constvol}& q &::=& (\cons{Const}?,\cons{Volatile}?) & \\
\hline
\end{array}\]\end{center}
\vskip-1ex\relax
\caption{The \coreC grammar.}
  \label{fig:corecinternal}
\end{figure}

The grammar uses the following conventions:
\begin{itemize}
\item[$c$] is any (non-string) constant.
\item[$id$] is a name from the original \ansiC program.
\item[$m$] is the ordinal number of a struct or union member.
\item[$x?$] is an optional $x$.
\item[$\vec x$] is a list of $x$. Lists are represented by the
        \syntax{Plist<}$T$\verb.>. template for a list of pointers
        to $T$, defined in \syntax{Plist.h}. These lists support
        STL-like iterators.
\item[$\coreXref x$] is a cross-link to
        an element somewhere else in the program.
        This notation is not used for types ($t$) because
        the structural sharing rules are not as simple as for the
        other classes. A detailed description appears in section
        \ref{sec:corecTypeSharing}.
\end{itemize}

The definitions of the C++ classes that make up the actual
\coreC representation are defined in the source file
\syntax{corec.h}. The auxilliary types whose names do not start
with \syntax{C\_} are defined in \syntax{tags.h}.

The \coreC structures use tags and
unions; not inheritance and virtual functions. The reason for this is
that the core language is not expected to change, whereas the various
kinds of analyses are very likely to change. In a traditional OO
design, every construct should contain a virtual method that carried
out the part of some analysis that applies for that particular
construct. This would result in each analysis being distributed into
several parts of the program, which would make \cmix hard to
understand and maintain.

\subsection{Programs}
A program is a collection of
\begin{itemize}
\item struct and union definitions.
\item enum definitions.
\item global variables (which include \syntax{static} and \syntax{extern}
        variables declared inside the \ansiC program's functions).
\item function definitions.
\item ``goal'' function definitions.
\end{itemize}

A struct and or definition consist of an optional tag and a list
of member descriptions. They also have a list of all the (\cons{StructUnion})
type nodes that reference them.

The member descriptions (denoted with $\epsilon$'s in the grammar)
only include the member's name, type qualifiers, and and optional
expression that denote a possible bitfield width.
If a definition for the aggregate needs to be written, one needs
to go via one of the referencing \cons{StructUnion} nodes to find
the type of each member. Aggregates with no referencing
\cons{StructUnion} should and need not exist in \coreC at all.

\subsection{Declarations}
Despite its name, the \syntax{C\_Decl} class represents not only variable
declaration but anything in the program the pointer analysis needs to
say a pointer may point to. This is the reason why variables and
functions are represented by the same construction.

The source file \syntax{ALoc.h} defines types and functions that
manage unordered sets of declaration references, used by several
of the analyses.

\subsubsection{Function declarations}

$\cons{Fun}(id,t,\vec d_p,\vec d_e,\vec b\,)$
declarations are simple. They have a name and a type (always
a \cons{Fun} type),
lists of formal parameter and local variable declarations, and a
nonempty list of basic blocks. Execution of the function body starts
at the first basic block in the list, and always ends with an
explicit \cons{Return} statement. Functions cannot return simply
by ``falling through the body''.

External functions are not separately represented in the \coreC
program; references to them are \cons{Const} expressions.

\subsubsection{Variable declarations}
\label{sec:corecVariables}

\cons{Var} declarations are more complex. They always have a type but
may have no name: sometimes the c2core phase
needs to introduce new temporary variables. Each variable has a
\emph{varmode} that defines its linkage and any user annotations on
it. A variable may have an \emph{initializer} (section
\ref{sec:coreInitializers}).

Additionally, if the variable consists of sub-objects that can be
individually pointed to (\emph{i.e.}, if it has \cons{Array} or
\cons{StructUnion} type), its declaration also includes one or
more \emph{subsidiary declarations} which represent the sub-objects
in the pointer analysis and other analyses that depend on that.
If the type is an array there is a single subsidiary declaration
which collectively represents the elements of the array. If the
type is a struct or union there is a subsidiary declaration for each
of its members.

A subsidiary declaration per convention have no name or
initializer; its varmodes specify internal linkage.
However, it may contain its own subsidiaries, according to its type.

An example on the use of subsidiary declaration appears on
figure \ref{fig:typesharing}, page \pageref{fig:typesharing}.

Declarations for formal parameters behave like variable declarations
but never have initializers. So does the name-less declaration that
is part of an \cons{Alloc} statement; it collectively represents every
block of memory that gets allocated by the statement.

\subsection{Initializers}
\label{sec:coreInitializers}

An initializer is an optional part of a local or global variable
declaration. Its structure resembles the C read syntax for
initializers.

The $\cons{FullyBraced}(\vec\imath\,)$ construction is used where it
can be guaranteed that each of the elements of $\vec\imath$
initialize exactly one of the immediate subobjects of the aggregate
initialized by the initializer.

Sometimes, however, that is not possible,
and the structure of the read \ansiC initializer is simply copied
using \cons{SloppyBraced} nodes. This reduces the precision of the
analyses and prevents splitting of the initialized object into
more than a single spectime or residual part.

A \cons{FullyBraced} initializer can have a \cons{SloppyBraced}
node as one of its subinitializers.

\begin{figure}\begin{center}\begin{tabular}{|l|l|}
\hline Declaration & Initializer \\ \hline \hline
\syntax{int a1[sizeof(int)][] = \{1,2,3,4,5,6\};} &
$\cons{SloppyBraced}(\syntax{1}, \syntax{2}, \syntax{3},
                     \syntax{4}, \syntax{5}, \syntax{6})$ \\
\syntax{int a2[4][] = \{1,2,3,4,5,6\};} &
$\cons{FullyBraced}(
        \cons{FullyBraced}(\syntax{1}, \syntax{2}, \syntax{3}, \syntax{4}),$ \\
& \hskip 2cm $
        \cons{FullyBraced}(\syntax{5},\syntax{6}))$ \\
\syntax{int a3[sizeof(int)][] = \{\{1,2,3,4\},\{5,6\}\};} &
$\cons{FullyBraced}(
        \cons{FullyBraced}(\syntax{1}, \syntax{2}, \syntax{3}, \syntax{4}),$ \\
& \hskip 2cm $
        \cons{FullyBraced}(\syntax{5},\syntax{6}))$ \\
\hline
\end{tabular}\end{center}
\caption{Examples of the initializer representation}
\end{figure}

\subsection{Statements}
All statements have an optional left-side expression that, if present,
points to the location where the result of the statement is stored. If
the left side is not present, the result is ignored. Actually the
left-side is only genuinely optional in the \cons{Call} statement, but
it is treated uniformly across all statement cases.

The \cons{Assign} statement is a simple assignment; its only side
effect is the assignment to the left-side. An \cons{Assign} with
an empty left-side would be a no-op, so is optimized away in the
c2core phase.

The \cons{Call} statement performs a call through a function pointer
expression. The statement has a \emph{callmode} attribute that
tells when (at spectime or at residual time)
the call should be made if it is a call to an external function.
Specializable functions (ones for which we know a body) are called
through static pointers, so for the purposes of callmode they are
considered spectime.

Calls to the \syntax{malloc()} and \syntax{calloc()} standard library
functions are intercepted in the c2core phase and translated into
$\cons{Alloc}(e,d)$ statements.
The declaration
$d$ is used in the analyses to represent collectively all the pieces
of memory allocated by that particular \cons{Alloc} statement.
The type of $d$ is always an \cons{Array} type; its size expression
specifies the number of objects to allocate on the heap. Contrary to
other array size expressions, this one does not need to be a constant
expression; it is evaluated in the context of the \cons{Alloc}
statement. If the \cons{Array} type has no size expression, a single
object is allocated.
The left-hand side of the \cons{Alloc} statement is never empty:
an allocation that turns out to be ignored gets optimized away
in the c2core phase.

The \cons{Free} statement is the translation of a call to the
\syntax{free()} standard library function. This function has a return
type of \syntax{void}, so the left side of the statement must be
empty.

\subsubsection{Control statements}
The definitions of control statements and basic blocks is fairly
straight-forward.

The \cons{Return} statements are normalized so that the expression
is present precisely if the function does not return \syntax{void}.
If necessary, an anonymous global variable of the proper type is
generated, and its (never assigned-to) value is returned.

\subsection{Expressions}

A \coreC expression has no side effects and always specifies a
value, never a location. An \ansiC expression that is used as
an lvalue is translated to a \coreC expression whose value is
a pointer to the location of the lvalue.

Every expression has a type as its first attribute; this is the
type of the expression's value. This attribute is treated uniformly
across the possible expression forms, so that analyses can act on
the type of an expression without doing a case analysis on its
form. See section \ref{sec:corecTypeSharing} for a description
of structural sharing between these types.

A $\cons{Var}(t,\coreXref d)$ expression evaluates to the
\emph{address} of the referenced variable or function. $d$ must be
a function, a global variable, or a formal
parameter or local variable in the same function the expression
appears in. It cannot be a subsidiary object declaration---pointers
to structure members and array elements are produced by the
\cons{Member} and \cons{Array} expressions.

\cons{EnumConst} expressions model enumeration constants. They are
different from \cons{Var} expression because enumeration constants
don't have addresses and can have different binding times each
time they are mentioned. They are different from \cons{Const} in that
they need to be name managed.

The $\cons{Const}(t,c)$ expression is used for numeric constants,
\emph{and} for identifiers declared as
\syntax{well-known constant:} by user annotations. These constants
can have arbitrarily complex types. This feature is used by the
shadow headers---\emph{e.g.}, by the \syntax{stdio.h} shadow which
defines \syntax{stderr} to be a well-known constant of type ``pointer
to \syntax{FILE}''. This differs from using a \syntax{const} variable
in that different occurences of the well-known constant can have
different binding times.
This situation is
similar to what should happen when external functions are mentioned.
Thus, \cons{Const} is also used for representing the expressions
consisting of the name of an external function.

A special case $\cons{Null}(t)$ exist for expressing null pointer
constants. This is different from $\cons{Const}(t,\syntax{NULL})$
in that a pointer \cons{Const} points to some unknown external
location, while \cons{Null} never points to anything.

The only \cons{Unary} operators in \coreC are the various senses
of negation. The operand is never a pointer---when the
\syntax{!}\ operator is applied to a pointer expression c2core
substitutes an explicit test against a null pointer constant.

The binary operators are translated to \cons{PtrArith},
\cons{PtrCmp}, or \cons{Binary} expressions, depending on the types
of the operands and result. If pointer expressions appear as
operands to the \verb.&&. and \verb.||. operators, explicit
tests against null pointer constants are inserted.
The short-circuiting behavior of the \verb.&&. and \verb.||. operators
in \ansiC is only partly relevant in \coreC because expressions do not
have side effects.

The operand ordering to \cons{PtrCmp} is normalized so the left
operand is always the pointer and the right always the integer.

In the $\cons{Member}(t,e,m)$ expression, $e$ must evaluate to
a pointer to a struct or union. The \cons{Member} expression
produces a pointer to the $m$th of its members. This is sufficient to
represent the dot operator of \ansiC\footnote
        {One might think that this assertation ignores the fact
        that there can be non-lvalue expression with struct or
        union type in \ansiC. However, this only happens with
        function calls, simple assignment, and the \syntax{?:}
        operator---neither of which are expressions in
        \coreC. In each of these cases the c2core translation
        already uses a translation strategy that automatically provides
        an addres to a struct or union object with the same contents
        as the non-lvalue struct or union.}.

The $\cons{Array}(t,e)$ expression produces a pointer to the
first element of the array pointed to by the pointer that $e$
evaluates to. The \ansiC counterpart is the implicit ``pointer
decay'' that happens to expressions with array type.

Type casts, $\cons{Cast}(t,e)$, can appear between
any pair of types $t_e$ and $t$. That is theory at least; in practise
programs with badly-behaved casts will not survive
the type checker to tell the tale.

% The following was a noble thought, but we need to reduce everything
% to SizeofExpr anyway so we can observe the scope rules as we emit
% pgen and pres.
%
%It is tempting to reduce \cons{SizeofExpr} to
%\cons{SizeofType}. After all, we know the type of the expression
%operand. However, in general---though not precisely in
%\cmix---having expressions as operands to \syntax{sizeof}
%can provide potentially valuable information to program analyses
%or program transformers, which might be more conservative
%when faced with a naked type expression. Thus it would be
%unjust of us to lose these connections in the residual program.

\subsubsection{Restrictions on expressions}

\begin{itemize}
\item No expressions have \cons{Fun} type.
\item Expressions with \cons{Array} type only appear as
        operands to \cons{SizeofExpr}.
\item Expressions with \cons{StructUnion} type only appear as
        operands to \cons{SizeofExpr}, as right-hand sides of
        \cons{Assign} statements, or in argument lists of
        \cons{Call} statements.
\item The expression in a \cons{Simple} initializer and the length
        of an \cons{Array} type are constant expressions. They may
        not contain any \cons{DeRef} expressions (except in operands to
        \cons{SizeofExpr}), and \cons{Var} expressions
        may not reference formal parameters or local variables.
\end{itemize}

\subsubsection{Displaying expressions}

The left-hand side of figure \ref{fig:corecdisplay} defines the
closest one gets to a fully compositional write syntax
for \coreC expression that could be parsed by an \ansiC
parser.

\begin{figure}\begin{center}
\def\can#1{(\!|#1|\!)}
\def\hum#1{\{\!|#1|\!\}}
\def\shum#1{{\hum{#1}\!}^*}
\small\[
\begin{array}{|r@{\:}c@{\:}l|r@{\:}c@{\:}l|}
\hline
\multicolumn{3}{|c|}{\mbox{Canonically}} &
\multicolumn{3}{c|}{\mbox{Optimizing}} \\
\hline
\hline
   \can{\cons{Var}(t,\coreXref d)} &=& \syntax{\&}id_d
 & \hum{e=\cons{Var}(\ldots)} &=& \syntax{\&}\shum{e}
\\ \can{\cons{EnumConst}(t,\coreXref\epsilon)} &=& id_\epsilon
 & \hum{\cons{EnumConst}(t,\coreXref\epsilon)} &=& id_\epsilon
\\ \can{\cons{Const}(t,c)} &=& c
 & \hum{\cons{Const}(t,c)} &=& c 
\\ \can{\cons{Null}(t)} &=& \syntax{NULL}
 & \hum{\cons{Null}(t)} &=& \syntax{NULL} 
\\ \can{\cons{Unary}(t,\diamond,e)} &=& \diamond\can e
 & \hum{\cons{Unary}(t,\diamond,e)} &=& \diamond\hum e 
\\ \can{\cons{PtrArith}(t,e_1,\circ,e_2)} &=& \can{e_1} \circ \can{e_2}
 & \hum{\cons{PtrArith}(t,e_1,\circ,e_2)} &=& \hum{e_1} \circ \hum{e_2} 
\\ \can{\cons{PtrCmp}(t,e_1,\circ,e_2)} &=& \can{e_1} \circ \can{e_2}
 & \hum{\cons{PtrCmp}(t,e_1,\circ,e_2)} &=& \hum{e_1} \circ \hum{e_2} 
\\ \can{\cons{Binary}(t,e_1,\circ,e_2)} &=& \can{e_1} \circ \can{e_2}
 & \hum{\cons{Binary}(t,e_1,\circ,e_2)} &=& \hum{e_1} \circ \hum{e_2} 
\\ \can{\cons{Member}(t,e,m)} &=& \syntax{\&}\can{e}\syntax{->}id^m_{\sigma_t}
 & \hum{e = \cons{Member}(\ldots)} &=& \syntax{\&}\shum{e} 
\\ \can{\cons{Array}(t,e)} &=& \syntax{\&(*}\can{e}\syntax{)[0]}
 & \hum{\cons{Array}(t,e)} &=& \shum{e} 
\\ \can{\cons{DeRef}(t,e)} &=& \syntax{*}\can{e}
 & \hum{\cons{DeRef}(t,e)} &=& \shum{e} 
\\ \can{\cons{Cast}(t,e)} &=& \syntax{(}t\syntax{)}\can{e}
 & \hum{\cons{Cast}(t,e)} &=& \syntax{(}t\syntax{)}\hum{e} 
\\ \can{\cons{SizeofType}(t,t')} &=& \syntax{sizeof(}t\syntax{)}
 & \hum{\cons{SizeofType}(t,t')} &=& \syntax{sizeof(}t\syntax{)} 
\\ \can{\cons{SizeofExpr}(t,e)} &=& \syntax{sizeof}\can{e}
 & \hum{\cons{SizeofExpr}(t,e)} &=& \syntax{sizeof}\hum{e} 
\\[1ex]
&&& \shum{\cons{Var}(t,\coreXref d)} &=& id_d \\
&&& \shum{\cons{PtrArith}(t,e_1,\syntax{+},e_2)}
                &=& \hum{e_1}\syntax{[}\hum{e_2}\syntax{]} \\[2pt]
\multicolumn{3}{|c|}{
    \setbox0\vbox{\setbox2\hbox{parentheses as required by precedence}
          \setbox4\hbox to\wd2{\hfil are implicit for both mappings\hfil}
          \box2\vskip1pt\box4}
    \ht0=0pt
    \box0}
  & \shum{\cons{Member}(t,e,m)} &=&
  \left\{\begin{array}{@{}l@{,}l@{}}
    \hum{e}\syntax{->}id^m_{\sigma_t} & \mbox{ for } e=\cons{DeRef}(\ldots) \\
    \shum{e}\syntax{.}id^m_{\sigma_t} & \mbox{ otherwise}
  \end{array}\right. \\
&&& \shum{\mbox{any other }e} &=& \syntax{*}\hum{e} \\[2pt]
\hline
\end{array}\]\end{center}
\caption{Translating \coreC expressions back to C.}
  \label{fig:corecdisplay}
\end{figure}

That mapping produces correct yet horribly unreadable expressions with a
lot of spurious address-of and dereference operators. Therefore, by
default the modules that unparse \coreC (outcore and gegen) use the
``optimizing'' rendering defined at the right side of figure
\ref{fig:corecdisplay}.

The downside of the ``optimizing'' format is that not all \coreC
constructs are individually visible. This means that the annotated
\coreC output in this notation cannot include the annotation
on every subexpression. When debugging \cmix it is sometimes
convenient to switch to the canonical representation in outcore; we provide
a \syntax{-S} switch that does that (almost. $\cons{Array}(t,e)$ is
rendered as a unary operator spelled \syntax{<DECAY>}).

\subsection{Types}
\label{sec:corecTypes}
The $\cons{Abstract}(id,q)$, $\cons{Arithmetic}(id,q)$,
and $\cons{Primitive}(id,q)$ types are
primitive types, at least as far as \cmix is concerned.

The \ansiC base types such as \syntax{signed char} and \syntax{float}
are all \cons{Primitive} types, with the $id$ of the type record containing
the type's full name. The exception is \syntax{void} which is represented
as $\cons{Abstract}(\syntax{void})$.

The user (or, more likely, shadow header files) can define additional
primitive types with the special syntax
\begin{verbatim}
  typedef __CMIX(bar) baz ;
\end{verbatim}
which declares \syntax{baz} as an \emph{abstract type} with the \syntax{b},
\syntax{a}, and \syntax{r} properties (the only one of these that
actually does something at the moment is \syntax{a}). The type
maps to $\cons{Primitive}(\syntax{baz})$ or
$\cons{Abstract}(\syntax{baz})$ in \coreC, depending on whether
it is numeric or not. The type checker differentiates between integral
and other arithmetic abstract types, and also maintains a guess on
whether the type is signed or not; those attributes are stored in
a ``hidden'' (which only means: not shown in the \coreC grammar)
field of the \cons{Primitive} type node.

\cons{Enum} types represent enumerated types. Except for the output
routines in \syntax{outcore} and \syntax{gegen}, they are treated
mostly like primitive types.

\cons{Pointer} and \cons{Array} are straight-forward---the optional
expression in the \cons{Array} type nodes is the length of the array.
Each of them contains the type qualifiers of the \emph{pointed-to}
type. That is, in \coreC, ``pointer to const int'' parses as
``(pointer to const) int'' rather than \ansiC's
``pointer to (const int)''. As far as partial evaluation is concerned,
this difference is not important, but the c2core translation works
best with having the qualifiers as part of the pointer.

\cons{Fun} is a function type with result and parameter types.
\cons{FunPtr} is a pointer to function; its single subitem is always
a \cons{Fun} type. This is also the only way \cons{Fun} types can be
used to form other types.

\cons{StructUnion} deserves some further explanation. The
master definition for the struct or union concerned is referenced
from the type node, so that the type tag and member names can be
accessed. However, the type has its own private list of member \emph{types},
allowing us to control the degree of structural sharing between types.
This is important because we attach binding-time information to type
nodes.

\subsubsection{Structural sharing of type representation}
\label{sec:corecTypeSharing}

For the purpose of defining the rules for sharing of type nodes, we
divide the references to type nodes in a program into ``owning'' and
``borrowing'' references, and say that a declaration or expression
``owns'' the type nodes that are reachable from its type field,
following only ``owning'' references. Then:
\begin{itemize}
\item A local or global variable declaration owns its
        entire type. The only borrowing references that can be
        reached from the type node is the back edges to
        \cons{StructUnion} nodes necessary to keep the
        representation finite.
\item A function declaration similarly owns its entire type.
\item A formal parameter declation borrows
        its type from the function's type's parameter type list.
\item The declaration in an \cons{Alloc} statement borrows its
        type from the relevant part of the statement's left-hand
        expression\footnote{This does not impose any risk of
                ``circular borrowing'', because only the
                alloc declaration's possible subsidiaries
                borrow from its type, and nothing outside the
                subsidiary tree borrows from it}.
\item A subsidiary declaration borrows
        its type from the appropritate part of its parent's type.
\item A \cons{Var}, \cons{Member}, \cons{Array}, or \cons{PtrArith} expression
        owns the topmost \cons{Pointer} node of its type. The
        pointed-to type is borrowed from the referenced declaraion
        or the relevant part of the operand's type, repectively.
\item A \cons{Const}, \cons{Unary}, \cons{PtrCmp}, \cons{Binary},
        \cons{Cast}, \cons{SizeofType}, or \cons{SizeofExpr}
        expression owns its entire type, and, in the case of
        \cons{SizeofType}, its entire operand type.
\item A \cons{DeRef} expression borrows its type from the
        relevant part of its operand's type.
\end{itemize}

Examples of the type sharing can be seen at figure \ref{fig:typesharing}

\begin{figure}[htbp]
  \begin{center}
     \ttfamily
     \begin{tabular}{l}
        struct node \{ \\
        ~~int ID:~24; \\
        ~~struct node *link; \\
        ~~char data[20]; \\
        \} a\_node, *a\_ptr; \\
        /* ...~*/ \\
        a\_ptr->data[5] \\
        \\
        \\
     \end{tabular}

     \epsfig{file=sharing.eps,height=16.5cm,width=15.5cm}
  \end{center}
  \caption{Type sharing example}
  \label{fig:typesharing}
\end{figure}

\subsection{Implementation level (1999-02-18)}
The above section describe the current implementation of Core C.
        
\end{docpart}
%%% Local Variables: 
%%% mode: latex
%%% TeX-master: "cmixII"
%%% End: 

% Edit Mode: -*- LaTeX -*-
% File: c2core.tex

\providecommand{\docpart}{\renewenvironment{docpart}{}{}
\end{docpart}
\documentclass[twoside]{cmixdoc}
%\bibliographystyle{apacite}

\makeatletter
\@ifundefined{@title}{\title{\cmix-documentation}}{}
\@ifundefined{@author}{\author{The \cmix{} Team}}%
{\expandafter\def\expandafter\@realauthor\expandafter{\@author}%
\author{The \cmix{} Team\\(\@realauthor)}}
\makeatother

\AtBeginDocument{%
\markboth{\hfill\today\quad\timenow\hfill\llap{\cmix\ documentation}}
{\hfill\today\quad\timenow\hfill}}

\renewcommand{\sectionmark}[1]{\markboth
{\hfill\today\quad\timenow\hfill\llap{\cmix\ documentation}}
{\rlap{\thesection. #1}\hfill\today\quad\timenow\hfill}}

%\newboolean{separate}
%\setboolean{separate}{true}

\renewenvironment{docpart}{\begin{document}}%
                          {\bibliography{cmixII}\printindex
                           \end{document}}
\begin{document}\shortindexingon
}

\title{From \ansiC to \coreC}
\author{Henning Makholm}
\begin{docpart}
\maketitle

\newcommand{\ctoc}{\texttt{c2core}\xspace}
\newcommand{\installleft}[2]{{#1}\curvearrowright{#2}}
\newcommand{\atomconnect}[2]{{#1}\rightsquigarrow{#2}}

\section{Transformation from C to Core C}
This section describes the \ctoc phase of \cmix. This phase
translates a type-checked \ansiC program into the \coreC
representation defined in the previous section.

\subsection{Types}
\label{sec:c2coreTypes}
The \ansiC program that is input to \ctoc has already been type
checked and each relevant \ansiC contruct annotated with type
information. However, this type information has to be translated
to the \coreC representation, and the rules for structural
sharing in the \coreC representation (section
\vref{sec:corecTypeSharing})
must be enforced.

The types in the \ansiC program are represented as a classical
object class hierarchy---\eg, ``a pointer type'' is a subclass
of ``some type'', and so forth. Each type object knows how to
translate itself into a fresh, fully owned, \coreC representation
of itself. This translation is used when a declaration or expression must
own its entire type, and is the source of most of the \coreC types.

For some expressions with pointer type,
the type sharing rules specify that the expression must own
the ``outermost'' pointer and borrow the rest of the type from
its subexpression. In these cases the pointer type is simply
generated and the \ansiC type representation is not consulted
at all.

\subsubsection{Translation of recursive types}

Locally, the type translation is a trivial recursive operation.
However, types
in C can be recursive, and we do not want to try to translate
them into infinite \coreC representations. We need some sort
of control over the recursion.

Any type recursion that can be expressed in \ansiC involves
a \syntax{struct} or \syntax{union} type, because
those are the only types that can be expressed before their successors
in the type graph have been specified\footnote
        {An \syntax{enum} type can be named without having been fully
        defined, too, but the translation of an \syntax{enum} type to
        \coreC does not involve translating other types, so it cannot
        be part of an infinite recursion.

        Note that this fact depends on the representation of \syntax{enum}
        types as primitive types in \coreC. Other imaginable
        representations of C's type system might need to treat
        recursion via \syntax{enum}s in pathological cases such as
        \texttt{enum foo \{ bar = sizeof(enum foo*) \};}}.
Therefore, we only check for type recursion while translating
these types, known in most of the \cmix documentation as
\emph{usertypes}.

The rule is that if a translation of a usertype is needed directly or
indirectly while tranlating the members of that usertype, the two
instances of the usertype are made the same by forming a loop.
Thus, given the definitions
\begin{verbatim}
  struct vertex ;
  struct neighbours {
     struct neighbours *next;
     struct vertex *neighbour;
  } ;
  struct vertex {
     struct neighbours *preds;
     struct neighbours *succs;
  } ;
  struct Vertex *foo ;
  struct Edgelist *bar ;
\end{verbatim}
the translation of \syntax{foo}'s type (shown on figure
\vref{fig:c2coreVertexExample}) contains a single copy
of \syntax{struct vertex} and two separate copies of
\syntax{struct neighbours}. The translation of \syntax{bar}'s type,
however, contains only a single copy of each \syntax{struct}.

\begin{figure}[htbp]
  \begin{frameit}
  \begin{center}
        This figure has not been drawn yet.
  %  \leavevmode
  %  \epsfig{file=struct1.eps,height=0.7\textheight,width=0.7\textwidth}
  \end{center}
  \end{frameit}
  \caption{Type recursion example}
  \label{fig:c2coreVertexExample}
\end{figure}

This strategy is implemented by each \ansiC usertype node\footnote
        {The implementation of this is a little more complicated,
        since there are actually two nodes involved in the \ansiC
        representation: one representing the \emph{definition} of
        the struct/union, and one representing the \emph{reference}
        to the definition. The definition holds the member types,
        so that is the place where the recursion prevention most
        naturally takes place. However, the reference node holds the
        type qualifiers that are to be put into the final type node,
        and we don't want to identify two instances with different
        qualifiers. Thus, it is the reference node that ``owns'' the
        current translation pointer; and it passes it as a reference
        parameter to the definition when translating itself.}
having a
pointer to a ``current'' \coreC translation, and translating it
with the following algorithm:
\begin{itemize}
\item[1.] If there is a current \coreC translation, return that.
\item[2.] Create a \cons{StructUnion} node with an empty member
          type list. Install it as the current \coreC translation.
\item[3.] Recursivly translate the types of the members into the
          member type list of the newly created node.
\item[4.] Set the current \coreC translation to ``none'', and
          return the node created in step 2.
\end{itemize}

Observe that the recursion resolution strategy directly effects
the precision of the binding-time analysis. Figure
\vref{fig:c2coreLinkedListBTs} shows two possible \coreC
representations of a linked list type. The one on the left
is able to represent a list where every other element is dynamic
and the rest are static. The \ctoc translation, however,
produces the one on the right, forcing all elements to have
the same binding time.

\begin{figure}[htbp]
  \begin{frameit}
  \begin{center}
        This figure has not been drawn yet.
  %  \leavevmode
  %  \epsfig{file=struct1.eps,height=0.7\textheight,width=0.7\textwidth}
  \end{center}
  \end{frameit}
  \caption{The translation strategy for recursive types affects the BTA}
  \label{fig:c2coreLinkedListBTs}
\end{figure}

The two depicted representations would be equally bad if the
application happened to need every \emph{third} element to
be dynamic. It is not at all trivial to select the best (finite)
number of iterations to unfold in the translation\footnote
        {It might be possible to implement an analysis that
        made good decisions about this. However, it would
        probably be complex and would \emph{need} to employ
        pointer analysis data---and in \cmix the pointer analysis
        comes \emph{after} the \ctoc translation},
so we simply chose to use the strategy that is easiest to implement.

\subsubsection{Typedef types}
\syntax{typedef} type synonyms are expanded in the syntax analysis phase.

\subsection{Global definitions and declarations}
\label{sec:c2coreGlobally}
It is tricky to find a safe overall strategy for the translation
of global definitions and declarations. \syntax{struct},
\syntax{union} and \syntax{enum} are easy:
\begin{itemize}
\item Apart from their role in the type translation (section
      \ref{sec:c2coreTypes}), \syntax{struct} and \syntax{union}
      definitions translate to \syntax{C\_UserDef}s that hold
      the type tag and the member names. These ``master definitions''
      must be present before we can translate any types. On the other
      hand, there are no preconditions for building them.
\item Likewise, \syntax{enum} definitions should be translated
      to \syntax{C\_EnumDef}s before any type translation begins.
      Because expression translation may involve type translation,
      this means that explicit value expressions for the enumeration
      constants cannot take place while building the
      \syntax{C\_EnumDef}.
\end{itemize}
Consequenty, the very first step in the \ctoc translation is to
traverse the \ansiC program\footnote
        {In the \ansiC representation usertype definitions can
        appear in the declaration part of compound statements
        as well as globally. In \coreC they are all moved to
        the global level.}
and translate any \syntax{struct}, \syntax{union}, and \syntax{enum}
definitions found there. The \ansiC definitions are annotated with
references to the \coreC definitions to aid the type translation.
Additionally, a list of original \syntax{enum} definitions is
collected so that as the very \emph{last} step of \ctoc, the
value expressions can be found and translated.

Global variable and function definitions are more complicated. A function
body usually contains many expressions, and variable definitions can contain
expressions in initializers. The translation of expressions may
contain \cons{Var} nodes that refer to---and must borrow the \coreC type
of---other functions or global variables.

This means that to fully translate a function or global variable, we
may need to already have created \coreC declarations for any other
global declaration, and to have translated their type.

This fact suggest that we should translate the program in two passes:
\begin{itemize}
\item[I] Make a \coreC declaration for each global declaration, and
        translate its type, but do not translate its function body,
        repectively its initializer.
\item[II] Make a second pass translating bodies and initializers.
\end{itemize}

Our problems are not totally solved by this, however, because even
the \emph{types} of declarations may contain expressions that need
translating, in the form of array size expressions. Among several
possible strategies, all complicated, we choose this:

\emph{When an array type is translated before pass I is finished, its
size is \emph{not} translated straight away but put into a list of
deferred array sizes.} The deferred array size expressions are then
translated before pass II begins.

It would be nice if the type translation could just keep deferring
all array sizes, even while the loop that tries to empty the deferred
list runs. Unfortunately it is possible to construct pathological
examples such as
\begin{verbatim}
    struct S { int array[sizeof (struct S*)]; } s ;
\end{verbatim}
where the deferred list would never shrink to size 0. When the type
translations translates the size expression right away, this
example will be caught by the recursive type strategy, and we will
end up with a type recursion that involves expressions.

\subsection{Variables}
\label{sec:c2coreVariables}
The same construct is used in the \ansiC representation to model
global variables, local variables, formal parameters, and usertype
members. Usertype members are not made directly into \coreC
declarations, but the former three cases are treated similarly
by the \ctoc translation.

The first step in the translation is to create a \syntax{C\_Decl}
object and supply it with a type. This step is handled differently
in the three cases:
\begin{description}
\item[Global variables] have this done in pass I of the translation,
        viz.\ section \ref{sec:c2coreGlobally}. The type is a fresh
        translation of the \ansiC type.
\item[Local varlaibles] are created by the translation of the
        compound statement that contains their definition. The type
        is a fresh translation of the \ansiC type.
\item[Formal parameters] are created by the translation of the
        function. Their type is borrowed from the function's type's
        parameter list.
\end{description}
As part of the first step, subsidiary declarations (see section
\vref{sec:corecVariables}) are generated for the variable if its
type demands it. The constructor for the \syntax{C\_Decl} class takes
care of this, recursively.

Then the variable's name and varmode is transferred to the \coreC
declaration.

If the variable has an initializer, that is translated and
installed in the \coreC declaration. As part of the type check,
the distinction between \cons{FullyBraced} and \cons{SloppyBraced}
initializer lists has already been resolved, so the translation
of initializers mostly consists of translating the initializer
expressions. Each initializer expression should translate to a
\cons{Value} (see section \vref{sec:c2coreTransExpr})
without pre- or post-sideeffects. If it does not, an error message
is emitted\footnote
        {The type checker does not check for constness of expressions,
        since a good approximation to the required check is
        automatically provided here.

        The approximation contains an error, however: a constant
        expression that uses the \syntax{?:} operator cannot be
        translated into a single \coreC expression, so it is
        erroneously flagged as non-constant.}.

\subsubsection{Well-known constants}
\label{sec:c2coreWellKnownConst}
The ``\syntax{well-known constant:}'' annotation can be applied to
a global variable to declare its name as a ``magic constant'' that
evaluates to an rvalue of the given type when it appears in an
expression.

This feature is used in shadow headers. \emph{E.g.},
\syntax{<stdio.h>} contains
\begin{verbatim}
extern FILE *const stderr, *const stdin, *const stdout ;
#pragma cmix well-known constant: stderr stdin stdout
\end{verbatim}
which causes \syntax{stderr} \emph{et al.}\ to be not
variables but primitive expressions of type ``pointer to
\syntax{FILE}''. The difference from a plain \syntax{const} variable
is that the binding times of different occurences of \syntax{stderr}
need not be equal. Thus, the program can contain calls of
\syntax{fprintf\discretionary{}{}{}(stderr,...)} that are performed
at spectime and other such calls that are residualized.

The implication for \ctoc is that the \ansiC variables that represent
these ``magic constants'' should not be translated into \coreC
variables. Leaf expressions referring to these variables
should be contained into \cons{Const} expressions instead of
\cons{Var}.

The annotation is implemented as a special varmode \cons{VarConstant}
which only appears in the \ansiC representation, not in \coreC.
In both translation passes, the global variables' varmodes are
inspected, and \cons{VarConstant} variables are simply ignored.
Similarly, the varmode is inspected at identifier leaf expressions
and decides how to translate the expression.

\subsection{Functions}
In pass I function definitions are treated little differently from
global variables. In pass II, however, the ``real'' translation to
\coreC takes place:

From pass I we have a \coreC type for the function. The parameter
types are located and used to generate \coreC declarations for
the formal parameters. These are finished off as described in
section \ref{sec:c2coreVariables}.

The body of the function is translated into a list of basic blocks as
described below and entered in the \coreC definition. As a by-product
of this translation the local variables in the body are translated
into the function's list of local variables, or (if they are
\syntax{static}) the list of global variables.

The translation of function bodies is akin to the intermediate
code generator in a compiler. \coreC has other caracteristics
than most compilers' intermediate code representations, so there
are also differences.
Most notably, \ctoc's attitude towards introducing new temporary
variables is much more cautious than the average intermediate code
generator's attitude to pseudo-registers.

The temporaries have proved
to be somewhat confusing to readers (human or machine) of the
residual program.
Additionally, while splitting \ansiC expressions into multiple
\coreC statements is often needed to fulfill the ``only one side
effect per statement'' principle, it also reduces the compiler's
freedom to choose the evaluation order of operands to fit its
optimization, so it should be avoided when possible.

\subsubsection{Proto-flowgraphs}
\label{sec:c2coreAtoms}

The body of a function is initially translated into a fine-grained
proto-flowgraph where each \coreC statement is a unit. Only when the
translation has been completed are the statements collected into
\coreC basic blocks with an ordinary graph traversal. This
collection process performs transition compression and dead
code elimination, so the number of basic
blocks in the final representation is minimized.

The elements that make up the proto-flowgraph are called
``atoms''. They come in several flavors, as depicted in
figure \vref{fig:c2coreAtoms}.
Every atom maintains a count $n$ of predecessors. When the basic-block
collection reach an atom with more than one predecessor, an
unconditional \cons{Goto} jump is generated and a new block
is started off.

\begin{figure}\begin{center}\small\begin{tabular}{|r@{~}c@{~}l|p{8cm}|}
\hline
\multicolumn{3}{|l|}{Grammar} & comments \\
\hline
\hline
$\alpha$ &::=& $\omega$ &
        \cons{StmtAtom}s and \cons{GotoAtom}s are \emph{simple} atoms.
        When they are created they have an empty slot where a
        reference to a successor atom must be plugged in
        before the proto-flowgraph is collected into basic blocks.\\
\cline{4-4}
&$|$& $\cons{IfAtom}(n,e,\coreXref{\alpha_{\mathrm{then}}},
                        \coreXref{\alpha_{\mathrm{else}}})$ &
        This atom models the \coreC \cons{If} jump. It does not
        contain the jump itself, because the pointers to the
        target basic blocks do not exist until the final collection.
        Rather, it contains the condition expression (translated
        to \coreC) and pointers to target atoms. \\
\cline{4-4}
&$|$& $\cons{ReturnAtom}(n,j)$ &
        This atom contains a \coreC \cons{Return} jump. \\
\cline{4-4}
&$|$& $\cons{StopperAtom}(n,t)$ &
        A single \cons{StopperAtom} is generated in each
        function translation; it represents returning without
        a value. \syntax{return;} statements jump to this, and
        a jump to it is inserted at the end of the function
        body (so it will be reached if control falls over the
        end of the function).

        This is different from the regular \cons{ReturnAtom}, in that
        it does not contain a \coreC \cons{Return} jump. The
        jump is only generated if the atom is actually reached
        in the collection phase. Then a suitable return
        expression must be invented. That relieves the
        following analyses from worrying about missing
        return expressions.

        The generated return expression depends on the function's
        return type as defined in figure \vref{fig:c2coreReturnExp}.
        As it is seen, the generation of a return expression sometimes
        involves generating a new variable. This is the reason why
        it is only generated if the \cons{StopperAtom} is reachable, and
        why the entire function shares a single \cons{StopperAtom}. \\
\hline
$\omega$ & $::=$ & $\cons{StmtAtom}(n,\coreXref \alpha ?,s)$ &
        This atom contains a \coreC statement.

        As a special rule, a \cons{Assign} and \cons{Alloc} statements
        may have an empty left side as long as it belongs to a
        \cons{StmtAtom}. However, if the empty left side has not been filled
        when the basic block generation begins, the statement is
        thrown away instead of being inserted into the basic block.
        This facilty is used in the ``Ignore III'' expression
        translation coercion (figure \vref{fig:c2coreCoerce}, page
        \pageref{fig:c2coreCoerce}). \\
\cline{4-4}
&$|$& $\cons{GotoAtom}(n,\coreXref \alpha ?)$ &
        A \cons{GotoAtom} is \emph{not} a representation of the \coreC
        \cons{Goto} jump. It is rather a ``no-op'' atom that is
        used as a jump target when the code that is jumped to
        has not yet been translated.

        When collecting atoms into basic blocks, a jump to a
        GotoAtom is treated just like a jump to its successor. \\
\hline
\end{tabular}\end{center}
\caption{Proto-flowgraph atom types}
\label{fig:c2coreAtoms}
\end{figure}

\begin{figure}\begin{center}\begin{tabular}{|l|l|}
\hline
Return type & Generated default return expression \\
\hline \hline
$\cons{Abstract}(\syntax{void},q)$ & none (in this case only)\\
$t = \cons{Liftable}(id,q)$ & $\cons{Const}(t,\syntax{0})$ \\
$t = \cons{Pointer}(id,q)$ & $\cons{Null}(t)$ \\
Any other $t$ & An anonymous global variable \\
& $\quad d = \cons{Var}(-,t,\cons{VarIntAuto},-,\ldots)$ \\
& is generated, and its value is returned \\
\hline
\end{tabular}\end{center}
\caption{Generation of default return expressions}
\label{fig:c2coreReturnExp}
\end{figure}

When the successor slot of a \cons{GotoAtom} is filled, it exports its
predecessor count to the successor, and any later predecessors
are also counted at the successor.

The exception is if an empty infinite loop is produced. The front
end of the last edge in the loop detects that it is trying to
propagate its predecessor count through itself, and sets a flag
that makes it act like a \cons{StmtAtom} with a discardable statement but
at least 2 predecessors. Thus in the collection phase an empty
basic block with an unconditional jump that points to itself is
generated.

\medskip
\noindent
When the basic blocks for a function have been collected, all the
generated atoms are freed.

\subsubsection{Proto-flowgraph intervals}
\label{sec:c2coreIntervals}
In texts about flow graph analysis, an ``interval'' is usually
a connected subgraph with a single entry node. However, as we
are here concerned with flow graph \emph{synthesis} it turns
out to be convenient to use the word for a less strict concept.

In the \ctoc context, an ``interval'' can be thought of simply
as a subset of the final flow graph with designated entry and
exit atoms. The intuition is that an interval models the
proto-flowgraph translation of an \ansiC statment. In
\begin{verbatim}
    a = 1; goto foo;
    while ( 1 ) {
      b = 42; goto bar;
    foo:
      c = a+117;
    }
  bar:
    d = b*c ;
\end{verbatim}
the body of the loop naturally translates to a piece of the
proto-flowgraph that is neither connected nor single-entry.

In practise an interval (represented by the C++ class
\syntax{Interval}) is just a shorthand for a pair of
references to an ``entry atom'' and an ``exit atom''.

An interval can have several different forms, shown on figure
\vref{fig:c2coreIntervals}

\begin{figure}\begin{center}\small\begin{tabular}{|r@{~}c@{~}l|p{10cm}|}
\hline
\multicolumn{3}{|l|}{Grammar} & comments \\
\hline
\hline
$I$ & $::=$ & $ [\coreXref\alpha:\coreXref\omega]$ &
        This is the most general form of an interval. The
        successor field $\alpha_\omega$ of $\omega$ is
        is yet empty. \\
\cline{4-4}
&$|$& $ [\coreXref\alpha,-]$ &
        This is equivalent for $[\alpha:\omega_0]$ where
        $\omega_0$ is a fresh \cons{GotoAtom} that does
        not have and will never get any predecessors---only
        this $\omega_0$ does not really exist. \\
\cline{4-4}
&$|$& $ [-] $ &
        The \emph{trivial interval}, equivalent to
        $[\alpha_0:\alpha_0]$ where $\alpha_0$ is
        a fresh \cons{GotoAtom}. \\
\hline
& & $ s $ &
        In the defintions of \coreC generation, a
        \coreC statement can appear where an interval is
        expected. That is a shorthand for $[\alpha:\alpha]$,
        where $\alpha=\cons{StmtAtom}(0,-,s)$ is a new atom. \\
\hline
\end{tabular}\end{center}
\caption{Interval types}
\label{fig:c2coreIntervals}
\end{figure}

A concatenation operator $+$ for intervals is defined by
$[\alpha_1:\omega_1] + [\alpha_2:\omega_2] = [\alpha_1:\omega_2]$
and has the side effect of connecting $\omega_1$ to $\alpha_2$.

\subsubsection{Translating \ansiC statements}
\label{sec:c2coreStatements}
Using the interval concept, \ansiC statements can translate
themselves to \coreC in a natural, recursive way. The translation is
parameterised on
\begin{itemize}
\item [$\coreXref{\alpha_R}$:] the atom to jump to if a
        \syntax{return;} without an expression is encountered.
\item [$\coreXref{\alpha_B}$:] the atom to jump to from a
        \syntax{break;} statement.
\item [$\coreXref{\alpha_C}$:] the atom to jump to from a
        \syntax{continue;} statement.
\item [$e_S$:] the expression that is tested in the innermost
        enclosing \syntax{switch} statement.
\item [$\coreXref{\coreXref{a_D}}$:] when translating the
        \syntax{default:} label, this reference is set to
        point to the atom it translates to. Later, the
        \syntax{switch} translation generates a jump to
        it if control falls through the test chain.
\item [$\coreXref{I_S}$:] references the interval in which
        the test chain of the innermost enclosing
        \syntax{switch} statement is being built. The
        translation of a \syntax{case} label adds a
        test to the end of the chain.
\item [$\coreXref{\vec d_L}$:] the list of declarations
        local \syntax{auto} or \syntax{register} variables
        and temporary variables should be added to when
        translated.
\item [$\coreXref{\vec d_G}$:] the list of declarations
        local \syntax{static} variables should be added to
        when translated.
\end{itemize}
The result of the the translation of a statement is an
interval, together with possible changes to the $a_D$, $I_S$,
$d_L$, and $d_G$ parameters.

The rest of this section summarizes the non-trivial features
of the translation:
\begin{description}
\item[\syntax{goto} and labeled statements:] For each label,
        a single \cons{GotoAtom} is generated to function
        as the target for \syntax{goto} jumps. This atom
        is generated when the label itself is translated,
        or when a \syntax{goto} statement referring to it
        is translated, whichever comes first.
\item[\syntax{switch}, \syntax{case}, and \syntax{default:}]
        The controlling expression in a \syntax{switch} is
        translated into a $\cons{Value}(I,e,-)$ (see section
        \vref{sec:c2coreExprRtn} for a description of different
        ways to translate expressions).

        Then the body is translated with $I_S$ initiallised
        to $I$ and $a_D$ initially empty. The translation of
        the \syntax{case:} labels then incrementally add
        tests to $I_S$ to form the test chain.

        If a \syntax{default:} label has not been met in the
        body, one is generated and appended to the translated
        body $I_B$. Then $\omega_{I_S}$ is connected to the
        \syntax{default} label and the translation of the
        whole \syntax{switch} statement is now $[\alpha_{I_S}:\omega_{I_B}]$.

\begin{figure}
  \begin{frameit}
  \begin{center}
        This figure has not been drawn yet.
  %  \leavevmode
  %  \epsfig{file=struct1.eps,height=0.7\textheight,width=0.7\textwidth}
  \end{center}
  \end{frameit}
  \caption{Translation of \syntax{switch} statements}
  \label{fig:c2coreSwitch}
\end{figure}

\item[\syntax{if} statements:] The expression is translated
        to a \cons{BoolRtn}; its ``then'' and ``else'' branches
        are connected to the translated then and else
        branches\footnote
                {The \ansiC parser generates empty
                else brances in else-less ifs}.
        A \cons{GotoAtom} is generated to serve as the exit
        of the translated statement; both branches are connected
        to it.
\item[loop statements:] The translation is obvious from the
        semantics of \ansiC loops, although there are many
        details to keep track of in the implementation.
\item[\syntax{return} statements] If the \syntax{return} statement
        has an expression, it is translated into
        $\cons{Value}(I,e,-)$. A \cons{ReturnAtom} is generated
        and appended to $I$.

        \syntax{return;} without an expression is translated to
        $[\alpha_R:-]$.
\end{description}

\subsection{Expression translation}
The following description of expression translations depend
on the ``atom'' and ``interval'' concepts defined in sections
\ref{sec:c2coreAtoms} and \ref{sec:c2coreIntervals}. In the
expression translation context an interval is always a
genuine single entry, single exit subgraph.

\subsubsection{Generating \coreC expressions}
\cons{PtrArith}, \cons{PtrCmp}, and \cons{Binary} expressions
are generated with the same constructor in C++. This constructor
can distinguish between the three cases by inspecting the
operands and the operator. In the following, applications
of this constructor are simply notated $\cons{Binary}(e_1,\circ,e_2)$.

The rules for structural sharing of types (section
\ref{sec:corecTypeSharing}) are enforced by the constructors for
\cons{Var}, \cons{PtrArith}, \cons{Member}, \cons{Array}, and
\cons{DeRef} expressions. These constructors do not take a type
parameter, and when one of those expressions are created in
this description, the type parameter is not shown.

Similarly, the constructors for \cons{Unary}, \cons{PtrCmp},
\cons{Binary}, \cons{SizeofType}, and \cons{SizeofExpr} knows
how to construct their types themselves. They are shown with
the type parameter omitted in the following, too.

\cons{Const}, \cons{Null}, and \cons{Cast} expressions are
always constructed with an explicit \coreC type.

\bigskip

\noindent\begin{tabular}{|l|l|}
\hline
\multicolumn{2}{|c|}{Pseudocode notation for building flow graph} \\
\hline
\hline
$x <- \mbox{blah blah blah}$
        & Define $x$ for the rest of the translation step \\
$\installleft{e}{\vec s}$
        & Set the left side of each $s$ to a fresh copy of $e$ \\
$\atomconnect{\omega}{\alpha}$
        & Establish atom $\alpha$ as the successor of $\omega$ \\
$==> \ldots$
        & The result of the translation is ``$\ldots$''. \\
\hline
\end{tabular}

\subsubsection{Translated expression representations}
\label{sec:c2coreExprRtn}
In order to minimize the need for introducing temporary
variables or allocating parts of the \coreC program that
later turn out to be unused, we do not fix a single canonical
shape for the result of translating an expression. Rather,
depending on the shape of an expression it can be translated
into one of the first four forms shown on figure \vref{fig:c2coreExprRtn}.

\begin{figure}\begin{center}\small\begin{tabular}{|r@{~}c@{~}l|p{10cm}|}
\hline
\multicolumn{3}{|l|}{Grammar} & comments \\
\hline
\hline
$\mathcal R$ & $::=$ & $\cons{Value}(I,e,\vec s)$ &
        $e$ is a \coreC expression that evaluates to
        the same value as the original expression,
        provided that $I$ has been executed first.
        $\vec s$ is a list of \coreC statements that
        perform side effects to take place \emph{after}
        $e$ has been evaluated. \\
\cline{4-4}
&$|$& $\cons{Place}(I,e,\vec s)$ &
        This is the same as \cons{Value}, except that
        $e$ evaluates to a pointer to the object that
        the original (lvalue) expression referred to,
        rather than its value. \\
\cline{4-4}
&$|$& $\cons{Statements}(I,\vec{\coreXref s})$ &
        Each possible execution path of $I$ contains
        exactly one of the statements $\vec s$, and the
        value assigned by that statement is the value
        of the original expression. The statements
        have yet empty left-hand sides which are filled
        according to the expression's context. \\
\cline{4-4}
&$|$& $\cons{BoolBranch}(\coreXref \alpha,
                \coreXref{\omega_1},\coreXref{\omega_0})$ &
        The original expression is a boolean value. When
        execution starts at $\alpha$ it comes out at
        $\omega_1$ or $\omega_0$ according to the expressions,
        value. \\
\cline{4-4}
&$|$& $\cons{Ignored}(I)$ &
        The expression's value is ignored.
        The interval $I$ performes its side effects. \\
\hline
\end{tabular}\end{center}
\caption{The different results of expression translation}
\label{fig:c2coreExprRtn}
\end{figure}

The \cons{Ignored} form is never spontaneously generated
by an expression but is required in certain contexts.

\subsubsection{Coercions}
If the context of an expressions requires another form than
the one it actually translates to, it is coerced into the
correct form by one or more of the primitive coercions in
figure \vref{fig:c2coreCoerce}. The sequence of primitive
coercions is determied uniquely by the rules
\begin{itemize}
\item[a)] An $\mathcal R$ is never coerced \emph{away from} a form
        that the context can accept without further coercion.
\item[b)] If generation is performed only when \cons{BoolBranch}
        is the only form accepted by the context.
\item[c)] Assignment generation is not performed where if generation
        applies.
\end{itemize}

\begin{figure}\begin{displaymath}\begin{array}{|l|l|l|}
\hline
\mbox{Name} & \mbox{input} & \mbox{action} \\
\hline
\hline
\mbox{Dereference} & \cons{Place}(I,e,\vec s) &
   ==> \cons{Value}(I,\cons{DeRef}(e),\vec s) \\
\hline
\mbox{Tempvar} & \cons{Statements}(I,\vec s) &
   \mbox{Create temporary variable $d$ of appropriate type} \\
&& \installleft{\cons{Var}(d)}{\vec s} \\
&& ==> \cons{Place}(I,\cons{Var}(d),-) \\
\hline
\mbox{Zero/one} & \cons{BoolBranch}(\alpha,\omega_1,\omega_0) &
   I_1 <- \cons{Assign}(-,\cons{Const}(\cons{Primitive}(-,\syntax{int}),
                        \syntax{1})) \\
&& I_0 <- \cons{Assign}(-,\cons{Const}(\cons{Primitive}(-,\syntax{int}),
                        \syntax{0})) \\
&& \omega <- \mbox{fresh \cons{GotoAtom}} \\
&& \forall i \in \{1,0\}: \atomconnect{\omega_i}{\alpha_{I_i}} \\
&& \forall i \in \{1,0\}: \atomconnect{\omega_{I_i}}{\omega} \\
&& ==> \cons{Statements}([\alpha:\omega],\{s_1,s_0\}) \\
\hline
\mbox{Assignment generation} & \cons{Value}(I,e,\vec s) &
   s' <- \cons{Assign}(-,e) \\
&& ==> \cons{Statements}(I+s'+\vec s,s') \\
\hline
\mbox{If generation} & \cons{Value}(I,e,\vec s) &
   I_{\neq 0} <- \mbox{new interval containg fresh copies of }\vec s \\
&& I_0 <- \mbox{new interval containing the statements }\vec s \\
&& \alpha <- \cons{IfAtom}(e,\alpha_{I_{\neq 0}},\alpha_{I_0}) \\
&& \atomconnect{\omega_I}{\alpha} \\
&& ==> \cons{BoolBranch}(\alpha_I,\omega_{I_{\neq 0}},\omega_{I_0}) \\
\hline
\mbox{Ignore I} & \cons{Value}(I,e,\vec s) &
   ==> \cons{Ignored}(I+\vec s) \\
\mbox{Ignore II} & \cons{Place}(I,e,\vec s) &
   ==> \cons{Ignored}(I+\vec s) \\
\mbox{Ignore III} & \cons{Statements}(I,\vec s) &
   \installleft{-}{\vec s} \\
&& ==> \cons{Ignored}(I) \\
\mbox{Ignore IV} & \cons{BoolBranch}(\alpha,\omega_1,\omega_0) &
   \omega <- \mbox{fresh \cons{GotoAtom}} \\
&& \forall i \in \{1,0\}: \atomconnect{\omega_i}{\omega} \\
&& ==> \cons{Ignored}([\alpha:\omega]) \\
\hline
\end{array}\end{displaymath}
\caption{Coercions for translated expressions}
\label{fig:c2coreCoerce}
\end{figure}

In some cases the
context wants a $\cons{Value}(I,e,-)$, \ie, with an empty
post-statement list. A
post-statement list is emptied by performing assignment generation,
temporary introduction, and dereferencing in that order.

\subsubsection{Translating \ansiC expressions}
\label{sec:c2coreTransExpr}
The translation of the various forms of \ansiC expressions are as
follows:

\begin{figure}\begin{displaymath}\begin{array}{|l|l|}
\hline
\mbox{Primitive} & \mbox{action} \\
\hline
\hline
\mbox{Variable identifier} &
  d <- \mbox{the \coreC declaration for the variable} \\
& ==> \cons{Place}([-],\cons{Var}(d),-) \\
\mbox{Internal function name} &
  d <- \mbox{the \coreC declaration for the function} \\
& ==> \cons{Value}([-],\cons{Var}(d),-) \\
\mbox{External function name} &
  t <- \mbox{fresh copy of \syntax{foo}'s type} \\
& ==> \cons{Value}([-],\cons{Const}(\cons{Pointer}(-,t),\syntax{foo}),-) \\
\mbox{Enumeration constant} &
  \epsilon <- \mbox{(look up the constant's definiton)} \\
& ==> \cons{Value}([-],\cons{EnumConst}(\epsilon),-) \\
\mbox{Well-known constant} &
  t <- \mbox{fresh copy of the constant's type} \\
& ==> \cons{Value}([-],\cons{Const}(t,\syntax{baz}),-) \\
\mbox{Numeric constant} &
  ==> \cons{Value}([-],\cons{Const}(\cons{Primitive}(-,...),\syntax{42}),-) \\
\mbox{String literal} &
  \mbox{Create initialized global array of \syntax{char} $d$} \\
& ==> \cons{Place}([-],\cons{Var}(d),-) \\
\hline
\end{array}\end{displaymath}
\caption{Translating primary expressions}
\label{fig:c2corePrimary}
\end{figure}

\begin{figure}\begin{displaymath}\begin{array}{|l|l|l|}
\hline
\mbox{Operator} & \mbox{\syntax{e} translates to} & \mbox{actions} \\
\hline
\hline
\syntax{e.id} & \cons{Place}(I,e,\vec s) &
   m <- \mbox{(look up the index of the member)} \\
&& ==> \cons{Place}(I,\cons{Member}(e,m),\vec s) \\
\syntax{e->id} & \cons{Value}(I,e,\vec s) &
   m <- \mbox{(look up the index of the member)} \\
&& ==> \cons{Place}(I,\cons{Member}(e,m),\vec s) \\
\hline
\syntax{e}\!\circ\!\circ & \cons{Place}(I,e,\vec s) &
   e' <- \mbox{fresh copy of }e \\
&& e'' <- \mbox{fresh copy of }e \\
&& s' <- \cons{Assign}(e',\cons{Binary}(\cons{DeRef}(e''),\circ,\syntax{1})) \\
&& ==> \cons{Place}(I,e,s' + \vec s) \\
\hline
\circ\!\circ\!\syntax{e} & \cons{Place}(I,e,\vec s) &
   e' <- \mbox{fresh copy of }e \\
&& e'' <- \mbox{fresh copy of }e \\
&& s' <- \cons{Assign}(e',\cons{Binary}(\cons{DeRef}(e''),\circ,\syntax{1})) \\
&& ==> \cons{Place}(I+e',e,\vec s) \\
\hline
\syntax{\&e} & \cons{Place}(I,e,\vec s) &
   ==> \cons{Value}(I,e,\vec s) \\
\syntax{*e} & \cons{Value}(I,e,\vec s) &
   ==> \cons{Place}(I,e,\vec s) \\
\syntax{+e} & {\mathcal R} &
   ==> {\mathcal R} \\
\syntax{-e} & \cons{Value}(I,e,\vec s) &
   ==> \cons{Value}(I,\cons{Binary}(\syntax{-},e),\vec s) \\
\syntax{\char`\~e} & \cons{Value}(I,e,\vec s) &
   ==> \cons{Value}(I,\cons{Binary}(\syntax{\char`\~},e),\vec s) \\
\syntax{!e} & \cons{Value}(I,e,\vec s) &
   ==> \cons{Value}(I,\cons{Binary}(\syntax{!},e),\vec s) \\
& \cons{BoolBranch}(\alpha,\omega_T,\omega_F) &
   ==> \cons{BoolBranch}(\alpha,\omega_F,\omega_T) \\
\hline
\syntax{sizeof e} & \cons{Value}(I,e,\vec s) &
   ==> \cons{Value}([-],\cons{SizeofExpr}(e),-) \\
\syntax{sizeof (T)} & - &
   t <- \mbox{translation of \syntax{T}} \\
&& ==> \cons{Value}([-],\cons{SizeofType}(t),-) \\
\hline
\syntax{(T) e} & \cons{Value}(I,e,\vec s) &
   t <- \mbox{translation of \syntax{T}} \\
&& ==> \cons{Value}(I,\cons{Cast}(t,e),\vec s) \\
\hline
\end{array}\end{displaymath}
\caption{Translating unary expressions}
\label{fig:c2coreUnary}
\end{figure}

\begin{figure}\begin{displaymath}\begin{array}{|l|ll|l|}
\hline
\mbox{Operator} & \multicolumn{2}{l|}{\mbox{operand translations}}
   & \mbox{actions} \\
\hline
\hline
\syntax{=} & \cons{Place}(I,e,\vec s) & \cons{Statements}(I',\vec {s'} &
    \installleft{e}{\vec{s'}} \\
&&& ==>\cons{Place}(I+I',e,\vec s) \\
\hline
\circ\syntax{=} & \cons{Place}(I,e,\vec s) & \cons{Value}(I',e',\vec{s'} &
    e_0 <- \mbox{fresh copy of }e \\
&&& e_1 <- \mbox{fresh copy of }e \\
&&& s_0 = \cons{Assign}(e,\cons{Binary}(\cons{DeRef}(e_0),\circ,e')) \\
&&& ==> \cons{Place}(I+I'+s_0,e_1,\vec s + \vec{s'} \\
\hline
\verb.&&. & \cons{Value}(I,e,\vec s) & \cons{Value}([-],e',-) &
    ==> \cons{Value}(I,\cons{Binary}(e,\syntax{\&\&},e'),\vec s) \\
& \cons{BoolBranch}(\alpha,\omega_T,\omega_F)
& \cons{BoolBranch}(\alpha',\omega'_T,\omega'_F) &
    \atomconnect{\omega_T}{\alpha'} \\
&&& \omega_0 <- \mbox{new \cons{GotoAtom}} \\
&&& \atomconnect{\omega_F}{\omega_0} \\
&&& \atomconnect{\omega'_F}{\omega_0} \\
&&& ==> \cons{BoolBranch}(\alpha,\omega'_T,\omega_0) \\
\verb.||. & & & \mbox{mutatis mutandi} \\
\hline
\syntax{,} & \cons{Ignored}(I_0) & \cons{Value}(I,e,\vec s) &
    ==> \cons{Value}(I_0+I,e,\vec s) \\
& \cons{Ignored}(I_0) & \cons{Place}(I,e,\vec s) &
    ==> \cons{Value}(I_0+I,e,\vec s) \\
& \cons{Ignored}(I_0) & \cons{Statements}(I,\vec s) &
    ==> \cons{Statements}(I_0 + I,\vec s) \\
& \cons{Ignored}(I_0) & \cons{BoolBranch}(\alpha,\omega_1,\omega_0) &
    \atomconnect{\omega_{I_0}}{\alpha} \\
&&& ==> \cons{BoolBranch}(\alpha_I,\omega_1,\omega_0) \\
\hline
\mbox{other } \circ & \cons{Value}(I,e,\vec s) & \cons{Value}(I',e',\vec{s'}) &
    ==> \cons{Value}(I+I',\cons{Binary}(e,\circ,e'),\vec s+\vec{s'}) \\
\hline
\syntax{[]} & \cons{Value}(I,e,\vec s) & \cons{Value}(I',e',\vec{s'}) &
    ==> \cons{Place}(I+I',\cons{Binary}(e,\syntax{+},e'),\vec s+\vec{s'}) \\
\hline
\end{array}\end{displaymath}
\caption{Translating binary expressions}
\label{fig:c2coreBinary}
\end{figure}

\begin{description}
\item[Primary expressions.] The translation is shown on figure
        \vref{fig:c2corePrimary}.
\item[Postfix expressions.] Except for array subscripting (which is
        shown on figure \vref{fig:c2coreBinary}) and function
        call, the translation is shown on figure
        \vref{fig:c2coreUnary}.
\item[Function calls.]
        \begin{displaymath}\begin{array}{|l|l|}
        \hline
        \syntax{e}_0\syntax{(e}_1\ldots\syntax{e}_n\syntax{)} &
          \forall i \leq n: \cons{Value}(I_i,e_i,-) <-
                \mbox{translation of \syntax{e}}_i \\
        & s <- \cons{Call}(-,e_0,e_1\ldots e_n) \\
        & ==> \cons{Statements}(I_0 + \cdots + I_n + s,s) \\
        \hline
        \end{array}\end{displaymath}
\item[Unary operators and casts.] The translation is shown on figure
        \vref{fig:c2coreUnary}.
\item[Binary operators.] The translation is shown on figure
        \vref{fig:c2coreBinary}.
\item[The conditional operator.]
        \begin{displaymath}\begin{array}{|l|l|}
        \hline
        \syntax{e}_0\syntax{?e}_1\syntax{:e}_2 &
          \cons{BoolBranch}(\alpha,\omega_2,\omega_3) <-
                \mbox{translation of \syntax{e}}_0 \\
        & \forall i \in \{1,2\}: \cons{Statements}(I_i,\vec{s_i}) <-
                \mbox{translation of \syntax{e}}_i \\
        & \forall i \in \{1,2\}: \atomconnect{\omega_i}{\alpha_{I_i}} \\
        & \omega <- \mbox{new \cons{GotoAtom}} \\
        & \forall i \in \{1,2\}: \atomconnect{\omega_{I_i}}{\omega} \\
        & ==> \cons{Statements}([\alpha,\omega],\vec{s_2}\cup\vec{s_3}) \\
        \hline
        \end{array}\end{displaymath}
\end{description}

\subsection{Implementation level (1998-11-16)}
Not very much of the \ctoc translation yet works as described
here. The old translation that is still used emits many temporary
variables that the translation described here avoids.

A full implementation of the described translation will have to
wait until the relevant parts of \coreC, especially the orthogonal
left-hand sides of statements, have been implemented.

\begin{itemize}
\item Sections \ref{sec:c2coreAtoms} and \ref{sec:c2coreIntervals}
have already been implemented, except for the \cons{StopperAtom},
and except that intervals are called \syntax{FlowGraph} in the code.
\item The basic principles of the statement translation in
section \ref{sec:c2coreStatements} have been implemented. Instead
of $I_S$ only an $\omega_S$ is part of the parameterization.
$\vec d_L$ and $\vec d_G$ are global variables, not translation
parameters.
\end{itemize}

\end{docpart}

%%% Local Variables: 
%%% mode: latex
%%% TeX-master: "cmixII"
%%% End: 

% Edit Mode: -*- LaTeX -*-
% File: callmode.tex
% Time-stamp: <98/06/08 12:33:53 panic>
% $Id: callmode.tex,v 1.2 1999/05/13 10:58:38 skalberg Exp $

\providecommand{\docpart}{\renewenvironment{docpart}{}{}
\end{docpart}
\documentclass[twoside]{cmixdoc}
%\bibliographystyle{apacite}

\makeatletter
\@ifundefined{@title}{\title{\cmix-documentation}}{}
\@ifundefined{@author}{\author{The \cmix{} Team}}%
{\expandafter\def\expandafter\@realauthor\expandafter{\@author}%
\author{The \cmix{} Team\\(\@realauthor)}}
\makeatother

\AtBeginDocument{%
\markboth{\hfill\today\quad\timenow\hfill\llap{\cmix\ documentation}}
{\hfill\today\quad\timenow\hfill}}

\renewcommand{\sectionmark}[1]{\markboth
{\hfill\today\quad\timenow\hfill\llap{\cmix\ documentation}}
{\rlap{\thesection. #1}\hfill\today\quad\timenow\hfill}}

%\newboolean{separate}
%\setboolean{separate}{true}

\renewenvironment{docpart}{\begin{document}}%
                          {\bibliography{cmixII}\printindex
                           \end{document}}
\begin{document}\shortindexingon
}
\title{Annotation of External Functions}
\author{Jens Peter Secher}
\begin{docpart}
\maketitle
%\documentclass[a4paper]{cmixdoc}

\section{Call-modes}
\label{sec:callmodes}

To avoid making worst-case assumptions about the effects of calls to
\emph{external} functions, the user can give a description to \cmix on
how (a call to) a particular external function behaves by annotating
external functions and/or individual call-sites.

\subsection{Function description}
The following \emph{function} annotations exists:

\begin{quote}
\begin{description}
\item[\texttt{pure}] means that the function is a function in a
  mathematical sense.  Example: \texttt{strcmp()}.
\item[\texttt{stateless}] means that the function possibly performs
  side-effects on its arguments, but otherwise does not depend on any
  external objects. Example: \texttt{strcpy()}.
\item[\texttt{ROstate}] means that the value returned depend on the
  state of some external object, but that this external state is not
  changed by the function. The function may, however, perform
  side-effects on its arguments. Example: \texttt{feof()}.
\item[\texttt{RWtate}] means that the function might change the state
  of some externally defined object, or perform side-effects on its
  arguments.  Example: \texttt{scanf()}. This is the default if an
  external function is unannotated.
\end{description}
\end{quote}
These categories form a proper-subset chain:
\[ \mathtt{pure} \subset \mathtt{stateless} \subset \mathtt{ROstate}
   \subset \mathtt{RWstate}
\] 

These annotations are primary a vehicle to describe standard
functions: \cmix is shipped with a set of shadow header files that
conform with the ISO/ANSI standard. Functions declared in these header
files are pre-annotated with the above mentioned annotations.

The annotations can, however, also be used by users when
{\specialis}ation of non-monolithic programs --- see the User Manual
section on separate {\specialis}ation.

\subsection{Call-site description}
Each call-site inherits the description of 
The following \emph{call-site} annotations exists:

\begin{quote}
\begin{description}
\item[\texttt{spectime}] instructs \cmix to make the call at specialization
  time.
\item[\texttt{residual}] instructs \cmix to make the call at residual time.
\item[(no annotation)] makes \cmix choose the call-time as it pleases.
\end{description}
\end{quote}

Any call-site can thus be classified according the above categories,
which are then used by the various analysis. The following subsections
will describe how each analysis responds to calls to external
functions.

\subsection{PA}
The points-to analysis maintains two special pools of objects: the
spectime pool and the residual pool. Any object can be contained in at
most one of these pools, depending on \emph{when} the object is alive.
Calls to external functions are classified thus:

\begin{tabular}{l|c|c|c|c|} 
                  & \texttt{pure} & \texttt{stateless} &
                  \texttt{ROstate} & \texttt{RWtate} \\
\hline
\texttt{spectime} & & & 3 & 4 \\
\cline{1-1}\cline{4-5}
\texttt{residual} & 1 & 2 & & \\
\cline{1-1}
(\cmix chooses) & & & 
                  \raisebox{1.5ex}[0cm][0cm]{5} & \raisebox{1.5ex}[0cm][0cm]{6} \\
\hline
\end{tabular}

\begin{enumerate}
\item Any object (with the right type) reachable from the arguments
  can be returned.
\item Any object (with the right type) reachable from the arguments
  can be returned. Any pointer object reachable from the arguments can
  be written to and thus point to any object (with the right type)
  reachable from the arguments.  
\item Any pointer object reachable from the arguments can be written
  to and thus point to any object (with the right type) in the
  spectime pool. 
\item Any pointer object reachable from the arguments can be written
  to and thus point to any object (with the right type) in the
  spectime pool. Furthermore, any object reachable from the
  arguments is added to the spectime pool.
\item Like 3, but with ``spectime pool'' replaced by ``residual pool''.
\item Like 4, but with ``spectime pool'' replaced by ``residual pool''.
\end{enumerate}


\subsection{BTA}
The binding-time analysis uses the classification of calls to external
functions thus:

\begin{tabular}{l|c|c|c|c|} 
                  & \texttt{pure} & \texttt{stateless} &
                  \texttt{ROstate} & \texttt{RWtate} \\
\hline
\texttt{spectime} & \multicolumn{3}{c|}{1} & 2 \\
\hline
\texttt{residual} & \multicolumn{4}{c|}{}  \\
\cline{1-3}
(\cmix chooses) & \multicolumn{2}{c|}{4} & 
                  \multicolumn{2}{c|}{\raisebox{1.5ex}[0ex][0ex]{3\quad}} \\
\hline
\end{tabular}

\begin{enumerate}
\item The call is made at {\specialis}ation time and is allowed to be
  under dynamic control. All arguments must be completely spectime.
\item The call is made at {\specialis}ation time and is \emph{not}
  allowed to be under dynamic control. All arguments must be
  completely spectime.
\item The call is made at residual time. All arguments must be
  residual or liftable.
\item All arguments are forced to be either completely spectime or
  completely residual. The call is made accordingly.
\end{enumerate}


\subsection{Dataflow}
The dataflow analysis uses the classification of calls to external
functions thus:


\begin{tabular}{l|c|c|c|c|} 
                  & \texttt{pure} & \texttt{stateless} & \texttt{ROstate} & \texttt{RWtate} \\
\hline
\texttt{spectime} & & & 3 & 4 \\
\cline{1-1}\cline{4-5}
\texttt{residual} & 1 & 2 & & \\
\cline{1-1}
(\cmix chooses) &  & & \raisebox{1.5ex}[0cm][0cm]{5} & \raisebox{1.5ex}[0cm][0cm]{6} \\
\hline
\end{tabular}


\begin{enumerate}
\item The call might read from objects reachable from the arguments.
\item The call might write to or read from objects reachable from the
  arguments. 
\item All objects in the spectime pool might be read. All objects
  reachable from the arguments might be read or written.
\item All objects in the spectime pool and objects reachable from the
  arguments might be read or written.
\item Like 3, but with ``spectime pool'' replaced by ``residual pool''.
\item Like 4, but with ``spectime pool'' replaced by ``residual pool''.
\end{enumerate}

\end{docpart}

%%% Local Variables: 
%%% mode: latex
%%% TeX-master: "cmixII"
%%% End: 

% Edit Mode: -*- LaTeX -*-
% File: pa.tex
% Time-stamp: <98/06/08 12:33:53 panic>
% $Id: pa.tex,v 1.3 1999/06/03 17:24:32 jpsecher Exp $

\providecommand{\docpart}{\renewenvironment{docpart}{}{}
\end{docpart}
\documentclass[twoside]{cmixdoc}
%\bibliographystyle{apacite}

\makeatletter
\@ifundefined{@title}{\title{\cmix-documentation}}{}
\@ifundefined{@author}{\author{The \cmix{} Team}}%
{\expandafter\def\expandafter\@realauthor\expandafter{\@author}%
\author{The \cmix{} Team\\(\@realauthor)}}
\makeatother

\AtBeginDocument{%
\markboth{\hfill\today\quad\timenow\hfill\llap{\cmix\ documentation}}
{\hfill\today\quad\timenow\hfill}}

\renewcommand{\sectionmark}[1]{\markboth
{\hfill\today\quad\timenow\hfill\llap{\cmix\ documentation}}
{\rlap{\thesection. #1}\hfill\today\quad\timenow\hfill}}

%\newboolean{separate}
%\setboolean{separate}{true}

\renewenvironment{docpart}{\begin{document}}%
                          {\bibliography{cmixII}\printindex
                           \end{document}}
\begin{document}\shortindexingon
}
\title{Pointer Analysis for Core C}
\author{Jens Peter Secher}
\begin{docpart}
\maketitle

\mathligsoff
\begin{inferencesymbols}
\renewcommand{\predicate}[1]{$ #1 $}

\begin{center}
  \fbox{\huge\textsl{THIS SECTION IS OUT-OF-DATE}}    
\end{center}

\section{Pointer analysis}
\label{sec:PointerAnalysis}

\subsection{Preliminaries}
\label{sec:PApreliminaries}

Core C is the internal representation of C programs in \cmix (cf.\ 
section~\vref{sec:TheCoreCLanguage}). The purpose of the pointer
analysis is to make a \emph{safe} approximation of which memory
locations pointers \emph{may} point to during program
execution~\cite{Andersen:1994:ProgramAnalysisAndSpecialization}. A
piece of memory will from here on be denoted an \emph{object} or
\emph{abstract location}.  Consider this program where \texttt{d} is
dynamic:

\begin{verbatim}
  int f(int d)
  {
    int x = 1;
    int *p = &x;
    *p = d;
    ...
  }
\end{verbatim}

\noindent
If no points-to information was present, the binding-time analysis
would wrongly assume that \texttt{x} was static, which is not the
case since a dynamic value is assigned to \texttt{x} through
the pointer \texttt{p}.

\subsection{The Analysis}
\label{sec:PAResult}
Given the domains of objects $\delta \in \DAloc$, definitions $d \in
\DD$, basic blocks $b \in \DB$, statements $s \in \DS$, control
statements (jumps) $j \in \DJ$ and expressions $e \in \DE$ described
in sections~\vref{sec:TheCoreCLanguage}, the point-to analysis also
uses the domains: sets of objects $\Delta \in \powset(\DAloc)$.

We will define a function $\OPT : \DD ->\Delta$ for determining
which objects may be pointed to by objects and expressions: $\delta'
\in \OPT(\delta)$ means that $\delta$ may point to $\delta'$; Likewise
for $\OPT(e)$.

For each object $\delta$, we will associate a \emph{points-to set}
$\Delta_{\delta}$. Informally $\Delta_p$ means the set of objects
pointed to by variable \texttt{p}. Likewise, for each expression $e$,
we will associate a points-to set $\Delta_{e}$.

The analysis proceeds in two phases: \emph{constraint generation} by
inference on the subject program, and \emph{constraint solving} by
fix-point iteration. Constraints have one of these forms:

\begin{description}
\item[$\Delta_p \supseteq \{x\}$] which means that variable $p$ may
  point to object $x$, as in \texttt{p = \&x;}
\item[$\Delta_p \supseteq \star\Delta_q$] which means that variable
  $p$ may point to any object that the set of objects $\Delta_q$ can
  point to, where $\Delta_q$ is set of the objects that variable $q$
  can point to, as in \texttt{ int **q; int *p; p = *q;}
\item[$\star\Delta_{f\!p} \supseteq \Delta_q\leftarrow
  (\Delta_{p_1},\dots,\Delta_{p_n})$] which means that all functions
  pointed to by $f\!p$ may be called with any tuple of objects in
  $\Delta_{p_1} \times \dots \times \Delta_{p_n}$. Furthermore,
  variable $q$ may point to any object that functions $f\!p$ may
  return a pointer to (via \syntax{return} statements), as in
  \texttt{q = fp(p1,...,pn);}
\end{description}

\noindent
Solving these constraints then involves updating sets of objects until
a least fix-point is reached.

In Core C, an object is (an approximation of) a set of memory
locations used during program execution, \ie a declaration \syntax{int
  x} in a possible recursive function $f$ represents all instances of
$x$.  Furthermore, we assume that each dynamic allocation point
originating from a \syntax{calloc} and string constants have been
given a unique placeholder declaration. Likewise, the transformation
from C to \coreC associates a special declaration with each array
object so that it can represent set of objects the array can contain.

\subsubsection{Constraint generation}
The inference rules create a set of constraints $\mathcal{C}$, written
$[c_1,c_2,\dots,c_n]$.  Constraint sets are unioned by juxtaposition:
$\mathcal{C}\mathcal{C}'$ means $\mathcal{C}$ extended with
$\mathcal{C}'$. Types are

\bigskip
% -------- Expressions ---------

\begin{figure}[htb]
  \begin{center}
    \[
    \begin{array}{ll}
      \textnormal{Expression } e & \Delta_e \\\hline
      \CRval(\delta)             & \Delta_e \supseteq \{\delta\} \\
      \CLval(\delta)            & \alpha = \mbox{new},
                                  \Delta_\alpha \supseteq \{\delta\},
                                  \Delta_e \supseteq \{\alpha\} \\
      \CConst(c)                & \{\}       \\
      \CUnary(\Vuop,e_1)        & \OU(e_1) \cup \OL(e) \\
      \CBinary(e_1,\Vbop,e_2)   & \OU(e_1) \cup \OU(e_2) \\
      \CCast(t,e_1)             & \OU(e_1)   \\
      \CMember(e_1,\varphi)     & \OU(e_1) \cup \OL(e) \\
    \end{array}
    \]
    \caption{Function $\OU$ for determining which objects may
      possibly be used in an expression}
    \label{fig:PAsomething}
  \end{center}
\end{figure}

\[\inference[const]
  {}
  {c : \Delta,[\Delta\supseteq\{\}]}
\qquad
  \inference[lvar]
  {}
  {x : \Delta, [\Delta\supseteq\{\&\},\Delta_\&\supseteq\{x\}]}
\]

\[\inference[rvar]
  {}
  {x : \Delta, [\Delta\supseteq\{x\}]}
\qquad
  \inference[struct]
  {e : \Delta' , \mathcal{C}}
  {e.i : \Delta, \mathcal{C}[\Delta\supseteq \Delta'\cdot i]}
\]

\[
  \inference[deref]
  {e : \Delta',\mathcal{C}}
  {\syntax{*}e : \Delta,\mathcal{C}[\Delta\supseteq\star\Delta']}
\qquad
  \inference[addr]
  {e : \Delta' ,\mathcal{C}}
  {\syntax{\&}^\alpha e : \Delta, \mathcal{C}
    [\Delta\supseteq\{\alpha\},\Delta_\alpha\supseteq\Delta']}
\]

\[\inference[unary]
  {e : \Delta,\mathcal{C}}
  {o_{uop}\ e : \Delta,\mathcal{C}}
\qquad
  \inference[binary]
  {e_1 : \Delta_1,\mathcal{C}_1 & e_2 : \Delta_2,\mathcal{C}_2}
  {e_1\ o_{bop}\ e_2 : \Delta,\mathcal{C}_1\mathcal{C}_2
    [\Delta\supseteq\Delta_1,\Delta\supseteq\Delta_2]}
\]

\[\inference[typesize]
  {}
  {\syntax{sizeof}(\tau) : \Delta,[\Delta\supseteq\{\}]}
\qquad
  \inference[expsize]
  {}
  {\syntax{sizeof}(e) : \Delta,[\Delta\supseteq\{\}]}
\]

\[\inference[cast]
  {e : \Delta',\mathcal{C} & (\mathcal{C}',\Delta)=\mbox{Cast}(\tau,\mbox{typeOf}(e),\Delta')}
  {\syntax{(}\tau\syntax{)} e : \Delta,\mathcal{C}\mathcal{C}'}
\]

\noindent
where the Cast function is defined below
% -------- Cast ---------

\[\begin{array}{l}
  \mbox{Cast}(to,from,\Delta) = \mbox{case}\ (to,from)\ \mbox{of}\\
  \begin{array}{lll}
  (\langle*\tau\rangle,\langle*\langle\mbox{struct S}\rangle\rangle) 
   & => \left\{\begin{array}{ll}
        ([\Delta'\supseteq \Delta\cdot 1],\Delta')
          & \mbox{if}\ \mbox{typeOf}(S.1)=\tau \\
        (\epsilon,\Delta)
          & \mbox{if}\ \mbox{typeOf}(S.1)\not=\tau \\
        \end{array}\right.\\
  (\langle*\tau\rangle,\langle[n]\tau\rangle)
   & => (\epsilon,\Delta) \\
  (\_\,,\_) & => (\epsilon,\Delta) \\
  \end{array}
  \end{array}
\]


% -------- Statements ---------

\noindent\fbox{Statements}\hfill\fbox{$s : \mathcal{C}$}

\[\inference[branch]
  {}
  {\syntax{if(}e\syntax{)}\ bb_1\ bb_2 : []}
\qquad
  \inference[jump]
  {}
  {\syntax{goto}\ bb : []}
\]


\[\inference[assign1]
  {e : \Delta,\mathcal{C}}
  {x\ \syntax{=}\ e : \mathcal{C}[\Delta_x\supseteq\star\Delta]}
\qquad
  \inference[assign2]
  {e : \Delta , \mathcal{C}}
  {\syntax{*}x\ \syntax{=}\ e : \mathcal{C}[\star\Delta_x\supseteq\star\Delta]}
\]

\[\inference[assign3]
  {e : \Delta' , \mathcal{C}}
  {x.i\ \syntax{=}\ e : \mathcal{C}[\Delta\supseteq\{x\},\Delta\cdot i\supseteq\star\Delta']}
\]

\[\inference[assign4]
  {e : \Delta , \mathcal{C}}
  {\syntax{(*}x\syntax{)}.i\ \syntax{=}\ e :
    \mathcal{C}[\Delta_x\cdot i\supseteq\star\Delta]}
\]

\[\inference[call]
  {f\!p : \Delta_{f\!p},\mathcal{C}  & e_i : \Delta_i,\mathcal{C}'_i}
  {x\ \syntax{=}\ f\!p \syntax{(}e_1,\dots,e_n\syntax{)} :
    \mathcal{C}\mathcal{C}'_i[\Delta_{f\!p}\supseteq\Delta_x
           \leftarrow (\Delta_1,\dots,\Delta_n)]}
\]

\[\inference[alloc]
  {}
  {x\syntax{ = calloc}^\alpha(\tau,e) :
    [\Delta_x\supseteq\{\alpha\},\Delta_\alpha\supseteq\{\}]}
\qquad
  \inference[free]
  {}
  {\syntax{free}\ e : []}
\]

\[\inference[return]
  {e : \Delta,\mathcal{C} & f\mbox{ is surrounding function}}
  {\syntax{return}\ e : \mathcal{C}[\Delta_{f_0}\supseteq\star\Delta]}
\]

\subsubsection{Constraint solving}
[Describe pre-normalization]
[The constraints make a term rewrite system]

\bigskip
% -------- Solving ---------

\begin{tabbing}
foreach $\Delta \in \nabla$ do touched$(\Delta):=$true  \\
repeat \= \\
       \> fixpoint $:=$ true \\
       \> foreach $\Delta \in \nabla$ do \= changed$(\Delta) := $ touched$(\Delta)$  \\
       \>                                \> touched$(\Delta) := $ false\\
       \> foreach $c \in \mathcal{C}$  \\
       \> do \= \\
       \>    \> case $c$ of \\
       \>    \> $\begin{array}{rclcl}
                \Delta&\supseteq&\{x\} &=>& \mbox{update}(\Delta,\{x\}) \\
                \Delta&\supseteq&\Delta' &=>& \mbox{update}(\Delta,\Delta') \\
                \Delta&\supseteq&\star\Delta' &=>& \mbox{update}(\Delta,\mbox{indr}(\Delta')) \\
                \Delta&\supseteq&\Delta'\cdot i &=>& \mbox{foreach}\
                s\in\Delta' \mbox{do update}(\Delta,\mbox{struct}(s,i)) \\
                \star\Delta&\supseteq&\star\Delta' &=>&
                                                     \mbox{if changed}(\Delta)\mbox{ or changed}(\Delta')\ \mbox{then} \\ 
                           &         &             &  & \mbox{foreach}\ x\in\Delta
                                                     \mbox{ do update}(\Delta_x,\mbox{indr}(\Delta')) \\
                \Delta\cdot i&\supseteq&\star\Delta' &=>&
                                      \mbox{foreach}\ s\in\Delta:
                                      \mbox{do update}(\mbox{struct}(s,i),\mbox{indr}(\Delta')) \\
                \Delta&\supseteq&\Delta_x \leftarrow
                                (\Delta_1\dots\Delta_n)&=>& \mbox{if changed}(\Delta)\ \mbox{then}\\
                           &&&& \mbox{foreach}\ (\Delta_1'\dots\Delta_n')->\Delta_{f_0}\in\mbox{params}(\Delta):\\
                           &&&& \mathcal{C} := \mathcal{C} \cup [\Delta_1'\supseteq\star\Delta_1,\dots,
                                \Delta_n'\supseteq\star\Delta_n,\Delta_x\supseteq\Delta_{f_0}] \\
                           &&&& \mbox{fixpoint:=false}\\
                \end{array}$\\
while fixpoint $=$ false \\
\\
indr$(\Delta) = \bigcup_{x\in\Delta}{\Delta_x}$ \\
\\
update$(\Delta,\Delta') = \mbox{foreach}\ x\in\Delta'\ $
                          do if $(x\not\in\Delta)$\ 
                          then\ \=$\Delta:=\Delta\cup\{x\}$ \\
                                       \>touched$(\Delta):=$ true\\
                                       \>fixpoint$:=$ false\\
\\
params$(\Delta) =\ $\=$\sigma:=\{\} $\\
          \> $\mbox{foreach}\ f\in\Delta\ \mbox{do}$\
             \=$\sigma:=\sigma\cup (\Delta_1\dots\Delta_n)->\Delta_{f_0}$ \\
          \> \>$\mbox{where}\ \Delta_i\ \mbox{represents parameter}\ d_i\ 
                             \mbox{in}\ f$ \\
          \> return $\sigma$ \\
\\
struct$(s,i) = \mbox{the}\ i\mbox{'th member of struct}\ s$ \\

\end{tabbing}

\subsection{Effects on other stuff}
\begin{itemize}
\item After the pointer analysis, struct types have to flow together
  because the type of a struct identifies what instance it is.
\item Free expressions can now be traversed to mark alloc expressions
  to be either freed or not freed.
\end{itemize}

\end{inferencesymbols}
\mathligson

\subsection{Implementation level (01.01.1998)}
\label{sec:PAImplementationLevel}
\index{implementation level!points-to analysis}

The in-use analysis has been implemented according to this chapter with the
following changes:

In the implementation the set $\Delta_p$ is explicitly connected with
the declaration of $p$; the formal definition assumes this implicitly.
This means that we can regard $\Delta_x$ as a unique representative of
the object $x$ and thus only work on $\Delta$-sets that contain
\emph{references} to other $\Delta$-sets.  When such a system
stabilizes (\ie the constraint solving reaches the least fix-point),
the result of the analysis is extracted from the relevant sets; \eg if
variable $p$ is connected with the set $\Delta_p =
\{\Delta_x,\Delta_y\}$, it means that $p$ may point to the objects
associated with $\Delta_x$ and $\Delta_y$, \ie the variables $x$ and
$y$.

\epsfig{file=extfunSE.eps}


\end{docpart}

%%% Local Variables: 
%%% mode: latex
%%% TeX-master: "cmixII"
%%% End: 

% File: calls.tex
% Time-stamp: 
% $Id: calls.tex,v 1.1 1999/03/02 17:34:28 jpsecher Exp $

\providecommand{\docpart}{\renewenvironment{docpart}{}{}
\end{docpart}
\documentclass[twoside]{cmixdoc}
%\bibliographystyle{apacite}

\makeatletter
\@ifundefined{@title}{\title{\cmix-documentation}}{}
\@ifundefined{@author}{\author{The \cmix{} Team}}%
{\expandafter\def\expandafter\@realauthor\expandafter{\@author}%
\author{The \cmix{} Team\\(\@realauthor)}}
\makeatother

\AtBeginDocument{%
\markboth{\hfill\today\quad\timenow\hfill\llap{\cmix\ documentation}}
{\hfill\today\quad\timenow\hfill}}

\renewcommand{\sectionmark}[1]{\markboth
{\hfill\today\quad\timenow\hfill\llap{\cmix\ documentation}}
{\rlap{\thesection. #1}\hfill\today\quad\timenow\hfill}}

%\newboolean{separate}
%\setboolean{separate}{true}

\renewenvironment{docpart}{\begin{document}}%
                          {\bibliography{cmixII}\printindex
                           \end{document}}
\begin{document}\shortindexingon
}
\begin{docpart}

\section{Function-call analysis}
\label{sec:Calls}\index{calls}

\end{docpart}
%%% Local Variables: 
%%% mode: latex
%%% TeX-master: "cmixII"
%%% End: 

% File: locals.tex
% Time-stamp: 
% $Id: locals.tex,v 1.1 1999/03/02 17:34:39 jpsecher Exp $

\providecommand{\docpart}{\renewenvironment{docpart}{}{}
\end{docpart}
\documentclass[twoside]{cmixdoc}
%\bibliographystyle{apacite}

\makeatletter
\@ifundefined{@title}{\title{\cmix-documentation}}{}
\@ifundefined{@author}{\author{The \cmix{} Team}}%
{\expandafter\def\expandafter\@realauthor\expandafter{\@author}%
\author{The \cmix{} Team\\(\@realauthor)}}
\makeatother

\AtBeginDocument{%
\markboth{\hfill\today\quad\timenow\hfill\llap{\cmix\ documentation}}
{\hfill\today\quad\timenow\hfill}}

\renewcommand{\sectionmark}[1]{\markboth
{\hfill\today\quad\timenow\hfill\llap{\cmix\ documentation}}
{\rlap{\thesection. #1}\hfill\today\quad\timenow\hfill}}

%\newboolean{separate}
%\setboolean{separate}{true}

\renewenvironment{docpart}{\begin{document}}%
                          {\bibliography{cmixII}\printindex
                           \end{document}}
\begin{document}\shortindexingon
}
\begin{docpart}

\section{Truly local variables}
\label{sec:TrulyLocals}\index{truly local variables}

\end{docpart}
%%% Local Variables: 
%%% mode: latex
%%% TeX-master: "cmixII"
%%% End: 

% File: dataflow.tex
% Time-stamp: 
% $Id: dataflow.tex,v 1.6 1999/07/19 12:01:00 makholm Exp $

\providecommand{\docpart}{\renewenvironment{docpart}{}{}
\end{docpart}
\documentclass[twoside]{cmixdoc}
%\bibliographystyle{apacite}

\makeatletter
\@ifundefined{@title}{\title{\cmix-documentation}}{}
\@ifundefined{@author}{\author{The \cmix{} Team}}%
{\expandafter\def\expandafter\@realauthor\expandafter{\@author}%
\author{The \cmix{} Team\\(\@realauthor)}}
\makeatother

\AtBeginDocument{%
\markboth{\hfill\today\quad\timenow\hfill\llap{\cmix\ documentation}}
{\hfill\today\quad\timenow\hfill}}

\renewcommand{\sectionmark}[1]{\markboth
{\hfill\today\quad\timenow\hfill\llap{\cmix\ documentation}}
{\rlap{\thesection. #1}\hfill\today\quad\timenow\hfill}}

%\newboolean{separate}
%\setboolean{separate}{true}

\renewenvironment{docpart}{\begin{document}}%
                          {\bibliography{cmixII}\printindex
                           \end{document}}
\begin{document}\shortindexingon
}
\begin{docpart}

\newcommand{\Wdef}{\ensuremath{W_{\!\mathrm{def}}}}
\newcommand{\Wmay}{\ensuremath{W_{\!\mathrm{may}}}}
\newcommand{\subdecl}[1]{\ensuremath{\widehat{#1}}}
\newcommand{\exactlyone}{\mbox{one}}
\newcommand{\keyword}[1]{\textbf{#1}}
\newcommand{\Use}{\ensuremath{U}}
\newcommand{\trulylocal}{\mbox{truelocals}}

\section{Dataflow Analysis}
The purpose of the dataflow analysis is to make a safe approximation
of how and where objects are used. The result of the analysis is used
during {\specialis}ation to avoid unnecessary {\memois}ation of
spectime data.

[Example: why in-use info makes a difference] [Example: why
do-not-alter info makes a difference]

\subsection{Write sets}
An object is \emph{written} when a statement writes a value to it. The
purpose of the write analysis is to calculate which objects are written
in each program point. The calculation of write sets are based on the
result of the pointer analysis (section~\ref{sec:PointerAnalysis})
which only gives imperfect knowledge about reference patterns. We are
therefore not able to calculate precise write information.

[Example: If $\OPT(p) |-> \{ x, y \}$, we are not able to say which
object \emph{definitely} gets written by the assignment \texttt{*p =
  1}; we can, however, say that the assigment \emph{may} write
\texttt{x} or \texttt{y}.]

We will calculate two functions that map each program point in the
program to two possibly different write sets: a definite-write set and
a may-write set, denoted $\Wdef$ and $\Wmay$, respectively. The
definite-write function is used to calculate which objects are in use
in each function.

The intuitive interpretation of $\Wdef$ and $\Wmay$ is that they
describe what may happen until the current function returns.

The may-write function is used to be sure to
{\memois}e spectime objects that may be written to by a function.
Even it the objects are not read by the function, {\memois}ation is
necessary in order to correctly replictate the side effects when the
function is later shared.

Let $D$ be the set of all objects in the program and define the
function $\subdecl{\bullet} : \powset(D) -> \powset(D)$ that
(recursively) calculates all subdeclarations for each declaration in
the argument set, \ie it calculates which objects a declaration
encompasses.

[Example: the objects in a struct are all the members and all members
of these members, and so forth. Consider the definition \texttt{struct
  S \{ int x, y; \} s;}. $\subdecl{\{s\}} = \{s, s.x, s.y\}$. ]



Define the function $\exactlyone : \powset(D) -> \powset(D)$ that
``rejects'' imprecise information:
\[
\begin{array}{lcl}
\exactlyone(\{d\}) &=& 
  \begin{array}[t]{ll}
    \keyword{if} & d \mbox{ is contained in an array, or} \\
                 & d \mbox{ is a heap-allocated object, or} \\
                 & d \mbox{ is a local variable but not truly-local} \\                 
    \keyword{then} & \emptyset \\
    \keyword{else} & \{ d \} \\                 
  \end{array} \\
  \exactlyone(\_\!\_) &=& \emptyset \\
\end{array}
\]

The write-functions can then be defined by the following recursive equations:
\[
\begin{array}{lll}
  \hfil \bullet & \hfil \Wdef(\bullet) & \hfil \Wmay(\bullet) \\
  \hrulefill & \hrulefill & \hrulefill \\
  \cons{If}(e,\coreXref{b_{\mathrm{then}}},
  \coreXref{b_{\mathrm{else}}}) & \Wdef(b_{\mathrm{then}}) \cap
  \Wdef(b_{\mathrm{else}}) & \Wmay(b_{\mathrm{then}}) \cup
  \Wmay(b_{\mathrm{else}}) \\
  \cons{Goto}(\coreXref b) & \Wdef(b) &
  \Wmay(b) \\
  \cons{Return}(e?) & \emptyset & \emptyset \\
  \cons{Assign}(e,e') & \subdecl{\exactlyone(\OPT(e))} &
  \subdecl{\OPT(e)} \\
  \cons{Call}(e?,\gamma,e_f,\vec e\,)\footnotemark
  & \subdecl{\exactlyone(\OPT(e))}
  \cup {\displaystyle \bigcap_{d_i \in \OPT(e_f)}} \Wdef(d_i) &
  \subdecl{\OPT(e)} \cup {\displaystyle \bigcup_{d_i \in \OPT(e_f)}} \Wmay(d_i) \\
  \cons{Alloc}(e,d,e') & \subdecl{\exactlyone(\OPT(e))} & \subdecl{\OPT(e)} \\
  \cons{Free}(,e') & \emptyset & \emptyset \\
  b = (\vec s,j) & \Wdef(j) \cup {\displaystyle \bigcup_{s_i \in \vec s}}
  \Wdef(s_i) & \Wmay(j) \cup {\displaystyle \bigcup_{s_i \in \vec s}}
  \Wmay(s_i) \\
  \cons{Fun}(id,t,\vec d_p,\vec d_v,\vec b\,) & \Wdef(b_1) \setminus
  \trulylocal(\bullet) &
  \Wmay(b_1) \setminus  \trulylocal(\bullet) \\
\end{array}
\]\footnotetext{If the return value is thrown away, the parts refering to
  $\OPT(e)$ are replaced by $\emptyset$. For an externally defined
  function, $\Wdef$ and $\Wmay$ is $\emptyset$ and the $\OPT$-closure
  of variables that are passed to external functions, respectively.}
The \trulylocal function defines the subset of the local variables
that are known to not be aliased between recursive calls to the function.
See section~\ref{sec:TrulyLocals}. 

\subsection{In-use sets}
The in-use information is used to avoid {\specialis}ing a function
or a program point with respect to spectime data that it does not use.

We will now define an in-use function $\Use$ that maps every \coreC
construct to a set of objects. For expressions, the in-use function
can be calculated directly from the points-to information:
\[
\begin{array}{ll}
  \hfil \bullet & \hfil \Use(\bullet) \\
  \hrulefill & \hrulefill \\
\cons{Var}(t,\coreXref d) & \emptyset \\
\cons{EnumConst}(t,\coreXref\epsilon) & \emptyset \\
\cons{Const}(t,c) & \emptyset \\
\cons{Null}(t) & \emptyset \\
\cons{Unary}(t,\diamond,e) & \Use(e) \\
\cons{PtrArith}(t,e_1,\circ,e_2) &  \Use(e_1) \cup \Use(e_2) \\
\cons{PtrCmp}(t,e_1,\circ,e_2) &  \Use(e_1) \cup \Use(e_2) \\
\cons{Binary}(t,e_1,\circ,e_2) &  \Use(e_1) \cup \Use(e_2) \\
\cons{Member}(t,e,m) &  \Use(e) \\
\cons{Array}(t,e) &  \Use(e) \\
\cons{DeRef}(t,e) &  \Use(e) \cup \subdecl{\OPT(e)} \\
\cons{Cast}(t,e) &  \Use(e) \\
\cons{SizeofType}(t,t') & \emptyset \\
\cons{SizeofExpr}(t,e) & \Use(e) \\
\end{array}
\]

For program points, the in-use function is defined by a set of
recursive equations. For each statement $s$, let $\Use(s)$ be the set
objects that are in use immediately before $s$ is executed, and let
$\overline{\Use}(s)$ be the set objects that are in use immediately
after $s$ is executed. [Example: Consider a statement $s$ and its
immediate successor $s'$; then $\overline{\Use}(s) = \Use(s')$.
\[
\begin{array}{ll}
  \hfil \bullet & \hfil \Use(\bullet)  \\
  \hrulefill & \hrulefill \\
  \cons{If}(e,\coreXref{b_{\mathrm{then}}},
  \coreXref{b_{\mathrm{else}}}) & \Use(e) \cup \Use(b_{\mathrm{then}}) \cup
  \Use(b_{\mathrm{else}}) \\
  \cons{Goto}(\coreXref b) & \Use(b) \\
  \cons{Return}(e?) & \Use(e) \\
  \cons{Assign}(e,e') & \Use(e) \cup \Use(e') \cup
  (\overline{\Use}(\bullet) \setminus \Wdef(\bullet)) \\
  \cons{Call}(e?,\gamma,e_f,\vec e\,) & \Use(e) \cup \Use(e_f) \cup
  {\displaystyle \bigcup_{e_i \in \vec e}} \Use(e_i) \cup
  {\displaystyle \bigcup_{d_i \in \OPT(e_f)}} \Use(d_i) \cup
  (\overline{\Use}(\bullet) \setminus
  \Wdef(\bullet)) \\  
  \cons{Alloc}(e,d,e') & \Use(e) \cup \Use(e') \cup
  (\overline{\Use}(\bullet) \setminus \Wdef(\bullet)) \\
  \cons{Free}(,e') & \Use(e') \cup (\overline{\Use}(\bullet) \setminus
  \Wdef(\bullet)) \\ 
  b = (\vec s,j) & \Use(s_1) \qquad (\mbox{or } \Use(j) \mbox{ if }
  \vec s = []) \\
  \cons{Fun}(id,t,\vec d_p,\vec d_v,\vec b\,) &
  \Use(b_1) \setminus \trulylocal(\bullet) \\
\end{array}
\]

\subsection{Constraint Formulation}
The recursive equations above can be stated as a constraint system
that can be solved by fixpoint iteration. The constraints employed are
\[
\begin{array}{ccccc}
X = \bigcup Y_i &
X = \bigcap Y_i &
X = Y \setminus Z &
X \subseteq Y &
X \supseteq Y
\end{array}
\]
where $X,Y,Z \in \powset(D)$. 

For the definitely-write function, we want to find the greatest
fixpoint\footnote
	{because otherwise a loop in the function would block
	the propagation of definitely-write information},
and we therefore start out with the assumption that all
objects are killed. The constraints generated are thus
\[
\begin{array}{lll}
  \hfil \bullet & \hfil \mbox{Initial value} & \hfil \mbox{Constraints} \\
  \hrulefill & \hrulefill & \hrulefill \\
  \cons{If}(e,\coreXref{b_{\mathrm{then}}},
  \coreXref{b_{\mathrm{else}}}) & \Wdef(\bullet) = D & \Wdef(\bullet) \subseteq
  \Wdef(b_{\mathrm{then}}) ,\hfil \Wdef(\bullet) \subseteq
  \Wdef(b_{\mathrm{else}}) \\
  \cons{Goto}(\coreXref b) & \Wdef(\bullet) = D & \Wdef(\bullet) \subseteq \Wdef(b) \\
  \cons{Return}(e?) & \Wdef(\bullet) = \emptyset &  \\
  \cons{Assign}(e,e') & \Wdef(\bullet) = \subdecl{\exactlyone(\OPT(e))} & \\
  \cons{Call}(e?,\gamma,e_f,\vec e\,)
  & \Wdef(\bullet) = D
  & \Wdef(\bullet) = {\displaystyle \bigcup} \{
  \subdecl{\exactlyone(\OPT(e))}, X_{\mathrm{tmp}} \} \\
  & X_{\mathrm{tmp}} = D & X_{\mathrm{tmp}} \subseteq \Wdef(d_i)
  \qquad \forall d_i \in \OPT(e_f)  \\
  \cons{Alloc}(e,d,e') & \Wdef(\bullet) = \subdecl{\exactlyone(\OPT(e))} \\ 
  \cons{Free}(,e') & \Wdef(\bullet) = \emptyset \\
  b = (\vec s,j) & \Wdef(\bullet) = D & \Wdef(\bullet) =
  {\displaystyle \bigcup} \{ \Wdef(j) \} \cup \{ \Wdef(s_i) |
  s_i \in \vec s \}  \\
  \cons{Fun}(id,t,\vec d_p,\vec d_v,\vec b\,)
  & \Wdef(\bullet) = D & \Wdef(\bullet) = \Wdef(b_1) \setminus
  \trulylocal(\bullet) \\
\end{array}
\]

For the may-write function, we want to find the least fixpoint, and we
therefore start out with the assumption that no objects are killed.
The constraints generated are thus
\[
\begin{array}{lll}
  \hfil \bullet & \hfil \mbox{Initial value} & \hfil \mbox{Constraints} \\
  \hrulefill & \hrulefill & \hrulefill \\
  \cons{If}(e,\coreXref{b_{\mathrm{then}}},
  \coreXref{b_{\mathrm{else}}}) & \Wmay(\bullet) = \emptyset & \Wmay(\bullet) \supseteq
  \Wmay(b_{\mathrm{then}}) ,\hfil \Wmay(\bullet) \supseteq
  \Wmay(b_{\mathrm{else}}) \\
  \cons{Goto}(\coreXref b) & \Wmay(\bullet) = \emptyset & \Wmay(\bullet) \supseteq \Wmay(b) \\
  \cons{Return}(e?) & \Wmay(\bullet) = \emptyset &  \\
  \cons{Assign}(e,e') & \Wmay(\bullet) = \subdecl{\OPT(e)} & \\
  \cons{Call}(e?,\gamma,e_f,\vec e\,)
  & \Wmay(\bullet) = \subdecl{\OPT(e)}
  & \Wmay(\bullet) \supseteq \Wmay(d_i) \qquad \forall d_i \in
  \OPT(e_f)  \\ 
  \cons{Alloc}(e,d,e') & \Wmay(\bullet) = \subdecl{\OPT(e)} \\ 
  \cons{Free}(,e') & \Wmay(\bullet) = \emptyset \\
  b = (\vec s,j) & \Wmay(\bullet) = \emptyset & \Wmay(\bullet) \supseteq
  \Wmay(j) , \Wmay(\bullet) \supseteq \Wmay(s_i) \quad
  \forall s_i \in \vec s  \\
  \cons{Fun}(id,t,\vec d_p,\vec d_v,\vec b\,)
  & \Wmay(\bullet) = \emptyset & \Wmay(\bullet) = \Wmay(b_1) \setminus
  \trulylocal(\bullet) \\
\end{array}
\]

When the definitely-write function has been calculated, the in-use
function can be calculated. We want to find the least fixpoint, so the
constraints generated are thus
\[
\begin{array}{lll}
  \hfil \bullet & \hfil \mbox{Initial value} & \hfil \mbox{Constraints} \\
  \hrulefill & \hrulefill & \hrulefill \\
  \cons{If}(e,\coreXref{b_{\mathrm{then}}},
  \coreXref{b_{\mathrm{else}}}) & \Use(\bullet) = \Use(e) & \Use(\bullet) \supseteq
  \Use(b_{\mathrm{then}}) ,\hfil \Use(\bullet) \supseteq
  \Use(b_{\mathrm{else}}) \\
  \cons{Goto}(\coreXref b) & \Use(\bullet) = \emptyset & \Use(\bullet) \supseteq \Use(b) \\
  \cons{Return}(e?) & \Use(\bullet) = \Use(e)   \\
  \cons{Assign}(e,e') & \Use(\bullet) = \Use(e) \cup \Use(e') & \Use(\bullet)
  \supseteq X_{\mathrm{tmp}} \\
  & X_{\mathrm{tmp}} = \emptyset & X_{\mathrm{tmp}} =
  \overline{U}(\bullet) \setminus \Wdef(\bullet) \\
  \cons{Call}(e?,\gamma,e_f,\vec e\,)
  & \Use(\bullet) = \Use(e) \cup \Use(e_f) \cup \Use(e_i) 
  & \Use(\bullet) \supseteq X_{\mathrm{tmp}}, \Use(\bullet) \supseteq \Use(d_i) \qquad \forall d_i \in
  \OPT(e_f)\\
  & X_{\mathrm{tmp}} = \emptyset 
  & X_{\mathrm{tmp}} = \overline{U}(\bullet) \setminus \Wdef(\bullet) \\
  \cons{Alloc}(e,d,e') & \Use(\bullet) = \Use(e) \cup \Use(e') & \Use(\bullet)
  \supseteq X_{\mathrm{tmp}} \\ 
  & X_{\mathrm{tmp}} = \emptyset 
  & X_{\mathrm{tmp}} = \overline{U}(\bullet) \setminus \Wdef(\bullet) \\
  \cons{Free}(,e') & \Use(\bullet) = \Use(e') & \Use(\bullet)
  \supseteq X_{\mathrm{tmp}} \\ 
  & X_{\mathrm{tmp}} = \emptyset 
  & X_{\mathrm{tmp}} = \overline{U}(\bullet) \setminus \Wdef(\bullet) \\
  b = (\vec s,j) & \Use(\bullet) = \emptyset & \Use(\bullet) \supseteq
  \Use(s_1) \qquad (\mbox{or } \Use(j)) \\
  \cons{Fun}(id,t,\vec d_p,\vec d_v,\vec b\,)
  & \Use(\bullet) = \emptyset & \Use(\bullet) = \Use(b_1) \setminus
  \trulylocal(\bullet) \\
\end{array}
\]

\subsection{Implementation level (1999-07-19)}
\label{sec:DataflowImplementationLevel}
\index{implementation level!dataflow analysis}

The dataflow analyses have been implemented according to the above
description.

\end{docpart}
%%% Local Variables: 
%%% mode: latex
%%% TeX-master: "cmixII"
%%% End: 


% File: bta.tex

\providecommand{\docpart}{\renewenvironment{docpart}{}{}
\end{docpart}
\documentclass[twoside]{cmixdoc}
%\bibliographystyle{apacite}

\makeatletter
\@ifundefined{@title}{\title{\cmix-documentation}}{}
\@ifundefined{@author}{\author{The \cmix{} Team}}%
{\expandafter\def\expandafter\@realauthor\expandafter{\@author}%
\author{The \cmix{} Team\\(\@realauthor)}}
\makeatother

\AtBeginDocument{%
\markboth{\hfill\today\quad\timenow\hfill\llap{\cmix\ documentation}}
{\hfill\today\quad\timenow\hfill}}

\renewcommand{\sectionmark}[1]{\markboth
{\hfill\today\quad\timenow\hfill\llap{\cmix\ documentation}}
{\rlap{\thesection. #1}\hfill\today\quad\timenow\hfill}}

%\newboolean{separate}
%\setboolean{separate}{true}

\renewenvironment{docpart}{\begin{document}}%
                          {\bibliography{cmixII}\printindex
                           \end{document}}
\begin{document}\shortindexingon
}
\title{Binding-time analysis in \cmix}
\author{Jens Peter Secher}
\begin{docpart}
\maketitle

\begin{center}
  \fbox{\huge\textsl{THIS SECTION IS OUT-OF-DATE}}    
\end{center}

\section{Binding-time analysis in \cmix}
This paper describes the binding-time analysis (BTA) on \coreC in \cmix.
The analysis is based on the one described in
\cite{Andersen:1994:ProgramAnalysisAndSpecialization} and extended with
ideas from \cite{Andersen:1997:PartiallyStaticBTA}. The analysis works on
annotated types, described in the next section. It consists of two phases:
constraint generation and constraint solving.  These are described in later
sections.

\subsection{Types and binding times}
The purpose of the BTA is to identify which parts of the program that
can be evaluated at compile time; these are the the \emph{static}
parts. All other parts must be suspended until the necessary data is
supplied; these are the \emph{dynamic} parts.  Figure~\ref{fig:btt}
defines binding times and annotated types.  Intuitively, $\beta +>
\beta'$ means that $\beta$ is at least as dynamic as $\beta'$.

\begin{figure*}%[htbp]
\begin{frameit}
\[\begin{array}{rcll}
  {\mathcal B} \ni \beta &::=& S & \mbox{Static} \\
                         & | & D & \mbox{Dynamic}\\
  \\
  +> &\subseteq& {\mathcal B} \times {\mathcal B} \\
  D & +> & D \\
  D & +> & S \\
  S & +> & S \\
  \\
    \tau_B &::=& \syntax{int}, \syntax{float}, \dots \\
  \\
  {\mathcal T} \ni T &::=& \btT{\tau_B}{\beta} & \mbox{Simple} \\
            & | & \btT{\mbox{struct }S}{\beta} & \mbox{Structure}  \\
            & | & \btT{\star}{\beta} T  & \mbox{Pointer to }T \\
            & | & \btT{[n]}{\beta} T  & \mbox{Array}[n]\mbox{ of }T \\
            & | & \btT{(T_1,...,T_n)}{\beta} T  & \mbox{Function
                         returning }T \\
\end{array}\]
\caption{binding times and types.}
\label{fig:btt}
\end{frameit}
\end{figure*}

\begin{Def}
  \label{def:wf-bttype}
  A binding-time type $T =
  \btT{\tau_1}{\beta_1}...\btT{\tau_n}{\beta_n}$ is \emph{well-formed}
  iff $\beta_n +> \dots +> \beta_1$ and for any function type $T =
  \btT{(\btT{\tau_1}{\beta_1},...,\btT{\tau_m}{\beta_m} )}{\beta} T_0$
  it must be the case that $\beta +> \beta_1 \wedge ... \wedge \beta
  +> \beta_m$.
\end{Def}

\noindent
The relation ${>>=} \subseteq \mathcal{T} \times \mathcal{T}$ is
defined below. Intuitively, $T >>= T'$ means that type $T$ is at least
as dynamic as $T'$.

\begin{Def}
  \label{def:bt-type-order}
  $T >>= T'$ iff $T = \btT{\tau_1}{\beta_1}...\btT{\tau_n}{\beta_n}$,
  $T' = \btT{\tau_1}{\beta_1'}...\btT{\tau_n}{\beta_n'}$ and $\beta_1
  +> \beta_1' \wedge ... \wedge \beta_n +> \beta_n'$. Also, for any
  pair of function types $T,T'$, we have $T >>= T'$ iff $T =
  \btT{(T_1,...,T_m)}{\beta}T_0, T' =
  \btT{(T_1',...,T_m')}{\beta}T_0'$, and $T_0 >>= T_0' \wedge T_1 >>=
  T_1' \wedge ... \wedge T_m >>= T_m'$.
\end{Def}

\begin{Def}
  \label{def:bt-of-type}
  The binding time of a type is defined thus: $\bt{T}=\beta$ iff $T =
  \btT{\tau}{\beta} \btT{\tau_1}{\beta_1} ...
  \btT{\tau_n}{\beta_n}$.
\end{Def}

\subsection{Constraint generation}
Constraints are generated to express dependencies between parts of the
program, \eg the binding time of an expression, which consists of an
operator and two operands, depends on the binding times of both
sub-expressions. 

There are three kinds of constraints: $T == T'$ means that the two
binding-time types must be identical; $T >>= T'$ is defined by
definition~\ref{def:bt-type-order}; $\beta +> \beta'$ is defined in
figure~\ref{fig:btt}; The short-hand notation $\bt{T} +> \bt{T'}$
means that if $\bt{T}=\beta$ and $\bt{T'}=\beta'$ then $\beta +>
\beta'$. The short-hand notation alldyn($e$) means that every part of
$e$ must be dynamic.

Initially, the types of the formal parameter declarations of the goal
function are assigned the initial binding-time division. All other
types are annotated with fresh binding-time variables. Since the
translation to \coreC has introduced a unique instance of every
structure declaration, each instance can have its own binding times.
Initialisers are static in Core C so they are not accounted for.
Parameter types of function pointers are made equal to parameters of
all functions they can point to, and so are the return types. Equality
constraints are denoted by $==$, which is realy a hort-hand notation
for two $>>=$ constraints.

\subsection{Expressions}
\label{sec:BTAExpressions}
The binding-time constraint generation for expressions is defined in
figure~\ref{fig:BTAExprConstraintGeneration}. During this phase we
implicitly assume that constraints are generated such that every type will
fulfil the well-formed criteria, see definition~\ref{def:wf-bttype}.

\bigskip
% -------- Expressions ---------

\begin{figure*}%[htbp]
\begin{frameit}
\noindent\fbox{Expressions}\hfill\fbox{$e : T,\mathcal{C}$}

\[\inference[const]
  {}
  {c : \btT{\tau_B}{\beta},\{\}}
\qquad
  \inference[rvar]
  {}
  {x : T, \{T == T_x\}}
\qquad
  \inference[lval]
  {\bt{T_x} = \beta_x}
  {\syntax{\&}x : \btT{\star}{\beta}T, \{T >>= T_x,\beta_x +> \beta\}}
\]


\[
  \inference[deref]
  {e : \btT{\star}{\beta} T', \mathcal{C} &
   \mbox{localglobal}(PT(e),\beta,\bt{T'}) = \mathcal{C}'}
  {\syntax{*}e : T, \mathcal{C}\mathcal{C}'\{T==T'\}}
\qquad
  \inference[addr]
  {e : T, \mathcal{C} & \bt{T} = \beta}
  {\syntax{\&}e : \btT{\star}{\beta'}T', \mathcal{C}
    \{T==T', \beta +> \beta'\}}
\]

\[\inference[unary]
  {e : T, \mathcal{C}}
  {uop\ e : T', \mathcal{C}\{\bt{T'} +> \bt{T}\}}
\qquad
  \inference[binary]
  {e_1 : T_1, \mathcal{C}_1 & e_2 : T_2, \mathcal{C}_2}
  {e_1\ bop\ e_2 : \btT{\tau}{\beta},
    \mathcal{C}_1\mathcal{C}_2\{\beta +> \bt{T_1}, \beta +> \bt{T_2}\}}
\]

\[
  \inference[struct]
  {e : T_S, \mathcal{C} & \mbox{refSet}(e.i) = \{ \alst{d} \}}
  {e.i : T, \mathcal{C}\{T >>= T_{d_1},...,T >>= T_{d_n}\}}
\qquad
  \inference[typesize]
  {}
  {\syntax{sizeof}(\tau) : \btT{\tau_B}{\beta},\{\beta +> D\}}
\]

\[
  \inference[expsize]
  {}
  {\syntax{sizeof}(e) : \btT{\tau_B}{\beta},\{\beta +> D, \mbox{alldyn}(e)\}}
\]

\[\inference[cast]
  {e : T, \mathcal{C}}
  {\syntax{(}T'\syntax{)} e : T', \mathcal{C} \cup \mbox{Cast}(T,T')}
\]

\[\begin{array}{lcl}
   \mbox{Cast}(\btT{\tau_B}{\beta},\btT{\tau_B'}{\beta'}) 
   & = & \{ \beta +> \beta' \} \\
  \mbox{Cast}(\btT{\star}{\beta}T,\btT{\star}{\beta'}T') 
   & = & \{ \beta=\beta' \} \cup \mbox{Cast}(T,T') \\
  \mbox{Cast}(\btT{\star}{\beta}T,\btT{[n]}{\beta'}T') 
   & = & \{ \beta=\beta' \} \cup \mbox{Cast}(T,T') \\
 \end{array}\]
 \caption{Constraint generation for expressions.}
 \label{fig:BTAExprConstraintGeneration}
\end{frameit}
\end{figure*}

Notice that if expression $e$ has a binding-time type $T$, then
$\bt{T}$ tells us the binding time of the \emph{result} of $e$ --- it
does not tell us when we can expect to evaluate various parts of
$e$. This information has to be extracted in a later phase,
see~\vref{sec:PartiallyStaticData}.

\subsection{Statements}
The binding-time type $T_f$ of a function $f$ has a binding time that
depends on all calls to $f$: If it is called under dynamic control,
the binding time is dynamic. The binding time $\alpha_f$ of a function
$f$ depends on the statements in that function: if it is static, no
dynamic statements are present in $f$ and thus any call to $f$ can be
eliminated.
  
The return type $T_{f_0}$ of a function $f$ tells us whether the
return value is static. It is thus possible to have a dynamic
function with a static return value, provided all return statements
are static and the control flow is totally static. Consider the
program

\begin{verbatim}
  int f(int d)
  {
    if (d) return 1; else return 2;
  }
\end{verbatim}

\noindent
where the return values are static, but under dynamic control since
\verb.d. is dynamic. If the return value of the function were to be
classified static, the function would have two different return
values. 

The constraint generation for statements is defined in
figure~\ref{fig:BTAStmtConstraintGeneration}. We implicitly assume that
each statement has a surrounding function $f$ with the type $T_f$ and
return type $T_{f_0}$.

\subsubsection{Non-local side-effects}
The binding-time analysis has to guarantee that static non-local
side-effects (NLS-effects) does not occur under dynamic control. To
ensure this, the BTA makes every NLS-effecting
statement\footnote{Recall that only statements can do side-effects in
  Core C.} dependent on the binding time of the preceding basic
block's control statement. Thus, if a basic block $A$ ends in a
conditional control statement of the form \syntax{if($e$) goto $B$
  else goto $C$}, then both blocks $B$ and $C$ will be dependent on
the expression $e$. To make this dependency transitive, every basic
block is dependend of all immediate preceding blocks; the first block
in a function is dependend on each call site's basic block.

Given a function $f$, a statement of the form \syntax{x=$e$} is
locally side-effecting (and thus harmless) if \syntax{x} is a local
variable in $f$. The same holds for statements of the form
\syntax{x.i=$e$}. When assignments are done through pointers, however,
we cannot rely on syntactical scope anymore: we need to decide whether
the objects reachable through a particular pointer are truly local
objects\footnote{A recursive function can have several sets of local
  variables at run-time, and thus a pointer can refer to a variable in
  any such set}.

Given the previously described points-to information (see
Section~\vref{sec:PointerAnalysis}), an approximation of the set of non-local
variables in a function $f$ can be calculated: the transitive closure
of the PA-info (denoted $PT^*$) is calculated for all global variables
and for each function $f$, such that $PT^*(f)$ is $PT^*$ of all formal
parameters, and $PT^*(\mbox{globals})$ is $PT^*$ of the globals. With
these sets in hand, a side-effecting statement of the form
\syntax{*p=$e$} is considered locally side-effecting if no object in
$PT(\syntax{p})$ is contained in $PT^*(\mbox{globals})$ or $PT^*(f)$.
We can thus define a function that returns a set of constraints when
given a declaration, the containing function and the binding-time
variable of the containing basic block.
\begin{eqnarray}
  \label{eqn:nonlocal}
  \mbox{nonlocal}(d,f,bb)
  &=& \mbox{if } d \in \mbox{locals}(f) \setminus (PT^*(\mbox{globals}) \cup PT^*(f)) \nonumber \\
  & & \mbox{then } \{ \} \nonumber  \\
  & & \mbox{else } \{ \bt{T_d} +> \bt{bb} \} \nonumber 
\end{eqnarray}

Each function should have a non-local side effect flag to remind the
generating extension that such a function needs a post-store
memoization.

\subsection{Static pointers to dynamic data}
Consider this program were \texttt{d} is dynamic.

\begin{verbatim}
int y,z;                    void swap(int* a, int *b)
                            {
int main(int d)               int tmp = *a;
{                             *a = *b, *b = tmp;
  int x = d+1;              }
  swap(&d,&x); /* 1 */  
  y = x;                
  z = d;                
  swap(&y,&z); /* 2 */  
  return d;             
}                       
\end{verbatim}

\noindent The first call to \texttt{swap} contains static pointers to
dynamic, non-local, non-global objects --- whereas the second contains
static pointers to dynamic, global objects. If this was allowed, the
residual program would wrongly be

\begin{verbatim}
int y,z;                    void swap_1()
                            {
int main(int d)               int tmp = d;     /* out of scope */
{                             d = x, x = tmp;
  int x = d+1;              }
  swap_1();                 
  y = x;                    void swap_2()
  z = d;                    {
  swap_2();                   int tmp = y;
  return d;                   y = z, z = tmp;
}                           }
\end{verbatim}

\noindent The problem is that dynamic, non-local, non-global objects
get out of scope during the first call. This can also happen with
global pointers to non-local objects. [So, we need to make constraints
such that if a dereference of static pointers results in a non-local,
non-global object, the pointer must be made dynamic.]

\begin{eqnarray}
  \label{eqn:localglobal}
  \mbox{localglobal}(\Delta,\beta,\beta_{*}) &=& \mbox{if }
  \exists\delta \in \Delta \mbox{ s.t. }
  \delta \not\in (\mbox{globals} \cup \mbox{locals}) \nonumber \\
  & & \mbox{then } \{ \beta_{*} +> \beta \} \nonumber  \\
  & & \mbox{else } \{ \} \nonumber
\end{eqnarray}

\bigskip
% -------- Statements ---------

\begin{figure*}%[htbp]
\begin{frameit}
\noindent\fbox{Statements}\hfill\fbox{$s : \mathcal{C}$}

\[\inference[if]
  {e : T, \mathcal{C}}
  {\syntax{if}(e)\ B\ B' : \mathcal{C}\{\bt{T_f} +> \bt{T},
                            \bt{B} +> \bt{T}, \bt{B'} +> \bt{e}\}}
\]

\[\inference[return]
  {e : T, \mathcal{C}}
  {\syntax{return}\ e : \mathcal{C}\{T_{f_0} >>= T,
     \bt{T_f} +> \bt{T}\}}
\qquad
  \inference[goto]
  {}
  {\syntax{goto}\ B : \{\}}
\]

\[\inference[assign]
  {e : T, \mathcal{C} & \mbox{nonlocal}(x,f,bb)=\mathcal{C}_x}
  {x\syntax{=}e : \mathcal{C}\mathcal{C}_x\{T_x >>= T, \bt{T_f} +> \bt{T}\}}
\]

\[\inference[passign]
  {e : T, \mathcal{C} & PT(x) = \{ \alst{d} \} & \mbox{nonlocal}(d_i,f,bb)=\mathcal{C}_i}
  {\syntax{*}x\syntax{=}e : \mathcal{C}\mathcal{C}_1 \cdots \mathcal{C}_n
    \{T_{d_1} >>= T,...T_{d_n} >>= T, \bt{T_f} +> \bt{T}\}}
\]

\[\inference[strassign]
  {e : T, \mathcal{C} & \mbox{nonlocal}(x,f,bb)=\mathcal{C}_x}
  {x\syntax{.}i\syntax{=}e : \mathcal{C}\mathcal{C}_x
    \{T_{x.i} >>= T, \bt{T_f} +> \bt{T}\}}
\]

\[\inference[pstrassign]
  {e : T, \mathcal{C} & PT(x) = \{ \alst{d} \}& \mbox{nonlocal}(d_i,f,bb)=\mathcal{C}_i}
  {\syntax{(*}x\syntax{).}i\syntax{=}e : \mathcal{C}\mathcal{C}_1 \cdots \mathcal{C}_n
    \{T_{d_1.i} >>= T,...T_{d_n.i} >>= T, \bt{T_f} +> \bt{T}\}}
\]

\[\inference[call]
  {e_i : T_i , \mathcal{C}_i &
    T_g = \btT{(T_1'\cdots T_n')}{\beta}T_0 & \mbox{nonlocal}(x,f,bb)=\mathcal{C}_x}
  {x\syntax{=}g(e_1,\dots,e_n) : \mathcal{C}_1 \cdots \mathcal{C}_n \mathcal{C}_x
    \{T_x >>= T_0, T_i' >>= T_i, 
      \bt{T_f} +> \bt{T_i}\}}
\]

\[\inference[pcall]
  {e_i : T_i , \mathcal{C}_i & PT(f\!p) = \{\alst[m]{d}\} &
   T_{f\!p} = \btT{\star}{\beta_{f\!p}}\btT{(T_1'\cdots
   T_n')}{\beta}T_0
   & \mbox{nonlocal}(x,f,bb)=\mathcal{C}_x}
  {x\syntax{=}f\!p(e_1,\dots,e_n) : \mathcal{C}_0 \cdots \mathcal{C}_n
    \mathcal{C}_x
    \{T_{d_j} == T_{f\!p}, T_x >>= T_0, T_i' >>= T_i, 
      \bt{T_f} +> \bt{T_i}\}}
\]

\[\inference[free]
  {e : \btT{\star}{\beta}T, \mathcal{C}}
  {\syntax{free}(e) : \mathcal{C}\{\bt{T_f} +> D, \beta +> D\}}
\]

\[\inference[calloc]
  {e : T_e, \mathcal{C} & T_x = \btT{\star}{\beta}T'
    & \mbox{nonlocal}(x,f,bb)=\mathcal{C}_x}
  {x\syntax{=calloc}(e,T,d) : \mathcal{C}\mathcal{C}_x
    \{ T==T',  T==T_d, \bt{T_f} +> \beta, \beta +> \bt{T_e} \}}
\]
 \caption{Constraint generation for statements.}
 \label{fig:BTAStmtConstraintGeneration}
\end{frameit}
\end{figure*}

\subsection{Assigning binding times}
Binding times can be induced directly from the types.

[Work-lists in \cite{KanamoriWeise:1994:WorklistManagement}]

[Each declaration $d$ that has pointer type: Make all the types of
objects $\in PT(d)$ depend on each other.]


\begin{tabular}{|c|c|c|l|}  \hline
  Allocation & Contents & Freed & Action \\ \hline
  S          & S        & +     & Ignore free \\
  S          & S        & -     & OK \\
  S          & D        & +     & $->$ DD \\
  S          & D        & -     & OK \\
  D          & D        & +/-   & Residualize \\ \hline
\end{tabular}

\end{docpart}
%%% Local Variables: 
%%% mode: latex
%%% TeX-master: "cmixII"
%%% hst: "Herrens Allermest Ultimative Naade"
%%% End: 


% File: sanity.tex
% Time-stamp: 
% $Id: sanity.tex,v 1.1 1999/04/21 13:12:24 jpsecher Exp $

\providecommand{\docpart}{\renewenvironment{docpart}{}{}
\end{docpart}
\documentclass[twoside]{cmixdoc}
%\bibliographystyle{apacite}

\makeatletter
\@ifundefined{@title}{\title{\cmix-documentation}}{}
\@ifundefined{@author}{\author{The \cmix{} Team}}%
{\expandafter\def\expandafter\@realauthor\expandafter{\@author}%
\author{The \cmix{} Team\\(\@realauthor)}}
\makeatother

\AtBeginDocument{%
\markboth{\hfill\today\quad\timenow\hfill\llap{\cmix\ documentation}}
{\hfill\today\quad\timenow\hfill}}

\renewcommand{\sectionmark}[1]{\markboth
{\hfill\today\quad\timenow\hfill\llap{\cmix\ documentation}}
{\rlap{\thesection. #1}\hfill\today\quad\timenow\hfill}}

%\newboolean{separate}
%\setboolean{separate}{true}

\renewenvironment{docpart}{\begin{document}}%
                          {\bibliography{cmixII}\printindex
                           \end{document}}
\begin{document}\shortindexingon
}
\begin{docpart}

\section{Sanity of BTA}



\end{docpart}

%%% Local Variables: 
%%% mode: latex
%%% TeX-master: "cmixII"
%%% End: 


% File: init.tex
% Time-stamp: 
% $Id: init.tex,v 1.1 1999/03/02 06:09:10 makholm Exp $

\providecommand{\docpart}{\renewenvironment{docpart}{}{}
\end{docpart}
\documentclass[twoside]{cmixdoc}
%\bibliographystyle{apacite}

\makeatletter
\@ifundefined{@title}{\title{\cmix-documentation}}{}
\@ifundefined{@author}{\author{The \cmix{} Team}}%
{\expandafter\def\expandafter\@realauthor\expandafter{\@author}%
\author{The \cmix{} Team\\(\@realauthor)}}
\makeatother

\AtBeginDocument{%
\markboth{\hfill\today\quad\timenow\hfill\llap{\cmix\ documentation}}
{\hfill\today\quad\timenow\hfill}}

\renewcommand{\sectionmark}[1]{\markboth
{\hfill\today\quad\timenow\hfill\llap{\cmix\ documentation}}
{\rlap{\thesection. #1}\hfill\today\quad\timenow\hfill}}

%\newboolean{separate}
%\setboolean{separate}{true}

\renewenvironment{docpart}{\begin{document}}%
                          {\bibliography{cmixII}\printindex
                           \end{document}}
\begin{document}\shortindexingon
}
\begin{docpart}

\section{Initializer conversion}
\label{sec:Initializer conversion}

In \ansiC, an initializer in a variable declaration must be regarded
as executable code. \textit{E.g.}, the following program must output
\texttt{abab} because the initializer is executed each time the
loop body is entered:
\begin{verbatim}
int main(void) {
   int i ;
   for(i=0; i<2; i++) {
     char j = 'a' ;
     printf("%c.",j);
     j = 'b' ;
     printf("%c.",j);
   }
}
\end{verbatim}

Not so in \coreC. First, in \coreC variables are can only be declared
at the beginning of a function. If, in the above program, we were to
just move the declaration and the associated initializer of \texttt{j}
to the beginning of the function, the meaning of the program would
silently change and it would output \texttt{abbb} instead.

Moreover, it is not even safe for variables that appear at the
beginning of a function to keep their initializers. In the above
program, the loop body could just as well be a separate function that
happened to be inlined during specialization---which would again
move the declaration of j to the top of the \emph{residual} function.

This section describes how \cmix avoids that sort of problems.

\subsection{Simple initialization}
When the initializer can be expressed as a single expression---\eg
when the initialized variable has arithmetic or pointer type---the
matter is handled in the c2core translation phase. It simply directly
converts the initializer to a regular assignment statement at the
right point in the flow graph and removes the initializer from the
\coreC declaration of the variable.

\subsection{Complex initialiation}
However, not all types that can be initialized in C do have a
syntax for constant \emph{expressions}\footnote{What a silly, filthy
unorthogonal language!}. For these types it is not simple to generate
an assignment statement.

One option would be to convert the initializer into multiple
assignment statements. That approach, however, does not apply
if the elements of the initializer cannot be resolved at analysis
time (which may happen if an array index cannot be interpreted by
the analyser). Thus we reject it.

The remaining option is to move the original initializer to a new
auxiliary variable of the same type as the original one, and copy
the values between the two initializers at the point where
initialization was originally supposed to take place.

This happens to be safe because \ansiC requires that the
expressions in a complex initializer does not depend on the
values of any object, so they will evaluate to the same value
at the top of the function as they would at the point of
initialization.

On the other hand it would not be safe to let the auxiliary
variable be global, because the expressions in the initializer may
need to refer to the \emph{addresses} of other local variables, as in
\begin{verbatim}
void foo(void) {
  int i,j ;
  struct { int *a, int j } bar = { &i, sizeof j };
  /* ... */
}
\end{verbatim}

\subsubsection{Selecting candidates for initializer movement}
The generation of auxiliary variables and assignments was felt
to be possibly impeding on the efficiency of the generated program,
and certainly confusing to its human readers. Therefore we want
to only do it when it is absolutely necessary.

We believe that in practise many variables with complex initializers
are not actually changed after initialization. They could have been
\texttt{const} but programmers do not always declare them thus.
These variables do not need initializer motion: because of the
requirement that the initializing expressions are constant the
value of the initializer cannot be different each time it is
``executed''.

This suggests that the initializer motion should be performed
after the pointer analysis, so that we can decide to keep
the initializers for variables that are never the target of
any side effects in the program.

\subsubsection{Special consideration for array initializers}
A special complication arises in the not uncommon case that the
initialized variable is an array. Because C has no array assignment\footnote
{Which in turn is because C has no array \emph{expressions} that
could go on the right-hand side of such an assignment} one cannot
simply generate an assignment statement between the original and
the auxiliary array.

The remedy is to wrap the array in a struct type, since C does have
struct assignment. That is, before initializer movement on
\begin{verbatim}
   T blarf[42] = { blah blah };
\end{verbatim}
we convert it to
\begin{verbatim}
struct blarf { T a[42] } ;
/*...*/
   struct blarf blarf = { { blah blah } };
\end{verbatim}
(notice the extra set of braces in the initializer).

The point where this gets complex is that we then need to replace
any $\cons{Var}(\texttt{blarf})$ expression with
$\cons{Member}(\cons{Var}(\texttt{blarf}),1)$. Fortunately the scope
rules guarantee that such expressions are only found within the same
function as \texttt{foo} itself.

\subsection{Implementation level, 1999-03-02}

The initializer conversions has been implemented as described.
The code for the complex initializer movement is in \texttt{init.cc}

\end{docpart}
%%% Local Variables: 
%%% mode: latex
%%% TeX-master: "cmixII"
%%% End: 


% File: strctsep.tex
% Time-stamp: 
% $Id: strctsep.tex,v 1.1 1999/03/02 17:34:47 jpsecher Exp $

\providecommand{\docpart}{\renewenvironment{docpart}{}{}
\end{docpart}
\documentclass[twoside]{cmixdoc}
%\bibliographystyle{apacite}

\makeatletter
\@ifundefined{@title}{\title{\cmix-documentation}}{}
\@ifundefined{@author}{\author{The \cmix{} Team}}%
{\expandafter\def\expandafter\@realauthor\expandafter{\@author}%
\author{The \cmix{} Team\\(\@realauthor)}}
\makeatother

\AtBeginDocument{%
\markboth{\hfill\today\quad\timenow\hfill\llap{\cmix\ documentation}}
{\hfill\today\quad\timenow\hfill}}

\renewcommand{\sectionmark}[1]{\markboth
{\hfill\today\quad\timenow\hfill\llap{\cmix\ documentation}}
{\rlap{\thesection. #1}\hfill\today\quad\timenow\hfill}}

%\newboolean{separate}
%\setboolean{separate}{true}

\renewenvironment{docpart}{\begin{document}}%
                          {\bibliography{cmixII}\printindex
                           \end{document}}
\begin{document}\shortindexingon
}
\begin{docpart}

\section{Structure separation}
\label{sec:StructSeparation}\index{Separating structures}

\end{docpart}
%%% Local Variables: 
%%% mode: latex
%%% TeX-master: "cmixII"
%%% End: 

% File: sharing.tex
% Time-stamp: 

\providecommand{\docpart}{\renewenvironment{docpart}{}{}
\end{docpart}
\documentclass[twoside]{cmixdoc}
%\bibliographystyle{apacite}

\makeatletter
\@ifundefined{@title}{\title{\cmix-documentation}}{}
\@ifundefined{@author}{\author{The \cmix{} Team}}%
{\expandafter\def\expandafter\@realauthor\expandafter{\@author}%
\author{The \cmix{} Team\\(\@realauthor)}}
\makeatother

\AtBeginDocument{%
\markboth{\hfill\today\quad\timenow\hfill\llap{\cmix\ documentation}}
{\hfill\today\quad\timenow\hfill}}

\renewcommand{\sectionmark}[1]{\markboth
{\hfill\today\quad\timenow\hfill\llap{\cmix\ documentation}}
{\rlap{\thesection. #1}\hfill\today\quad\timenow\hfill}}

%\newboolean{separate}
%\setboolean{separate}{true}

\renewenvironment{docpart}{\begin{document}}%
                          {\bibliography{cmixII}\printindex
                           \end{document}}
\begin{document}\shortindexingon
}
\begin{docpart}

\section{Function sharing}
\label{sec:Sharing}\index{Sharing}

The code sharing analysis computes a set of functions that should be
inlined by the generating extention. This can have
various reasons, e.g. they contain impure spectime
actions or it is judged that their residuals would be
too trivial to warrant a separate residual function.

By default, functions are \emph{not} inlined.
The description below details the conditions why a function
may be marked for inlining.

\subsection{Impure static actions}

Any function that, perhaps indirectly, contain a call
with callmode \cons{CallOnceSpectime} or a static allocation
must be unsharable.

This is because these operations must be expected to yield different
result even when performed in identical (as far as \cmix can see)
static states. In the case of \cons{CallOnceSpectime} calls we
explicitly promise the user (in the user manual---this callmode
corresponds to the \texttt{spectime} user annotation on function
calls) that the generating extension will not duplicate or optimize
away the calls.

\subsection{Functions with static return values}

A function with a static, non-void, return value must be
unsharable. This is because gegen doesn't yet know how to memoise
function return values.

\subsection{Functions that are probably ``small'' residually}

We implement the following heuristics for identifying functions whose
residuals are probably small enough that we should inline them. A
function will be inlined if
\begin{itemize}
\item[1)] it contains no dynamic conditionals, and
\begin{itemize}
\item[2a)] it contains no dynamic assignments, or
\item[2b)] it it not recursive and contains no loops
\end{itemize}\end{itemize}
 In theory it should be ``safe'' (in the sense that it
 only introduces infinite specialization if there are
 statically-controlled infinite recursions without
 dynamic exits) to inline every function that has no
 dynamic conditionals. Rules (2a) and (2b) try to
 restrict inlining to functions that will be ``small''
 residually, in an attempt to reduce code blowup.

   Functions that match (2a) will be \emph{empty} in the
 residual programs, modulo other function calls, and
 are obvious candidates for inlining.

Functions that match (2b) can at least be trusted
not to grow without bounds.

\subsubsection{Implementation notes}

This is how we check whether the flow graph of a function
may be cyclic. The algorithm is derived from the description
of depth-first orderings in \citeA{AhoSethiUllman:1986:Compilers}.

   We do a depth-first traversal of the flow graph,
 maintaining along the way two sets of \emph{pending} and
 \emph{finished} blocks. Once a block gets marked as
 finished we know that no cycle can be reached from
 that block. The pending blocks are those that are
 ancestors to the one currently processed.

   At the scanning time each edge can lead to either
\begin{itemize}
\item an unprocessed node, which we descend to. If it
     finished, then all is fine
\item a finished node---then we know that the edge
     cannot be part of cycle.
\item a pending node, in which case we've spotted a
     cycle and abort the search.
\end{itemize}

\subsection{Implementation level 1999-03-16}

At the time of this writing the implementation of the function
sharing analysis is in sync with the above description.

\end{docpart}
%%% Local Variables: 
%%% mode: latex
%%% TeX-master: "cmixII"
%%% End: 

% File: partstat.tex

\providecommand{\docpart}{\renewenvironment{docpart}{}{}
\end{docpart}
\documentclass[twoside]{cmixdoc}
%\bibliographystyle{apacite}

\makeatletter
\@ifundefined{@title}{\title{\cmix-documentation}}{}
\@ifundefined{@author}{\author{The \cmix{} Team}}%
{\expandafter\def\expandafter\@realauthor\expandafter{\@author}%
\author{The \cmix{} Team\\(\@realauthor)}}
\makeatother

\AtBeginDocument{%
\markboth{\hfill\today\quad\timenow\hfill\llap{\cmix\ documentation}}
{\hfill\today\quad\timenow\hfill}}

\renewcommand{\sectionmark}[1]{\markboth
{\hfill\today\quad\timenow\hfill\llap{\cmix\ documentation}}
{\rlap{\thesection. #1}\hfill\today\quad\timenow\hfill}}

%\newboolean{separate}
%\setboolean{separate}{true}

\renewenvironment{docpart}{\begin{document}}%
                          {\bibliography{cmixII}\printindex
                           \end{document}}
\begin{document}\shortindexingon
}
\title{Binding Time Dependency Information}
\author{Arne John Glenstrup}
\begin{docpart}
\maketitle

\MakeShortVerb{"}


\begin{center}
  \fbox{\huge\textsl{THIS SECTION IS OUT-OF-DATE}}    
\end{center}

\section{Partially Static Data}
\label{sec:PartiallyStaticData}
\index{partially static data}
\index{data!static!partially}
\index{struct!splitting}
\index{array!splitting}

The binding time analysis may result in some complex data structures like
structs or arrays  being classified as \emph{partially static}, e.g.~some
members of a struct are static and some dynamic. In the case of arrays, the
length might be known (static), but the contents is unknown (dynamic). In
these cases it is desirable to remove the static parts of the data
structures, thus in effect \emph{changing their type}: structs in the
residual program have fewer members, and arrays are split up into
individual variables.

This separation of static and dynamic parts of data structures can be done
as a preprocessing before the actual specialization phase,
and~\cite[section~3.3.1]{Andersen:1997:StaticMemoryManagementInCMix}
points out that it is a key transformation for keeping the time spent
during the specialization phase low. He gives an example of the splitting
transformation:\bigskip

\noindent
\begin{minipage}[c]{.45\textwidth}
\begin{alltt}
struct P \{ int x, y; \}; /*\rlap{\hspace{6em}*/} \(S\x{D}\)     
                      
 
struct P pair;          /*\rlap{\hspace{6em}*/} \(S\x{D}\)


struct P *p;            /*\rlap{\hspace{6em}*/} \(*(S\x{D})\)


int a[42];              /*\rlap{\hspace{6em}*/} \([42]D\) 

struct P b[42];         /*\rlap{\hspace{6em}*/} \([42](S\x{D})\)


pair.x;
pair.y;
p->x;
p->y;
a[3];
b[4].x;
b[4].y;
\end{alltt}
\end{minipage}
\hfil
$\displaystyle\mathop{==>}\limits\sp{\mathit{split}}$
\hfil
\begin{minipage}[c]{.40\textwidth}
\begin{alltt}
struct P\(\sb{s}\) \{ int x; \};
struct P\(\sb{d}\) \{ int y; \};

struct P\(\sb{s}\) pair\(\sb{s}\);
struct P\(\sb{d}\) pair\(\sb{d}\);

struct P\(\sb{s}\) *p\(\sb{s}\);
struct P\(\sb{d}\) *p\(\sb{d}\);

int a0, a1, ..., a42;

struct P\(\sb{s}\) b0\(\sb{s}\), b1\(\sb{s}\), ..., b41\(\sb{s}\);
struct P\(\sb{d}\) b0\(\sb{d}\), b1\(\sb{d}\), ..., b41\(\sb{d}\);

pair\(\sb{s}\).x;
pair\(\sb{d}\).y;
p\(\sb{s}\)->x;
p\(\sb{d}\)->y;
a3;
b4\(\sb{s}\).x;
b4\(\sb{d}\).y;
\end{alltt}
\end{minipage}\medskip

Although it looks tempting to split arrays of known length in
a preprocessing phase, it is hard to see that it would be sensible
in practise: expressions like "d = a[s]" (where "s" is statically known but
not a $\Ppann$ constant) would have to be translated into
\begin{center}
"switch (s) { 0 : d = a0; break;  1 : d = a1; break;  ... ;  41 : d = a41; break; }"
\end{center}
in $\Ppann$, and furthermore translated into \coreC's
"if"-"goto" constructions, resulting in myriads of basic blocks! 
Instead,
we will content ourselves with letting the gegen phase_{phase!gegen}
translate staic-length arrays with dynamic content into
\begin{alltt}
/* \Ppgen \textnormal{\emph{source code}} */
int a0, a1, ..., a41;
Code a[42] = \{ \emit{a0}, \emit{a1}, ..., \emit{a41} \};

struct P\(\sb{s}\) b\(\sb{s}\)[42];
struct P\(\sb{d}\) b0\(\sb{d}\), b1\(\sb{d}\), ..., b41\(\sb{d}\);
Code b\(\sb{d}\)[42] = \{ \emit{b0\(\sb{d}\)}, \emit{b1\(\sb{d}\)}, ..., \emit{b41\(\sb{d}\)} \};

d = a[s];
s = b\(\sb{s}\)[4].x;
d = b\(\sb{d}\)[4].y;
\end{alltt}
where $\emit{\cdot} : \DObject -> \TCode$ is some function returning the
$\TCode$ representation of its argument.  Doing array splitting this way,
static-length arrays still end up as individual variables in the residual
program, and $\Ppann$ has not exploded.

\paragraph{Structs in functions returns.} When passing a partially static 
struct as a return value from a function call, somehow both the static and
dynamic struct part must be returned, but in C, only one value can be
returned. We therefore \emph{extend} \coreC_{Core C!extension beyond C}
\emph{beyond} C
so that the "return" statement always takes a tuple containing 
the static and dynamic return value. If either or both of these are empty,
"null" is passed as argument: \bigskip

\noindent
\begin{minipage}[c]{.30\textwidth}
\begin{alltt}
struct P pair; /* \(S\x{D}\) */


struct P f(...) \{
  ...
  return pair;
\}    
\end{alltt}
\end{minipage}
\hfil
$\displaystyle\mathop{==>}\limits\sp{\mathit{split}}$
\hfil
\begin{minipage}[c]{.55\textwidth}
\begin{alltt}
struct P\(\sb{s}\) pair\(\sb{s}\);
struct P\(\sb{d}\) pair\(\sb{d}\);

struct P f(...) \{
  ...
  return (pair\(\sb{s}\), pair\(\sb{d}\));
\}
\end{alltt}
\end{minipage}
\bigskip

\noindent
The tuple is eliminated during the gegen phase_{phase!gegen}, so $\Ppgen$
will of course contain legal C code.  But as this needs special treatment,
\emph{the $\Ppann$ splitting does not change function return
  types_{type!function return}_{type!splitting}.} We have also chosen that
\emph{function declarations_{declaration!function} are never
  split_{splitting!function} into two declarations}, as this could
dramatically increase the overhead in administrating the static
store_{store!static} in the specialization phase_{phase!specialization}.


\subsection{Splitting transformation}
\label{sec:PSDSplittingTransformation}

The two principles in splitting is that
\begin{enumerate}
\item every object (e.g.~variable, struct, array) should be either 
  static_{data!static} or dynamic_{data!dynamic} data.
\item every pointer that can point to partially static data is transformed
  into a struct with two members, "s" and "d", that are static pointers to
  static and dynamic data.
\item Assignments that copy both static and dynamic data are split into two 
  assignments.
\end{enumerate}
When we split a struct assignment, we might have to split recursively into
nested structs_{struct!nested}. 
If nested struct types are viewed as trees, we must perform the split such that
all static leaf nodes end up in the static substruct and all the dynamic
leaf nodes in the dynamic substruct.  
Figure~\vref{fig:SplittingNested} shows how a struct containing an array of
known length (3) of partially static structs is represented in the
generating extension.
\begin{figure}[htb]
  \begin{center}\lineskip 1em
\leavevmode
\setlength{\unitlength}{0.0025\textwidth}%
\begin{tabular}[b]{c}
\begin{picture}(160,90)(-60,-90)
\thinlines
\put(  0,  0){\circle{4}}
  \put(  0,  0){\line(-4,-3){40}}
  \put(  0,  0){\line( 0,-1){30}}
  \put(  0,  0){\line( 1,-1){30}}

\put(-40,-30){\circle{4}}
  \put(-40,-30){\line(-1,-3){10}}
  \put(-40,-30){\line( 1,-3){10}}
\put(  0,-30){\circle*{4}}
  {\thicklines
  \put(  0,-30){\line(-1,-3){10}}
  \put(  0,-30){\line( 1,-3){10}}}
\put( 30,-30){\circle{4}}
  \put( 30,-30){\line( 0,-1){30}}

\put(-50,-60){\circle{4}}
  \put(-50,-60){\line(-1,-3){10}}
  \put(-50,-60){\line( 1,-3){10}}
\put(-30,-60){\circle{4}}
\put(-10,-60){\circle*{4}}
  {\thicklines
  \put(-10,-60){\line(-1,-3){10}}
  \put(-10,-60){\line( 1,-3){10}}}
\put( 10,-60){\circle*{4}}
\put( 30,-60){\circle{4}}
  \put( 30,-60){\line( 1, 0){30}}
  \put( 30,-60){\line(-1,-3){10}}
  \put( 30,-60){\line( 1,-3){10}}
\put( 60,-60){\circle{4}}
  \put( 60,-60){\line( 1, 0){30}}
  \put( 60,-60){\line(-1,-3){10}}
  \put( 60,-60){\line( 1,-3){10}}
\put( 90,-60){\circle{4}}
  \put( 90,-60){\line(-1,-3){10}}
  \put( 90,-60){\line( 1,-3){10}}

\put(-60,-90){\circle{4}}
\put(-40,-90){\circle{4}}
\put(-20,-90){\circle*{4}}
\put(  0,-90){\circle*{4}}
\put( 20,-90){\circle{4}}
\put( 40,-90){\circle*{4}}
\put( 50,-90){\circle{4}}
\put( 70,-90){\circle*{4}}
\put( 80,-90){\circle{4}}
\put(100,-90){\circle*{4}}
\end{picture}
\\
$\Ppann$ struct $S$
\end{tabular}
\begin{tabular}[b]{c}{\small
\begin{picture}(160,90)(-60,-90)
\thinlines
\put(  0,  0){\circle{4}}
  \put(  0,  0){\line(-4,-3){40}}
  \multiput(  0,  0)( 0,-3){10}{\makebox(0,0){.}}
  \put(  0,  0){\line( 1,-1){30}}

\put(-40,-30){\circle{4}}
  \put(-40,-30){\line(-1,-3){10}}
  \put(-40,-30){\line( 1,-3){10}}
%\put(  0,-30){\circle*{4}}
  \multiput(  0,-30)(-1,-3){11}{\makebox(0,0){.}}
  \multiput(  0,-30)( 1,-3){11}{\makebox(0,0){.}}
\put( 30,-30){\circle{4}}
  \put( 30,-30){\line( 0,-1){30}}

\put(-50,-60){\circle{4}}
  \put(-50,-60){\line(-1,-3){10}}
  \put(-50,-60){\line( 1,-3){10}}
\put(-30,-60){\circle{4}}
%\put(-10,-60){\circle*{4}}
  \multiput(-10,-60)(-1,-3){11}{\makebox(0,0){.}}
  \multiput(-10,-60)( 1,-3){11}{\makebox(0,0){.}}
%\put( 10,-60){\circle*{4}}
\put( 30,-60){\circle{4}}
  \put( 30,-60){\line( 1, 0){30}}
  \put( 30,-60){\line(-1,-3){10}}
  \multiput( 30,-60)( 1,-3){11}{\makebox(0,0){.}}
\put( 60,-60){\circle{4}}
  \put( 60,-60){\line( 1, 0){30}}
  \put( 60,-60){\line(-1,-3){10}}
  \multiput( 60,-60)( 1,-3){11}{\makebox(0,0){.}}
\put( 90,-60){\circle{4}}
  \put( 90,-60){\line(-1,-3){10}}
  \multiput( 90,-60)( 1,-3){11}{\makebox(0,0){.}}

\put(-60,-90){\circle{4}}
\put(-40,-90){\circle{4}}
%\put(-20,-90){\circle*{4}}
%\put(  0,-90){\circle*{4}}
\put( 20,-90){\circle{4}}
%\put( 40,-90){\circle*{4}}
\put( 50,-90){\circle{4}}
%\put( 70,-90){\circle*{4}}
\put( 80,-90){\circle{4}}
%\put(100,-90){\circle*{4}}
\end{picture}}
\\
static $\Ppgen$ substruct $S_s$
\end{tabular}
\begin{tabular}[b]{c}
\begin{picture}(160,90)(-60,-90)
\thinlines
\put(  0,  0){\circle{4}}
  \multiput(  0,  0)(-2.5,-1.9){16}{\makebox(0,0){.}}
  \put(  0,  0){\line( 0,-1){30}}
  \put(  0,  0){\line( 1,-1){30}}

%\put(-40,-30){\circle{4}}
  \multiput(-40,-30)(-1,-3){11}{\makebox(0,0){.}}
  \multiput(-40,-30)( 1,-3){11}{\makebox(0,0){.}}
\put(  0,-30){\circle*{4}}
  {\thicklines
  \put(  0,-30){\line(-1,-3){10}}
  \put(  0,-30){\line( 1,-3){10}}}
\put( 30,-30){\circle{4}}
  \put( 30,-30){\line( 0,-1){30}}

%\put(-50,-60){\circle{4}}
  \multiput(-50,-60)(-1,-3){11}{\makebox(0,0){.}}
  \multiput(-50,-60)( 1,-3){11}{\makebox(0,0){.}}
%\put(-30,-60){\circle{4}}
\put(-10,-60){\circle*{4}}
  {\thicklines
  \put(-10,-60){\line(-1,-3){10}}
  \put(-10,-60){\line( 1,-3){10}}}
\put( 10,-60){\circle*{4}}
\put( 30,-60){\circle{4}}
  \put( 30,-60){\line( 1, 0){30}}
  \multiput( 30,-60)(-1,-3){11}{\makebox(0,0){.}}
  {\thicklines
  \put( 30,-60){\line(1,-3){10}}}
\put( 60,-60){\circle{4}}
  \put( 60,-60){\line( 1, 0){30}}
  \multiput( 60,-60)(-1,-3){11}{\makebox(0,0){.}}
  {\thicklines
  \put( 60,-60){\line( 1,-3){10}}}
\put( 90,-60){\circle{4}}
  \multiput( 90,-60)(-1,-3){11}{\makebox(0,0){.}}
  {\thicklines
  \put( 90,-60){\line( 1,-3){10}}}

%\put(-60,-90){\circle{4}}
%\put(-40,-90){\circle{4}}
\put(-20,-90){\circle*{4}}
\put(  0,-90){\circle*{4}}
%\put( 20,-90){\circle{4}}
\put( 40,-90){\circle*{4}}
%\put( 50,-90){\circle{4}}
\put( 70,-90){\circle*{4}}
%\put( 80,-90){\circle{4}}
\put(100,-90){\circle*{4}}
\end{picture}
\\
dynamic $\Ppgen$ substruct $S_d$
\end{tabular}
\begin{tabular}[b]{c}
\begin{picture}(160,90)(-60,-90)
\thicklines
\put(  0,  0){\circle*{4}}
  \multiput(  0,  0)(-2.5,-1.9){16}{\makebox(0,0){.}}
  \put(  0,  0){\line( 0,-1){30}}
  \put(  0,  0){\line( 1,-2){30}}
  \put(  0,  0){\line( 1,-1){60}}
  \put(  0,  0){\line( 3,-2){90}}

%\put(-40,-30){\circle{4}}
  \multiput(-40,-30)(-1,-3){11}{\makebox(0,0){.}}
  \multiput(-40,-30)( 1,-3){11}{\makebox(0,0){.}}
\put(  0,-30){\circle*{4}}
  {\thicklines
  \put(  0,-30){\line(-1,-3){10}}
  \put(  0,-30){\line( 1,-3){10}}}
%\put( 30,-30){\circle{4}}
  \multiput( 30,-30)( 0,-3){10}{\makebox(0,0){.}}

%\put(-50,-60){\circle{4}}
  \multiput(-50,-60)(-1,-3){11}{\makebox(0,0){.}}
  \multiput(-50,-60)( 1,-3){11}{\makebox(0,0){.}}
%\put(-30,-60){\circle{4}}
\put(-10,-60){\circle*{4}}
  {\thicklines
  \put(-10,-60){\line(-1,-3){10}}
  \put(-10,-60){\line( 1,-3){10}}}
\put( 10,-60){\circle*{4}}
\put( 30,-60){\circle*{4}}
  \multiput( 30,-60)( 3, 0){10}{\makebox(0,0){.}}
  \thicklines
  \multiput( 30,-60)(-1,-3){11}{\makebox(0,0){.}}
  \put( 30,-60){\line(1,-3){10}}
\put( 60,-60){\circle*{4}}
  \multiput( 60,-60)( 3, 0){10}{\makebox(0,0){.}}
  \multiput( 60,-60)(-1,-3){11}{\makebox(0,0){.}}
  \put( 60,-60){\line( 1,-3){10}}
\put( 90,-60){\circle*{4}}
  \multiput( 90,-60)(-1,-3){11}{\makebox(0,0){.}}
  \put( 90,-60){\line( 1,-3){10}}

%\put(-60,-90){\circle{4}}
%\put(-40,-90){\circle{4}}
\put(-20,-90){\circle*{4}}
\put(  0,-90){\circle*{4}}
%\put( 20,-90){\circle{4}}
\put( 40,-90){\circle*{4}}
%\put( 50,-90){\circle{4}}
\put( 70,-90){\circle*{4}}
%\put( 80,-90){\circle{4}}
\put(100,-90){\circle*{4}}
\end{picture}
\\
residual $\Ppres$ substruct $S_{\mathit{res}}$
\end{tabular}
    \caption{Splitting nested structures into static and dynamic components}
    \label{fig:SplittingNested}
  \end{center}
\end{figure}
Notice how a layer of indirection has been specialized away in the residual 
program: the constant-length array has been replaced by direct references. 

As we must change the type of some pointers we assume given a predicate,
$\FpsdPtr : \DE -> \Dbool$, to determine whether a pointers points to
partially static data. In general, we say a type $t$ is
\emph{splitable}_{type!splitable} if it contains both static and dynamic
parts. For instance, in Figure~\ref{fig:SplittingNested} only the topmost
node and its rightmost child are splittable. Note that a pointer is
\emph{not} splitable: it is either static, pointing to static/partially
static or dynamic data, or dynamic, pointing to dynamic data.

The output of the split phase is a program where each assignment copies
only fully static or fully dynamic values, where every parameter is either
fully static or fully dynamic, and where the types are binding-time
annotated with subscripts $s$ or $d$ according to with the following
notation:
\begin{center}
\begin{tabular}{ll}\label{tab:PSDannotatedTypes}
$\CPointer_s$   & $\CPointer_s^s$ or $\CPointer_s^d$ \\
$\CPointer_s^s$ & static pointer to static data, $\CArray_s^s$ or $\CUser_s^s$
\\
$\CPointer_s^d$ & static pointer to dynamic data, $\CArray_s^d$ or $\CUser_s^d$ \\
$\CArray_s^s$ & array of known length of static elements \\
$\CArray_s^d$ & array of known length of dynamic elements \\
$\CUser_s^s$ & static struct with only static members \\
$\CUser_s^d$ & static struct with only dynamic members \\
\end{tabular}
\end{center}

In the following, we define a number of split functions_{split function}
for various syntactical constructs:
\[
\begin{array}{lcl}
  \Fsplitt &:& \DT -> (\DT \x \DT + \DT \x \DNil + \DNil \x \DT) \\
  \Fsplitd &:& \DD -> (\DD \x \DD + \DD \x \DNil + \DNil \x \DD) \\
  \Fsplite &:& \DE -> (\DE \x \DE + \DE \x \DNil + \DNil \x \DE) \\
  \Fsplit_{=} &:& \DAssignment -> \DAssignment \DList \\
  \Fsplits &:& \DS -> \DS \DList \\
  \Fsplitj &:& \DJ -> \DJ \\
  \Fsplitb &:& \DB -> \DB \\
  \Fsplitu &:& \DU -> \DU \DList \\
  \Fsplitp &:& \DP -> \DP, \\
\end{array}
\]
where $\DAssignment = \{\CAssign(\delta,e), \CPAssign(\delta,e),
\CStrtAssign(\delta,e,\varphi), \CPStrtAssign(\delta,e,\varphi)  \}$ and
$\DNil = \{\CNil\}$.

The functions for splitting types is given
in Figure~\ref{fig:PDSSplitTypes}.%
\begin{figure}[htbp]
  \begin{center}
    \leavevmode\hbox{\vbox{%
    \begin{pseudocode}
      $\Fsplitt(t) = \Kif\ \Fbt(t) = D\ \Kthen\ (\CNil, t)\ 
      \Kelse$ \+\\
        $\Kcase\ t\ \Kof$ \+\\
          $\CInt(q), \CFloat(q), ... : (t, \CNil)$ 
              \\[1ex]
          $\CPointer(q, t') : {}$ \+\\
            $\Kcase\ \Fsplitt(t')\ \Kof$ \+\\
              $(t_s,t_d)$ \ \=:\ \=$(\CUser_s^s(q, \Fsplit_{\sigma}
                                    ([], t, t_s, t_d)), \CNil)$\\
              $(t_s,\CNil) $ \>:\> $(\CPointer_s^s(q, t_s), \CNil)$ \\
              $(\CNil,t_d) $ \>:\> $(\CPointer_s^d(q, t_d), \CNil)$ 
              \-\-\\[1ex]
          $\CArray(q, t', \Voe) : {}$ \+\\
            $\Kcase\ \Fsplitt(t')\ \Kof$ \+\\
              $(t_s,t_d) $ \ \= : \= 
              $(\CArray_s^s(q, t_s, \Voe), 
                \CArray_s^d(q, t_d, \Voe))$ \\
              $(t_s,\CNil) $\>:\>$(\CArray_s^s(q, t_s, \Voe), \CNil)$\\
              $(\CNil,t_d) $\>:\>$(\CNil, \CArray_s^d(q, t_d, \Voe))$ 
              \-\-\\[1ex]
          $\CFunT(q, t', t_1 ... t_n) : 
             (\CFunT(q, \Fsplitt(t'), \FsplitArgst(t_1, ..., t_n)),
              \CNil)$ 
           \\[1ex]
          $\CUser(q, \sigma, d_1 ... d_n) : $ \+\\
            $\Kcase\ \FsplitMems(d_1, ..., d_n)\ \Kof$ \+\\
              $(\Vss,\Vdd)$ \= : \= 
                $(\CUser_s^s(q, \sigma_s, \Vss), 
                  \CUser_s^d(q, \sigma_d, \Vdd))$ \\
              $(\Vss,[])$ \>:\> $(\CUser_s^s(q, \sigma, \Vss), \CNil)$ \\
              $([],\Vdd)$ \>:\> $(\CNil, \CUser_s^d(q, \sigma, \Vdd))$ 
              \-\-\\[1ex]
          $\CAbstract(q, \Vid) : 
            \Kif\ \Fbt(t) = D\ \Kthen\ (\CNil, t)\ \Kelse\ (t, \CNil)$
          \-\-\\[1em]
      $\FsplitMems(d_1, ..., d_n) = {}$ \+\\
        $\Kfor\ i \in \{1, ..., n\}\ \Kdo\
          (\Vss_i, \Vdd_i) := \Fsplitd(d_i)$\\
        $\Kreturn\ (\Fconcat [\Vss_1, ..., \Vss_n],
                    \Fconcat [\Vdd_1, ..., \Vdd_n])$ 
          \-\\[1em]
      $\FsplitArgs_{\chi}(x_1, ..., x_n) = {}$ \+\\
        $\Vas := []$ \\
        $\Kfor\ i = 1, ..., n\ \Kdo$ \+\\
          $\Kcase\ \Fsplit_{\chi}(x_i)\ \Kof$ \+\\
            $(x_{s}, x_{d})$ \ \= : \=
              $\Vas := \Vas \append [x_{s},x_d]$ \\
            $(x_s, \CNil)$  \>:\> $\Vas := \Vas \append [x_s]$ \\
            $(\CNil, x_d)$  \>:\> $\Vas := \Vas \append [x_d]$ \-\-\\
        $\Kreturn\ \Vas$ 
        \-\\[1em]
      $\Fsplit_{\sigma}(\Vms, t, t_s, t_d) = {}$ \+\\
        $\Kcase\ t\ \Kof$ \+\\
        $\CUser(q, \sigma, d_1 ... d_n) : {} $ \+\\
\quad\=\quad\=\quad\=\quad\=\quad\=\quad\=\quad\=\quad\=\quad\=\quad\=\kill
           $\Kif\ \sigma_{\Vms} = \CNull\ \Kthen$ \+\\
            $\sigma_{\Vms} := \Ffresh();\
             \delta_s := \Ffresh();\ \delta_d := \Ffresh()$ \\
            $(\sigma_{\Vms}, \Void, \CStruct_s,
            \begin{array}[t]{@{}l@{}}
              \CStructMem(\delta_s, "s", \CPointer_s^s(q, t_s)), \\
              \CStructMem(\delta_d, "d", \CPointer_s^d(q, t_d)))
            \end{array} $ \-\\
          $\Kreturn\ \sigma_{\Vms}$ \-\\
        $\CArray(q, t, e) : 
          \Fsplit_{\sigma}([m_1, ..., m_k, e], t, t_s, t_d)$
    \end{pseudocode}}}
    \caption{Splitting transformation for \coreC types}
    \label{fig:PDSSplitTypes}
  \end{center}
\end{figure}
Note that each time a pointer to a (possibly multidimensional) array of
partially static structs is encountered, we must generate a new struct type
definition for a struct with two static pointers to arrays of the static and
dynamic struct parts, e.g.: 
\begin{center}\def\d{\(\sb{d}\)}\def\s{\(\sb{s}\)}%
\leavevmode
\hbox to .3\textwidth{\hss$p$\hss\hss}\hfil
\hbox to .68\textwidth{\hss$\Ppsplit$\hss\hss}\nopagebreak[4]\smallskip
\nopagebreak[4]

\small
\begin{minipage}[t]{.3\textwidth}
\begin{alltt}
struct T \verb"{" int s; int d; \verb"}";







struct T  *t0;
struct T (*t1)[42];
struct T (*t2)[17];
struct T (*t3)[17][42];
struct T (*t4)[42];
\end{alltt}
\end{minipage}\hfil
\begin{minipage}[t]{.68\textwidth}
\begin{alltt}
struct T\s \verb"{" int s; \verb"}";
struct T\d \verb"{" int d; \verb"}";

struct T0    \verb"{" struct T\s  *s;          struct T\d  *d; \verb"}";
struct T42   \verb"{" struct T\s (*s)[42];     struct T\d (*d)[42]; \verb"}";
struct T17   \verb"{" struct T\s (*s)[17];     struct T\d (*d)[17]; \verb"}";
struct T1742 \verb"{" struct T\s (*s)[17][42]; struct T\d (*d)[17][42]; \verb"}";

struct T0    t0;
struct T42   t1;
struct T17   t2;
struct T1742 t3;
struct T42   t4;
\end{alltt}
\end{minipage}
\end{center}
For this reason, we keep for each user type $\sigma$ a set of split pointer
types indexed on the dimension sizes: $\sigma_{\Vms}$ where $\Vms =
[m_1,...,m_k]$. The function $\Fsplit_{\sigma}$ makes sure to reuse any
previously defined split pointer type (like "t1" and "t4" in the example
above)---otherwise assignments like "t1 = t4" would become illegal after
the split.

The split functions for declarations and expressions are
shown in
Figures~\ref{fig:PDSSplitDeclarations}--\ref{fig:PDSSplitExpressionsII}.%
\begin{figure}[htbp]
  \begin{center}
    \leavevmode\hbox{\vbox{%
    \begin{pseudocode}
      $\Fsplitd(d) = {}$ \+\\
        $\Kcase\ d\ \Kof$ \+\\
          $\CVar(\delta, \Void, t, \Vocs) : {}$ \+\\
            $\Kcase\ \Fsplitt(t)\ \Kof$ \+\\
              $(t_s, t_d) : $ \+\\
                $(\Vocs_s, \Vocs_d) := \FsplitInits(t,\Vocs)$ \\
                $\delta_s := \Ffresh();\ \delta_d := \Ffresh()$ \\
                $\Kreturn\ (\CVar_s(\delta_s, \Void, t_s, \Vocs_s),
                            \CVar_d(\delta_d, \Void, t_d, \Vocs_d))$ \-\\
              $(t_s,\CNil) : 
              (\CVar_s(\delta, \Void, t_s, \Vocs), \CNil)$ \\
              $(\CNil,t_d) : 
              (\CNil, \CVar_d(\delta, \Void, t_d, \Vocs))$ 
              \-\-\\[1ex]
          $\CFun(\delta, \Void, t, d_1 ... d_n, d'_1 ... d'_m,
                 b_1 ... b_k,) : {}$ \+\\
            $\CFun(\delta, \Void, t, \FsplitArgsd(d_1, ..., d_n),
            \FsplitArgsd(d'_1, ..., d'_m), 
            \Fsplitb(b_1) ... \Fsplitb(b_k))$
            \-\\[1ex]
          $\CStructMem(\delta, \Void, t, \Voc) : $ \+\\
            $\Kcase\ \Fsplitt(t)\ \Kof$ \+\\
              $(t_s, t_d) : $ \+\\
                $\delta_s := \Ffresh();\ \delta_d := \Ffresh()$ \\
                $(\CStructMem_s(\delta_s, \Void_s, t_s, \CNoConst),
                  \CStructMem_d(\delta_d, \Void_d, t_d, \CNoConst))$
                  \-\\
              $(t_s, \CNil) : 
                (\CStructMem_s(\delta, \Void_s, t_s, \Voc),
                 \CNil)$ \\
              $(\CNil, t_d) :  
                (\CNil, \CStructMem_d(\delta, \Void_d, t_d, \Voc))$ 
              \-\-\\[1ex]
          $\CEnumMem(\delta, \Vid, t, \Voc) : {}$ \+\\
            $\Kif\ \Fbt(\delta) = S\ \Kthen\
               (\CEnumMem_s(\delta, \Vid, t, \Voc), \CNil)\
             \Kelse\ \Kerror$
    \end{pseudocode}}}
    \caption{Splitting transformation for \coreC declarations}
    \label{fig:PDSSplitDeclarations}
  \end{center}
\end{figure}%
\begin{figure}[htbp]
  \begin{center}
    \leavevmode\hbox{\vbox{%
    \begin{pseudocode}
      $\FsplitInits(t, \Vcs) =
         \Kcase\ \FsplitInits'(t, \Vcs, [], [])\ \Kof\ 
         (\_\!\_, \Vcs_s, \Vcs_d) : (\Vcs_s, \Vcs_d)$ \\[1em]
      $\FsplitInits'(t, c:\Vcs, \Vcs_s, \Vcs_d) = {}$ \+\\
         $\Kcase\ t\ \Kof$ \+\\
           $\CInt(q), \CFloat(q), ... : 
            \Kif\ \Fbt(t) = S\ \Kthen\ 
              (\Vcs, \Vcs_s \append [c], \Vcs_d)\
            \Kelse\
              (\Vcs, \Vcs_s, \Vcs_d \append [c])$ 
              \\[1ex]
           $\CPointer(q,t') : $ \+\\
             $\Kcase\ \Fsplitt(t)\ \Kof$ \+\\
               $(\CUser(...), \CNil) : {}$ \+\\
                 $\Kcase\ c\ \Kof$ \+\\
                   $\CBinary(e_1, \pm, e_2)$ \= : \= \kill
                   $\CRval(\delta)$ \> : \> 
                     $(\Vcs, \Vcs_s \append 
                     [\CRval(\delta)], \Vcs_d)$ \\
                   $\CLval(\delta)$ \> : \>
                     $(\Vcs, \Vcs_s \append 
                     [\CLval(\delta_s),
                      \CLval(\delta_d)], \Vcs_d)$ \\
                   $\CConst("NULL")$ \> : \>
                     $(\Vcs, \Vcs_s \append ["NULL", "NULL"],
                             \Vcs_d)$ \\
                   $\CUnary("&", \CRval(\delta))$ \> : \> 
                     $(\Vcs, \Vcs_s  \append 
                     [\CUnary("&", \CRval(\delta_s)),
                      \CUnary("&", \CRval(\delta_d))], \Vcs_d)$ \\
                   $\CUnary(\Vuop, e_1)$ \> : \> ? \\
                   $\CBinary(e_1, \pm, e_2)$ \> : \>
                     $(\Vcs, \Vcs_s \append 
                        [
                        \begin{array}[t]{@{}l@{}}
                          \CBinary(\CMember(e_1, "s"), \pm, e_2), \\
                          \CBinary(\CMember(e_1, "d"), \pm, e_2)]),
                          \Vcs_d)
                        \end{array}$ \\
                   $\CCast(t, e_1)$ \> : \> ? \\
                   $\CMember(e_1, \varphi)$ \> : \> ? \-\-\\
               $(\CPointer_s, \CNil) :
                  (\Vcs, \Vcs_s \append [c], \Vcs_d)$ \\
               $(\CNil, \CPointer_d) :
                  (\Vcs, \Vcs_s, \Vcs_d \append [c])$ \-\-
             \\[1ex]
           $\CArray(q,t',e) : {}$ \+\\
             $\Vcs := c:\Vcs$\\
             $\Kfor\ i = 1, ..., \Fvalue(e)\ \Kdo\
               (\Vcs, \Vcs_s, \Vcs_d) := 
                 \FsplitInits'(t', \Vcs, \Vcs_s, \Vcs_d)$ \\
             $\Kreturn\ (\Vcs, \Vcs_s, \Vcs_d)$ 
             \-\\[1ex]
%           $\CFunT(q, t, t_1 ... t_n) : 
%            \CFunT(q, t, \FsplitArgs(t_1, ..., t_n))$ 
%           \\[1ex]
           $\CUser(q, \sigma, d_1 ... d_n) : {}$ \+\\
             $\Kfor\ \varphi = 1, ..., n\ \Kdo\
               (\Vcs, \Vcs_s, \Vcs_d) := 
                 \FsplitInits'(t_{d_\varphi}, \Vcs, \Vcs_s, \Vcs_d)$ \\
             $\Kreturn\ (\Vcs, \Vcs_s, \Vcs_d)$ 
             \-\\[1ex]
           $\CAbstract(q, \Vid) : ???$
    \end{pseudocode}}}
    \caption{Splitting transformation for \coreC initializers}
    \label{fig:PDSSplitInitializers}
  \end{center}
\end{figure}%
\begin{figure}[htbp]
  \begin{center}\leavevmode\hbox{\vbox{%
    \begin{pseudocode}
      $\Fsplite(e) = \Kif\ \Fbt(e) = D\ \Kthen\ (\CNil, e)\ \Kelse$ \+\\
        $\Kcase\ e\ \Kof$ \+\\
          $\CRval(\delta) : $ \+\\
            $\Kcase\ \Fsplitt(t_\delta)\ \Kof$ \+\\
              $(t_s,t_d)$ \ \= : \= $(\CRval(\delta_s), \CRval(\delta_d))$ \\
              $(t_s,\CNil)$ \> : \> $(\CRval(\delta), \CNil))$ \\
              $(\CNil,t_d)$ \> : \> $(\CNil, \CRval(\delta))$
              \-\-\\[1ex]
\quad\=\quad\=\quad\=\quad\=\quad\=\quad\=\quad\=\quad\=\quad\=\quad\=\kill
          $\CLval(\delta) : $ \+\\
            $\Kcase\ \Fsplitt(t_\delta)\ \Kof$ \+\\
              $(t_s,t_d) : {}$ \+\\
                $\Vtemp := \Ffresh()$ \\
                $\VpreStmt := \VpreStmt \append
                [{\Vtemp}".s = "\CLval(\delta_s),\
                 {\Vtemp}".d = "\CLval(\delta_d)]$ \\
                $\Kreturn\ (\CRval(\delta_{\Vtemp}), \CNil)$ \-\\
              $(t_s,\CNil)$ \= : \= $(\CLval_s(\delta), \CNil)$ \\
              $(\CNil,t_d)$ \> : \> $(\CLval_s(\delta), \CNil)$
              \-\-\\[1ex]
\quad\=\quad\=\quad\=\quad\=\quad\=\quad\=\quad\=\quad\=\quad\=\quad\=\kill
          $\CConst(c) : {}$ \+\\
            $\Kif\ \FpsdPtr(e)\ \Kthen$ \+\\
               $(\CConst_s("NULL"_{\Vpsd}), \CNil)$ \-\\
             $\Kelse$ \+\\
               $\Kif\ \Fbt(e) = D\ \Kthen\ 
                  (\CNil, \CConst_d(c))\
                \Kelse\
                  (\CConst_s(c), \CNil)$ 
              \-\-\\[1ex]
          $\CUnary("*",e_1) : {}$ \+\\
            $\Kif\ \FpsdPtr(e_1)\ \Kthen$ \+\\
              $(\CUnary_s("*", \CMember(e_1, "s")), 
                \CUnary_s("*", \CMember(e_1, "d")))$ \-\\
            $\Kelse$ \+\\
              $\Kcase\ \Fsplite(e_1)\ \Kof$ \+\\
                $(e_s, \CNil)$ \=: \= $(\CUnary_s("*", e_s), \CNil)$ \\
                $(\CNil, e_d)$ \>: \> $(\CNil, \CUnary_d("*", e_d))$ 
              \-\-\-\\[1ex]
          $\CUnary("&",e_1) : $ \+\\
            $\Kcase\ \Fsplite(e_1)\ \Kof$ \+\\
              $(e_s,e_d) : {}$ \+\\
                $\Vtemp := \Ffresh()$ \\
                $\VpreStmt := \VpreStmt \append
                [{\Vtemp}".s = "\CUnary_s("&", e_s),\
                 {\Vtemp}".d = "\CUnary_s("&", e_d)]$ \\
                $\Kreturn\ (\CRval(\delta_{\Vtemp}), \CNil)$ \-\\
              $(e_s, \CNil)$ \=: \= $(\CUnary_s("&", e_s), \CNil)$ \\
              $(\CNil, e_d)$ \>: \> $(\CUnary_s("&", e_d), \CNil)$ 
              \-\-\\[1ex]
          $\CUnary(\Vuop,e_1) : $ \+\\
            $\Kcase\ \Fsplite(e_1)\ \Kof$ \+\\
              $(e_s,e_d)$ \ \= : \= 
                $(\CUnary_s(\Vuop, e_{s}), \CUnary_d(\Vuop, e_{d}))$ \\
              $(e_s, \CNil)$ \>:\> $(\CUnary_s(\Vuop, e_s), \CNil)$ \\
              $(\CNil, e_d)$ \>:\> $(\CNil, \CUnary_d(\Vuop, e_d))$ 
              \-\-\\[1ex]
    \end{pseudocode}}}
    \caption{Splitting transformation for \coreC expressions, part I}
    \label{fig:PDSSplitExpressionsI}
  \end{center}
\end{figure}%
\begin{figure}[htbp]
  \begin{center}\leavevmode\hbox{\vbox{%
    \begin{pseudocode}
          $\CBinary(e_1,\Vbop,e_2) : $ \+\\
            $\Kif\ \Vbop = {\pm} \ \land \ {\FpsdPtr}(e_1)\ \Kthen\ 
              \Kcase\ (\Fsplite(e_1), \Fsplite(e_2))\ \Kof\
              ((e'_1, \CNil), (e'_2, \CNil)) : {}$ \+\\
              $\Vtemp := \Ffresh()$ \\
              $\VpreStmt := \VpreStmt \append
              [\Vtemp'" = "e'_2,\
               {\Vtemp}".s = "e'_1".s + "{\Vtemp'},\
               {\Vtemp}".d = "e'_1".d + "{\Vtemp'}]$ \\
              $\Kreturn\ (\CRval_s(\delta_{\Vtemp}), \CNil)$ \-\\
            $\Kelse\ 
             \Kif\ \Vbop \in \{"-","==","!="\} \ \land \ 
             {\FpsdPtr}(e_1) \land {\FpsdPtr}(e_2)\ \Kthen\ $ \+\\
              $\Kcase\ (\Fsplite(e_1), \Fsplite(e_2))\ \Kof$ \+\\
                 $((e'_1, \CNil), (e'_2, \CNil)) : 
                  (\CBinary_s(\CMember(e'_1, "s"), \Vbop,
                               \CMember(e'_2, "s")), \CNil)$ \-\-\\
            $\Kelse$ \+\\
              $\Kcase\ (\Fsplite(e_1), \Fsplite(e_2))\ \Kof$ \+\\
                $((e_{1s}, \CNil), (e_{2s}, \CNil))$ \= : \=
                  $(\CBinary_s(e_{1s}, \Vbop, e_{2s}), \CNil)$ \\
                $((\CNil, e_{1d}), (\CNil, e_{2d}))$ \> : \> 
                  $(\CNil, \CBinary_d(e_{1d}, \Vbop, e_{2d}))$ \\
                $((\CNil, e_{1d}), (e_{2s}, \CNil))$ \> : \>
                  $(\CNil, \CBinary_d(e_{1d}, \Vbop, \Flift(e_{2s})))$ \\
                $((e_{1s}, \CNil), (\CNil, e_{2d}))$ \> : \> 
                  $(\CNil, \CBinary_d(\Flift(e_{1s}), \Vbop, e_{2d}))$ 
                \-\-\-\\[1ex]
          $\CCast(t,e_1) : $ \+\\
            $\Kcase\ (\Fsplitt(t), \Fsplite(e_1))\ \Kof$ \+\\
              $((t_s, t_d), (e_s, e_d))$ \quad \= : \= 
                 $(\CCast_s(t_s, e_s), \CCast_d(t_d, e_d))$ \\
              $((t_s, \CNil), (e_s, \CNil))$ \>:\> 
                $(\CCast_s(t_s,e_s), \CNil)$ \\
              $((\CNil, t_d), (\CNil, e_d))$ \>:\> 
                $(\CNil, \CCast_d(t_d,e_d))$
                \-\-\\[1ex]
\quad\=\quad\=\quad\=\quad\=\quad\=\quad\=\quad\=\quad\=\quad\=\quad\=\kill
          $\CMember(e_1,\varphi) : $ \+\\
            $\Kcase\ \Fsplite(e_1)\ \Kof$ \+\\
              $(e_s, e_d) : $ \+\\
                $\Kcase\ \Fsplitt(t_{(e_1.\varphi)})\ \Kof$ \+\\
                  $(t_s,t_d)$ \ \= : \= 
                    $(\CMember_s(e_s, \varphi), 
                      \CMember_d(e_d, \varphi))$ \\
                  $(t_s,\CNil)$ \> : \> 
                    $(\CMember_s(e_s, \varphi), \CNil)$ \\
                  $(\CNil,t_d)$ \> : \> 
                    $(\CNil, \CMember_d(e_d, \varphi))$ 
                    \-\-\\[1ex]
              $(e_s, \CNil) : $ \+\\
                $\Kcase\ \Fsplitt(t_{(e_1.\varphi)})\ \Kof$ \+\\
                  $(t_s,\CNil) : (\CMember_s(e_s, \varphi), \CNil)$
                    \-\-\\[1ex]
              $(\CNil, e_d) : $ \+\\
                $\Kcase\ \Fsplitt(t_{(e_1.\varphi)})\ \Kof$ \+\\
                  $(\CNil,t_d) : (\CNil, \CMember_d(e_d, \varphi))$
    \end{pseudocode}}}
    \caption{Splitting transformation for \coreC expressions, part II}
    \label{fig:PDSSplitExpressionsII}
  \end{center}
\end{figure}%

Function $\Fsplite$_{splite@$\Fsplite$} takes an expression $e$ and returns
either
\begin{enumerate}
\item a  pair of expressions $(e_s, e_d)$ if $e$ has a splitable type, or
\item a pair  $(e_s, \CNil)$ if $e$'s type is not splitable and
  is static, or
\item a pair $(\CNil, e_d)$ if $e$'s type is not splitable and
  is dynamic.
\end{enumerate}
The $e_s$ and $e_d$ components are the expressions by which the static and
dynamic parts of $e$ should be referenced. For instance,
\begin{center}
  $\Fsplite("t") = ("t"_s,"t"_d)$ \qquad 
  $\Fsplite("t.s") = ("t"_s".s", \CNil)$ \qquad
  $\Fsplite("t.d") = (\CNil,"t"_d".d")$,
\end{center}
assuming "t" is a struct with staic member "s" and dynamic member "d". 

Note that in the cases where an expression is a pointer to partially static
data (by the \& operator, by adding an integer to a pointer or by the
constant "NULL"), we must create a new temporary struct containing the two
versions of the pointer. This is done by adding some struct assignment
statements as a side effect to a list, $\VpreStmt$.


The
split functions for types and declarations work analogously.

The function $\Fsplit_{=}$_{split=@$\Fsplit_{=}$} shown in
Figure~\vref{fig:PDSSplittingAssignments} splits any of the four types of
assignment operations.%
\begin{figure}[htb]
  \begin{center}\leavevmode\hbox{\vbox{%
      \begin{pseudocode}
$\Fsplit_{=}[[e" = "e']] = {}$ \+\\
  $\Kcase\ \Fsplite(e')\ \Kof$ \+\\
    $(e'_s,e'_d)$\ \= : \= $[e\{\delta:=\delta_s\}" ="_s" "e'_s,
                             e\{\delta:=\delta_d\}" ="_d" "e'_d]$ \\
    $(e'_s,\CNil)$ \> : \> $[e\{\delta:=\delta_s\}" ="_s" "e'_s]$ \\
    $(\CNil,e'_d)$ \> : \> $[e\{\delta:=\delta_d\}" ="_d" "e'_d]$ \\
      \end{pseudocode}}}
    \caption{Algorithm for splitting assignments into static and dynamic
      parts. `$e\{\delta:=\delta'\}$' denotes substitution.} 
    \label{fig:PDSSplittingAssignments}
  \end{center}
\end{figure}

These functions can now be used to define splitting functions for
statements, control statements and basic blocks in
Figure~\ref{fig:PDSSplitStatementsBBs}.
\begin{figure}[htbp]
  \begin{center}
\[
    \begin{array}{lll}
      \textnormal{Statement } s       & \Fsplits(s) \\\hline
      \CAssign(\delta,e)              & \VpreStmt \append 
                                        \Fsplit_{=}[[\delta" = "e]] \\
      \CPAssign(\delta,e)             & \VpreStmt \append 
                                        \Fsplit_{=}[["*"\delta" = "e]] \\
      \CStrtAssign(\delta,e,\varphi)  
         & \VpreStmt \append 
           \Fsplit_{=}[[\delta"."\varphi" = "e]] \\
      \CPStrtAssign(\delta,e,\varphi) 
         & \VpreStmt \append 
           \Fsplit_{=}[["(*"\delta")."\varphi" = "e]] \\
      \CCall(\optdelta,\delta,e_1...e_n)
         & \VpreStmt \append 
           [\CCall(\optdelta,\delta, \FsplitArgse(e_1,...e_n))] \\
      \CPCall(\optdelta,\delta,e_1...e_n)
         & \VpreStmt \append 
           [\CPCall(\optdelta,\delta, \FsplitArgse(e_1,...,e_n))] \\
      \CMalloc(\delta,t)
         & [\CMalloc(\delta_s,t_s), \CMalloc(\delta_d,t_d)], 
         & \textnormal{if $t$ splitable} \\
      \CMalloc(\delta,t)
         & [\CMalloc(\delta,t)], 
         & \textnormal{else}\\
      \CCalloc(\delta,t,e)            
         & [\CCalloc(\delta_s,t_s,e), \CCalloc(\delta_d,t_d,e)],
         & \textnormal{if $t$ splitable} \\
      \CCalloc(\delta,t,e)            
         & [\CCalloc(\delta,t,e)],
         & \textnormal{else} \\
      \CFree(e)                       & ??? \\
      \CSequence                      & ??? 
      \\[1em]
      \textnormal{Control statement } j & \Fsplitj(j) \\\hline
      \CIf(e,l_1,l_2)   & \CIf(\tilde{e},l_1,l_2) \\
      \CGoto(l)         & \CGoto(l) \\
      \CReturn(\CNoExp) & \CReturn(\CNoExp, \CNoExp) \\
      \CReturn(e)  & \CReturn(\Fsplite(e)) \\[1ex]
      \mc{3}{l}{\textnormal{where } \tilde{e} = 
        \begin{array}[t]{@{}l@{}}
          \Kcase\ \Fsplite(e)\ \Kof \\
          \quad (e_s, e_d)   : \Kerror \\
          \quad (e_s, \CNil) : e_s \\
          \quad (\CNil, e_d) : e_d \\
        \end{array}}\\
      \\
      \textnormal{Basic block } b & \Fsplitb(b) \\\hline
      (l, \Void,s_1...s_n,j) &
      \mc{2}{l}{
        (l, \Void, \Fconcat(\Fsplits(s_1), ..., \Fsplits(s_n),
        \VpreStmt_{\mathcal{J}}),
        \Fsplitj(j))}
      \\[1em]
      \textnormal{User type definition } u & \Fsplitu(u) \\\hline
       (\sigma, \Void, \CEnum, d_1...d_n)  &
      [(\sigma, \Void, \CEnum, d_1...d_n)] \\
       (\sigma, \Void, \CUnion, d_1...d_n)  & 
      [(\sigma, \Void, \CUnion_s, d_1...d_n)]
      & \textnormal{if }\Fbt(d_1) = S \\
       (\sigma, \Void, \CUnion, d_1...d_n)  & 
      [(\sigma, \Void, \CUnion_d, d_1...d_n)]
      & \textnormal{if }\Fbt(d_1) = D \\
       (\sigma, \Void, \CStruct, d_1...d_n) &
      [(\sigma_s, \Void_s, \CStruct_s, \Vss),
       (\sigma_d, \Void_d, \CStruct_d, \Vdd)] 
      & \textnormal{if $u$ splitable} \\ 
       (\sigma, \Void, \CStruct, d_1...d_n) &
      [(\sigma, \Void, \CStruct_s, \FsplitArgsd(d_1, ..., d_n))] & 
      \textnormal{else if } \Fbt(d_1) = S \\ 
       (\sigma, \Void, \CStruct, d_1...d_n) &
      [(\sigma, \Void, \CStruct_d, \FsplitArgsd(d_1, ..., d_n))] & 
      \textnormal{else} \\[1ex]
      \mc{3}{l}{\textnormal{where } 
        (\Vss,\Vdd) = \FsplitMems(d_1, ..., d_n)}
      \\[1em]
      \textnormal{Program } p & \Fsplitp(p) \\\hline
      (
      \begin{array}[t]{@{}l@{}}
        u_1 ... u_n, d^v_1 ... d^v_m, \\
        d^f_1 ... d^f_l, d^{ve}_1 ... d^{ve}_k, \\
        d^{fe}_1 ... d^{fe}_j, b_1 ... b_h)
      \end{array}
      & \mc{2}{l}{
        \begin{array}[t]{@{}l@{}}
          \Klet\\
          \quad \Vus := \Fsplitu(u_1) \append \cdots 
                                      \append \Fsplitu(u_n) \\
          \quad \Vds^v := \FsplitArgsd(d^v_1, ..., d^v_m) \\
          \quad \Vds^f := \FsplitArgsd(d^f_1, ..., d^f_m) \\
          \quad \Vus':= [(\sigma_{\Vms}, \Void_{\sigma_{\Vms}},
                        \CStruct_s, d_s, d_d) \mid
            \sigma_{\Vms} \ne \CNull ] \\
          \Kin\\
          \quad (\Vus \append \Vus', \Vds^v, \Vds^f, 
          d^{ve}_1 ... d^{ve}_k, d^{fe}_1 ... d^{fe}_j, 
                 \Fsplitb(b_1) ...\Fsplitb(b_1))
        \end{array}}
      \end{array}
    \]
    \caption{Splitting transformation for \coreC statements, control
      statements,  basic blocks, userdefined types and programs}
    \label{fig:PDSSplitStatementsBBs}
  \end{center}
\end{figure}%
Function $\Fsplits$_{splits@$\Fsplits$} takes a statement and returns a
list of statements that perform the same operation, but split into totally
static_{assignment!totally static} and totally dynamic_{assignment!totally
  dynamic} operations.


\subsection{Implementation level (22.04.1998)}
\label{sec:PSDImplementationLevel}

The splitting algorithms in this chapter have not been implemented in
\cmix yet.

\end{docpart}

%%% Local Variables: 
%%% mode: latex
%%% TeX-master: "cmixII"
%%% End: 

% File: structsort.tex
% Time-stamp: 
% $Id: structsort.tex,v 1.2 1999/03/16 22:21:19 makholm Exp $

\providecommand{\docpart}{\renewenvironment{docpart}{}{}
\end{docpart}
\documentclass[twoside]{cmixdoc}
%\bibliographystyle{apacite}

\makeatletter
\@ifundefined{@title}{\title{\cmix-documentation}}{}
\@ifundefined{@author}{\author{The \cmix{} Team}}%
{\expandafter\def\expandafter\@realauthor\expandafter{\@author}%
\author{The \cmix{} Team\\(\@realauthor)}}
\makeatother

\AtBeginDocument{%
\markboth{\hfill\today\quad\timenow\hfill\llap{\cmix\ documentation}}
{\hfill\today\quad\timenow\hfill}}

\renewcommand{\sectionmark}[1]{\markboth
{\hfill\today\quad\timenow\hfill\llap{\cmix\ documentation}}
{\rlap{\thesection. #1}\hfill\today\quad\timenow\hfill}}

%\newboolean{separate}
%\setboolean{separate}{true}

\renewenvironment{docpart}{\begin{document}}%
                          {\bibliography{cmixII}\printindex
                           \end{document}}
\begin{document}\shortindexingon
}
\begin{docpart}

\section{Structure definition sorting}
\label{sec:StructureDefinitionSorting}
The structure sorting phase is a helper phase that takes place just
before the gegen phase. Its purpose is to perform a ``topological
sort'' on the final list of structure or union definition. This is
done to ensure that the generating extension will declare the
structures in the correct order, such that \eg the definition
\begin{verbatim}
struct foo {
   struct bar baz ;
}
\end{verbatim}
will not be written out before struct bar has been declared.

\subsection{The ``sequence number'' strategy}
Theoretically the order we want is a topological sort according to
the ``containment ordering'' defined by ``struct $A$
precedes struct $B$ iff struct $B$constains a member whose type is
`(array of)$^n$ struct $A$' for
some $n\geq 0$''.

This is implemented by assigning \emph{sequence numbers} to the struct 
definitions so that if struct $A$ precedes struct $B$, then $A$ will have a
lower sequence number than $B$. Then we can perform the topological
sort simply by sorting by sequence number.

We assign the sequence numbers in the C parser, then check in the type
checking phase that the connection between the sequence numbers and
the ``contaiment ordering'' hold. The sequence numbers are then
carried through to \coreC and simply copied when copies of structures
are made during the various phases. After struct definition have been
manipulated we have a different ``containment ordering'' but it is
easily seen, by inspection of the different phase definitions, that
the projection of the final containment ordering to the original
source definitions is a subset of the contaiment ordering that was
used in the type cheker. Thus the sequence numbers still have the
required property with respect to the final containment ordering.

\subsection{Assigning sequence numbers}

In the parser phase we simply maintain a counter and assign ascending
sequence numbers in the order we see the closing brace of the
definition of each structure. We may see the same structure defined
multiple times if there are more than one source file; in that case
the first definition we see is the one that counts.

This strategy works if the input is correct according to \ansiC;
\ansiC requires the types of members to be object types, and
`(array of)$^n$ struct T' is an object type only if a definition
of struct T is in scope at the point of the member declaration.
Being in scope implies that a definition appears textually earlier in
the file, thus the connection to the containment ordering holds.

\subsection{Checking the sequence numbers}

We do not implement the \ansiC semantics regarding the scope of
structure definitions---though we do (try to) implement the scope
rules of structure \emph{declarations}. This means that we need to
check the connection between sequence numbers and containment ordering
directly in the type checking phase.

Thus, we accept some programs that are not correct \ansiC, but since
we are able to resolve structures and generate correct generating
extensions even for those programs, we do not view this as much of
a problem.

\subsection{The implementation of the sorting phase}

We leave the actual sorting to the \texttt{qsort()} library function.

\subsection{Alternative strategies}

Another strategy that was considered was coding the parser phase such
that the proper topological ordering held from the beginning.
This strategy was rejected because a number of subsequent phases
make changes to the userdecl list, sometimes rather drastic.

There would be a lot of possibilities for inadvertantly introducing bugs
while restructuring code that didnt \emph{seem} to have anything to
with the order of structure declarations. Indeed, even getting the
parser to produce the list correctly would depend very subtly on a
lot of minor details that look rather unimportant in their immediate
context.

\subsection{Implementation level, 1999-03-16}

The above description fits the current implementation of the structure
sorting phase.

\end{docpart}
%%% Local Variables: 
%%% mode: latex
%%% TeX-master: "cmixII"
%%% End: 


\providecommand{\docpart}{\renewenvironment{docpart}{}{}
\end{docpart}
\documentclass[twoside]{cmixdoc}
%\bibliographystyle{apacite}

\makeatletter
\@ifundefined{@title}{\title{\cmix-documentation}}{}
\@ifundefined{@author}{\author{The \cmix{} Team}}%
{\expandafter\def\expandafter\@realauthor\expandafter{\@author}%
\author{The \cmix{} Team\\(\@realauthor)}}
\makeatother

\AtBeginDocument{%
\markboth{\hfill\today\quad\timenow\hfill\llap{\cmix\ documentation}}
{\hfill\today\quad\timenow\hfill}}

\renewcommand{\sectionmark}[1]{\markboth
{\hfill\today\quad\timenow\hfill\llap{\cmix\ documentation}}
{\rlap{\thesection. #1}\hfill\today\quad\timenow\hfill}}

%\newboolean{separate}
%\setboolean{separate}{true}

\renewenvironment{docpart}{\begin{document}}%
                          {\bibliography{cmixII}\printindex
                           \end{document}}
\begin{document}\shortindexingon
}
\title{The Generating Extension and How To Generate It}
\author{Henning Makholm}
\begin{docpart}
\maketitle

\MakeShortVerb{/}
\section{The Gegen phase}
\label{sec:HowToGenerate}
\index{gegen}\index{generating extension}

\def\Pgen{$\hbox{P}_{\textit{\small gen}}$\xspace}

The \emph{generating extension generator}, or \emph{gegen} for short, takes
as input the binding-time annotated Core C program (and other analysis
results) and produces a generating extension on the basis of the
annotations. For brevity, the generating extension will be called
\Pgen.

Gegen is divided into several submodules which share information
through the common header file /gegen.h/. The header file defines
class /GegenEnv/ to which most of the submodules contribute member
functions.

The submodules are, --
\begin{description}
\item[/gg-expr/] handles the translation of \coreC types and
expressions into the representations they have in \Pgen. It defines
a special, iostream-like, class /GegenStream/ which keeps an internal
state for managing the output of p-gen code that uses the speclib to
create residual code snippets.
\item[/gg-cascades/] is a small submodule that defines useful idioms
that other modules use to support statically indexed arrays.
\item[/gg-struct/] performs various jobs related to struct, union,
and enum types. It sees to that they are declared properly in \Pgen
and the residual program. It manages the names of residual structures'
members. It creates functions that helps initialize the \Pgen
counterparts of residual struct and union objects, etc.
\item[/gg-memo/] takes care of all work related to the memoisation
of static values.
\item[/gg-decl/] writes out code to construct proper declarations
for residual variables. It also manages the declaration of the subject
program's global variables in \Pgen.
\item[/gg-code/] writes code for \coreC statements and functions. It
takes care of function inlining, and -- with help from /gg-memo/ for
the dirty parts -- writes out pending-list loop code and function
sharing code where necessary.
\item[/gegen/] contains code to tie all of the above together, and
code for some small helper functions that weren't naturally placed
otherwhere.
\end{description}

The detailed description of gegen that follows is organised by these
submodules. Each section begin by describing the external interface
of the submodule as seen by the other submodules, \emph{e.g.}, the
actual C code emitted is largely considered an ``implementation
detail'' in this description. Then what the emitted C code looks
like is described. At the end of each section there are tips for
understanding the source code for the submodule.

Previous to the split into submodules, the documentation
tried to follow the order in which things appear in the \Pgen source,
which was not very easy to follow after all. Hopefully this logical
organisation is easier to read.

\subsection{The type mapping}
\label{sec:gegen:TheTypeMapping}

Before beginning to describe the individual submodules it is important
to describe the fundamental relationship between the binding-time
annotated \coreC types of things, and the types the same things will
have in \Pgen.

\begin{figure}[htb]\begin{center}\begin{tabular}{|c|c|c|c|}
\hline
Binding-time & Variable type & Expression type & Type $[\![T]\!]$ in the \\
annotated type $T$) & $[T]$ in \Pgen & $[T]^*$ in \Pgen & residual program \\
\hline
\hline
/int/$_s$ & /int/ & /int/ &  \\ \hline
/int/$_d$ & /Code/ & /Code/ & /int/ \\ \hline \hline
pointer$_s$ to $T'$ & pointer to $[T']$ & pointer to $[T']$ & \\ \hline
pointer$_d$ to $T'$ & /Code/ & /Code/ & pointer to $[\![T']\!]$ \\ \hline\hline
array$_s[c]$ of $T'$ & array$[c]$ of $[T']$ &
    & $c$ copies of $[\![T']\!]$ \\ \hline
array$_d[c]$ of $T'$ & /Code/ &
    & array$[c]$ of $[\![T']\!]$ \\ \hline\hline
struct S$_s$ & /struct S/ & /struct S/ & \\ \hline
struct S$_d$ & shadow struct$^\dag$ & /Code/ & /struct S/ \\ \hline\hline
union S$_s$ & /union S/ & /union S/ &  \\ \hline
union S$_d$ & shadow struct$^\dag$ & /Code/ & /union S/ \\ \hline\hline
function$_s$ from & function from & &  \\
$(T'\ldots)$ to $T''$ & $([T']^*\ldots)$ to $[T'']^*$ & & \\ \hline
function$_d$ from & & & function from \\
$(T'\ldots)$ to $T''$ & & & $([\![T']\!]\ldots)$ to $[\![T'']\!]$ \\ \hline
\end{tabular}\end{center}
\caption{The type mapping}
\label{fig:typeRelationship}
\end{figure}

This is depicted in figure \ref{fig:typeRelationship}, and what this 
figure says about /int/ also applies to other ``simple'' type, that
is, basic types, abstract types, and enum types.

The reader will notice that the figure has two columns for the
residual version of a type. The $[T]$ column is the standard mapping;
it is the one used for \emph{variables} declared in the
\coreC program. In this mapping, when a type maps to /Code/ this
/Code/ object will at specialization time contain the name of the
corresponding residual variable.

The complex point about the variable type mapping is the way struct
and union types are handled. Each residual struct or union type maps
to a \Pgen struct (even if it was originally a union) whose members'
types are the ``variable'' versions of the subject structure's member
types. In addition to this, the shadow struct also contains a member
called /cmix/ of type /Code/. At specialization time this member
contains the name of the entire residual struct or union, while the
/Code/s inside the other members will be residual expressions denoting
individual members of the struct or union. The idea behind this is
that static pointers will be able to point to the members structure
even though the the structure itself is ``dynamic''.

The other type mapping for \Pgen is used for the value of \coreC
\emph{expressions}. Expressions cannot have array or function type,
which is why $[T]^*$ is not defined for these types. For static values
the two mappings are the same, so the only place the expression type
mapping is important is in the case of dynamic parameters and return
values of specializable functions.

The tricky thing about dynamic parameters to specializable functions
is that there \emph{is} something present at their place in the
corresponding \Pgen function's parameter list, but this ``something''
does \emph{not} represent a residual parameter object. Rather it
represents the actual expression used as argument in the function
call. This is because the \Pgen function decides by itself whether
to unfold the function call during specialization or create a residual
function to be called. In either case a real residual variable will
be created (and it will be represented by a standard local variable in
the \Pgen function) but the way this is done in practise differs.
So it is more convenient to simply hand the actual argument expression
to the \Pgen function and let that do the right thing with it.

The practical consequence of this is that dynamic function parameters
should always be /Code/ and never shadow structs---even though the
parameter is a dynamic struct S, the caller may have to create a
complex residual expression for finding the real struct S residually,
and no shadow struct may be at hand.

Similar reasoning lead to the decision that shadow structs should
not be used as function return types.

\smallskip

Turning to the rightmost column of figure \ref{fig:typeRelationship},
the mapping to types in the residual program should be
straightforward. 

Empty space in the $[\![T]\!]$ column means that values and variables
of the corresponding type are specialized completely away from the
residual program.

Note, though, that statically indexed arrays \emph{may} cause
something to exist in the residual program. For example, a variable
of type ``array$_s$[5] of /int/$_d$'' will produce 5 separate /int/
variables in the residual program.

If, however, the type is, \eg, ``array$_s[42]$ of /short/$_s$'', then the
variable maps to 42 times nothing in the residual program, \ie, it
still disappears. This is characteristic of our handling of arrays
with static subscripting: many aspects of types are transitive
through ``array$_s$'' components.

\subsection{gg-expr: printing expressions and types}

Before we can describe the interface between /gg-expr/ and the
rest of Gegen we need to summarize how \Pgen is expected to
build residual expressions.

\subsubsection{The code generation model in speclib}

The statements and expressions in the residual program as it is
built by the generating functions consist of trees (or rather, DAGs)
with the following types of nodes:
\begin{description}
\item[name] nodes are leaves and are replaced with a ``suitably
    uniqe'' indentifier. They are filled into the \Pgen variables
    that represent residual variables as these are initialized.
\item[lift] leaves of various kind represent numerical and character
    constants that originate in lifts.
\item[inner] nodes can have children. They are labeled with a
    string in which question marks define ``holes'' where the
    node's children are insert when the residual program is
    eventually written out. The string can be arbitrary and is
    often complex, such as /"?*(?-42)/ /+/ /sizeof(int/ /(*)[?])"/.
\end{description}

Apart from the primitives that create ``name'' and ``lift'' nodes,
the speclib functions that generate code---such as /cmixStmt/ which
emits a simple statement to the current function, or /cmixIf/ which
produces a residual conditional---all come bundled with the
creation of an ``inner'' node. The parameter list for these functions
end in ``/,char const*,.../'', where the last explicit parameter
is the label string and is followed by as many /Code/ arguments
as there are question marks in the string.

This design minimizes the extent to which \Pgen has to manipulate
the expression and declaration syntax of C. However, it also places
special requirements in gegen, as expressions cannot simply
be printed with a normal inorder traversal. For example, the
expression /1==x*y+z/, where /x/ and /y/ are static and /z/ dynamic,
must end up something like
$$/cmixMkExp("1==?+?",cmixLiftSigned(x*y),z)/.$$
One sees that the expression subsection of gegen needs to be able
to postpone parts of expressions until after the skeleton string
for the entire code bit has been constructed.

\subsubsection{External interface to gg-expr}

The /GegenStream/ class is /gg-expr/s main interface to the rest of
the world. A /GegenStream/ behaves in some contexts simlarly to
a standard C++ /ostream/: one can print strings, characters, and
numbers to it with the /<</ operator. However, there are also other
expressions one can print:
\begin{description}
\item[/enter/] ``Printing'' this word (the mechanism is similar to
  the standard library's /endl/ construct) outputs a quote and
  switches an internal flag from ``\Pgen mode'' to ``residual-code
  mode''. In residual-code mode the /GegenStream/ maintains a
  list of postponed subexpressions which have yet only been rendered
  as holes in the skeleton string.
\item[/EmitExpr(/\textit{\mdseries e}/,/\textit{\mdseries{precedence}}/)/]
  Prints the \coreC expression \textit{e} according to the current
  mode: if in \Pgen mode the entire expression is printed; in
  residual mode parts of the expression may be postponed and replaced
  by lifts. The \textit{precedence} parameter controls whether the
  expression will be enclosed in parantheses; it can be omitted when
  the client knows that parentheses are never necessary.
\item[/EmitDeref(/\textit{\mdseries e}/,/\textit{\mdseries{precedence}}/)/]
  This is equivalent to /EmitExpr/, but implicitly applies the unary
  /*/ operator to the expression. This way the usual simplifications
  of the \coreC rendering algorithm may succeed in merging this
  /*/ with the topmost operator of the expression, so the generated
  code makes full use of C's lvalue expression syntax.
\item[\textit{\mdseries i}]
  where \textit{i} is a \coreC initializer.
\item[/pushtype(/\textit{\mdseries t}/,/\textit{\mdseries q}/)/]
  Prints the part that comes before the identifier in a declaration
  with type \textit{t} and qualifiers \textit{q}. The type is
  interpreted according to the $[\cdot]$ or $[\![\cdot]\!]$ type
  mapping, depending on the mode of the /GegenStream/.
\item[/poptype/]
  Prints the second half of declaration for a waiting type that has
  been printed with /pushtype/. As the names imply, a stack discipline
  is maintained for waiting types.
\item[/EmitType(/\textit{\mdseries t}/,/\textit{\mdseries{string}}/,/%
	\textit{\mdseries q}/)/]
  Equivalent to printing /pushtype(/\textit{t}/,/\textit{q}/)/, then
  printing \textit{string}, then /poptype/.
\item[/AbstractDecl(/\textit{\mdseries t}/)/]
  Short form for /EmitType(/\textit{t}/,"",constvol())/.
\item[/Unqualified(/\textit{\mdseries t}/,/\textit{\mdseries{string}}/)/]
  Short form for /EmitType(/\textit{t}/,/\textit{string}/,constvol())/.
\item[\textit{\mdseries d}]
  where \textit{d} is a \coreC declaration node. Prints a declarator
  for the variable. Equivalent to /pushtype/ + name + /poptype/.
  ``name of variable'' will be the \Pgen name or a name reques
  according to the mode of the /GegenStream/.
\item[/alwayscode/]
  Instructs the /GegenStream/ that the next /pushtype/ should interpret
  the type according to the rules for expressions (i.e., function
  parameters and return values) rather than the normal rules for
  objects.
\item[/exit/]
  Ends ``residual-code mode'', printing a quote followed by the
  postponed expression parts that have been collected since the
  /enter/.
\item[/userhole/]
  This can be used in residual mode to print a ``hole'' character
  to the current skeleton string and leave a special marker in
  the list of postponed expressions. When /exit/ meets this special
  marker it returns to its caller which can then output whatever
  \Pgen it wants to fill this hole with. Then the rest of the
  postponed expressions are printed with a \emph{second} /exit/.
  There can be only one /userhole/ for each /enter/.
\end{description}

/GegenStream/ has one member function, /EmitLiftingHole/ which takes
a \coreC type parameter. It prints an appropriately decorated
/userhole/ and returns to its caller a string containing the name
of a speclib lifting function that works together with the said
decorations to lift values of the given type. The return string
contains the opening parenthesis of the lifting call. The intended
use is in a sequence like
\begin{verbatim}
  gegenstream << enter ;
  // ...
  char const *lf = gegenstream.EmitLiftingHole(type) ;
  // ...
  gegenstream << exit << ',' << lf << ... << ')' << exit ;
\end{verbatim}

\subsubsection{Output aspects}

\subsubsection{Implementation notes}

\subsection{gg-cascades: iterating over statically indexed arrays}

The submodule defines the /ForCascade/ class which other submodules
use for iterating through all elements of statically indexed arrays.
Constructing a single /ForCascade/ is used for iterating through
an arbitrary number of array levels, starting at zero---thus the
common case that an object is not a staticlly indexed array is
handled transparently.

The constructor for a /ForCascade/ takes three arguments:
\begin{itemize}
\item a \coreC type. The loop will iterate through as many levels of
      statically indexed arrays as it finds at the top of this type.
\item a reference to a /GegenEnv/
\item an integer specifying the number of spaces the loop must
      be indented in the output.
\end{itemize}
As the /ForCascade/ is constructed it outputs zero or more nested
loop statements to the output stream, and also outputs appropritate
indentation for the body statement of the loop.

Once the /ForCascade/ is created, the client can do these things
with it:
\begin{itemize}
\item inspect its /t2/ member which is the type after all the
      levels of statically indexed arrays has been stripped of.
      That is, the type of the actual array elements.
\item inspect of change its /cv/ member which is a /constvol/
      structure for holding type qualifiers. This member is
      initialized with the conjunction of the type qualifiers
      in the array types which was stripped off to produce /t2/.
\item output the /ForCascade/ to a /GegenStream/ using an
      overloaded /<</ operator. This causes a series of subscripts
      to be written to the /GegenStream/.
\item call the /addline()/ member function. This causes a comma or
      semicolon to be output to the output stream, followed by
      a newline and appropriate indentation for a second action in the
      loop body.
\end{itemize}

\subsubsection{Output aspects}

\subsubsection{Implementation notes}

\subsection{gg-struct: struct and union related issues}

This submodule defines one global function and several member
functions for /GegenEnv/.

The global function is /dotcmix(/\textit{type}/)/. It assumes the
type is dynamic and returns the string /".cmix"/ if the type is
a struct or union type and the empty string otherwise. This return
value can be used as a postfix operator to convert the ``object''
representation of a dynamic type and the ``expression'' representation.

The /GegenEnv/ member functions are:
\begin{description}
\item[/pgen_usertype_fwds()/]
	This function outputs forward declarations for the static
	versions of usertypes to \Pgen.
\item[/pgen_usertype_decls()/]
	This function outputs actual definitions for the static
	versiond of usertypes to \Pgen.
\item[/define_cmixMember()/]
	This function outputs a definition of an object called
	/cmixMember/ which is to hold the residual name nodes for
	members of residual usertypes. After this object has
	been properly initialized the expression /cmixMember.S.t[i]/
	evaluates to a /Code/ snippet containing the residual name
	of the /i/th of the residual members corresponding to the
	member /t/ of struct or union /S/. (In this example /t/'s
	type is supposed to be a one-level statically indexed array).
\item[/define_cmixPutNameX()/]
	This functions output definitions of functions that are
	used to intialize \Pgen object whose types map to
	``shadow structs''. The functions are called /cmixPutName/
	$X$ where $X$ is a positive integer.

	Every /cmixPutName/$X$ takes two arguments. The first is
	a reference to the \Pgen object to be initialized; the second
	is a /Code/ that should contain the residual name that has
	been created for the object.

	The speclib defines a /cmixPutName0/ which is used for types
	that map to plain /Code/ in \Pgen:
\begin{verbatim}
void cmixPutName0(Code &obj,Code name) {
  obj = name ;
}
\end{verbatim}
\item[/putnameseq(/\textit{\mdseries t}/)/]
	Returns the number $X$ in the /cmixPutName$X$/ that is
	appropriate for the given type.
\item[/init_struct()/]
	Outputs whatever usertype-related code should go in the
	/cmixGennit/ function. This code must initialize the
	/cmixMember/ structure and may pass on usertype definitions
	to the speclib.
\item[/exit_struct()/]
	Outputs whatever usertype-related code should go in the
	/cmixGenExit/ function. Apart from interacting with the
	speclib, this code may write text to the residual program
	using the /FILE/ pointer /fp/. This text will come before
	the residual program text that has been collected during
	specialization.
\end{description}

\subsubsection{Output aspects}

\subsubsection{Implementation notes}

\subsection{gg-memo: generating memoisation code}

This submodule defined the class /GegenMemo/ which is the type of
the /memo/ member of /GegenEnv/. /GegenMemo/ contains data that
are private to the task of generating memoization code.

Its publicly accessible member functions are
\begin{description}
\item[/follow_functions()/]
	Outputs to \Pgen any auxiliary functions that is needed for
	the memoisation code to work in addition to speclib. Currently
	this is functions that help the memoisation engine locate
	pointer members in usertype objects.

	Also this function makes global decisions about memoisation
	stategies, so it should be called before other member
	functions in /GegenMemo/.
\item[/emit_globalmemodata()/]
	Outputs to \Pgen any data declarations necessary for memoising
	global variables.
\item[/emit_globalmemocode()/]
	Outputs any code that must go into the body of /cmixGenInit/
	in order to initialize the global variable memoisation.
\item[/emit_localmemodata(/\textit{\mdseries{fun}}/,/%
	\textit{ec}/,/\textit{pp}/)/]
	Outputs those declarations that are necessary for supporting
	memoisation of local variables and parameters in the
	generating function corresponding to \coreC function
	\textit{fun}.

	\textit{\bfseries{ec}} and \textit{\bfseries{pp}} are booleans
	which specify whether it may be necessary to memoise,
	repectively, the function entry conditions, and program points
	within the function.

	/emit_localmemodata/ returns a number which the generator
	function must pass to the speclib function /cmixPopObjects/
	before returning.
\item[/emit_pendinsert(/\textit{\mdseries{fun}}/,/%
	\textit{\mdseries{bb}}/)/]
	Outputs a \Pgen expression which inserts the given program
	point in the pending list and evaluates to a reference to
	the label resulting from this.
\item[/emit_funmemo(/\textit{\mdseries{fun}}/)/]
	Outputs the memoization-related arguments in calls to
	the speclib function /cmixFunctionSeenB4/ or /cmixPushFun/
\end{description}

\subsubsection{Output aspects}

\subsubsection{Implementation notes}

\subsection{gg-decl: constructing residual declarations}

\subsubsection{Output aspects}

\subsubsection{Implementation notes}

\subsection{gg-code: constructing generating functions}

\subsubsection{Output aspects}

\subsubsection{Implementation notes}

\subsection{gegen itself: tying all of it together}

\subsubsection{Implementation notes}

\textbf{\Large NB! the rest of this chapter has not yet been updated to match
the submodule split of gegen, or even the latest changes to the
speclib interface definition. Trust the material below at your own
risk. Documentation updates will follow shortly (I hope).}

\subsection{Chapter I: Header inclusions}
\label{sec:gegen:HeaderInclusions}
This first chapter includes the header files needed for \Pgen.

First, the files specified in /header:/ specializer directives
appear. Each of the ``shadow header'' files we use for standard
headers include an appropriate /header:/ directive. Thus, when the
user specifies, \eg, ``/#include <math.h>/'' \Pgen will also contain
``/#include <math.h>/''.

After these user-specified headers, we include the header that defines
the specialization support library and check that its version is the
same as the one Gegen was written to work with:
\begin{verbatim}
#include <cmix/speclib.h>
static cmixNameManager cmixNameMan;
#if cmixSPECLIB_VERSION != 2002
#error Wrong speclib version
#endif
\end{verbatim}
The default file name for the speclib header is
\verb|<cmix/speclib.h>|; if the environment variable CMIX_SPECLIB_H is
set, its value is interpolated instead.

Why this chapter is the right place to define the name manager used
for constructing residual identifiers seems to have been lost in the
mists of history. The fact is that it happens here...

\subsection{Chapter II: Forward declarations for usertypes}
\label{sec:gegen:ForwardDeclarationsForUsertypes}
The usertypes used in the program can be mutually recursive through
pointers. This means that before we begin to define their contents we
need to declare their \emph{names} so that we can define members of
type ``pointer to struct S'' before the definition of struct S itself.

That is what happens in this phase. The only slight complications
are
\begin{itemize}
\item per the type mapping, an immortal union type maps to a
      \emph{struct} in \Pgen, not a union.
\item enum types cannot be forward declared; so we have a choice of
      defining them here or among the real definitions in the next
      chapter. We choose to define them early.
\end{itemize}

\subsection{Chapter III: Definitions for usertypes}
\label{sec:gegen:DefinitionsForUsertypes}
Here are the real definitions for the usertypes. This is completely
straightforward; the members of each usertype is declared according to
the type mapping.

The immortal usertypes always map to structs here, even if they are
really unions. Consider a source declaration
\begin{verbatim}
union U { int a, int b } u ;
\end{verbatim}
This will map to these definitions in \Pgen:
\begin{verbatim}
struct U : Code {
  Code a;
  Code b;
};
/* ... */
struct U u;
\end{verbatim}
At specialzation time, a residual name will be assigned to /u/---say,
/u42/---and the fields /u.a/ and /u.b/ will contain the residual
expressions ``/u42.a/'' and ``/u42.b/''.

\subsection{Chapter IV: Names for dynamic structutes}
\label{sec:gegen:NamesForDynamicStructures}
Recall (section \ref{sec:gegen:TheTypeMapping}) that the \emph{tags}
of dynamic structs and unions are fixed at Gegen time. We cannot do
the same thing for the members, because the number of members in each
structure type need not be known at Gegen time. Consider
\begin{verbatim}
#include <limits.h>
struct T { int a[UCHAR_MAX], b; };
\end{verbatim}
where the binding times of T.a resolve to a statically indexed array
of dynamic values. The residual program should contain a structure
like
\begin{verbatim}
struct T {
  int a, a0, a1, a2, a3, a4, a5, a6, ... a254, b;
}
\end{verbatim}

Clearly, we can only generate names for these fields when we know
how many of them we need. The shadow header for /<limits.h>/ declares
/UCHAR_MAX/ to be a `well-known constant' of type /unsigned/
/char/. Gegen cannot know its numeric value, but of course it will
be available when \Pgen{} is compiled. This means that \Pgen{} will
have to keep track of the member names itself.

This chapter declares variables in \Pgen{} to hold these member names.
If struct T from the above example was the only immortal struct in the
program, the chapter would look like
\begin{verbatim}
static struct {
  struct {
    const char *a[UCHAR_MAX];
    const char *b;
  } T;
} cmixMember;
\end{verbatim}

The /cmixMember/ struct is there purely to conserve namespace. It
contains the name data for all the immortal structs. (If there are
none, the entire chapter is omitted in \Pgen).

The types for the second-level members of /cmixMember/ are formed
by a process analogous to the \Pgen{} type mapping, replacing /Code/
with /char/ /const/ /*/.

The initialization function chapter (section
\ref{sec:gegen:InitializationFunction}) contains code to generate
names for the residual fields and store them into /cmixMember/.

\subsection{Pseudochapter: For cascades}
\label{sec:gegen:ForCascades}
Throughout \Pgen{} the need often arises to do something to a \Pgen{}
object unless it is an array in which case the something should be
done to each of the array's elements.

One main case of this is the handling of the /Code/ objects that make
up the parts of objects with immortal types. Recall (the type mapping,
section \ref{sec:gegen:TheTypeMapping}) that an immortal type is
mapped in \Pgen{} to the /Code/ type inside zero or more levels of
statically indexed arrays.

This pseudochapter defines an idiom for iterating through levels of
arrays. Say we have an object /x/ whose \Pgen{} type is array[42] of
array[117] of /short/ and we want to add 5 to each of the /short/s.
Then we could emit
\begin{verbatim}
  { unsigned cmixI0 ;
    for(cmixI0=0; cmixI0<42; cmixI0++ ) {
      unsigned cmixI1 ;
      for(cmixI1=0; cmixI1<117; cmixI1++ ) {
        x[cmixI0][cmixI1] = x[cmixI0][cmixI1] + 5 ;
  } } }
\end{verbatim}

This pattern will be known as a \emph{for cascade}. If the type of /x/
had just been /short/ the same for cascade would degenerate into
simply
\begin{verbatim}
  x = x + 5;
\end{verbatim}

In the descriptions that follow, the possible presence of a for
cascade will usually just be hinted at, and both the above examples
would be written as \texttt{x = x + 5}.

\subsubsection{Implemenation notes}
\begin{itemize}
\item The /for(...)/ lines of a for cascade is emitted by the function
      /EmitForCascade()/. This function is given a binding-time
      annotated type $T$ and itself calculates how many levels of
      arrays $[T]$ contains. A global variable records this number so
      that subsequent calls to /EmitIndexCascade()/ write out the
      right number of indexes.
\item /EmitForCascade()/ takes care of the indentation of the
      /for(...)/ lines \emph{and} the body line. The indentation
      will be wrong if the body statement spans multiple lines.
\item For convenience, /EmitForCascade()/ returns the binding-time
      annotated type stripped of the array levels. (Any dynamically
      indexed array levels are still present).
\item For cascades cannot be nested.
\item The same index variables /cmixI0/, /cmixI1/, /cmixI2/,$\ldots$
      are used in each for cascade.
\end{itemize}

\subsection{Chapter V: Initialization functions for dynamic usertypes}
\label{sec:gegen:InitializationFunctionsForDynamicUsertypes}
This chapter (which partially serves as a pseudochapter too) concerns
itself with name management for the residual program and the
initialization of those /Code/ objects in \Pgen{} that represent
objects in the residual program.

The principal role of the /Code/ type is to represent residual
\emph{expressions}. However each of the objects that will exist in the
residual program is represented as a /Code/ object in \Pgen{}. When
viewed as an expressions, these representative /Code/ objects are
lvalue expressions that refer to the residual objects.

The strategy for the management of residual variable names is that
when a name is generated it is put into the appropritate
representative /Code/ object. It finds its way into actual residual
expressions when the representative /Code/ object is used for
constructing them.

Finding ``the appropriate representative /Code/s'' and installing the
freshly generated name
there can be more or less complicated, according to the type of the
variable:
\begin{description}
\item[/short/$_d$:] This is easy; there is a single /Code/ to worry
about. We can simply construct an expression referring to the residual
variable (/Code/ has a built-in conversion from /char*/) and assign
it into the representative /Code/.
\item[pointer$_d$ to $T$:] Same as the /short/$_d$ case.
\item[array$_d$ of $T$:] Same as the /short/$_d$ case.
\item[array$_s$ of something:] This case never occurs; we generate
different residual name for each of the array elements and install
them separately.
\item[struct$_s$ S:] Here is where the trouble is. \Pgen's version of
struct S is derived from /Code/ and need to be initialized; but in
addition to this the members of the structure are also representative
/Code/ objects and need to be initialized with member selection
expressions. This task is left over to the special initialization
functions that gives this chapter its name.
\end{description}

\subsubsection{Initialization functions}
For each immortal usertype a custom initialization function is
generated. Say we have an fully dynamic struct originally defined as
\begin{verbatim}
struct S {
  int a ;
  struct T *b ;
  struct T c ;
}
\end{verbatim}
The initialization function that gets generated for this is
\begin{verbatim}
static void cmix_putName(S &cmixThis,const Code &cmixIt) {
  (Code&)cmixThis = cmixIt ;
  cmixThis.a=cmixStruct(cmixIt,cmixMember.S.a);
  cmixThis.b=cmixStruct(cmixIt,cmixMember.S.b);
  cmix_putName(cmixThis.c,cmixStruct(cmixIt,cmixMember.S.c));
}
static inline void cmix_putName(S &cmixThis,const char *cmixIt)
  { cmix_putName(cmixThis,Code(cmixIt)); }
\end{verbatim}

All initialization functions are overloaded onto the single
name /cmix_putName/. /cmixStruct/ is an expression constructor
defined in /speclib.h/; it applies a ``dot'' operator to an
expression. /cmixMember/ is discussed in section
\ref{sec:gegen:NamesForDynamicStructures}. Note the different way
of (pseudo)recursively initializing the /c/ field; this is because it
is not plainly a /Code/ but a \Pgen{} version of another immortal
struct.

For members that are statically indexed arrays for cascades are insert.

\subsubsection{Naming a dynamic object}
This chapter also serves as pseudochapter: its implementation contains
a function that emits \Pgen{} code to generate a name and install it
properly. This way all decisions related to the initialization of
representative /Code/ objects are collected in a single chapter.

Initializing a representative object consists of constructing a name
based on the variable's original name (if the variable is generated by
\cmix, a pseudo-original name is invented) and installing it according
to the object's type. A typical piece of code is:
\begin{verbatim}
  cmix_putName(x,cmixNameman.freshLocalName("x"));
\end{verbatim}
with the obvious variations if /x/ is not a struct. Well, nearly
obvious. Instead of
\begin{verbatim}
  x = cmixNameman.freshLocalName("x");
\end{verbatim}
the current Gegen emits
\begin{verbatim}
  x.putVarName(cmixNameman.freshLocalName("x"));
\end{verbatim}
where /putVarName/ is a member function of /Code/ that essentially just
converts its argument to /Code/ and assigns it to /*this/.

If /x/ were a statically indexed array the initialization code would
be wrapped in a for cascade.

Depending on the name management strategy required for the variable
type, /cmixNameman.freshLocalName/ can be
/cmixNameman.freshGlobalName/ or can be totally absent for variables that
\emph{must} keep their original names in the residual program.

\subsubsection{Implementation notes}
\begin{itemize}
\item The generation of /cmix_putName()/ functions should be
straightforward.
\item The function /NameDynamicObject()/ emits the initialization code
described above, as well as any needed for cascade. Its second
parameter is the function to call to generate a new name based on
the original name.
\item The generation of pseudo-original names take place in
/C_Decl::get_name()/ which is defined in /coreC.cc/.
\item As a convenience feature, and to share ``for cascades'',
/NameDynamicObject()/ can also emit code that emits the definition
of the residual object. When this is wanted the third parameter
to /NameDynamicObject()/ is the name of the function in speclib that
should be called to emit the definition. Its calling interface should
be similar to /cmixGlobal/. When the third parameter is /NULL/ the
caller takes the responsibility of defining the variable in the
residual program.
\end{itemize}

\subsection{Chapter VI: Memoization helper functions}
This chapter consists of functions that are called by the
specialization library when it needs to follow the
pointers to mortal data in a spectime object.

\subsubsection{Speclib-Pgen interface conventions}
Each object
has an associated \emph{pointer following function} declared
like this:
\begin{verbatim}
bool cmix_follow117(cmixPointerCallback &,void *,unsigned);
\end{verbatim}
The /void*/
parameter to the pointer following function points to the object
itself, and the /unsigned/ parameter is the number of elements in
the object if the object is an array.

The /cmixPointerCallback/ class is defined in /speclib.h/ and
has an overloaded function call operator with a /void*/ argument,
returning /bool/. The
pointer following function is supposed to call that operator once
for every pointer to mortal data in the object. If the
/cmixPointerCallback/ ever returns /false/ the pointer following
function must return /false/ immediately---otherwise it must
return /true/.

\subsubsection{Anatomy of pointer following functions}
If there are any struct types in the program containing pointers to
mortal data, one overloaded helper function
\begin{verbatim}
static inline bool cmix_followH
  (cmixPointerCallback &cmixCallback,T *cmixSrc)
\end{verbatim}
is defined for each of these. Its job is to follow pointers in a
single object of that struct or union type.

For each member /x/ of the struct, the helper function contains:
\begin{verbatim}
  if (!cmix_followH(cmixCallback,&cmixSrc->x)) return false;
\end{verbatim}
\noindent if /x/ is itself a struct containing pointers to mortal
data, or
\begin{verbatim}
  if (!cmix_followH(cmixCallback,(T*)&cmixSrc->x)) return false;
\end{verbatim}
\noindent if /x/ is a union containing pointers to motal data. The
BTA should only let this happen if all members of the union are
structs, and if all pointers to mortal data are duplicated in every
member of the struct. /T/ is then the type of the first member of
the union, or
\begin{verbatim}
  if (!cmixCallback(cmixSrc->x)) return false;
\end{verbatim}
\noindent if /x/ is a pointer to mortal data, or \\
\emph{nothing} if none of these cases hold. Each of these
constructs is called a \emph{memoizer statement}.

After the /cmix_followH/ functions the primary pointer following
functions follow. They all look like
\begin{verbatim}
static bool cmix_follow117
    (cmixPointerCallback &cmixCallback, void *cmixVoid, unsigned cmixI) {
  T* const cmixSrc = (T*)cmixVoid ;
  while ( cmixI-- )
    .. memoizer statement for cmixSrc[cmixI] .. ;
  return true;
}
\end{verbatim}

Several objects can share the same pointer following function if they
have the same (binding-time annotated) type. The connection between
pointer following functions and data objects is made in later
chapters.

The specialization library predeclares the function /cmix_follow1()/
which is used for objects that do not contain any pointers to mortal
data.

\subsubsection{Implementation notes}
\begin{itemize}
\item The function /AnyPointersToFollow()/ investigates if a given
	declaration contains pointers that should be followed. If
	it does, it also sets up a couple of global variables that
	makes the next call to /EmitMemoizer()/ emit a memoizer
	statement for the type specified by the declaration.
\item /AnyPointersToFollow()/ relies on the /SimpleMemoization/
	annotation on struct declarations. This annotation is
	set up while generating the /cmix_followH/ functions;
	a process which itself uses the /AnyPointersToFollow()/
	function. It is essential for correct results that the
	struct declarations are processed in ``topological order''
	(cv. section \ref{sec:StructureDefinitionSorting}).
\item /AlocMemodata()/ is called for each object in the program
	and does several things:
	\begin{itemize}
	\item decides if the object is mortal or immortal
	\item assigns a sequential ID number to each mortal object.
		Immortal objects get the pseudo-ID 0.
	\item for each mortal object, decides if a previously
		generated pointer following function can be used for
		it. Generates one if not.
	\item annotates the object with the number of the pointer
		following function that fits it.
	\end{itemize}
	The second argument to /AlocMemodata/ is a list that the
	function builds of the objects that have had specialized
	pointer following functions generated. It is used to share
	pointer following functions between objects.
\item /AlocMemodata(NULL,NULL,NULL)/ initializes the object
	sequence counter.
\item /AlocMemodata(NULL,NULL,streamp)/ emits a definition of
	/cmixMWCount/ which is an array of /unsigned/ with as many
	elements as there are object IDs. The specialization
	library uses this for some arcane purpose.
\end{itemize}


\subsection{Chapter VII: Global variables}
This chapter contains declaration for the program's global variables.

Most variables are simply declared as
\begin{verbatim}
static T x;
\end{verbatim}
according to the type mapping,
but external variables that have been annotated /dangerous spectime:/ and
variables with /visible spectime:/ annotations need to be able to link
together with external C code that is linked into \Pgen. They are
defined as
\begin{verbatim}
extern "C" T x;    // x is dangerous spectime
\end{verbatim}
or
are declared as
\begin{verbatim}
extern "C" {
  T x;             // x is visible spectime
}
\end{verbatim}

Mortal variables\footnote{with the exception of \texttt{dangerous
spectime} variables} must be registered with the specialization
library's memoization engine. Immediately after each definition
a /cmixDataObject/ is constructed to take care of that. This looks
like
\begin{verbatim}
static cmixDataObject cmixDO1235(1234,&x,sizeof x,cmix_follow29, 1, false);
\end{verbatim}
/1233/ is the object's ID number in the specialization library (which is
one less than the ID number used internally in gegen; that number is
0 for objects that should not be memoized, but the specialization
library wants the numbers to start at 0).

The number /1/ is the number of array elements that should be given to
the pointer following function. It is always 1 even if the object is
an array, this number is different only for heap allocations where the
number of array elements is determined at spectime.

The last argument /false/ to the constructor signals that the control
object should not register itself as a local variable with the
specialization library.


After all global variables from the subject have been declared, a
constant array /cmix_globals/ containing pointers to all the
/cmixDataObject/s is defined. This array is used for referring to
the set og global variables in calls to the specialization library.

\subsubsection{Implementation notes}
\begin{itemize}
\item The /DeclareDO()/ function that emits the declaration of
	/cmixDataObjects/ is also used by the generator functions
	to declare local /cmixDataObject/s.
\end{itemize}

\subsection{Pseudochapter: Generator function conventions}
\label{sec:gegen:GeneratorFunctionConventions}
This pseudochapter specifies the calling conventions among generating
functions.

A generating function corresponds to a function definition in the
subject program. It is called when a call to that function needs to
be specialized. The generating function is given the mortal arguments
to the call. It can then either decide to re-use a previously
generated specialized version, in which case it reproduces the
memoized static side effect, or to specialize the function anew.

The generating function's return value is a /Code/ which represents
the \emph{name} of the residual function that shold be used when
specializing the call. It is the caller's job to then construct
a function call expression.

The parameters to a generating function are those of the original
function's parameters whose types are mortal. If the original
function's return type is mortal and different from /void/ an
additional parameter is appended \emph{in front} of the others.
The additional parameter has type ``pointer to the return type'', and
locates the place where the caller wants the return value deposited.

Generating functions are always /static/ in the \Pgen{} source.

\subsubsection{Implementation notes:}
\begin{itemize}
\item The pseudochapter's primary contribution is an /EmitType/
	deriviate called /GeneratorProto/ whose job is to generate
	prototypes for generating functions. These are used in
	chapters VIII and IX and need to be identical in those
	two chapters. /GeneratorProto/ does not really use its
	/EmitType/ ancestry for anything serious but might
	change to do it when the calling convention changes to
	accomodate inlining at some time in the future.
\item Located in this pseudochapter is also the function
	/EmitThingsForCall/ which produces the \Pgen equivalent
	of a Core C call statement. The intention of having the
	function defined here was that is must be changed when
	the generator function interface changes, so is would
	be smart to have the two pieces of code close to each
	other. Yeah, right.
\end{itemize}

\subsection{Chapter VIII: Function prototypes}
This chapter contains prototypes for every external function declared
in the subject program. We may not need to call all of them at
spectime, but it does not hurt to mention all of them. And we have
no analysis to show us which ones we do need.

External functions whose name is mentioned in a /well-known:/
directive are explicitly excluded from the prototype list. This
feature is used by shadow header files where we want the appropriate
real header to be included in Chapter I instead of having our own
prototypes written out. If the real header file defines the function
as a macro it could be disasterous to try to declare it explicitly.

Also, since the generating functions may be mutually recursive, we
also declare prototypes for all of them.

\subsection{Chapter XI: Generator functions}
Here at last we reach the goal of all the previous preparations:
the functions that actually specialize code.

There is one such function for each function in the subject program;
their calling conventions have been discussed in an earlier section
(\ref{sec:gegen:GeneratorFunctionConventions}).

\subsubsection{Sharing strategy}
Before going into the details of the generating function we shall
discuss the strategy for sharing residual functions between multiple
residual calls. There are three possible strategies:
\begin{itemize}
\item \textbf{Non-shared:} the function is only used once at the
	call that prompted its initial generation. Even if a
	perfect match for the arguments and static state later
	arises, the function is not reused.
\item \textbf{Semi-shared:} The function
	can be shared, but only for calls that are specialized
	after it has been fully specialized---\ie, recursive
	and indirecly recursive calls cannot be shared.
\item \textbf{Fully shared:} Even recursive calls can be shared.
	For this to be possible, the function must have no
	static side effects---when a call is shared any side
	effects must be reproduced, and naturally that can't
	happen until we actually know that all side effects have
	been \emph{performed} in the first place.
\end{itemize}
\cmix{} selects sharing strategy on a per-function basis.

The only functions that are currently handled as non-shared are those
that have a static return value different from /void/. Such a return
value is equivalent to a static side effect, so they can never be
fully shared. However, all it would take to make them semi-shared
would be working code to memoize their return value.

We now turn to functions without static return values.
Naturally it is unsafe to specify ``fully shared'' for
a function that actually contains static nonlocal side
effects. On the other hand, there is a cost to using the
semi-shared stategy: an attepmt to create a recursive
residual program will loop.

If would appear that we need an analysis that tells which functions
may cause static side effects. It turns out, however, that there is
a clever way to use existing BTA information to select a sharing
strategy:

Assume the specialization process actually tries
to create a recursive residual program. The execution path
from this call to the recursive call either has a dynamic
branch in it, or it has not. If there is no dynamic branch,
then the subject program would also loop. It is commonly
accepted for partial evaluators to loop when faced with
an infinte loop under purely static control.

Hence we can assume that there \emph{is} a dynamic branch
somewhere in the recursion. But that means that the entire
recursion will be classified as ``under dynamic control'', which
again means that this function will not have been \emph{allowed}
by the BTA to have any static nonlocal side-effects.

The net result is that when the function is under dynamic
control it is safe to use the fully-shared strategy, and
when the function is not under dynamic control it is safe
to use a semi-shared strategy.

A function that was not under dynamic control, yet could
be shown to be free from static nonlocal side-effects, can
safely use either strategy. However, the dynamic control
information is readily available the only cost of letting
the function be fully shared is a slight waste of time while
detecting that there were no static side effects that need
to be reproduced.

Therefore, we choose to rely on the dynamic control information
for selecting sharing strategy.

\subsubsection{The generation function preamble}
Now we describes what happen in the generating function until
we reach the part that is directly derived from the subject
function's flow graph.

\begin{description}
\item[Local variables] \Pgen{} versions of all the function's
	local variables are declared, according to the type
	mapping. /cmixDataObject/s are constructed for the
	mortal ones. The /cmixDataObject/ constructor automatically
	registers itself with the specialization library.
\item[Parameters] The mortal parameters have already been declared
	in the function headings, but still /cmixDataObject/s
	need to be constructed for them. \Pgen{} versions of
	the immortal parameters are declared, again according
	to the type mapping.
\item[Locals array] An array called /cmix_locals/ that contains
	pointers to the /cmixDataObject/s just constructed for
	mortal local variables and parameters, is constructed.
\item[State object] A /cmix_state/ referring to the local and
	global objects is constructed. This will be used when
	memoizing program points within the function.
\item[Second state object] Another state object /cmix_funmemo/
	that includes the global variables and arguments, but
	not the local variables, is declared. If there are
	no mortal local variables it is an alias for /cmix_state/;
	else it is constructed by skipping past the first elements
	of /cmix_locals/.

	This state object is used when memoizing
	the function call itself---thus function calls will not
	be prevented from being shared just because they start
	with different random garbage in their local variables.
\item[Function memoization] The speclib function /cmixFunctionSeenB4/
	is called to determine whether a previously specialized
	function can be used (the sharing strategy is not used
	here: non-sharable functions are simply not found because
	they are never entered into the memoization pool).

	If a suitable function is found, the /cmixDataObject/s are
	de-registered (/cmixPopObjects/) and the generating
	function returns.
\item[Name managers] A name manager /cmixLabelNM/ is created
	to take care of goto labels in the residual functions.

	A name manager /cmixLocalNames/ is created to take care
	of the names of local variables and parameters in the
	residual function. This name manager works together
	with the master name manager to ensure that local names
	avoid names that have been used globally and, conversely,
	that later global namings avoid names that have been used
	at the local level.
\item[Residual function headning] The /cmixPushFun()/ is called
	to create a new residual functions and install that as
	the target of subsequent residual code emissions. The
	specialization library also gets told the sharing mode
	at this point.

	The specialization library only supports function return
	types that can be written \emph{before} the function name.
	That is, the residual return type cannot be pointer to array
	or pointer to function. \textbf{We do not check this!}---\Pgen{}
	simply contains a literal string with an abstract declaration
	as the ``type'', and un-compilable residual programs result
	when something goes wrong here.
\item[Residual parameters] The immortal parameters of the function
	map to parameters of the residual function. They get a name
	from the local name manager and are entered in the residual
	function's parameter list by /cmixParam()/. See 
	\ref{sec:gegen:InitializationFunctionsForDynamicUsertypes}
	for more on naming dynamic objects.
\item[Residual local variables] The immortal residual variables
	are initialized in the same way as the parameters.
\item[The pending loop] We create a fresh pending list with
	/cmixPushPend()/ and insert the function's first
	basic block into it. Since it is also the first block
	to be pulled out again, its code will be at the start
	of the residual function, too.

	The pending loop uses a GCC extension that allows goto
	labels in \Pgen{} to be stored in /void*/ pointers and
	jumped to later. A more portable solution would be
	to identify basic blocks numerically and have a /switch/
	statement, but that could be slow on C compilers that
	don't create jump tables for switches.

	When there are no more pending points for the function,
	we pop off the current function, the local memoization
	object, and the pending list, and return the name of
	the newly finished residual function.
\end{description}

\subsubsection{The actual generating code}
Now comes the code where residual production code is generated and
spectime actions done, one basic block at a time.

Each basic block consists of a series of transformed statements
and a transformed jump.

\begin{description}
\item[Assignment statements] are transformed straightforwardly
	according to the expression mapping. A residual assignment
	is generated iff the assigned value has a immortal type.
\item[Call statements] come in several flavors:
	\begin{description}
	\item[external pure calls] have been resolved by the
		btdebug phase to either external spectime calls
		or external residual calls.
	\item[external spectime calls] are made at spectime.
		The return value can be ignored or assigned to
		either an immortal or a mortal variable.
	\item[external residual calls] are simply constructed
		and emitted to the residual code. Again, the
		return value can be ignored or assigned
		to an immortal variable.
	\item[specializable calls] are more complex. They always
		have their own compound statement in \Pgen, so there
		can be local variables in their specialization.

		If the called function returns a mortal
		value it wants to write it to an actual mortal
		variable. If we have not a mortal variable at
		hand (because the call statement specifies
		it should be ignored, or lifted and assigned to an immortal
		variable), we need a temporary location to copy
		it into. If necessary, this location is declared
		with the name /cmixSR/.

		After possibly declaring a /cmixSR/ the generating
		function is called with the mortal of the arguments,
		and its return value (a /Code/ representing the
		function the residual program should call) is saved
		in a temporary names /cmixF/.

		Then a residual function call is constructed, and
		if the return value is genuinely immortal and not
		ignored the function call makes up the right-hand side
		of a residual assignment---else it is just a residual
		expression statement.

		Finally, if the return value should be lifted, a
		residual assignment is generated containing the
		lifted value of /cmixSR/.
	\end{description}
\item[Call through function pointers] are not handled yet and result
	in an error message if they ever make it to the Gegen phase.
\item[Alloc statements] depend on the binding time of the allocated pointer:
	\begin{description}
	\item[dynamic pointer] always points to immortal data. The
		call to {\catcode`\/=11 malloc/calloc} is emitted
		to the residual function.
	\item[static pointer to immortal data] This allocation
		makes a heap allocation in \Pgen{} but a global
		variable in the residual program. The \Pgen{}
		counterpart of the object is heap allocated,
		but is then named with a name from the global
		name manager and registered as a global residual
		variable.

		\textbf{Possible error:}In effect the allocation
		becomes a statically indexed dynamic array. Do
		all the separate dynamic variables get initialized
		correctly?
	\item[static pointer to mortal data] The allocation is
		performed by a special specialization library helper
		routine that allocates the memory and constructs and
		registers a /cmixDataObject/ for memoizing its
		contents.
	\end{description}
\item[Free statements] are ignored unless freeing a dynamic pointer.
	In that case a call to /free()/ is constructed and emitted
	to the residual program.
\item[Conditional jumps] If the condition is static the jump is
	performed at spectime, bypassing the pending list and jumping
	directly to the next basic block.

	If the condition is dynamic a residual /if/ is generated by
	a speclib function that also searches for the two targets in
	the pending list and enters them if they are not found.
	Then control resumes to the pending loop.
\item[Unconditional jumps] normally bypasses the pending list: an
	unconditional jump results in \Pgen.

	However the added potential for code sharing that can result
	from memoizing unconditional jumps is sometimes needed so much
	that it makes up for the very significant spectime slowdown
	that it causes. \cmix{} contains a switch that makes it
	generate \Pgen{}s that memoizes unconditional jumps when they
	are under dynamic control---it is not safe to memoize jumps
	that are not under dynamic control: that might violate the
	guarantees of execution sequence we make about /spectime:/
	annotated external calls.
\item[Return statements] generate a residual return statement. If
	the function's return value is static an assignment to
	the caller's static return value is generated before returning
	control to the pending loop.
\end{description}

\subsubsection{Implementation notes}
\begin{itemize}
\item The code for the translation of call statements is located in
	the pseudochapter ``Generator function conventions''.
\end{itemize}

\subsection{Chapter X: Initialization function}
This chapter contains the functions /cmixGenInit()/ which initializes
various spectime data before specialization can begin, and
/cmixGenExit()/ which writes the final residual program to a file
\subsubsection{cmixGenInit}
\begin{description}
\item[Taboo words] The global name manager is initialized with a
	list of words that must not be used for autogenerated residual
	identifiers. The list includes:
	\begin{itemize}
	\item words from explicit /taboo:/ directives
	\item tags of enums and dynamic structs and unions
	\item names of external and visible global residual variables
	\item names of all external functions
	\item C keywords
	\end{itemize}

	\textbf{Possible Error!} Shouldn't the names of enum constants
	appear in the taboo list too?

\item[Speclib version check] The initialization function calls the
	function /cmixSpeclibVersion/. This name is defined as a
	macro in /speclib.h/, and its expansion changes each time
	the \Pgen-speclib interface is modified. It has already been
	checked that the /speclib.h/ that is included agrees with
	the generated \Pgen; this call checks that the compiled
	specialization library that is linked in agrees with the
	header file. A version mismatch causes a linker error if
	one tries to link with an out-of-date speclib.

	The /cmixSpeclibVersion()/ function itself does nothing.
\item[Member names for dynamic structures] The /cmixMember/ structure
	described in section \ref{sec:gegen:NamesForDynamicStructures}
	is filled with freshly generated member names.

	A name manager is created for each usertype. \textbf{This is
	probably wrong}: member names should avoid at least some
	taboo words which may be defined as macros in header files.

\item[Global varibales] Initialization for global variables. This is
	different according to whether the variable is mortal or
	immortal:
	\begin{description}
	\item[Mortal variables] have their controlling
		/cmixDataObject/ (which was declared along with the
		variable itself).
	\item[Immortal variables] first have names assigned as
		descibed in section
		\ref{sec:gegen:InitializationFunctionsForDynamicUsertypes}.
		The names are either generated by the global name
		manager or (for external and visible variables) simply
		identical to the original names.

		After that, and unless the variable is /well-known:/
		annotated, a declaration is emitted to the residual
		program by calling the /cmixGlobal/ function.
		This is slightly messy because the specialization
		library does not have any independen concept of
		storage modes, so the residual variable storage modes
		must be included in the string that speclib thinks
		is the type.

		Also, the variable may have an intializer which is
		given as an optional third argument to /cmixGloba()/.
	\end{description}
\item[Start-up code] At the end of /cmixGenInit/ is a possibly empty
	series of fully spectime Core C statements that initialize
	global spectime variables. Initializers for global variables
	can result either in initializers in the variable's definition
	or in a statement here. The Core C generator tries to minimize
	the use of statements in this block to initialize global
	variables, because immortal variables cannot be initialized
	here. (If a statement with immortal parts occurs here, Gegen
	exits with an error message).
\end{description}

\subsection{cmixGenExit}
The /cmixGenExit()/ function is called after specialization is
complete to write out the residual program is complete.
/cmixGenExit()/ itself contains a \emph{preabmle} which it writes
to the specified files. Then it calls the specialization library
to write out the residual code that has been collected.

The preamble for the residual program contains:
\begin{description}
\item[Business card] A statement that the residual program was
	automatically generated by \cmix.
\item[Header inclusions] All headers which have been mentioned
	by /header:/ directives.
\item[Usertypes] Definitions of immortal structs and unions with
	the field names computed by /cmixGenInit()/. Definitions
	for every enum type.
\item[External functions] Prototypes for all external functions
	in the program, except those mentioned i /well-known:/
	directives.
\end{description}



\subsection{Implementation level (1998-03-02)}
\label{sec:GEGENImplementationLevel}

This chapter has become somewhat out of date after the speclib
interface has been reworked. The same general principles still
apply, though.

\DeleteShortVerb{/}

\end{docpart}

%%% Local Variables: 
%%% mode: latex
%%% TeX-master: /cmixII/
%%% End: 



% File: speclib.tex

\providecommand{\docpart}{\renewenvironment{docpart}{}{}
\end{docpart}
\documentclass[twoside]{cmixdoc}
%\bibliographystyle{apacite}

\makeatletter
\@ifundefined{@title}{\title{\cmix-documentation}}{}
\@ifundefined{@author}{\author{The \cmix{} Team}}%
{\expandafter\def\expandafter\@realauthor\expandafter{\@author}%
\author{The \cmix{} Team\\(\@realauthor)}}
\makeatother

\AtBeginDocument{%
\markboth{\hfill\today\quad\timenow\hfill\llap{\cmix\ documentation}}
{\hfill\today\quad\timenow\hfill}}

\renewcommand{\sectionmark}[1]{\markboth
{\hfill\today\quad\timenow\hfill\llap{\cmix\ documentation}}
{\rlap{\thesection. #1}\hfill\today\quad\timenow\hfill}}

%\newboolean{separate}
%\setboolean{separate}{true}

\renewenvironment{docpart}{\begin{document}}%
                          {\bibliography{cmixII}\printindex
                           \end{document}}
\begin{document}\shortindexingon
}
\title{Specialization library functions for \cmix}
\author{Arne John Glenstrup}
\begin{docpart}
\maketitle

\MakeShortVerb{"}

\begin{center}
  \fbox{\huge\textsl{THIS SECTION IS VERY OUT-OF-DATE}}    
\end{center}


\section{Specialization Library Functions}
\emph{NB:
Restructuring code has been added to the library, and is turned on as
default.  To switch off restructuring, set the integer "cmixRestruct"
to 0 (zero).  The restruturing algorithms used are slightly tweaked versions
of the algorithms presented by Cifuentes in her
thesis~\cite{cifuentes:rev-comp-tech}
}
\label{sec:SpecializationLibraryFunctions}
\index{speclib}\index{function!library!specialization}

This section describes the auxillary functions that are used by the
generating extension. They are grouped into three main classes:
\begin{description}
\item[Memory management functions] that take care of saving, restoring and
  comparing the static store,
\item[Memoization functions] that handle the sets of specializations points 
  that are waiting to be specialized, and checks whether a specialization
  point with the same static store has been seen before
\item[Code generation functions] that simply create and gather up the
  residual code, including lifting functions and functions for generating
  fresh identifiers.
\item[Miscellaneous helper functions] including functions for handling
  static return values.
\end{description}
Section~\ref{sec:SLFMemoryManagement} is based on an excerpt from a report
by~\cite{Andersen:1997:StaticMemoryManagementInCMix}.

\subsection{Memory management}
\label{sec:SLFMemoryManagement}
\index{management!memory}\index{store!static}

The present description does not yet discuss the management of
heap-allocated objects.

\paragraph{Objects.} On the one hand, it is desirable to have a
fine-grained representation of the active store during specialization
to enable memoization with high precision (e.g., the memoization
algorithm may want to treat the members of a struct as separate
objects, so that it can memoize only the members that are actually
in use at a given program point).

On the other hand, each object must be registered somewhere, so it is
desirable to have as coarse-grained representation as possible to
minimize the storage overhead.

We believe that this stage of the implementation calls for simplicity,
so we choose a relatively coarse-grained representation: The storage
that is allocated by the ``execution'' of a (static) declaration is
perceived as one object. This implies, for instance, that all elements of
an array are treated together.

At specialization time a static object is thus allocated by the
``execution'' of a declaration. Global declarations are executed at startup
and local declarations are executed when the function they appear in is
entered. Due to recursive calls one local declaration may thus give rise to
several  objects that are allocated at different locations on the
stack at a given time.

At specialization time we need to keep some book-keeping information
for each object. We shall refer to this as \emph{the object
description} and as the C++ class $\TDataObject$_{DataObject@$\TDataObject$}.

\paragraph{Copies of Objects.} Since the operations needed
for objects in the active store are different from those needed for copies,
we distinguish between the two kinds of objects. The book-keeping
information for a copied object is referred to as \emph{the description of
  a copied object} or as the C++ class
$\TDataObjectCopy$_{DataObjectCopy@$\TDataObjectCopy$}.


\subsubsection{Memory management functions and data structures}
In general, we need to make three operations available on objects:
\begin{center}\def\arraystretch{1.2}
{\setbox0=\hbox{$\Tint\ \Fcmp(\TDataObject\ *\Vobj,\ \TDataObjectCopy
  *\Vcpy)$}
\newdimen\commentwidth\commentwidth=-\wd0
\advance\commentwidth by .9\textwidth
\ifdim\commentwidth<5em\commentwidth=5em\fi
\begin{tabular}{lp{\commentwidth}}
$\TDataObjectCopy\ {*}\Fcopy(\TDataObject\ {*}\Vobj)$
&
copies the object described by $\Vobj$ and
returns a description of the copied object.
\\
$\Tvoid\ \Frestore(\TDataObjectCopy\ {*}\Vcpy)$
&
takes a description
of a copy and restores it to its original location.
This requires that the original
location of the object must be part of the description of the copied
object\footnotemark
\\
$\Tbool\ \Fcmp(\TDataObject\ *\Vobj,\ \TDataObjectCopy *\Vcpy)$
&
compares an object in the active
store ($\Vobj$) with a copied object ($\Vcpy$),  returning true
if they match, false otherwise.
\end{tabular}\footnotetext{Alternatively, the restore function could take the
  description of the original object as parameter, thus reducing the
  memory overhead in the object description, but this makes it more
  complicated to use the $\Frestore$ function}}
\end{center}

The C++ data structures that describe an object and a copy of an object 
must contain the following fields:
\begin{center}\def\arraystretch{1.1}
\begin{tabular}{|l|l|}\hline
\hfil $\TDataObject$        & \hfil $\TDataObjectCopy$ \\\hline
A pointer to the object           & A pointer to the original 
                                    location of the copy \\
                                  & A pointer to the copy \\
The size of the object            & The size of the copy \\
A pointer to its copy function    & \\
A pointer to its compare function & \\\hline
\end{tabular}
\end{center}
The first two fields are specific to the object, whereas the rest are
specific to the declaration (type) of the object.  
%The visit flag is used to mark already processed objects, thus avoiding
%infinite looping on cyclic data-structures when doing a recursive copy.

The size can be hard-coded (generated) into the copy and compare
functions, and the two functions can be joined into a general function
which takes an extra argument specifying whether it should copy or
compare, but we leave this for future optimizations.

Note that the original location is only needed for restoring
and not for comparing.

The data structure used when copying an object must in fact contain
copies of several objects, because if the object contains a pointer,
we must also copy whatever object it points to. This data structure must
support restoring and comparing. Restoring of the state does not
require the copied objects to be structured in any way, since it
amounts to restoring all copied objects. Comparing is a bit more
difficult, since we must make sure that the right objects are compared
to the right copies.

We thus arrange the copied objects in a list where the copy of an
object comes immediately before the copies of the objects it
references. This list structure incurs an overhead of two pointers per
copied object. Alternatively, we could use an array of references to
objects, which would reduce the overhead to one pointer per object,
but at the same time it complicates the implementation, because the
number of elements is unknown when the copying begins. Another
possibility is to store the copied objects contiguously in memory,
which does not involve any overhead, but it also complicates the
implementation and makes it difficult to share identical copies. For
simplicity we have chosen to use a list, and leave trimming of the
memory usage to the future.

\subsubsection{The Standard and Pointers} Four rules from the Standard
are important to note when dealing with pointers:

\begin{enumerate}
\item \label{item-array} Strictly conforming \ansiC allows pointer
arithmetic to move a pointer around inside the same array, and to move
it one past the last element of the array.
\item It is guaranteed that a pointer to an object may be converted to
a pointer to an object whose type requires less or equally strict
storage alignment and back again without change.
\item\label{rule-first-member} If a pointer to a structure is cast
to the type of a pointer to its first member, the result refers to the
first member.
\item A cast of a pointer to an integral type\footnote{Types {\tt
char}, and {\tt int} of all sizes, each with or without sign, and also
enumerations types, are collectively called \emph{integral}
types.~\cite{Kernighan:1988:CProgrammingLanguage}.} or vice versa is implementation
defined and therefore not strictly conforming to the Standard.
\end{enumerate}
The first rule implies that it may be necessary to copy an
entire array when a pointer refers to it. The second rule means that
the type of the pointer may not be sufficient to determine how many
bytes to copy (due to casts). The third rule implies that it may be
necessary to copy the entire struct when copying a pointer to the
first element of the struct. Finally, the fourth rule drastically
limits the number of objects a pointer may point to. Without the forth
rule we would have to copy the entire store in the worst case.

\cmix allows a pointer to an object to be cast to another pointer to an
object, as long as the level of indirection remains unchanged, whereas
other casts from or to pointers are residualized, cf.\ 
Section~\vref{sec:BTAExpressions}. This implies that a cast of a pointer to
an integral type or vice versa is made dynamic by \cmix.


\subsubsection{Type specific memory management functions}
\label{sec:TypeSpecificMemoryFunctions}

We now consider each C type in turn to see what is
required to implement memory management support for objects of that
particular type.

We divide the types into \emph{arithmetic} (char, int, float, double,
enumerations, and derived forms; i.e., signed, unsigned, short, long),
\emph{function pointers}, \emph{pointers to objects}, \emph{structs},
\emph{unions}, and \emph{arrays}.


\paragraph{Arithmetic Types}\label{sec:SLFArithmeticTypes}
 are easy:~given a variable $x$ of type $T$,
we simply copy $\Fsizeof(T)$ bytes starting at $\&x$. 
Comparing is equally simple. 

Thus, the specialization library must contain two functions for
copying and comparing simple objects:
\[
\begin{array}{rl}
\TDataObjectCopy & \FcopySimple(\TDataObject\ *\Vobj); \\
\Tbool & \FcmpSimple(\TDataObject\ *\Vobj, \TDataObjectCopy\ *\Vcpy); 
\end{array}
\]

\paragraph{Function
  Pointers}\label{sec:SLFFunctionPointers}\index{pointer!function}
can be treated in exactly the same manner as arithmetic types. The
referenced function cannot change, and therefore there is no need to copy
it.

\paragraph{Pointers to Objects}
\label{sec:PointersToObjects}
\index{pointer!object}

In this section we describe how we at specialization time can
determine exactly which object a given pointer refers to.

Consider a static pointer \texttt{p}. If \texttt{p} points to dynamic
data, we simply copy the pointer. If, however, \texttt{p} is a pointer
to static data, we must also copy what it points to. In general it is
not sufficient just to copy the integer \texttt{*p}, though, since
other addresses may be referenced as well, either via pointer
arithmetic or via casts.

If \texttt{p} refers to an element of an array, then we must copy the
entire array. This implies that the copy function must know the size
of the array.

When we later during the specialization compare the memoized array
with a new array referenced by \texttt{p}, the new array may be of
another size, so before comparing the two their sizes must be found to
match.  However, this is still not sufficient:~it could be the case
that the pointers match and the referenced arrays match, but the
offset of the pointers into the arrays differ, as in this example with a
4-element array:
\begin{center}
  \begin{tabular}{l|c|r}\cline{2-2}
    "0x4020": & 42 \\\cline{2-2}
    "0x4016": & 17 \\\cline{2-2}
    "0x4012": & 42 & $<--$"p" \\\cline{2-2}
    "0x4008": & 17 \\\cline{2-2}
    \mc{1}{l}{"0x4004":} & \mc{1}{c}{--} \\
    \mc{1}{l}{"0x4000":} & \mc{1}{c}{--} \\
  \end{tabular}
\hfil
  \begin{tabular}{l|c|r}
    \mc{1}{l}{"0x4020":} & \mc{1}{c}{--} \\
    \mc{1}{l}{"0x4016":} & \mc{1}{c}{--} \\\cline{2-2}
    "0x4012": & 42 & $<--$"p" \\\cline{2-2}
    "0x4008": & 17 \\\cline{2-2}
    "0x4004": & 42 \\\cline{2-2}
    "0x4000": & 17 \\\cline{2-2}
  \end{tabular}
\end{center}
In this case, the two states should not compare equal; hence the offset
of of a pointer into an array must be memoized as well.
Similar considerations apply to pointers to structs: we must copy the
entire struct and the pointer's offset relative to the start address
of the struct.
This way, what we later compare are not the absolute
locations_{location!absolute}, but the \emph{structure} of the
data_{data!structure of}.

When copying a pointer $p$ at specialization time we therefore examine
each object in turn to determine whether or not $p$ may refer to it.
One could take an offline approach, and memoize all objects that a
pointer may point to as reported by the pointer analysis. However, we
fear that this would be overly conservative. For example, the pointer
analysis does not distinguish between different objects allocated by
the same declaration, and it is non-trivial to extend it to do so.

If a pointer points to an array, then it may point one past the last
element of the array (cf.\ Rule~\vref{item-array}). This implies that
we are not able to determine whether a pointer refers to an array or
the object allocated right after the array. A safe approximation would
be to copy both objects, but this may give redundant specialization,
it may waste storage, and it complicates the comparison algorithm, so
to avoid ambiguities we \emph{increase the size of all static arrays
  by one prior to specialization_{array!length!increasing}.} Since the
original program is strictly conforming to the Standard then we know
that pointer arithmetic will only move pointers around inside the
slightly larger arrays and never outside any. The only exceptions are
expressions like "sizeof(a)" where "a" is an array; in this case we
transform it into "(sizeof(a) - sizeof(a[0]))".

Now it is the case that a (non-function) object $x$ is referenced by
pointer $p$ if $\&x =< p < \&x + \Fsize(x)$.
%The following condition determines whether $p$ may refer
%to the object:
%\begin{center}
%\begin{tabular}{ll}
% \emph{Object $x$ of type} & \emph{is referenced by pointer $p$ if} \\[.8ex]
% arithmetic & $\&x = p$ \\
% pointer    & $\&x = p$ \\
% struct     & $\&x =< p < \&x + \Fsize(x)$ \\
% union      & $\&x =< p < \&x + \Fsize(x)$ \\
% array      & $\&x =< p < \&x + \Fsize(x)$
%\end{tabular}
%\end{center}
As mentioned above, pointers to functions do not require any object to be
copied.

Thus, the specialization library must contain two functions for copying and
comparing a referenced object and the offset of the reference (i.e.\ $p -
\&x$)%
_{copyWithOffset@$\FcopyWithOffset$}%
_{cmpWithOffset@$\FcmpWithOffset$}:
\[
\begin{array}{rl}
\TDataObjectCopy & \FcopyWithOffset(\TDataObject\ {*}\Vobj,\ \Tvoid\
{*}\Vref); \\
\Tbool & \FcmpWithOffset(\TDataObject\ {*}\Vobj,\ \TDataObjectCopy\
{*}\Vcpy,\ \Tvoid\ {*}\Vref); \\
\end{array}
\]

\paragraph{Pointers to Strings.} The Standard specifies that the
behavior of a program that attempts to alter a string literal is
undefined. This implies that string literals need not be copied.
 
\paragraph{Extern Pointers.} Since the binding-time analysis of C-Mix
residualizes all external pointers, the object referred to by a static
pointer will always stem from a static declaration in the program.


\paragraph{Structs}
\label{sec:SLFStructs}\index{struct!static!memoizing}
are copied like arithmetic types by copying $\Fsize(x)$
bytes starting at $\&x$, but additionally, if the struct contains
pointers then we must also recursively copy objects referenced by
pointers in the struct.

Since the static data is represented directly in the generating extension,
all we have is a void pointer to a collection of bytes inside which one or
more pointers are ``hidden''. The simplest way to extract the pointers is
to cast the void pointer to a pointer of the appropriate struct type. To
this end we generate \emph{type-specific copy and compare
  functions}_{function!copy!type specific}_{function!copy!struct}%
_{function!compare!type specific}_{function!compare!struct} for each
user defined type in use by the static objects~\cite[Section
3.10.4]{Andersen:1994:ProgramAnalysisAndSpecialization}.  These functions
are generated in the gegen phase, prior to specialization, and added to the
generating extension.

\begin{example}[Copying structs]
\index{struct!static!memoizing}
\label{exm:StructCopy}
Suppose the subject program contains a static variable of  type
"struct U":
\begin{verbatim}
struct T { int a; int *p; };
struct U { int *q; struct T t[42]; };
\end{verbatim}
When copying an object of type "struct U" with object
description "obj", the function first copies the struct itself,
then it casts the void pointer denoting the location of the object
"obj->loc" to a "struct U" pointer, and extracts the 42 pointers
from it:
\begin{verbatim}
DataObjectCopy *copy_struct_U(DataObject *obj, void *ref) {
  if (obj->visit) return NULL;

  DataObjectCopy *copy = copyWithOffset(obj, ref);   /* Copy struct U itself */

  struct U *pu = (struct U *) (obj->loc);            /* Copy any referenced objects */
  copy = append(copy, copyReferencedObject(pu->q));
  for (i = 0; i < 42; i++) {
    struct T *pt = &(pu->t[i]);
    copy = append(copy, copyReferencedObject(pt->p));
  }
  return copy;
}
\end{verbatim}
The compare function will simply compare the copied objects in the
same order:
\begin{verbatim}
bool cmp_struct_U(DataObject *obj, DataObjectCopy *copy, void *ref) {
  if (obj->visit) return true;

  if (!cmpWithOffset(obj, copy, ref)) return false; /* Compare struct U itself */

  struct U *pu = (struct U *) (obj->loc);           /* Compare any referenced objects */
  if (!cmpReferencedObject(pu->q, copy)) return false;
  for (i = 0; i < 42; i++) {
    struct T *pt = &(pu->t[i]);
    if (!cmpReferencedObject(pt->p, copy)) return false;
  }
  return true;
}
\end{verbatim}
Note that no copy or compare functions are generated for "struct T"
unless a static variable is declared with this type.
\end{example}


The copy and compare functions are generated by inserting a list of
transformations $\gegen{u_{i_1}}_{\Vcopy}$; \ldots
; $\gegen{u_{i_g}}_{\Vcopy}$; \ldots;
$\gegen{u_{i_1}}_{\Vcmp}$; \ldots
; $\gegen{u_{i_g}}_{\Vcmp}$ after transforming the user defined types
in Figure~\vref{fig:GEGENPrograms}. This is only done for those user types
$u_{i_1}, ..., u_{i_g}$ for which variable declarations exist. These
transformations_{<.>copy@$\gegen{\cdot}_{\Vcopy}$}%
_{<.>cmp@$\gegen{\cdot}_{\Vcmp}$}_{<.>@$\gegen{\cdot}$} are defined in
Figures~\ref{fig:SLFUserTypesCopyFun}
and~\ref{fig:SLFUserTypesCompareFun}%
_{copys@$\Fcopy_{\sigma}$}_{cmps@$\Fcmp_{\sigma}$}.
\begin{figure}[htbp]
  \begin{center}
    \small
    \[\def\arraystretch{1.2}
    \begin{array}{@{}lcl@{}}
      \gegen{(\sigma, \Void, \CStruct_s, d_1... d_n)}_{\Vcopy} &
      =& \TDataObjectCopy\ \widehat{\Fcopy_{\sigma}}
           (\TDataObject\ {*}\Vobj, \Tvoid\ {*}\Vref)\ \{ \\
      && \quad \Kif\ \Fvisited(\Vobj)\ \Kthen\ \Kreturn\ [] \\
      && \quad \Vcopy
               := \FcopyWithOffset(\Vobj, \Vref) \\
      && \quad \Tstruct\ \Void_{\sigma}\ *\widehat{x} 
               := (\Tstruct\ \Void_{\sigma}*)\Fobject(\Vobj) \\
      && \quad \gegen*{(*x), d_1}_{\Vcopy}; ...; 
               \gegen*{(*x), d_n}_{\Vcopy}\\
      && \quad \Kreturn\ \Vcopy \\
      && \} \\
      \gegen{(\sigma, \Void, \CStruct_d, d_1... d_n)}_{\Vcopy} &
      =& \mbox{---}
      \\
      \gegen{(\sigma, \Void, \CUnion_s, d_1... d_n)}_{\Vcopy} &
      =& \TDataObjectCopy\ \widehat{\Fcopy_{\sigma}}
           (\TDataObject\ {*}\Vobj, \Tvoid\ {*}\Vref)\ \{ \\
      && \quad \Kif\ \Fvisited(\Vobj)\ \Kthen\ \Kreturn\ [] \\
      && \quad \Vcopy
               := \FcopyWithOffset(\Vobj, \Vref) \\
      && \quad \Tstruct\ \Void_{\sigma}\ *\widehat{x} 
               := (\Tstruct\ \Void_{\sigma}*)\Fobject(\Vobj) \\
      && \quad \gegen*{(*x), d_1}_{\Vcopy} \\
      && \quad \Kreturn\ \Vcopy \\
      && \} \\
      \gegen{(\sigma, \Void, \CUnion_d, d_1... d_n)}_{\Vcopy} &
      =& \mbox{---}\\[1em]
      \gegen*{x, \CStructMem(\delta, \Void, t_s^s, \Voc)}_{\Vcopy} &
      =& \mbox{---}
      \\
      \gegen*{x, \CStructMem(\delta, \Void, 
        \CPointer_s^s(q, t), \Voc)}_{\Vcopy} &
      =& \Vcopy := \Vcopy \append \FcopyReferencedObject(x.\Void)
      \\
      \gegen*{x, \CStructMem(\delta, \Void, 
        \CPointer_s^d(q, t), \Voc)}_{\Vcopy} &
      =& \mbox{---}
      \\
      \gegen*{x, \CStructMem(\delta, \Void, 
        \CArray_s^s(q, t, e), \Voc)}_{\Vcopy} &
      =& \gegen*{x.{\Void}, [e], t}_{\Vcopy}
      \\
%      \gegen*{ 
%        \CStructMem(\delta, \Void, 
%                    \CFunT(q, t, t_1 ... t_n), \Voc)}_{\Vcopy} &
%      =& \Kerror
%      \\
      \gegen*{x, \CStructMem(
        \begin{array}[t]{@{}l@{}}
          \delta, \Void, \\
          \CUser_s^s(q, \sigma, \CStruct, d_1 ... d_n),
          \Voc)}_{\Vcopy} 
      \end{array} &
      =& \gegen*{x.\Void_{d_1}, d_1}_{\Vcopy}; ...; 
         \gegen*{x.\Void_{d_n}, d_n}_{\Vcopy}
      \\
      \gegen*{x, \CStructMem(
        \begin{array}[t]{@{}l@{}}
          \delta, \Void, \\
          \CUser_s^s(q, \sigma, \CUnion, d_1 ... d_n), \Voc)}_{\Vcopy}
      \end{array} &
      =& \gegen*{x.\Void_{d_1}, d_1}_{\Vcopy}
      \\
      \gegen*{x, \CStructMem(\delta, \Void, 
        \CAbstract(q, \Vid), \Voc)}_{\Vcopy} &
      =& ???
      \\[1ex]
      \gegen*{x, \Vms,  t_s^s}_{\Vcopy} &
      =& \mbox{---}
      \\
      \gegen*{x, \Vms, \CPointer_s^s(q, t_d)}_{\Vcopy} &
      =& \Kfor\ (i_1, ..., i_k) \in 
           \{0, ..., m_1\} \x \cdots \x \{0, ..., m_k\}\ \Kdo\\
      && \quad \Vcopy := \Vcopy \append 
           \FcopyReferencedObject(x[i_1]\cdots[i_k])
      \\
      \gegen*{x, \Vms, \CPointer_s^d(q, t)}_{\Vcopy} &
      =& \mbox{---}
      \\
      \gegen*{x, \Vms, \CArray_s^s(q, t, e)}_{\Vcopy} &
      =& \gegen*{x, [m_1, ..., m_k,e], t}_{\Vcopy}
      \\
%      \gegen*{ 
%        \Vms, \delta, \Void, \CFunT(q, t, t_1 ... t_n), \Voc}_{\Vcopy} &
%      =& \Kerror
%      \\
      \gegen*{x, \Vms, 
          \CUser_s^s(q, \sigma, \CStruct, d_1 ... d_n)}_{\Vcopy} &
      =& \Kfor\ (i_1, ..., i_k) \in 
           \{0, ..., m_1\} \x \cdots \x \{0, ..., m_k\}\ \Kdo\\
      && \quad \Tstruct\ \Void_{\sigma}\ *\widehat{x'} 
         := \&(x[i_1]\cdots[i_k]) \\
      && \quad \gegen*{x'->\Void_{d_1}, d_1}_{\Vcopy}; ...; 
               \gegen*{x'->\Void_{d_n}, d_n}_{\Vcopy}
      \\
      \gegen*{x, \Vms, 
          \CUser_s^s(q, \sigma, \CUnion, d_1 ... d_n)}_{\Vcopy} &
      =& \Kfor\ (i_1, ..., i_k) \in 
           \{0, ..., m_1\} \x \cdots \x \{0, ..., m_k\}\ \Kdo\\
      && \quad \Tstruct\ \Void_{\sigma}\ *\widehat{x'} 
         := \&(x[i_1]\cdots[i_k]) \\
      && \quad \gegen*{x'->\Void_{d_1}, d_1}_{\Vcopy}
      \\
      \gegen*{x, \Vms, \CAbstract(q, \Vid)}_{\Vcopy} &
      =&
    \end{array}
    \]
    \caption{Adding copy functions for user defined type to the gegen
      transformation}
    \label{fig:SLFUserTypesCopyFun}
  \end{center}
\end{figure}

\begin{figure}[htbp]
  \begin{center}
    \small
    \[\def\arraystretch{1.2}
    \begin{array}{@{}lcl@{}}
      \gegen{(\sigma, \Void, \CStruct_s, d_1... d_n)}_{\Vcmp} &
      =& \Tbool\ \widehat{\Fcmp_{\sigma}}
          (\begin{array}[t]{@{}l@{}}
             \TDataObject\ {*}\Vobj, \\
             \TDataObjectCopy\ {*}\Vcpy, \Tvoid\ {*}\Vref)\ \{ 
           \end{array}\\
      && \quad \Kif\ \Fvisited(\Vobj)\ \Kthen\ \Kreturn\ \Ctrue \\
      && \quad \Kif\ \Fnot(\FcmpWithOffset(\Vobj, \Vcpy, \Vref))\
               \Kthen\\
      && \qquad \Kreturn\ \Cfalse \\
      && \quad \Tstruct\ \Void_{\sigma}\ *\widehat{x} 
               := (\Tstruct\ \Void_{\sigma}*)\Fobject(\Vobj) \\
      && \quad \gegen*{(*x), d_1}_{\Vcmp}; ...; 
               \gegen*{(*x), d_n}_{\Vcmp}\\
      && \quad \Kreturn\ \Ctrue \\
      && \} \\
      \gegen{(\sigma, \Void, \CStruct_d, d_1... d_n)}_{\Vcmp} &
      =& \mbox{---}
      \\
      \gegen{(\sigma, \Void, \CUnion_s, d_1... d_n)}_{\Vcmp} &
      =& \Tbool\ \widehat{\Fcmp_{\sigma}}
          (\begin{array}[t]{@{}l@{}}
             \TDataObject\ {*}\Vobj, \\
             \TDataObjectCopy\ {*}\Vcpy, \Tvoid\ {*}\Vref)\ \{ 
           \end{array}\\
      && \quad \Kif\ \Fvisited(\Vobj)\ \Kthen\ \Kreturn\ \Ctrue \\
      && \quad \Kif\ \Fnot(\FcmpWithOffset(\Vobj, \Vcpy, \Vref))\
               \Kthen\\
      && \qquad \Kreturn\ \Cfalse \\
      && \quad \Tstruct\ \Void_{\sigma}\ *\widehat{x} 
               := (\Tstruct\ \Void_{\sigma}*)\Fobject(\Vobj) \\
      && \quad \gegen*{(*x), d_1}_{\Vcmp}\\
      && \quad \Kreturn\ \Ctrue \\
      && \} \\
      \gegen{(\sigma, \Void, \CUnion_d, d_1... d_n)}_{\Vcmp} &
      =& \mbox{---}\\[1em]
      \gegen*{x, \CStructMem(\delta, \Void, t_s^s, \Voc)}_{\Vcmp} &
      =& \mbox{---}
      \\
      \gegen*{x, \CStructMem(\delta, \Void, 
        \CPointer_s^s(q, t), \Voc)}_{\Vcmp} &
      =& \Kif\ \Fnot(\FcmpReferencedObject(x.\Void, \Vcpy))\ \Kthen\\
      && \quad \Kreturn\ \Cfalse
      \\
      \gegen*{x, \CStructMem(\delta, \Void, 
        \CPointer_s^d(q, t), \Voc)}_{\Vcmp} &
      =& \mbox{---}
      \\
      \gegen*{x, \CStructMem(\delta, \Void, 
        \CArray_s^s(q, t, e), \Voc)}_{\Vcmp} &
      =& \gegen*{x.{\Void}, [e], t}_{\Vcmp}
      \\
%      \gegen*{ 
%        \CStructMem(\delta, \Void, 
%                    \CFunT(q, t, t_1 ... t_n), \Voc)}_{\Vcmp} &
%      =& \Kerror
%      \\
      \gegen*{x, \CStructMem(
        \begin{array}[t]{@{}l@{}}
          \delta, \Void, \\
          \CUser_s^s(q, \sigma, \CStruct, d_1 ... d_n), \Voc)}_{\Vcmp}
      \end{array} &
      =& \gegen*{x.\Void_{d_1}, d_1}_{\Vcmp}; ...;
         \gegen*{x.\Void_{d_n}, d_n}_{\Vcmp}
      \\
      \gegen*{x, \CStructMem(
        \begin{array}[t]{@{}l@{}}
          \delta, \Void, \\
          \CUser_s^s(q, \sigma, \CUnion, d_1 ... d_n), \Voc)}_{\Vcmp}
      \end{array} &
      =& \gegen*{x.\Void_{d_1}, d_1}_{\Vcmp}
      \\
      \gegen*{x, \CStructMem(\delta, \Void, 
        \CAbstract(q, \Vid), \Voc)}_{\Vcmp} &
      =& ???
      \\[1ex]
      \gegen*{x, \Vms, t_s^s}_{\Vcmp} &
      =& \mbox{---}
      \\
      \gegen*{x, \Vms, \CPointer_s^s(q, t_d)}_{\Vcmp} &
      =& \Kfor\ (i_1, ..., i_k) \in 
           \{0, ..., m_1\} \x \cdots \x \{0, ..., m_k\}\ \Kdo\\
      && \quad \Kif\ \Fnot
          (\FcmpReferencedObject(x[i_1]\cdots[i_k], \Vcpy))\
         \Kthen\\
      && \qquad \Kreturn\ \Cfalse
      \\
      \gegen*{x, \Vms, \CPointer_s^d(q, t)}_{\Vcmp} &
      =& \mbox{---}
      \\
      \gegen*{x, \Vms, \CArray_s^s(q, t, e)}_{\Vcmp} &
      =& \gegen*{x, [m_1, ..., m_k,e], t}_{\Vcmp}
      \\
%      \gegen*{ 
%        \Vms, \delta, \Void, \CFunT(q, t, t_1 ... t_n), \Voc}_{\Vcmp} &
%      =& \Kerror
%      \\
      \gegen*{x, \Vms, 
          \CUser_s^s(q, \sigma, \CStruct, d_1 ... d_n)}_{\Vcmp} &
      =& \Kfor\ (i_1, ..., i_k) \in 
           \{0, ..., m_1\} \x \cdots \x \{0, ..., m_k\}\ \Kdo\\
      && \quad \Tstruct\ \Void_{\sigma}\ *\widehat{x'} 
         := \&(x[i_1]\cdots[i_k]) \\
      && \quad \gegen*{x'->\Void_{d_1}, d_1}_{\Vcmp}; ...; 
               \gegen*{x'->\Void_{d_n}, d_n}_{\Vcmp}
      \\
      \gegen*{x, \Vms, 
          \CUser_s^s(q, \sigma, \CUnion, d_1 ... d_n)}_{\Vcmp} &
      =& \Kfor\ (i_1, ..., i_k) \in 
           \{0, ..., m_1\} \x \cdots \x \{0, ..., m_k\}\ \Kdo\\
      && \quad \Tstruct\ \Void_{\sigma}\ *\widehat{x'} 
         := \&(x[i_1]\cdots[i_k]) \\
      && \quad \gegen*{x'->\Void_{d_1}, d_1}_{\Vcmp}
      \\
      \gegen*{x, \Vms, \CAbstract(q, \Vid)}_{\Vcmp} &
      =& ???
    \end{array}
    \]
    \caption{Adding compare functions for user defined type to the gegen
      transformation}
    \label{fig:SLFUserTypesCompareFun}
  \end{center}
\end{figure}


\paragraph{Unions}\index{union!static!memoizing}
are a bit more tricky, since a member of pointer type might
share storage with other members. Consider  the following
union:
\begin{verbatim}
union U { int a; double x; int *p; } u;
\end{verbatim}
In general the Standard does not allow a member of a union
to be inspected unless the value of the union has been assigned using
that same member. The exception from the rule is that ``if a union
contains several structures that share a common initial sequence, and
if the union currently contains one of these structures, it is
permitted to refer to the common initial part of any of the contained
structures'' (an example can be found in
\cite[p.~214]{Kernighan:1988:CProgrammingLanguage}). Thus, when we
want to perform the memoizing, we can copy "u" itself, but we do not
know how to interpret the data: as an int, double or pointer?

Two possible ways of handling unions are:

\begin{enumerate}
\item[1.] keeping track of which members are the currently valid, or
\item[2.] splitting the union, so that members of pointer type are
allocated at private memory locations (keeping in mind the exception
mentioned above).
\end{enumerate}

\noindent The first approach should be avoided since it conflicts with
the idea directly representing static objects in the generating extension.
The second approach can be implemented with some storage overhead, but to
keep things simple, we will instead make a restriction for the present
version of \cmix:
\begin{constraint}[Unions]\index{union!static!constraint on}%
\index{constraint!union!static}%
\label{cns:SLFUnion}
In static unions, pointers to static data are only allowed if the
union exclusively contains structures, and then only in the initial
part that is common for \emph{all} the structures\footnote{Recall
  from Seciton~\ref{sec:BTAUnions} that the binding times of the
  members that are in an intial common part for two or more struct
  members must be identical.}.
\end{constraint}
This implies that pointers "u.t1.pt1" and "u.t2.pt2", but not pointer
"u.t1.qt" in this example can become pointers to static data:
\begin{verbatim}
struct T1 { int tag1; struct T *pt1; struct T *qt; };
struct T2 { int tag2; struct T *pt2; };
union U { struct T1 t1; struct T2 t2; } u;
\end{verbatim}
Note also that pointers "u.p" and "u.q" in this example can \emph{not}
become pointers to static data because they are not contained in
structs:
\begin{verbatim}
union U { int *p; int *q; } u;
\end{verbatim}
Most programs that do not satisfy this restriction can be rewritten to 
do so.

\paragraph{Arrays}\index{array!static!memoizing}
\label{sec-arrays}
are handled similarly to structs. If an array contains
pointers to static data then the referenced objects must be copied as
well as the array itself.

The type-specific copy and compare functions for arrays
containing pointers to static data are also placed in the generating
extension by gegen transformations shown in
Figure~\ref{fig:SLFArrayCopyCmpFun}.
\begin{figure}[htb]
  \begin{center}
    \small
    \[\def\arraystretch{1.2}
    \begin{array}{@{}lcl@{}}
      \gegen{t' == \CArray_s^s(q, t, e)}_{\Vcopy} &
      =& \TDataObjectCopy\ \widehat{\Fcopy_{t'}}
        (\TDataObject\ {*}\Vobj, \Tvoid\ {*}\Vref)\ \{ \\
      && \quad \Kif\ \Fvisited(\Vobj)\ \Kthen\ \Kreturn\ [] \\
      && \quad \Vcopy := \FcopyWithOffset(\Vobj, \Vref) \\
      && \quad t'\ {*}\widehat{x} := (t'*)\Fobject(\Vobj) \\
      && \quad \gegen*{x,[e],t}_{\Vcopy} \\
      && \}
      \\[1em]
      \gegen{t' == \CArray_s^s(q, t, e)}_{\Vcmp} &
      =& \Tbool\ \widehat{\Fcmp_{t'}}
        (\TDataObject\ {*}\Vobj, \TDataObjectCopy\ {*}\Vcpy,
         \Tvoid\ {*}\Vref)\ \{ \\
      && \quad \Kif\ \Fvisited(\Vobj)\ \Kthen\ \Kreturn\ \Ctrue \\
      && \quad \Kif\ \Fnot(\FcmpWithOffset(\Vobj, \Vcpy, \Vref)\
                \Kthen\ \Kreturn\ \Cfalse\\
      && \quad t'\ {*}\widehat{x} := (t'*)\Fobject(\Vobj) \\
      && \quad \gegen*{x,[e],t}_{\Vcmp} \\
      && \quad \Kreturn\ \Ctrue \\
      && \}
    \end{array}
    \]
    \caption{Adding copy and compare functions for array types to the
      gegen transformation}
    \label{fig:SLFArrayCopyCmpFun}
  \end{center}
\end{figure}


\subsection{Memoization functions}
\label{sec:SLFMemoizationFunctions}
\index{set!pending}\index{memoization}

The generating extension as described in
Section~\ref{sec:GeneratingTheGeneratingExtension} makes use of several
functions for maintaining a set of pending specialization points:
\begin{description}
\item[{$[\cdot] : \DLabel -> \DResidualLabel$}]_{[.]@$[\cdot]$} 
  Calls $\FpendInsert$ with the current state.
\item[$\FpendInsert:\DLabel \x \DState ->
  \DResidualLabel$.]_{pendInsert@$\FpendInsert$} When a program
  point_{point!program} $p$ represented by a label in the generating
  extension is to be memoized with respect to the current state $S$,
  $\FpendInsert$ generates a fresh label $l$ for the residual program and
  inserts a tuple $(p,S,l)$ called a \emph{specialization
    point}_{point!specialization} into the pending set, along with a copy,
  $\sigma$, of the part of the active store that is reachable from $S$.
\item[$\Fpending:\Dvoid -> \Dbool$.]_{pending@$\Fpending$}
  Checks whether there are any specialization
  points_{point!specialization!pending} waiting to be specialized.
\item[$\FpopSP : \Dvoid -> (\DProgramPoint \x \DState \x
  \DResidualLabel)$.]_{popSP@$\FpopSP$} Removes a pending specialization
  point from the pending set and returns it.
\item[$\Frestore:\DState -> \Dvoid$.]_{restore@$\Frestore$} Given a state,
  $S$, $\Frestore(S)$ copies the static data $\sigma$ associated with $S$
  to the active store.
\item[$\Fcopy:\DState -> \DState$.]_{copy@$\Fcopy$} Given the current
  state, $S$, $\Fcopy(S)$ returns a copy of it, along with the static
  data $\sigma$ in the active store associated with $S$.
\item[$\Fequal : \DState \x \DState$.]_{equal$\Fequal$} Given two states
  $S_1, S_2$, $\Fequal(S_1, S_2)$ checks whether they are equal and whether the
  static data reachable from them is equal.
\end{description}
During specialization of a function, \emph{two} sets of specialization
points are maintained: the set of \emph{pending}
 points_{set!specialization point!pending} and the set of
\emph{processed} points_{set!specialization point!processed}. Whenever a
specialization point is to be added to the pending set, we check whether it
has already been processed, and if so, simply return the residual label
associated with that specialization point_{memoization}. This way, residual 
code sharing_{sharing!code} is obtained.

There are some  opportunities for ``peephole''
optimizations_{optimizations!peephole}: 
\begin{itemize}
\item The state consists of local (+parameter) variables and global
  variables. In functions that perform no non-local side 
  effects_{side effects}, we do not need to compare the global
  variables. This can 
  be implemented by passing an extra $\VsideEffects$ parameter to
  $\FpendInsert$.
\item Consider specializing a dynamic "if" statement, cf.\ 
  Figure~\vref{fig:GEGENControlStatements}. If the current state has not
  been seen before, the following happens:
  \begin{enumerate}
  \item The current state + label $l_1$ and the current state + label $l_2$ 
    is added to the pending set, involving two copyings of the store
  \item In the pending loop, a state is removed from the pending set,
    involving a restoring of the store
  \end{enumerate}
  By observing that the static store is unchanged between these two
  operations, we can skip the restoring in those cases where we remove a
  specialization point we have just added. As store copying is one of the
  most time-consuming operations during specialization, this can
  potentially lead to a visible reduction of specialization time.
  
  This can be implemented by keeping a set of recently added specialization
  points.
\end{itemize}
The function for inserting specialization points into the pending set
is shown in Figure~\ref{fig:SLFpendInsert}.%
\begin{figure}[htb]
\begin{center}\leavevmode\hbox{\vbox{%
\begin{pseudocode}
  $\FpendInsert(\VpgenLabel, \Vstate, \VsideEffects) = {}$ \+\\
    $\VpendSet := \Ftop(\VpendSetStack)$ \\
    $\Kif\ \VsideEffects\ \Kthen$ \+\\
      $\Kfor\ (p,S,l) \in \VpendSet\ \Kdo$ 
      \> \hskip 15em $/*$ check the pending specialization points $*/$ \+\\
        $\Kif\ p = \VpgenLabel \land \Fequal(S, \Vstate)\ \Kthen\
          \Kreturn\ l$ \-\\
      $\Kfor\ (p,S,l) \in \VprocessedSet\ \Kdo$ 
      \> \hskip 15em $/*$ check the processed specialization points $*/$ \+\\
        $\Kif\ p = \VpgenLabel \land \Fequal(S, \Vstate)\ \Kthen\
          \Kreturn\ l$ \-\-\\
    $\Kelse$ \+\\
      $\Kfor\ (p,S,l) \in \VpendSet\ \Kdo$ 
      \> \hskip 15em $/*$ check the pending specialization points $*/$ \+\\
        $\Kif\ p = \VpgenLabel \land \FlocalEqual(S, \Vstate)\ \Kthen\
          \Kreturn\ l$ \-\\
      $\Kfor\ (p,S,l) \in \VprocessedSet\ \Kdo$ 
      \> \hskip 15em $/*$ check the processed specialization points $*/$ \+\\
        $\Kif\ p = \VpgenLabel \land \FlocalEqual(S, \Vstate)\ \Kthen\
          \Kreturn\ l$ \-\-\\[1ex]
    $\VresLabel := \FfreshResidLabel()$ 
    \>\> \hskip 15em $/*$ the specialization point has not been seen before $*/$ \\
    $\Kif\ \VsideEffects\ \Kthen$ \+\\
      $\Vsp := (\VpgenLabel, \Fcopy(\Vstate), \VresLabel)$ \-\\
    $\Kelse$ \+\\
      $\Vsp := (\VpgenLabel, \FlocalCopy(\Vstate), \VresLabel)$ \-\\
    $\Fadd(\VpendSet, \Vsp)$ \\
    $\Fadd(\VrecentSet, \Vsp)$ \\
    $\Kreturn\ \VresLabel$
\end{pseudocode}}}
    \caption{Specialization library function for inserting specialization points into the pending set }
    \label{fig:SLFpendInsert}
  \end{center}
\end{figure}

Functions $\Fpending$ and $\FpopSP$ are in fact always used together (in
the pending loop_{loop!pending} of each generating
function_{function!generating}, cf.\
Figure~\vref{fig:GEGENFunctionsDynamic}), so
they can be merged into one, that also
takes care of restoring the state and emitting the residual
label_{label!residual}, cf.\ Figure~\ref{fig:SLFpendingSP}.
\begin{figure}[htb]
\begin{center}\leavevmode\hbox{\vbox{%
\begin{pseudocode}
  $\FpendingSP() = {}$ \+\\
    $\Kif\ \Ftop(\VpendSetStack) = \{\}\ \Kthen\ \Kreturn\ \CNull$ \\
    $\Kif\ \VrecentSet \ne \{\}\ \Kthen$ \+\\
      $\Vsp := \Fremove(\VrecentSet)$ \\                          
      $(p,S,l) := \Fremove(\Vsp, \Ftop(\VpendSetStack))$ \-\\
    $\Kelse$ \+\\
      $(p,S,l) := \Fremove(\Ftop(\VpendSetStack))$ \\
      $\Frestore(S)$ \-\\
    $\VrecentSet := \{\}$ \\
    $\Fadd(\VprocessedSet, (p,S,l))$ \\
    $\emit{\bqt{l}":"}$ \\
    $\Kreturn\ p$
\end{pseudocode}}}
    \caption{Specialization library function for finding a
      specialization point in the pending set}
    \label{fig:SLFpendingSP}
  \end{center}
\end{figure}


\subsubsection{Sharing residual functions}

Another form of code sharing in the residual program is obtained by
sharing residual functions_{sharing!function} across several call
sites.

Due to non-local side effects_{side effect!non-local} during
specialization, sharing of
residual functions becomes complicated. Suppose we have
specialized a function $f$ to the input state $s$ and the end state
after $f$ is $e$. If we later during specialization encounter a call
to $f$ with the same input store $s$, then we must generate a call to
the 
specialized function and restore the end 
configuration_{end configuration!function} $e$ to mimic
the effect of the static non-local side-effects in $f$.

\paragraph{Functions Allocating Heap Storage.}_{allocation!heap}
Recall that restoring a
copied store means copying the data back to their original location in
the store. This makes sharing of a function that allocates heap
storage almost impossible, because the end configuration may refer to
some memory allocations that are unavailable when a new call to the
function is encountered.

However, if the code subsequent to the allocation does not depend on
the {\em location} of the allocated store, then restoring of a store
is allowed to copy the heap allocated objects to a new location, so
functions that allocate heap storage can be shared. Whether the code
may depend on the location of an object can be determined by a special
in-use analysis.

Since we are not implementing such an analysis at this stage we will
mark functions that allocate heap storage as unsharable.

\paragraph{Functions Deallocating Heap Storage.}_{deallocation!heap} A
function that
deallocates heap storage can be shared provided that the memoization
algorithm also keeps a record of which objects are deallocated.

Since calls to functions that may refer to a heap-allocated object can
only be shared if the object is allocated at the same location, it is
unlikely that two calls may be shared if the object has been
deallocated and allocated in between the calls.

Since functions that deallocate heap storage are almost always
unsharable, practically nothing is lost by marking all functions that
deallocate heap storage as unsharable, so we have chosen to do this.

\paragraph{Recursive Functions.}_{function!recursive} A function with
a non-local side
effect can be shared even if it is called recursively, as long as we
only allow sharing {\em after} it has been specialized, since the end
configuration is unknown until specialization of the function has been
completed.

Allowing functions to call themselves recursively without memoizing
the calls involves a risk of non-termination, but rather than
automatically suspending all non-local side effects in recursive
functions, we make them static unless the user specifies otherwise.

Another possibility is to compute the end configuration as a fix point
of the recursion, but this complicates the specialization process
considerably~\cite{Bulyonkov:1996:PracticalAspects}.


\paragraph{Implementation.} The changes required to implement function 
sharing amounts to adding a library function call to each generating
function, i.e.\ by augmenting Figure~\ref{fig:GEGENFunctionsDynamic}
as shown in
Figure~\vref{fig:SLFFunctionSharingGegen},
and defining  library functions
$\FfunctionSeenBefore$_{functionSeenBefore@$\FfunctionSeenBefore$},
$\FpushFun$_{pushFun@$\FpushFun$}
and $\FpopFun$_{popFun@$\FpopFun$} to handle function memoization, cf.\
Figure~\vref{fig:SLFFunctionSharingLibrary}. The function
$\FnonLocalCopy$_{nonLocalCopy@$\FnonLocalCopy$}_{copy!non-local}
copies exactly those non-local objects that are reachable
(via pointers etc.) from the global \emph{and} local part of the
state_{state!local}_{state!global}.

\subsection{Utilizing in-use information}
\index{in-use analysis!using results from}
\label{sec:SLFUsingInUseInformation}

Consider the following program fragment where "x", "y" and "n" are
static, "z" and "total" dynamic:
\begin{verbatim}
total = (x + y) * (z - x);           /* basic block 1 */
if (total >= 0)
  printf("%d: profit!  x=%d", n, x); /* basic block 2 */
else
  printf("%d: deficit!", n);         /* basic block 3 */

return;                              /* basic block 4 */
\end{verbatim}
If, during specialization, "x" and "y" can take many different static
values, basic block 2 will result in many identical specialized versions
(one for each value of "y"), and basic block 3 will result in an identical
specialized version for each $("x","y")$, increasing both specialization
time and residual program size.  The problem is that we compare \emph{all}
the static values, even though their values cannot influence the following
specialization.

We reduce this problem by utilizing the in-use information computed in
Section~\ref{sec:In-UseAnalysis},  only comparing values for variables
that are actually in use.

After the split phase described in Section~\ref{sec:PartiallyStaticData}
has completed, we can give each static variable and parameter declaration
in the program a unique identification number, its $\VID$.
Whenever a call to $\FpendInsert$ is made, an extra parameter is passed,
containing (a representation of the $\VID$'s of) those variables that are
actually in use at the present program point. This parameter is then passed 
on when comparing or copying the static store.

We can represent the set of variables in use as a bit string $s$,
where $s[\VID] = $`1' if variable $v_{\VID}$ is in use, $s[\VID] = $`0'
otherwise. This is probably not a space-efficient
representation\footnote{If it is a problem in practice, the string can be
  compressed by a factor 4 (or more) by encoding it as hexadecimal digits},
but it is a simple way of hardcoding the in-use information into the
generating extension as string constants.

Assuming $\VID_{\mathtt{x}} = 0$, $\VID_{\mathtt{y}} = 1$ and
$\VID_{\mathtt{n}} = 2$, the above example would now translate into the
following:
\begin{center}\def\d{\(\sb{d}\)}\def\s{\(\sb{s}\)}%
\leavevmode\small
\hbox to .44\textwidth{\hss$p$\hss\hss}\hfil
\hbox to .55\textwidth{\hss$\Ppgen$\hss\hss}\nopagebreak[4]\smallskip
\nopagebreak[4]

\begin{minipage}[t]{.44\textwidth}\normalsize
\begin{alltt}
total = (x + y) * (z - x);
if (total >= 0)
  printf("%d: profit!  x=%d", n, x);
else 
  printf("%d: deficit!", n); 




return;
\end{alltt}  
\end{minipage}\hfil
\begin{minipage}[t]{.55\textwidth}\def\arraystretch{1.08}
\begin{array}[t]{rl}
l_1: & \emit{\bqt{\Vtotal}" = "\bqt{\Flift(x + y)}" * ("
       \bqt{z}" - "\bqt{\Flift(x)}");"} \\
     & \emitstmtbegin"if ("\bqt{\Vtotal}" >= 0)" \\
     & \quad "goto "\bqt{[l_2,\mbox{``101''}]}";" \\
     & \ "else" \\
     & \quad "goto "\bqt{[l_3,\mbox{``001''}]}";"\emitstmtend;\
       \Kgoto\ \LpendLoop \\
l_2: & \emitstmtbegin\verb+printf("%d: profit! x=%d", +
       \bqt{\Flift(n)}", "\bqt{\Flift(x)}");"\emitstmtend;\\
     & \emitstmtbegin"goto "\bqt{[l_4,
       \mbox{``000''}]}";"\emitstmtend;\
       \Kgoto\ \LpendLoop\\
l_3: & \emitstmtbegin\verb+printf("%d: deficit!", +
       \bqt{\Flift(n)}");"\emitstmtend;\\
     & \emitstmtbegin"goto "\bqt{[l_4,
       \mbox{``000''}]}";"\emitstmtend;\
       \Kgoto\ \LpendLoop\\
l_4: & ...
\end{array}
\end{minipage}\hfil
\end{center}



\paragraph{The memory management functions and data structures}
must be augmented to accommodate and check the in-use parameter before copying as shown in
Figure~\ref{fig:SLFMemoryManagementAdditions}:%
\begin{figure}[htbp]
\begin{center}\leavevmode\hbox{\vbox{%
\begin{pseudocode}
  $\TDataObjectCopy\ {*}
    \FcopySimple(\TDataObject\ {*}\Vobj,\ \TIUInfo\ \VinUse)$ \+\\
    $\Kif\ \Fnot(\FisInUse(\FID(\Vobj), \VinUse))\ \Kthen\ \Kreturn\ \CNull$ \\
      $...$ \-
  \\[1em]
  $\Tbool\ \FcmpSimple(\TDataObject\ *\Vobj,\
                       \TDataObjectCopy *\Vcpy,\ \TIUInfo\ \VinUse)$ \+\\
    $\Kif\ \Fnot(\FisInUse(\FID(\Vobj), \VinUse))\ \Kthen\ \Kreturn\ \Ctrue$ \\
    $...$ \-
  \\[1em]
  $\TDataObjectCopy\ {*}
    \FcopyWithOffset(\TDataObject\ {*}\Vobj,\ \Tvoid\ {*}\Vref,\
                     \TIUInfo\ \VinUse)$ \+\\
    $\Kif\ \Fnot(\FisInUse(\FID(\Vobj), \VinUse))\ \Kthen\ \Kreturn\ \CNull$ \\
      $...$ \-
  \\[1em]
  $\Tbool\ \FcmpWithOffset(\TDataObject\ *\Vobj,\
                       \TDataObjectCopy *\Vcpy, \
                       \Tvoid\ {*}\Vref,\ \TIUInfo\ \VinUse)$ \+\\
    $\Kif\ \Fnot(\FisInUse(\FID(\Vobj), \VinUse))\ \Kthen\ \Kreturn\ \Ctrue$ \\
    $...$ \-
\end{pseudocode}}}
    \caption{Additions to memory management functions to make use of in-use 
      information}
    \label{fig:SLFMemoryManagementAdditions}
  \end{center}
\end{figure}
just before copying or comparing an object, we check whether the
object is in use by using the object's ID to index into the in-use
string: type $\TIUInfo$ is a synonym for $\Tchar*$ and $\FisInUse :
\Tint \x \TIUInfo -> \Tbool$_{isInUse@$\FisInUse$} is defined by
$\FisInUse(\Tint\ \VID, \Tchar\ *\VinUse) = (\VinUse[\VID] =
{}$`1'$)$. If the object turns out not to be in use, we return
immediately; any other objects it might point to need \emph{not} be
copied due to this object. If some other object in use points to these
objects, \emph{they} will take care of the copying: If some object is
in use, it must either be a local or global variable which is in use,
or be reachable via local or global pointers that are in use.

Quite  similar additions apply to the copy and compare
functions_{function!copy!type specific}%
_{function!compare!type specific} for structs and arrays generated by
gegen in
Figures~\ref{fig:SLFUserTypesCopyFun}--\ref{fig:SLFArrayCopyCmpFun}.

The library functions for handling residual function sharing must be
augmented so that  in-use information is passed along when comparing and
copying the state, cf.\ Figure~\ref{fig:SLFFunctionSharingLibrary}.%
\begin{figure}[hbtp]
\begin{center}\leavevmode\hbox{\vbox{%
\begin{pseudocode}
  $\FfunctionSeenBefore(\Vid, \Vstate, \VinUse, \VtmpRetVal) = {}$ \+\\
    $\Kfor\ (\Vid_f, \Vstate_f, \VendState_f, \VtmpRetVal_f, f_{\Vresid}) 
             \in \VseenBefore\ \Kdo$ \+\\
      $\Kif\ \Vid_f = \Vid \mathrel{\land} 
             \Fequal(\Vstate_f, \Vstate, \VinUse)\ \Kthen$ \+\\
        $\Frestore(\VendState_f)$ \\
        $\VretVal := \VtmpRetVal_f$ \\
        $\Kreturn\ f_{\Vresid}$ \-\-\\
    $\Kreturn\ \CNoName$ \-
  \\[1em]
  $\FpushFun(\Vid, \Vstate, \VinUse) = {}$ \+\\
    $f_{\Vresid} := \FfreshResidFun(\Vid)$ \\
    $\Fpush(\VfunStack, (\Vid, f_{\Vresid}, \VemptyBody))$ \\
    $\Kif\ \Fsharable(\VcurrentFun)\ \Kthen$ \+\\
      $\VcurrentFun := (\Vid, \Fcopy(\Vstate, \VinUse), 
                        \CNull, \CNull, f_{\Vresid})$ \\
      $\Fadd(\VseenBefore, \VcurrentFun)$ \-\\
    $\Kelse$ \+\\
      $\VcurrentFun := (\Vid, \CNull, 
                        \CNull, \CNull, f_{\Vresid})$ \-\\
    $\Kreturn\ f_{\Vresid}$ \-
  \\[1em]
  $\FpopFun(\Vstate, \VtmpRetVal) = {}$ \+\\
    $(\Vid_f, \Vstate_f, \VendState_f, \VtmpRetVal_f, f_{\Vresid})
     == \VcurrentFun$ \\
    $\Kif\ \Fsharable(\VcurrentFun)\ \Kthen$ \+\\
      $\VendState_f := \FnonLocalCopy(\Vstate)$ \\
      $\VtmpRetVal_f := \VtmpRetVal$ \-\\
    $\Kelse$ \+\\
      $\Ffree(\VtmpRetVal)$ \-\\
    $\Kreturn\ f_{\Vresid}$
\end{pseudocode}}}
    \caption{Specialization library functions supporting
      residual function sharing and in-use information} 
    \label{fig:SLFFunctionSharingLibrary}
  \end{center}
\end{figure}
Note that in-use information is \emph{not} needed at
function return, because we do not know at gegen time what objects are 
in use at the call site that has invoked the function; all we can do
is to reduce the copying to copy only non-local objects.

\paragraph{The gegen transformations}
for dynamic "if" and "goto"  in
Figure~\ref{fig:GEGENControlStatements} must be augmented thus:_{IU@$\OIU$}
\[
\begin{array}{lcl}
      \gegen{\CIf_d(e,l_1,l_2)} & 
      =& \emit{"if ("\bqt{\gegen{e}}") goto "
         \bqt{[l_1, \widetilde{\textsl{iuCst}_{l_1}}]}
        "; else goto "\bqt{[l_2, \widetilde{\textsl{iuCst}_{l_2}}]}";"};\ 
        \hfill\Kgoto\ \LpendLoop \mathpunct{}
      \\[1.2ex]
      \gegen{\CGoto_d(l)} &
      =& \emit{"goto "\bqt{[l, \widetilde{\textsl{iuCst}_l}]}";"};\ 
        \hfill\Kgoto\ \LpendLoop\mathpunct{\rlap{,}}
  \\[1em]
  \mc{3}{l}{\textnormal{where $\widetilde{\textsl{iuCst}_l} =
      \FtoStr(\OIU(b_l))$ is a constant computed at gegen time}}
\end{array}
\]
and the gegen transformation for dynamic functions must be augmented as
shown in Figure~\ref{fig:SLFFunctionSharingGegen}_{<.>@$\gegen{\cdot}$}.
\begin{figure}[hbt]
\begin{center}\leavevmode\hbox{\vbox{%
\begin{pseudocode}
  \>$\quad\vdots$ \+\\
  $\Kif\ \VcallType = \CMem\ \Kthen$ \+\\
    $f_{\Vresid} := 
     \FfunctionSeenBefore(\Void, \VactiveState,
     \widetilde{\textsl{iuCst}}, \VtmpRetVal)$ \\
    $\Kif\ f_{\Vresid} \ne \CNoName\ \Kthen$ \+\\
      $\FassignRetVal(\VretVal_s, \VtmpRetVal, \Fsizeof(t_s))$ \\
      $\FpopObjects(q)$ \\
      $\Kreturn\ f_{\Vresid}$ \-\\
    $\Kelse$ \+\\
      $f_{\Vresid} := 
      \FpushFun(\Void, \VactiveState, \widetilde{\textsl{iuCst}})$ \\
      $\emitstmtbegin\bqt{t_d\ f_{\Vresid}}"("
           \bqt{\gegen{d_1}_{\Vpini}}", " ... ", "
           \bqt{\gegen{d_n}_{\Vpini}}
           "){"\emitstmtend$ \-\-\\
  $\Kelse$ \+\\
    $\vdots$ \-\-\\[1em]
  where $\widetilde{\textsl{iuCst}} = \FtoStr(\OIU(b_1))$ is a
  constant computed
  at gegen time 
\end{pseudocode}}}
    \caption{Additions to gegen for supporting residual function
      sharing and in-use information}
    \label{fig:SLFFunctionSharingGegen}
  \end{center}
\end{figure}

\subsection{Code generation}
\label{sec:SLFCodeGeneration}
\index{[.]@$\emit{\cdot}$|(} 

The overloaded code generating function, $\emit{\cdot}$, used in the
generating extension is simply implemented by various "cmix"$x$ library
functions: "cmixIf"_{cmixIf@"cmixIf"}, "cmixGoto"_{cmixGoto@"cmixGoto"},
"cmixAssign"_{cmixAssign@"cmixAssign"}, "cmixExpr"_{cmixExpr@"cmixExpr"},
\ldots

\index{[.]@$\emit{\cdot}$|)}

\subsubsection{Lift functions}
\label{sec:SLFLiftFunctions}
\index{lifting}

During specialization, constant values need to be exported to the residual
program---this is called \emph{lifting} static values into $\Ppres$.  The
(type overloaded) library function for lifting simple types is called
"cmixLift"_{cmixLift@"cmixLift"}. For function pointers, we need a special
``dereference-and-lift'' function: given a \emph{pointer} we want the
residual representation (name) of a \emph{function}. This is done by the
library function $\FliftFP$, which is passed a list of pointers to all the
functions in the generating extension, so that it can compare a pointer
that must be dereferenced-and-lifted:
\begin{center}\leavevmode\hbox{\vbox{%
\begin{pseudocode}
  $\TCode\ \FliftFP(\Tstruct\ \{\ \Tvoid\ {*}f;\ \TCode\ \Vname\ \}\
   {*}\Vfcns,\ \Tvoid\ {*}f)\ \{$ \+\\
    $\quad \Kreturn\ \Flookup(\Vfcns,\ f)$ \-\\
  $\}$
\end{pseudocode}}}
\end{center}


\subsubsection{Namespace Management}
\label{sec:SLFNamespaceManagement}
\index{namespace!$\Ppres$}

$\Ppgen$'s namespace manager for $\Ppres$ identifiers is quite similar to
gegen's (cf.\ Section~\vref{sec:GEGENNamespaceManagement}). 


\subsection{Miscellaneous helper functions}
\label{sec:SLFMiscellaneous}

The functions for handling the static part of return
values_{value!return!static} (cf.\ Figures~\ref{fig:GEGENControlStatements}
and~\ref{fig:GEGENFunctionsDynamic}) are defined
by_{setRetVal@$\FsetRetVal$}_{assignRetVal@$\FassignRetVal$}:
\begin{center}\leavevmode\hbox{\vbox{%
  \begin{pseudocode}
    $\FsetRetVal(\Tvoid\ {*}\Vval,\ \Tvoid\ {*}\VtmpRetVal,\ 
                 \Tint\ \Vsize) = {}$
    \+\\
      $\Kif\ \VtmpRetVal \ne \CNull\ \Kthen$ \+\\
        $\Kif\ \Fmemcmp(\VtmpRetVal,\ \Vval,\ \Vsize) \ne 0\ \Kthen$ \+\\
           $\Kerror($``Two different function 
                       end configurations!''$)$ \-\-\\
      $\Kelse$ \+\\
        $\VtmpRetVal := \Fmalloc(\Vsize)$ \\
        $\Fmemcpy(\VtmpRetVal,\ \Vval,\ \Vsize)$ \-\-
    \\[1em]
    $\FassignRetVal(\Tvoid\ {*}\VretVal,\ \Tvoid\ {*}\VtmpRetVal,\ 
                 \Tint\ \Vsize) = {}$
    \+\\
      $\Kif\ \VtmpRetVal \ne \CNull\ \Kthen$ \+\\
        $\Fmemcpy(\VtmpRetVal,\ \Vval,\ \Vsize)$ \\
  \end{pseudocode}}}
\end{center}



\subsection{Implementation level (1999-03-16)}
\label{sec:SLFImplementationLevel}

This chapter has not been updated in a year and is wildly incorrect.

\end{docpart}

%%% Local Variables: 
%%% mode: latex
%%% TeX-master: "cmixII"
%%% End: 

% File: output.tex
% Time-stamp: <98/03/04 16:02:37 panic>
% $Id: output.tex,v 1.3 1999/04/21 13:12:12 jpsecher Exp $

\providecommand{\docpart}{\renewenvironment{docpart}{}{}
\end{docpart}
\documentclass[twoside]{cmixdoc}
%\bibliographystyle{apacite}

\makeatletter
\@ifundefined{@title}{\title{\cmix-documentation}}{}
\@ifundefined{@author}{\author{The \cmix{} Team}}%
{\expandafter\def\expandafter\@realauthor\expandafter{\@author}%
\author{The \cmix{} Team\\(\@realauthor)}}
\makeatother

\AtBeginDocument{%
\markboth{\hfill\today\quad\timenow\hfill\llap{\cmix\ documentation}}
{\hfill\today\quad\timenow\hfill}}

\renewcommand{\sectionmark}[1]{\markboth
{\hfill\today\quad\timenow\hfill\llap{\cmix\ documentation}}
{\rlap{\thesection. #1}\hfill\today\quad\timenow\hfill}}

%\newboolean{separate}
%\setboolean{separate}{true}

\renewenvironment{docpart}{\begin{document}}%
                          {\bibliography{cmixII}\printindex
                           \end{document}}
\begin{document}\shortindexingon
}
\begin{docpart}

\begin{center}
  \fbox{\huge\textsl{THIS SECTION IS OUT-OF-DATE}}    
\end{center}

\section{Representing output}
\label{sec:output}
The \verb|Output| class is an abstract representation of a piece of
data and a number of annotations on this data. For instance, a piece
of data could be a program, and the annotations could be results of
various analyses on this program. 

The purpose of this representation is to be able to represent a piece
of data together with all its annotations, and thus postpone the
decision of which annotations will be relevant. For instance, a
hypertext browser can view a program together with information about
what objects each declaration can point to; or a program can be
annotated with colours that represent the binding-times of each
construct in the program. In the case of points-to information, it is
highly desirable for an annotation to be able to refer back to the
main piece of data. This is illustrated in
figure~\ref{fig:pgm-pa-bta}.

\begin{figure}[htbp]
  \begin{center}
    \leavevmode
    \epsfig{file=pgm-pa-bta.eps,width=0.7\textwidth}
    \caption{A program with two sets of annotations.}
    \label{fig:pgm-pa-bta}
  \end{center}
\end{figure}

To be able to select a particular annotation set (and thus ignore
other annotation sets), it must be a requirement that annotations only
refer to the main piece of data and/or other annotations in the same
annotation set.

The selection of relevant annotations and how these are visualised is
carried out by a \emph{filter} or \emph{viewer}, see
section~\ref{sec:filter}.

\subsection{The output classes}
What we want to represent is essentially annotated text, so an
abstract output representation must have the following
characteristics.

\begin{itemize}
\item Data consists of a sequence of smaller pieces of information.
\item A piece of data can be a string of text or a newline.
\item Several pieces of data can be grouped together and their
  (visual) relationship can be expressed.
\item Each piece of data can have a number of annotations attached.
  Each such annotation has a specific type.
\item All annotations associated with a piece of data must have
  different type.
\item Each annotation consists of smaller pieces of data. Such pieces
  can refer to portions of the main data and/or to other annotations
  of the same type.
\end{itemize}

Hence it is natural to represent output as trees, where the leaves
are text string, and the interior nodes nodes express a relationship
between subtrees. It would also be handy to be able to annotate text
strings with attributes such as bold, italics, \etc  

Entities of the \texttt{Output} class can be categorised as follows.

\begin{itemize}
\item A block is a sequence of output trees and It can optionally be
  indented by a given amount. If the output interpreter need to break
  lines inside a block, it can be done in two ways: If the block is
  \emph{consistent}, any shortage of line space will force all
  elements of the block to appear on a separate line. If the block is
  \emph{inconsistent}, line breaks are only inserted to avoid over-full
  lines.
\item A break code can be put inside a block, which means that the
  output interpreter can insert space or break an over-full line here.
  If the whole block fits on the line, each break code is replaced by
  the specified number of spaces.
  
  If the enclosing block does not fit on the line and the block is
  \begin{description}
  \item [consistent,] each break code is replaced by a newline and
    each line is indented to the indentation level plus the block
    offset. [Maybe this is not quite right...]
  \item [inconsistent,] if the element fits on the current line, the
    specified number of spaces is output. If not, a newline is issued
    and the new line is indented to the current indentation level plus
    the block offset plus the break offset.
  \end{description}
\item A text is a character string with an attribute
  (see~\ref{sec:textattribute}).
\item A linked output is a piece of output that is associated to
  a list of anchors (see~\ref{sec:anchors}).
\item An anchored output is a piece of output that has an unique
  anchor attached to it.
\item A hard newline is a piece of output.
\end{itemize}

  \begin{verbatim}
      struct Output {
        enum Mark { Block,Break,HardNL,Text,Anno,Label };
        enum BlockType { Consistent, Inconsistent };

        Output(list<Output*>*, unsigned = 0, BlockType = Consistent); // Block.
        Output(unsigned offset, unsigned spaces); // Break.
        Output(const char*, TextAttribute*); // Text.
        Output(Output*, list<Anchor*>*); // Associate a list of anchors
                                         // with the subtree.
        Output(Anchor*,Output*); // Attach an existing anchor to a tree.
        Output(); // HardNL

        // Lookup
        BlockType type();
        unsigned level();
        list<Output*>* blocks();
        unsigned break_offset();
        const char* text();
        TextAttribute* attribute();
        Output* anno_subtree();
        list<Anchor*>* anno_anchors();
        Anchor* label();
        Output* label_subtree();

        friend Output* oconcat(Output*,Output*);
        friend Output* oconcat(list<Output*>*);
      };
      \end{verbatim}

\subsubsection{Output types}
\label{sec:outputtypes}
Output is collected in an \texttt{OutputContainer} class.  As
described earlier, it should be possible to group output into several
groups, depending on the nature and/or origin of the output. This is
done by demanding that each piece of output appended to an
\texttt{OutputContainer} must be accompanied by an \emph{output type}.
Such a type consists of a symbolic name and a parent identification
such that the types are organized in a tree. The output type tree for
figure~\ref{fig:pgm-pa-bta} could be

\[
\xymatrix{
  & Program \ar[ld] \ar[rd] & & \\
  BT\!A \ar[d] \ar[rd] & & P\!A \\
  Static & Dynamic & \\
}
\]

\noindent
Notice that the output type $BT\!A$ is used solely to group the two
different binding-times under. As stated earlier, the purpose of this
is to be able to select a certain \emph{view} of the program: It
should be possible to instruct the output formatter only to consider
(and thus view) a portion of the output. Such a portion must be a
substree from the root to some depth in the output type tree. This
restriction means that a piece of output only can refer to other
pieces of output of the same type or of ancestor type. Reference
points are called \emph{anchors}.

\begin{verbatim}
struct OType {
    OType(const char*, OType*);
    const OType* parent();
    const list<OType*>& children();
    const char* name();
    unsigned number();
    friend bool operator==(OType&,OType&);
};

\end{verbatim}

\subsubsection{Anchors}
\label{sec:anchors}
An anchors is thus a unique label in some piece of output and a type
that tells us which group of output we should search for this
label. The associated type can also be used to ``trim'' a set of
outputs when a certain view has been decided upon.

\begin{verbatim}
struct Anchor {
    Anchor(OType*);
    OType* isIn();
    unsigned anchor();
};
\end{verbatim}

\subsubsection{Text Attributes}
\label{sec:textattribute}
A text attribute is simply a symbolic name that can be used to make
distinctions between pieces of text, \eg variable names, keywords,
\etc

\begin{verbatim}
struct TextAttribute {
    TextAttribute(const char*, unsigned);
};
\end{verbatim}

\subsubsection{Output containers}
\label{sec:outputcontainer}
An output container holds a set of typed output trees and can export
such a set to a linear \textsc{ascii} format, \eg a file. To create an
output container, one needs to specify (the root of) the output type
tree that the container can work on.

\begin{verbatim}
struct OutputContainer {
    OutputContainer(OType*);
    void add(Output*,OType*);
    friend ostream& operator<<(ostream&,OutputContainer*);
};
\end{verbatim}

\subsection{Filters}
\label{sec:filter}
A filter is something that interprets the contents of an output
container. One could imagine an interactive tool that lets the user
select a type subtree, \ie which annotations she wants to see, and
possibly select the appearance of text by mapping text attributes to
specific fonts; or map subtypes to specific colours.

\end{docpart}
%%% Local Variables: 
%%% mode: latex
%%% TeX-master: "cmixII"
%%% End: 



\appendix

% $Id: infra.tex,v 1.3 1999/04/21 13:12:10 jpsecher Exp $

\providecommand{\docpart}{\renewenvironment{docpart}{}{}
\end{docpart}
\documentclass[twoside]{cmixdoc}
%\bibliographystyle{apacite}

\makeatletter
\@ifundefined{@title}{\title{\cmix-documentation}}{}
\@ifundefined{@author}{\author{The \cmix{} Team}}%
{\expandafter\def\expandafter\@realauthor\expandafter{\@author}%
\author{The \cmix{} Team\\(\@realauthor)}}
\makeatother

\AtBeginDocument{%
\markboth{\hfill\today\quad\timenow\hfill\llap{\cmix\ documentation}}
{\hfill\today\quad\timenow\hfill}}

\renewcommand{\sectionmark}[1]{\markboth
{\hfill\today\quad\timenow\hfill\llap{\cmix\ documentation}}
{\rlap{\thesection. #1}\hfill\today\quad\timenow\hfill}}

%\newboolean{separate}
%\setboolean{separate}{true}

\renewenvironment{docpart}{\begin{document}}%
                          {\bibliography{cmixII}\printindex
                           \end{document}}
\begin{document}\shortindexingon
}
\begin{docpart}

\begin{center}
  \fbox{\huge\textsl{THIS SECTION IS OUT-OF-DATE}}    
\end{center}

\section{Boring infrastructure details}
\label{sec:Infrastructure}

This section describes a number of infrastructure details that work
behind the scenes and support the analyses.

\subsection{Command line and option processing}

\subsection{Specializer directive parser}

\subsection{Connection between C and Core C representations}

\subsection{Source code positions}

\subsection{Error and Warning messages}

\subsection{Lists and sets}

One of these days some better documentation than the following
will be written.

Traversing a list is very close to running through an array. If you
\eg want to traverse a structure \syntax{decls} of type
\syntax{list<C\_Decl*>*}, you can do it by using an \emph{iterator} in
a simple \syntax{for}-loop:
\begin{verbatim}
    for(list<C_Decl*>::iterator i = decl->begin(); i++; i!=decl->end())
    {
         (*i)->get_name() ...
    }
\end{verbatim}

A macro is defined in \verb|auxilary.h| which shorens the above to
\begin{verbatim}
    FOREACH(i, decl, C_Decl*)
    {
         (*i)->get_name() ...
    }
\end{verbatim}

Observe that an iterator works as a pointer to an element of an array.
There are several operations on lists, whereof a few are listed in
figure~\ref{fig:listops}.

\begin{figure}[htbp]
  \begin{frameit}
    \leavevmode
    A list is parameterized over a type $T$.

    \noindent{\small
    \begin{tabular}{ll}
    \syntax{iterator  begin();}
      & Iterator to the first element. \\
    \syntax{iterator  end();}
      & Iterator to the last element. \\
    \syntax{unsigned size();} & \\
    \syntax{bool      empty();} & \\
    \syntax{reference front();} 
      & Reference to the first element. \\
    \syntax{reference back();}
      & Reference to the last element. \\
    \syntax{void      push\_front($T$*);}
      & Insert as first element. \\
    \syntax{void      push\_back($T$*);}
      & Insert as last element. \\
    \syntax{iterator  insert(iterator p, $T$*);}
      & Insert an element before position p. \\
    \syntax{void      splice(iterator p, list<$T$>\& x);}
      & Insert list x before position p. \\
    \end{tabular}}
  \caption{Some members of the \syntax{Plist} class.}
  \label{fig:listops}
  \end{frameit}
\end{figure}

\subsection{Associative arrays}

\subsection{Inverse Sets}
Inverse sets (or complement sets) are implemented by a regular set and
a boolean value that tells whether the set is inversed or not. The
union, intersection and set difference can be calculated as follows:
\[
\begin{array}{rclrclrcl}
  & \bigcup & & & \bigcap & & & \setminus \\
  \complement X \cup \complement Y &=& \complement(X \cap Y)
  & \complement X \cap \complement Y &=& \complement(X \cup Y) 
  & \complement X \setminus \complement Y &=& Y \setminus X \\
  \complement X \cup Y &=& \complement(X \setminus Y)
  & \complement X \cap Y &=& Y \setminus X 
  & \complement X \setminus Y &=& \complement(X \cup Y) \\
  X \cup \complement Y &=& \complement(Y \setminus X)
  & X \cap \complement Y &=& X \setminus Y 
  & X \setminus \complement Y &=& X \cap Y \\
\end{array}
\]
Inverse sets are used in the dataflow analysis.

\end{docpart}
%%% Local Variables: 
%%% mode: latex
%%% TeX-master: "cmixII"
%%% End: 

% Edit Mode: -*- LaTeX -*-
% File: end.tex
% Time-stamp: <98/06/08 12:33:53 panic>
% $Id: end.tex,v 1.1.1.1 1999/02/22 13:50:28 makholm Exp $

\providecommand{\docpart}{\renewenvironment{docpart}{}{}
\end{docpart}
\documentclass[twoside]{cmixdoc}
%\bibliographystyle{apacite}

\makeatletter
\@ifundefined{@title}{\title{\cmix-documentation}}{}
\@ifundefined{@author}{\author{The \cmix{} Team}}%
{\expandafter\def\expandafter\@realauthor\expandafter{\@author}%
\author{The \cmix{} Team\\(\@realauthor)}}
\makeatother

\AtBeginDocument{%
\markboth{\hfill\today\quad\timenow\hfill\llap{\cmix\ documentation}}
{\hfill\today\quad\timenow\hfill}}

\renewcommand{\sectionmark}[1]{\markboth
{\hfill\today\quad\timenow\hfill\llap{\cmix\ documentation}}
{\rlap{\thesection. #1}\hfill\today\quad\timenow\hfill}}

%\newboolean{separate}
%\setboolean{separate}{true}

\renewenvironment{docpart}{\begin{document}}%
                          {\bibliography{cmixII}\printindex
                           \end{document}}
\begin{document}\shortindexingon
}
\title{Tables}
\author{Jens Peter Secher}
\begin{docpart}
\maketitle

\bibliography{cmix-bib}

\printindex

\end{docpart}

%%% Local Variables: 
%%% mode: latex
%%% TeX-master: "cmixII"
%%% End: 


\end{document}

%%% Local Variables: 
%%% mode: latex
%%% TeX-master: t
%%% End: 

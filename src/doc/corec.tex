% File: corec.tex
% Time-stamp: 
% $Id: corec.tex,v 1.5 1999/06/10 15:45:28 makholm Exp $

\providecommand{\docpart}{\renewenvironment{docpart}{}{}
\end{docpart}
\documentclass[twoside]{cmixdoc}
%\bibliographystyle{apacite}

\makeatletter
\@ifundefined{@title}{\title{\cmix-documentation}}{}
\@ifundefined{@author}{\author{The \cmix{} Team}}%
{\expandafter\def\expandafter\@realauthor\expandafter{\@author}%
\author{The \cmix{} Team\\(\@realauthor)}}
\makeatother

\AtBeginDocument{%
\markboth{\hfill\today\quad\timenow\hfill\llap{\cmix\ documentation}}
{\hfill\today\quad\timenow\hfill}}

\renewcommand{\sectionmark}[1]{\markboth
{\hfill\today\quad\timenow\hfill\llap{\cmix\ documentation}}
{\rlap{\thesection. #1}\hfill\today\quad\timenow\hfill}}

%\newboolean{separate}
%\setboolean{separate}{true}

\renewenvironment{docpart}{\begin{document}}%
                          {\bibliography{cmixII}\printindex
                           \end{document}}
\begin{document}\shortindexingon
}
\begin{docpart}

\section{The \coreC language}
\label{sec:TheCoreCLanguage}\index{Core C language}
Input to \cmix is any strictly confoming \ansiC
program~\cite{ANSI:1990:ANSIC}. In most analyses\footnote{The
only exceptions are an initial conventional type check and the
resolution of identifiers in user annotation directives.}
we represent it
internally in the \cmix system in a reduced representation
that we call \coreC. The reason for
this is to make it easier both to implement and to ensure correctness
of various analyses and transformations.

\coreC is sufficiently close to C that one can view it as a
restricted subset of \ansiC. However, several details are
treated more orthogonally than the real language.
The most important differences are:
\begin{itemize}
\item \coreC has no structured control constructs apart from
        function calls. The control statements of \ansiC are
        translated into a ``flat'' flow graph where only
        \syntax{goto} statements and ``\syntax{if ($e$) goto
        $l_1$; else goto $l_2$;}''
        appear.
\item \coreC \emph{expressions} have no side effects. Side effects
        are produced by primitive \emph{statements}.
        (Expressions can still have the ``side effect'' of
        trying to derefence a null pointer and killing the program).
\item \coreC has no lvalue expressions. Thus there is no
        generic ``address-of'' operator, but the constructs
        that in \ansiC produce lvalues (such as naming a variable
        or selecting a member of a struct) have \coreC counterparts
        that evaluate to pointers to the location the lvalue
        references.
\item \coreC has no implicit conversions or coercions. Every
        such thing---ranging from promoting an \syntax{int} to
        \syntax{long} so one can add it to another \syntax{long},
        to the ``pointer decay rule'' for arrays---is represented
        as an explicit construct.
\item \coreC has only two lifetimes for variables: they are either
        global or local to a function. \syntax{static} variables
        are made global, and \syntax{auto} or \syntax{register}
        variables declared in compound statements are collected
        in a single list of the function's local variables.
\end{itemize}

\coreC also has an element of ``non-ortogonality'': where a
construction can easily be replaced with an equivalent
representation in a certain context, we generally
disallow it in this description.

Examples of this strategy are the decisions that pointer-valued
expressions cannot be direct operands to the logical operators
\syntax{!}, \syntax{||}, and \verb.&&., or the rule that a
suitable expression \emph{must} be present in the \syntax{return}
statements in a nonvoid function.

This is intended to simplify the analyses: \eg, one could imagine
an analysis that has to do something special when a pointer value
is ``consumed''; such an analysis would benefit from not having to
check the operand types of the logical operators for pointerness.

Internally \coreC is represented as heavily interlinked heap
allocated C++ objects. However, it is instructive to think of
a \coreC program as a tree structure where cross-links occasionally
appear as payload data in the nodes---even if the cross-links are
physically represented as the same types of pointers that parent
nodes use to link to their children. The grammar for this tree
appears in figure \ref{fig:corecinternal}.

\begin{figure}\begin{center}
\small\[
\begin{array}{|l|r@{~}c@{~}l|l|}
\hline
  \mbox{C++ class} & \multicolumn{3}{l|}{\mbox{grammar}} & \mbox{comments} \\
\hline\hline
\syntax{C\_Pgm} & p &::=& (\vec\sigma, \vec\eta, \vec d_v,
                                \vec d_f ,\vec d_{gf}) & \\
\syntax{C\_UserDef} & \sigma
        &::=& \cons{Struct}(id?,\vec{\epsilon},\vec{\coreXref t}\,) & \\
\mbox{~~(struct/union defn)}
        &&|& \cons{Union}(id?,\vec{\epsilon},\vec{\coreXref t}) & \\
\syntax{C\_EnumDef} & \eta   &::=& \cons{Enum}(id?,\vec\epsilon\,) & \\
\mbox{~~(enum definition)} & \epsilon &::=& (id,q,e?)
                & \mbox{enum constant with possible value} \\
\hline
\syntax{C\_Decl} & d &::=& \cons{Var}(id?,q,t,\delta,i?,\vec d\,) & \\
\mbox{~~(declaration)}&&|& \cons{Fun}(id,t,\vec d_p,\vec d_v,\vec b\,)
                        & \mbox{empty } \vec b \mbox{ means prototype} \\
\syntax{C\_Init} & i &::=& \cons{Simple}(e) & \\
\mbox{~~(initializer)}&&|& \cons{FullyBraced}(\vec\imath\,) & \\
                 &   & | & \cons{SloppyBraced}(\vec\imath\,) & \\
		 &   & | & \cons{StringInit}(\mathit{string}) & \\
\syntax{VariableMode} & \delta
  &::=& \cons{VarIntAuto}     & \mbox{Internal variable} \\
  &&| & \cons{VarIntResidual} & \mbox{Internal, declared residual} \\
  &&| & \cons{VarIntSpectime} & \mbox{Internal, declared spectime} \\
  &&| & \cons{VarVisResidual} & \mbox{Declared visible residual} \\
  &&| & \cons{VarVisSpectime} & \mbox{Declared visible spectime} \\
  &&| & \cons{VarExtDefault} ~|~
        \cons{VarExtResidual} & \mbox{External residual variable} \\
  &&| & \cons{VarExtSpectime} & \mbox{External spectime variable} \\
\hline
\syntax{C\_BasicBlock}& b &::=& (\vec s,j) & \\
\syntax{C\_Jump}& j &::=& \cons{If}(e,\coreXref{b_{\mathrm{then}}},
                                \coreXref{b_{\mathrm{else}}})
        & \syntax{if (}e\syntax{) goto }b_{\mathrm{then}}
          \syntax{; else goto }b_{\mathrm{else}}\syntax{;} \\
\mbox{~~(control stmt)}&&|& \cons{Goto}(\coreXref b) & \\
                &   & | & \cons{Return}(e?) & \\
\syntax{C\_Stmt}    & s &::=& \cons{Assign}(e,e') & \syntax{*}e\syntax{=}e' \\
\mbox{~~(statement)}&   & | & \cons{Call}(e?,\gamma,e_f,\vec e\,)
        & \syntax{*}e\syntax{=(*}e_f\syntax{)(}\avec{e}\syntax{)} \\
                    &   & | & \cons{Alloc}(e,d)
        & \syntax{*}e\syntax{=calloc(}e',\syntax{sizeof (}t_d\syntax{))} \\
                    &   & | & \cons{Free}(,e')
        & \syntax{free(}e'\syntax{)} \\
\syntax{CallMode}&\gamma&::=& \cons{CallAtSpectime}
                        & \mbox{call external function at spectime} \\
                    &   & | & \cons{CallOnceSpectime}
                        & \mbox{ditto but illegal if dyn. control} \\
                    &   & | & \cons{CallAuto}
                        & \mbox{pure external call} \\
                    &   & | & \cons{CallResidual} ~|~ \cons{CallDefault}
                        & \mbox{residual call to external function} \\
\hline
\syntax{C\_Expr} & e &::=& \cons{Var}(t,\coreXref d)
                                & \mbox{the address of the variable} \\
\mbox{~~(expression)}&&| & \cons{EnumConst}(t,\coreXref\epsilon)
                        & \mbox{an enum constant (always an \syntax{int})} \\
                 &   & | & \cons{Const}(t,c)
                                & \mbox{(constants can have any type)} \\
                 &   & | & \cons{Null}(t)
                                & \mbox{null pointer constant} \\
                 &   & | & \cons{Unary}(t,\diamond,e) & \\
                 &   & | & \cons{PtrArith}(t,e_1,\circ,e_2) 
                                & \mbox{ptr}\circ\mbox{num}=\mbox{ptr} \\
                 &   & | & \cons{PtrCmp}(t,e_1,\circ,e_2)
                                & \mbox{ptr}\circ\mbox{ptr}=\mbox{num} \\
                 &   & | & \cons{Binary}(t,e_1,\circ,e_2)
                                & \mbox{num}\circ\mbox{num}=\mbox{num} \\
                 &   & | & \cons{Member}(t,e,m)
                & \mbox{ptr to struct}\rightarrow\mbox{ptr to member} \\
                 &   & | & \cons{Array}(t,e)
                & \mbox{ptr to array}\rightarrow\mbox{ptr to first element} \\
                 &   & | & \cons{DeRef}(t,e)
                                & \mbox{contents of pointed-to address} \\
                 &   & | & \cons{Cast}(t,e) & \\
                 &   & | & \cons{SizeofType}(t,t') & \\
                 &   & | & \cons{SizeofExpr}(t,e) & \\
\syntax{UnOp}&\diamond&::=&\syntax{-}~|~\verb.~.~|~\syntax{!} & \\
\syntax{BinOp}&\circ &::=& \syntax{*}~|~\syntax{/}~|~\syntax{\%}~|~
                                \syntax{+}~|~\syntax{-}~|~\syntax{<<}~|~
                                \syntax{>>}~|~\syntax{<}~|~\syntax{>} & \\
                 &   & | & \syntax{<=}~|~\syntax{=>}~|~\syntax{==}~|~
                                \syntax{!=}~|~\syntax{\&}~|~\syntax{|}~|~
                                \verb.^.~|~\syntax{\&\&}~|~\syntax{||} & \\
\hline
\syntax{C\_Type} & t &::=& \cons{Abstract}(id)
                                & \mbox{Non-arithmetic primitive type} \\
\mbox{~~(type)}  &   & | & \cons{Primitive}(id)
                                & \mbox{Arithmetic primitive type} \\
                 &   & | & \cons{Enum}(\coreXref\eta)
                                & \mbox{Enumerated type} \\
                 &   & | & \cons{Array}(t,e?,q)
                                & \mbox{Array of }q~t\mbox{, length }e \\
                 &   & | & \cons{Pointer}(t,q)
                                & \mbox{Pointer to } q ~ t \\
                 &   & | & \cons{FunPtr}(\cons{Fun}(t,\vec t\,))
                                & \mbox{Pointer to function} \\
                 &   & | & \cons{Fun}(t,\vec t\,)
                        & \mbox{Function of }\vec t\mbox{ returning }t\\
                 &   & | & \cons{StructUnion}(q,\coreXref\sigma,\vec t\,)
                        & \mbox{Struct or union with member types }\vec t \\
\syntax{constvol}& q &::=& (\cons{Const}?,\cons{Volatile}?) & \\
\hline
\end{array}\]\end{center}
\vskip-1ex\relax
\caption{The \coreC grammar.}
  \label{fig:corecinternal}
\end{figure}

The grammar uses the following conventions:
\begin{itemize}
\item[$c$] is any (non-string) constant.
\item[$id$] is a name from the original \ansiC program.
\item[$m$] is the ordinal number of a struct or union member.
\item[$x?$] is an optional $x$.
\item[$\vec x$] is a list of $x$. Lists are represented by the
        \syntax{Plist<}$T$\verb.>. template for a list of pointers
        to $T$, defined in \syntax{Plist.h}. These lists support
        STL-like iterators.
\item[$\coreXref x$] is a cross-link to
        an element somewhere else in the program.
        This notation is not used for types ($t$) because
        the structural sharing rules are not as simple as for the
        other classes. A detailed description appears in section
        \ref{sec:corecTypeSharing}.
\end{itemize}

The definitions of the C++ classes that make up the actual
\coreC representation are defined in the source file
\syntax{corec.h}. The auxilliary types whose names do not start
with \syntax{C\_} are defined in \syntax{tags.h}.

The \coreC structures use tags and
unions; not inheritance and virtual functions. The reason for this is
that the core language is not expected to change, whereas the various
kinds of analyses are very likely to change. In a traditional OO
design, every construct should contain a virtual method that carried
out the part of some analysis that applies for that particular
construct. This would result in each analysis being distributed into
several parts of the program, which would make \cmix hard to
understand and maintain.

\subsection{Programs}
A program is a collection of
\begin{itemize}
\item struct and union definitions.
\item enum definitions.
\item global variables (which include \syntax{static} and \syntax{extern}
        variables declared inside the \ansiC program's functions).
\item function definitions.
\item ``goal'' function definitions.
\end{itemize}

A struct and or definition consist of an optional tag and a list
of member descriptions. They also have a list of all the (\cons{StructUnion})
type nodes that reference them.

The member descriptions (denoted with $\epsilon$'s in the grammar)
only include the member's name, type qualifiers, and and optional
expression that denote a possible bitfield width.
If a definition for the aggregate needs to be written, one needs
to go via one of the referencing \cons{StructUnion} nodes to find
the type of each member. Aggregates with no referencing
\cons{StructUnion} should and need not exist in \coreC at all.

\subsection{Declarations}
Despite its name, the \syntax{C\_Decl} class represents not only variable
declaration but anything in the program the pointer analysis needs to
say a pointer may point to. This is the reason why variables and
functions are represented by the same construction.

The source file \syntax{ALoc.h} defines types and functions that
manage unordered sets of declaration references, used by several
of the analyses.

\subsubsection{Function declarations}

$\cons{Fun}(id,t,\vec d_p,\vec d_e,\vec b\,)$
declarations are simple. They have a name and a type (always
a \cons{Fun} type),
lists of formal parameter and local variable declarations, and a
nonempty list of basic blocks. Execution of the function body starts
at the first basic block in the list, and always ends with an
explicit \cons{Return} statement. Functions cannot return simply
by ``falling through the body''.

External functions are not separately represented in the \coreC
program; references to them are \cons{Const} expressions.

\subsubsection{Variable declarations}
\label{sec:corecVariables}

\cons{Var} declarations are more complex. They always have a type but
may have no name: sometimes the c2core phase
needs to introduce new temporary variables. Each variable has a
\emph{varmode} that defines its linkage and any user annotations on
it. A variable may have an \emph{initializer} (section
\ref{sec:coreInitializers}).

Additionally, if the variable consists of sub-objects that can be
individually pointed to (\emph{i.e.}, if it has \cons{Array} or
\cons{StructUnion} type), its declaration also includes one or
more \emph{subsidiary declarations} which represent the sub-objects
in the pointer analysis and other analyses that depend on that.
If the type is an array there is a single subsidiary declaration
which collectively represents the elements of the array. If the
type is a struct or union there is a subsidiary declaration for each
of its members.

A subsidiary declaration per convention have no name or
initializer; its varmodes specify internal linkage.
However, it may contain its own subsidiaries, according to its type.

An example on the use of subsidiary declaration appears on
figure \ref{fig:typesharing}, page \pageref{fig:typesharing}.

Declarations for formal parameters behave like variable declarations
but never have initializers. So does the name-less declaration that
is part of an \cons{Alloc} statement; it collectively represents every
block of memory that gets allocated by the statement.

\subsection{Initializers}
\label{sec:coreInitializers}

An initializer is an optional part of a local or global variable
declaration. Its structure resembles the C read syntax for
initializers.

The $\cons{FullyBraced}(\vec\imath\,)$ construction is used where it
can be guaranteed that each of the elements of $\vec\imath$
initialize exactly one of the immediate subobjects of the aggregate
initialized by the initializer.

Sometimes, however, that is not possible,
and the structure of the read \ansiC initializer is simply copied
using \cons{SloppyBraced} nodes. This reduces the precision of the
analyses and prevents splitting of the initialized object into
more than a single spectime or residual part.

A \cons{FullyBraced} initializer can have a \cons{SloppyBraced}
node as one of its subinitializers.

\begin{figure}\begin{center}\begin{tabular}{|l|l|}
\hline Declaration & Initializer \\ \hline \hline
\syntax{int a1[sizeof(int)][] = \{1,2,3,4,5,6\};} &
$\cons{SloppyBraced}(\syntax{1}, \syntax{2}, \syntax{3},
                     \syntax{4}, \syntax{5}, \syntax{6})$ \\
\syntax{int a2[4][] = \{1,2,3,4,5,6\};} &
$\cons{FullyBraced}(
        \cons{FullyBraced}(\syntax{1}, \syntax{2}, \syntax{3}, \syntax{4}),$ \\
& \hskip 2cm $
        \cons{FullyBraced}(\syntax{5},\syntax{6}))$ \\
\syntax{int a3[sizeof(int)][] = \{\{1,2,3,4\},\{5,6\}\};} &
$\cons{FullyBraced}(
        \cons{FullyBraced}(\syntax{1}, \syntax{2}, \syntax{3}, \syntax{4}),$ \\
& \hskip 2cm $
        \cons{FullyBraced}(\syntax{5},\syntax{6}))$ \\
\hline
\end{tabular}\end{center}
\caption{Examples of the initializer representation}
\end{figure}

\subsection{Statements}
All statements have an optional left-side expression that, if present,
points to the location where the result of the statement is stored. If
the left side is not present, the result is ignored. Actually the
left-side is only genuinely optional in the \cons{Call} statement, but
it is treated uniformly across all statement cases.

The \cons{Assign} statement is a simple assignment; its only side
effect is the assignment to the left-side. An \cons{Assign} with
an empty left-side would be a no-op, so is optimized away in the
c2core phase.

The \cons{Call} statement performs a call through a function pointer
expression. The statement has a \emph{callmode} attribute that
tells when (at spectime or at residual time)
the call should be made if it is a call to an external function.
Specializable functions (ones for which we know a body) are called
through static pointers, so for the purposes of callmode they are
considered spectime.

Calls to the \syntax{malloc()} and \syntax{calloc()} standard library
functions are intercepted in the c2core phase and translated into
$\cons{Alloc}(e,d)$ statements.
The declaration
$d$ is used in the analyses to represent collectively all the pieces
of memory allocated by that particular \cons{Alloc} statement.
The type of $d$ is always an \cons{Array} type; its size expression
specifies the number of objects to allocate on the heap. Contrary to
other array size expressions, this one does not need to be a constant
expression; it is evaluated in the context of the \cons{Alloc}
statement. If the \cons{Array} type has no size expression, a single
object is allocated.
The left-hand side of the \cons{Alloc} statement is never empty:
an allocation that turns out to be ignored gets optimized away
in the c2core phase.

The \cons{Free} statement is the translation of a call to the
\syntax{free()} standard library function. This function has a return
type of \syntax{void}, so the left side of the statement must be
empty.

\subsubsection{Control statements}
The definitions of control statements and basic blocks is fairly
straight-forward.

The \cons{Return} statements are normalized so that the expression
is present precisely if the function does not return \syntax{void}.
If necessary, an anonymous global variable of the proper type is
generated, and its (never assigned-to) value is returned.

\subsection{Expressions}

A \coreC expression has no side effects and always specifies a
value, never a location. An \ansiC expression that is used as
an lvalue is translated to a \coreC expression whose value is
a pointer to the location of the lvalue.

Every expression has a type as its first attribute; this is the
type of the expression's value. This attribute is treated uniformly
across the possible expression forms, so that analyses can act on
the type of an expression without doing a case analysis on its
form. See section \ref{sec:corecTypeSharing} for a description
of structural sharing between these types.

A $\cons{Var}(t,\coreXref d)$ expression evaluates to the
\emph{address} of the referenced variable or function. $d$ must be
a function, a global variable, or a formal
parameter or local variable in the same function the expression
appears in. It cannot be a subsidiary object declaration---pointers
to structure members and array elements are produced by the
\cons{Member} and \cons{Array} expressions.

\cons{EnumConst} expressions model enumeration constants. They are
different from \cons{Var} expression because enumeration constants
don't have addresses and can have different binding times each
time they are mentioned. They are different from \cons{Const} in that
they need to be name managed.

The $\cons{Const}(t,c)$ expression is used for numeric constants,
\emph{and} for identifiers declared as
\syntax{well-known constant:} by user annotations. These constants
can have arbitrarily complex types. This feature is used by the
shadow headers---\emph{e.g.}, by the \syntax{stdio.h} shadow which
defines \syntax{stderr} to be a well-known constant of type ``pointer
to \syntax{FILE}''. This differs from using a \syntax{const} variable
in that different occurences of the well-known constant can have
different binding times.
This situation is
similar to what should happen when external functions are mentioned.
Thus, \cons{Const} is also used for representing the expressions
consisting of the name of an external function.

A special case $\cons{Null}(t)$ exist for expressing null pointer
constants. This is different from $\cons{Const}(t,\syntax{NULL})$
in that a pointer \cons{Const} points to some unknown external
location, while \cons{Null} never points to anything.

The only \cons{Unary} operators in \coreC are the various senses
of negation. The operand is never a pointer---when the
\syntax{!}\ operator is applied to a pointer expression c2core
substitutes an explicit test against a null pointer constant.

The binary operators are translated to \cons{PtrArith},
\cons{PtrCmp}, or \cons{Binary} expressions, depending on the types
of the operands and result. If pointer expressions appear as
operands to the \verb.&&. and \verb.||. operators, explicit
tests against null pointer constants are inserted.
The short-circuiting behavior of the \verb.&&. and \verb.||. operators
in \ansiC is only partly relevant in \coreC because expressions do not
have side effects.

The operand ordering to \cons{PtrCmp} is normalized so the left
operand is always the pointer and the right always the integer.

In the $\cons{Member}(t,e,m)$ expression, $e$ must evaluate to
a pointer to a struct or union. The \cons{Member} expression
produces a pointer to the $m$th of its members. This is sufficient to
represent the dot operator of \ansiC\footnote
        {One might think that this assertation ignores the fact
        that there can be non-lvalue expression with struct or
        union type in \ansiC. However, this only happens with
        function calls, simple assignment, and the \syntax{?:}
        operator---neither of which are expressions in
        \coreC. In each of these cases the c2core translation
        already uses a translation strategy that automatically provides
        an addres to a struct or union object with the same contents
        as the non-lvalue struct or union.}.

The $\cons{Array}(t,e)$ expression produces a pointer to the
first element of the array pointed to by the pointer that $e$
evaluates to. The \ansiC counterpart is the implicit ``pointer
decay'' that happens to expressions with array type.

Type casts, $\cons{Cast}(t,e)$, can appear between
any pair of types $t_e$ and $t$. That is theory at least; in practise
programs with badly-behaved casts will not survive
the type checker to tell the tale.

% The following was a noble thought, but we need to reduce everything
% to SizeofExpr anyway so we can observe the scope rules as we emit
% pgen and pres.
%
%It is tempting to reduce \cons{SizeofExpr} to
%\cons{SizeofType}. After all, we know the type of the expression
%operand. However, in general---though not precisely in
%\cmix---having expressions as operands to \syntax{sizeof}
%can provide potentially valuable information to program analyses
%or program transformers, which might be more conservative
%when faced with a naked type expression. Thus it would be
%unjust of us to lose these connections in the residual program.

\subsubsection{Restrictions on expressions}

\begin{itemize}
\item No expressions have \cons{Fun} type.
\item Expressions with \cons{Array} type only appear as
        operands to \cons{SizeofExpr}.
\item Expressions with \cons{StructUnion} type only appear as
        operands to \cons{SizeofExpr}, as right-hand sides of
        \cons{Assign} statements, or in argument lists of
        \cons{Call} statements.
\item The expression in a \cons{Simple} initializer and the length
        of an \cons{Array} type are constant expressions. They may
        not contain any \cons{DeRef} expressions (except in operands to
        \cons{SizeofExpr}), and \cons{Var} expressions
        may not reference formal parameters or local variables.
\end{itemize}

\subsubsection{Displaying expressions}

The left-hand side of figure \ref{fig:corecdisplay} defines the
closest one gets to a fully compositional write syntax
for \coreC expression that could be parsed by an \ansiC
parser.

\begin{figure}\begin{center}
\def\can#1{(\!|#1|\!)}
\def\hum#1{\{\!|#1|\!\}}
\def\shum#1{{\hum{#1}\!}^*}
\small\[
\begin{array}{|r@{\:}c@{\:}l|r@{\:}c@{\:}l|}
\hline
\multicolumn{3}{|c|}{\mbox{Canonically}} &
\multicolumn{3}{c|}{\mbox{Optimizing}} \\
\hline
\hline
   \can{\cons{Var}(t,\coreXref d)} &=& \syntax{\&}id_d
 & \hum{e=\cons{Var}(\ldots)} &=& \syntax{\&}\shum{e}
\\ \can{\cons{EnumConst}(t,\coreXref\epsilon)} &=& id_\epsilon
 & \hum{\cons{EnumConst}(t,\coreXref\epsilon)} &=& id_\epsilon
\\ \can{\cons{Const}(t,c)} &=& c
 & \hum{\cons{Const}(t,c)} &=& c 
\\ \can{\cons{Null}(t)} &=& \syntax{NULL}
 & \hum{\cons{Null}(t)} &=& \syntax{NULL} 
\\ \can{\cons{Unary}(t,\diamond,e)} &=& \diamond\can e
 & \hum{\cons{Unary}(t,\diamond,e)} &=& \diamond\hum e 
\\ \can{\cons{PtrArith}(t,e_1,\circ,e_2)} &=& \can{e_1} \circ \can{e_2}
 & \hum{\cons{PtrArith}(t,e_1,\circ,e_2)} &=& \hum{e_1} \circ \hum{e_2} 
\\ \can{\cons{PtrCmp}(t,e_1,\circ,e_2)} &=& \can{e_1} \circ \can{e_2}
 & \hum{\cons{PtrCmp}(t,e_1,\circ,e_2)} &=& \hum{e_1} \circ \hum{e_2} 
\\ \can{\cons{Binary}(t,e_1,\circ,e_2)} &=& \can{e_1} \circ \can{e_2}
 & \hum{\cons{Binary}(t,e_1,\circ,e_2)} &=& \hum{e_1} \circ \hum{e_2} 
\\ \can{\cons{Member}(t,e,m)} &=& \syntax{\&}\can{e}\syntax{->}id^m_{\sigma_t}
 & \hum{e = \cons{Member}(\ldots)} &=& \syntax{\&}\shum{e} 
\\ \can{\cons{Array}(t,e)} &=& \syntax{\&(*}\can{e}\syntax{)[0]}
 & \hum{\cons{Array}(t,e)} &=& \shum{e} 
\\ \can{\cons{DeRef}(t,e)} &=& \syntax{*}\can{e}
 & \hum{\cons{DeRef}(t,e)} &=& \shum{e} 
\\ \can{\cons{Cast}(t,e)} &=& \syntax{(}t\syntax{)}\can{e}
 & \hum{\cons{Cast}(t,e)} &=& \syntax{(}t\syntax{)}\hum{e} 
\\ \can{\cons{SizeofType}(t,t')} &=& \syntax{sizeof(}t\syntax{)}
 & \hum{\cons{SizeofType}(t,t')} &=& \syntax{sizeof(}t\syntax{)} 
\\ \can{\cons{SizeofExpr}(t,e)} &=& \syntax{sizeof}\can{e}
 & \hum{\cons{SizeofExpr}(t,e)} &=& \syntax{sizeof}\hum{e} 
\\[1ex]
&&& \shum{\cons{Var}(t,\coreXref d)} &=& id_d \\
&&& \shum{\cons{PtrArith}(t,e_1,\syntax{+},e_2)}
                &=& \hum{e_1}\syntax{[}\hum{e_2}\syntax{]} \\[2pt]
\multicolumn{3}{|c|}{
    \setbox0\vbox{\setbox2\hbox{parentheses as required by precedence}
          \setbox4\hbox to\wd2{\hfil are implicit for both mappings\hfil}
          \box2\vskip1pt\box4}
    \ht0=0pt
    \box0}
  & \shum{\cons{Member}(t,e,m)} &=&
  \left\{\begin{array}{@{}l@{,}l@{}}
    \hum{e}\syntax{->}id^m_{\sigma_t} & \mbox{ for } e=\cons{DeRef}(\ldots) \\
    \shum{e}\syntax{.}id^m_{\sigma_t} & \mbox{ otherwise}
  \end{array}\right. \\
&&& \shum{\mbox{any other }e} &=& \syntax{*}\hum{e} \\[2pt]
\hline
\end{array}\]\end{center}
\caption{Translating \coreC expressions back to C.}
  \label{fig:corecdisplay}
\end{figure}

That mapping produces correct yet horribly unreadable expressions with a
lot of spurious address-of and dereference operators. Therefore, by
default the modules that unparse \coreC (outcore and gegen) use the
``optimizing'' rendering defined at the right side of figure
\ref{fig:corecdisplay}.

The downside of the ``optimizing'' format is that not all \coreC
constructs are individually visible. This means that the annotated
\coreC output in this notation cannot include the annotation
on every subexpression. When debugging \cmix it is sometimes
convenient to switch to the canonical representation in outcore; we provide
a \syntax{-S} switch that does that (almost. $\cons{Array}(t,e)$ is
rendered as a unary operator spelled \syntax{<DECAY>}).

\subsection{Types}
\label{sec:corecTypes}
The $\cons{Abstract}(id,q)$, $\cons{Arithmetic}(id,q)$,
and $\cons{Primitive}(id,q)$ types are
primitive types, at least as far as \cmix is concerned.

The \ansiC base types such as \syntax{signed char} and \syntax{float}
are all \cons{Primitive} types, with the $id$ of the type record containing
the type's full name. The exception is \syntax{void} which is represented
as $\cons{Abstract}(\syntax{void})$.

The user (or, more likely, shadow header files) can define additional
primitive types with the special syntax
\begin{verbatim}
  typedef __CMIX(bar) baz ;
\end{verbatim}
which declares \syntax{baz} as an \emph{abstract type} with the \syntax{b},
\syntax{a}, and \syntax{r} properties (the only one of these that
actually does something at the moment is \syntax{a}). The type
maps to $\cons{Primitive}(\syntax{baz})$ or
$\cons{Abstract}(\syntax{baz})$ in \coreC, depending on whether
it is numeric or not. The type checker differentiates between integral
and other arithmetic abstract types, and also maintains a guess on
whether the type is signed or not; those attributes are stored in
a ``hidden'' (which only means: not shown in the \coreC grammar)
field of the \cons{Primitive} type node.

\cons{Enum} types represent enumerated types. Except for the output
routines in \syntax{outcore} and \syntax{gegen}, they are treated
mostly like primitive types.

\cons{Pointer} and \cons{Array} are straight-forward---the optional
expression in the \cons{Array} type nodes is the length of the array.
Each of them contains the type qualifiers of the \emph{pointed-to}
type. That is, in \coreC, ``pointer to const int'' parses as
``(pointer to const) int'' rather than \ansiC's
``pointer to (const int)''. As far as partial evaluation is concerned,
this difference is not important, but the c2core translation works
best with having the qualifiers as part of the pointer.

\cons{Fun} is a function type with result and parameter types.
\cons{FunPtr} is a pointer to function; its single subitem is always
a \cons{Fun} type. This is also the only way \cons{Fun} types can be
used to form other types.

\cons{StructUnion} deserves some further explanation. The
master definition for the struct or union concerned is referenced
from the type node, so that the type tag and member names can be
accessed. However, the type has its own private list of member \emph{types},
allowing us to control the degree of structural sharing between types.
This is important because we attach binding-time information to type
nodes.

\subsubsection{Structural sharing of type representation}
\label{sec:corecTypeSharing}

For the purpose of defining the rules for sharing of type nodes, we
divide the references to type nodes in a program into ``owning'' and
``borrowing'' references, and say that a declaration or expression
``owns'' the type nodes that are reachable from its type field,
following only ``owning'' references. Then:
\begin{itemize}
\item A local or global variable declaration owns its
        entire type. The only borrowing references that can be
        reached from the type node is the back edges to
        \cons{StructUnion} nodes necessary to keep the
        representation finite.
\item A function declaration similarly owns its entire type.
\item A formal parameter declation borrows
        its type from the function's type's parameter type list.
\item The declaration in an \cons{Alloc} statement borrows its
        type from the relevant part of the statement's left-hand
        expression\footnote{This does not impose any risk of
                ``circular borrowing'', because only the
                alloc declaration's possible subsidiaries
                borrow from its type, and nothing outside the
                subsidiary tree borrows from it}.
\item A subsidiary declaration borrows
        its type from the appropritate part of its parent's type.
\item A \cons{Var}, \cons{Member}, \cons{Array}, or \cons{PtrArith} expression
        owns the topmost \cons{Pointer} node of its type. The
        pointed-to type is borrowed from the referenced declaraion
        or the relevant part of the operand's type, repectively.
\item A \cons{Const}, \cons{Unary}, \cons{PtrCmp}, \cons{Binary},
        \cons{Cast}, \cons{SizeofType}, or \cons{SizeofExpr}
        expression owns its entire type, and, in the case of
        \cons{SizeofType}, its entire operand type.
\item A \cons{DeRef} expression borrows its type from the
        relevant part of its operand's type.
\end{itemize}

Examples of the type sharing can be seen at figure \ref{fig:typesharing}

\begin{figure}[htbp]
  \begin{center}
     \ttfamily
     \begin{tabular}{l}
        struct node \{ \\
        ~~int ID:~24; \\
        ~~struct node *link; \\
        ~~char data[20]; \\
        \} a\_node, *a\_ptr; \\
        /* ...~*/ \\
        a\_ptr->data[5] \\
        \\
        \\
     \end{tabular}

     \epsfig{file=sharing.eps,height=16.5cm,width=15.5cm}
  \end{center}
  \caption{Type sharing example}
  \label{fig:typesharing}
\end{figure}

\subsection{Implementation level (1999-02-18)}
The above section describe the current implementation of Core C.
        
\end{docpart}
%%% Local Variables: 
%%% mode: latex
%%% TeX-master: "cmixII"
%%% End: 

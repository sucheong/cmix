% File: structsort.tex
% Time-stamp: 
% $Id: structsort.tex,v 1.2 1999/03/16 22:21:19 makholm Exp $

\providecommand{\docpart}{\renewenvironment{docpart}{}{}
\end{docpart}
\documentclass[twoside]{cmixdoc}
%\bibliographystyle{apacite}

\makeatletter
\@ifundefined{@title}{\title{\cmix-documentation}}{}
\@ifundefined{@author}{\author{The \cmix{} Team}}%
{\expandafter\def\expandafter\@realauthor\expandafter{\@author}%
\author{The \cmix{} Team\\(\@realauthor)}}
\makeatother

\AtBeginDocument{%
\markboth{\hfill\today\quad\timenow\hfill\llap{\cmix\ documentation}}
{\hfill\today\quad\timenow\hfill}}

\renewcommand{\sectionmark}[1]{\markboth
{\hfill\today\quad\timenow\hfill\llap{\cmix\ documentation}}
{\rlap{\thesection. #1}\hfill\today\quad\timenow\hfill}}

%\newboolean{separate}
%\setboolean{separate}{true}

\renewenvironment{docpart}{\begin{document}}%
                          {\bibliography{cmixII}\printindex
                           \end{document}}
\begin{document}\shortindexingon
}
\begin{docpart}

\section{Structure definition sorting}
\label{sec:StructureDefinitionSorting}
The structure sorting phase is a helper phase that takes place just
before the gegen phase. Its purpose is to perform a ``topological
sort'' on the final list of structure or union definition. This is
done to ensure that the generating extension will declare the
structures in the correct order, such that \eg the definition
\begin{verbatim}
struct foo {
   struct bar baz ;
}
\end{verbatim}
will not be written out before struct bar has been declared.

\subsection{The ``sequence number'' strategy}
Theoretically the order we want is a topological sort according to
the ``containment ordering'' defined by ``struct $A$
precedes struct $B$ iff struct $B$constains a member whose type is
`(array of)$^n$ struct $A$' for
some $n\geq 0$''.

This is implemented by assigning \emph{sequence numbers} to the struct 
definitions so that if struct $A$ precedes struct $B$, then $A$ will have a
lower sequence number than $B$. Then we can perform the topological
sort simply by sorting by sequence number.

We assign the sequence numbers in the C parser, then check in the type
checking phase that the connection between the sequence numbers and
the ``contaiment ordering'' hold. The sequence numbers are then
carried through to \coreC and simply copied when copies of structures
are made during the various phases. After struct definition have been
manipulated we have a different ``containment ordering'' but it is
easily seen, by inspection of the different phase definitions, that
the projection of the final containment ordering to the original
source definitions is a subset of the contaiment ordering that was
used in the type cheker. Thus the sequence numbers still have the
required property with respect to the final containment ordering.

\subsection{Assigning sequence numbers}

In the parser phase we simply maintain a counter and assign ascending
sequence numbers in the order we see the closing brace of the
definition of each structure. We may see the same structure defined
multiple times if there are more than one source file; in that case
the first definition we see is the one that counts.

This strategy works if the input is correct according to \ansiC;
\ansiC requires the types of members to be object types, and
`(array of)$^n$ struct T' is an object type only if a definition
of struct T is in scope at the point of the member declaration.
Being in scope implies that a definition appears textually earlier in
the file, thus the connection to the containment ordering holds.

\subsection{Checking the sequence numbers}

We do not implement the \ansiC semantics regarding the scope of
structure definitions---though we do (try to) implement the scope
rules of structure \emph{declarations}. This means that we need to
check the connection between sequence numbers and containment ordering
directly in the type checking phase.

Thus, we accept some programs that are not correct \ansiC, but since
we are able to resolve structures and generate correct generating
extensions even for those programs, we do not view this as much of
a problem.

\subsection{The implementation of the sorting phase}

We leave the actual sorting to the \texttt{qsort()} library function.

\subsection{Alternative strategies}

Another strategy that was considered was coding the parser phase such
that the proper topological ordering held from the beginning.
This strategy was rejected because a number of subsequent phases
make changes to the userdecl list, sometimes rather drastic.

There would be a lot of possibilities for inadvertantly introducing bugs
while restructuring code that didnt \emph{seem} to have anything to
with the order of structure declarations. Indeed, even getting the
parser to produce the list correctly would depend very subtly on a
lot of minor details that look rather unimportant in their immediate
context.

\subsection{Implementation level, 1999-03-16}

The above description fits the current implementation of the structure
sorting phase.

\end{docpart}
%%% Local Variables: 
%%% mode: latex
%%% TeX-master: "cmixII"
%%% End: 


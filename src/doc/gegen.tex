\providecommand{\docpart}{\renewenvironment{docpart}{}{}
\end{docpart}
\documentclass[twoside]{cmixdoc}
%\bibliographystyle{apacite}

\makeatletter
\@ifundefined{@title}{\title{\cmix-documentation}}{}
\@ifundefined{@author}{\author{The \cmix{} Team}}%
{\expandafter\def\expandafter\@realauthor\expandafter{\@author}%
\author{The \cmix{} Team\\(\@realauthor)}}
\makeatother

\AtBeginDocument{%
\markboth{\hfill\today\quad\timenow\hfill\llap{\cmix\ documentation}}
{\hfill\today\quad\timenow\hfill}}

\renewcommand{\sectionmark}[1]{\markboth
{\hfill\today\quad\timenow\hfill\llap{\cmix\ documentation}}
{\rlap{\thesection. #1}\hfill\today\quad\timenow\hfill}}

%\newboolean{separate}
%\setboolean{separate}{true}

\renewenvironment{docpart}{\begin{document}}%
                          {\bibliography{cmixII}\printindex
                           \end{document}}
\begin{document}\shortindexingon
}
\title{The Generating Extension and How To Generate It}
\author{Henning Makholm}
\begin{docpart}
\maketitle

\MakeShortVerb{/}
\section{The Gegen phase}
\label{sec:HowToGenerate}
\index{gegen}\index{generating extension}

\def\Pgen{$\hbox{P}_{\textit{\small gen}}$\xspace}

The \emph{generating extension generator}, or \emph{gegen} for short, takes
as input the binding-time annotated Core C program (and other analysis
results) and produces a generating extension on the basis of the
annotations. For brevity, the generating extension will be called
\Pgen.

Gegen is divided into several submodules which share information
through the common header file /gegen.h/. The header file defines
class /GegenEnv/ to which most of the submodules contribute member
functions.

The submodules are, --
\begin{description}
\item[/gg-expr/] handles the translation of \coreC types and
expressions into the representations they have in \Pgen. It defines
a special, iostream-like, class /GegenStream/ which keeps an internal
state for managing the output of p-gen code that uses the speclib to
create residual code snippets.
\item[/gg-cascades/] is a small submodule that defines useful idioms
that other modules use to support statically indexed arrays.
\item[/gg-struct/] performs various jobs related to struct, union,
and enum types. It sees to that they are declared properly in \Pgen
and the residual program. It manages the names of residual structures'
members. It creates functions that helps initialize the \Pgen
counterparts of residual struct and union objects, etc.
\item[/gg-memo/] takes care of all work related to the memoisation
of static values.
\item[/gg-decl/] writes out code to construct proper declarations
for residual variables. It also manages the declaration of the subject
program's global variables in \Pgen.
\item[/gg-code/] writes code for \coreC statements and functions. It
takes care of function inlining, and -- with help from /gg-memo/ for
the dirty parts -- writes out pending-list loop code and function
sharing code where necessary.
\item[/gegen/] contains code to tie all of the above together, and
code for some small helper functions that weren't naturally placed
otherwhere.
\end{description}

The detailed description of gegen that follows is organised by these
submodules. Each section begin by describing the external interface
of the submodule as seen by the other submodules, \emph{e.g.}, the
actual C code emitted is largely considered an ``implementation
detail'' in this description. Then what the emitted C code looks
like is described. At the end of each section there are tips for
understanding the source code for the submodule.

Previous to the split into submodules, the documentation
tried to follow the order in which things appear in the \Pgen source,
which was not very easy to follow after all. Hopefully this logical
organisation is easier to read.

\subsection{The type mapping}
\label{sec:gegen:TheTypeMapping}

Before beginning to describe the individual submodules it is important
to describe the fundamental relationship between the binding-time
annotated \coreC types of things, and the types the same things will
have in \Pgen.

\begin{figure}[htb]\begin{center}\begin{tabular}{|c|c|c|c|}
\hline
Binding-time & Variable type & Expression type & Type $[\![T]\!]$ in the \\
annotated type $T$) & $[T]$ in \Pgen & $[T]^*$ in \Pgen & residual program \\
\hline
\hline
/int/$_s$ & /int/ & /int/ &  \\ \hline
/int/$_d$ & /Code/ & /Code/ & /int/ \\ \hline \hline
pointer$_s$ to $T'$ & pointer to $[T']$ & pointer to $[T']$ & \\ \hline
pointer$_d$ to $T'$ & /Code/ & /Code/ & pointer to $[\![T']\!]$ \\ \hline\hline
array$_s[c]$ of $T'$ & array$[c]$ of $[T']$ &
    & $c$ copies of $[\![T']\!]$ \\ \hline
array$_d[c]$ of $T'$ & /Code/ &
    & array$[c]$ of $[\![T']\!]$ \\ \hline\hline
struct S$_s$ & /struct S/ & /struct S/ & \\ \hline
struct S$_d$ & shadow struct$^\dag$ & /Code/ & /struct S/ \\ \hline\hline
union S$_s$ & /union S/ & /union S/ &  \\ \hline
union S$_d$ & shadow struct$^\dag$ & /Code/ & /union S/ \\ \hline\hline
function$_s$ from & function from & &  \\
$(T'\ldots)$ to $T''$ & $([T']^*\ldots)$ to $[T'']^*$ & & \\ \hline
function$_d$ from & & & function from \\
$(T'\ldots)$ to $T''$ & & & $([\![T']\!]\ldots)$ to $[\![T'']\!]$ \\ \hline
\end{tabular}\end{center}
\caption{The type mapping}
\label{fig:typeRelationship}
\end{figure}

This is depicted in figure \ref{fig:typeRelationship}, and what this 
figure says about /int/ also applies to other ``simple'' type, that
is, basic types, abstract types, and enum types.

The reader will notice that the figure has two columns for the
residual version of a type. The $[T]$ column is the standard mapping;
it is the one used for \emph{variables} declared in the
\coreC program. In this mapping, when a type maps to /Code/ this
/Code/ object will at specialization time contain the name of the
corresponding residual variable.

The complex point about the variable type mapping is the way struct
and union types are handled. Each residual struct or union type maps
to a \Pgen struct (even if it was originally a union) whose members'
types are the ``variable'' versions of the subject structure's member
types. In addition to this, the shadow struct also contains a member
called /cmix/ of type /Code/. At specialization time this member
contains the name of the entire residual struct or union, while the
/Code/s inside the other members will be residual expressions denoting
individual members of the struct or union. The idea behind this is
that static pointers will be able to point to the members structure
even though the the structure itself is ``dynamic''.

The other type mapping for \Pgen is used for the value of \coreC
\emph{expressions}. Expressions cannot have array or function type,
which is why $[T]^*$ is not defined for these types. For static values
the two mappings are the same, so the only place the expression type
mapping is important is in the case of dynamic parameters and return
values of specializable functions.

The tricky thing about dynamic parameters to specializable functions
is that there \emph{is} something present at their place in the
corresponding \Pgen function's parameter list, but this ``something''
does \emph{not} represent a residual parameter object. Rather it
represents the actual expression used as argument in the function
call. This is because the \Pgen function decides by itself whether
to unfold the function call during specialization or create a residual
function to be called. In either case a real residual variable will
be created (and it will be represented by a standard local variable in
the \Pgen function) but the way this is done in practise differs.
So it is more convenient to simply hand the actual argument expression
to the \Pgen function and let that do the right thing with it.

The practical consequence of this is that dynamic function parameters
should always be /Code/ and never shadow structs---even though the
parameter is a dynamic struct S, the caller may have to create a
complex residual expression for finding the real struct S residually,
and no shadow struct may be at hand.

Similar reasoning lead to the decision that shadow structs should
not be used as function return types.

\smallskip

Turning to the rightmost column of figure \ref{fig:typeRelationship},
the mapping to types in the residual program should be
straightforward. 

Empty space in the $[\![T]\!]$ column means that values and variables
of the corresponding type are specialized completely away from the
residual program.

Note, though, that statically indexed arrays \emph{may} cause
something to exist in the residual program. For example, a variable
of type ``array$_s$[5] of /int/$_d$'' will produce 5 separate /int/
variables in the residual program.

If, however, the type is, \eg, ``array$_s[42]$ of /short/$_s$'', then the
variable maps to 42 times nothing in the residual program, \ie, it
still disappears. This is characteristic of our handling of arrays
with static subscripting: many aspects of types are transitive
through ``array$_s$'' components.

\subsection{gg-expr: printing expressions and types}

Before we can describe the interface between /gg-expr/ and the
rest of Gegen we need to summarize how \Pgen is expected to
build residual expressions.

\subsubsection{The code generation model in speclib}

The statements and expressions in the residual program as it is
built by the generating functions consist of trees (or rather, DAGs)
with the following types of nodes:
\begin{description}
\item[name] nodes are leaves and are replaced with a ``suitably
    uniqe'' indentifier. They are filled into the \Pgen variables
    that represent residual variables as these are initialized.
\item[lift] leaves of various kind represent numerical and character
    constants that originate in lifts.
\item[inner] nodes can have children. They are labeled with a
    string in which question marks define ``holes'' where the
    node's children are insert when the residual program is
    eventually written out. The string can be arbitrary and is
    often complex, such as /"?*(?-42)/ /+/ /sizeof(int/ /(*)[?])"/.
\end{description}

Apart from the primitives that create ``name'' and ``lift'' nodes,
the speclib functions that generate code---such as /cmixStmt/ which
emits a simple statement to the current function, or /cmixIf/ which
produces a residual conditional---all come bundled with the
creation of an ``inner'' node. The parameter list for these functions
end in ``/,char const*,.../'', where the last explicit parameter
is the label string and is followed by as many /Code/ arguments
as there are question marks in the string.

This design minimizes the extent to which \Pgen has to manipulate
the expression and declaration syntax of C. However, it also places
special requirements in gegen, as expressions cannot simply
be printed with a normal inorder traversal. For example, the
expression /1==x*y+z/, where /x/ and /y/ are static and /z/ dynamic,
must end up something like
$$/cmixMkExp("1==?+?",cmixLiftSigned(x*y),z)/.$$
One sees that the expression subsection of gegen needs to be able
to postpone parts of expressions until after the skeleton string
for the entire code bit has been constructed.

\subsubsection{External interface to gg-expr}

The /GegenStream/ class is /gg-expr/s main interface to the rest of
the world. A /GegenStream/ behaves in some contexts simlarly to
a standard C++ /ostream/: one can print strings, characters, and
numbers to it with the /<</ operator. However, there are also other
expressions one can print:
\begin{description}
\item[/enter/] ``Printing'' this word (the mechanism is similar to
  the standard library's /endl/ construct) outputs a quote and
  switches an internal flag from ``\Pgen mode'' to ``residual-code
  mode''. In residual-code mode the /GegenStream/ maintains a
  list of postponed subexpressions which have yet only been rendered
  as holes in the skeleton string.
\item[/EmitExpr(/\textit{\mdseries e}/,/\textit{\mdseries{precedence}}/)/]
  Prints the \coreC expression \textit{e} according to the current
  mode: if in \Pgen mode the entire expression is printed; in
  residual mode parts of the expression may be postponed and replaced
  by lifts. The \textit{precedence} parameter controls whether the
  expression will be enclosed in parantheses; it can be omitted when
  the client knows that parentheses are never necessary.
\item[/EmitDeref(/\textit{\mdseries e}/,/\textit{\mdseries{precedence}}/)/]
  This is equivalent to /EmitExpr/, but implicitly applies the unary
  /*/ operator to the expression. This way the usual simplifications
  of the \coreC rendering algorithm may succeed in merging this
  /*/ with the topmost operator of the expression, so the generated
  code makes full use of C's lvalue expression syntax.
\item[\textit{\mdseries i}]
  where \textit{i} is a \coreC initializer.
\item[/pushtype(/\textit{\mdseries t}/,/\textit{\mdseries q}/)/]
  Prints the part that comes before the identifier in a declaration
  with type \textit{t} and qualifiers \textit{q}. The type is
  interpreted according to the $[\cdot]$ or $[\![\cdot]\!]$ type
  mapping, depending on the mode of the /GegenStream/.
\item[/poptype/]
  Prints the second half of declaration for a waiting type that has
  been printed with /pushtype/. As the names imply, a stack discipline
  is maintained for waiting types.
\item[/EmitType(/\textit{\mdseries t}/,/\textit{\mdseries{string}}/,/%
	\textit{\mdseries q}/)/]
  Equivalent to printing /pushtype(/\textit{t}/,/\textit{q}/)/, then
  printing \textit{string}, then /poptype/.
\item[/AbstractDecl(/\textit{\mdseries t}/)/]
  Short form for /EmitType(/\textit{t}/,"",constvol())/.
\item[/Unqualified(/\textit{\mdseries t}/,/\textit{\mdseries{string}}/)/]
  Short form for /EmitType(/\textit{t}/,/\textit{string}/,constvol())/.
\item[\textit{\mdseries d}]
  where \textit{d} is a \coreC declaration node. Prints a declarator
  for the variable. Equivalent to /pushtype/ + name + /poptype/.
  ``name of variable'' will be the \Pgen name or a name reques
  according to the mode of the /GegenStream/.
\item[/alwayscode/]
  Instructs the /GegenStream/ that the next /pushtype/ should interpret
  the type according to the rules for expressions (i.e., function
  parameters and return values) rather than the normal rules for
  objects.
\item[/exit/]
  Ends ``residual-code mode'', printing a quote followed by the
  postponed expression parts that have been collected since the
  /enter/.
\item[/userhole/]
  This can be used in residual mode to print a ``hole'' character
  to the current skeleton string and leave a special marker in
  the list of postponed expressions. When /exit/ meets this special
  marker it returns to its caller which can then output whatever
  \Pgen it wants to fill this hole with. Then the rest of the
  postponed expressions are printed with a \emph{second} /exit/.
  There can be only one /userhole/ for each /enter/.
\end{description}

/GegenStream/ has one member function, /EmitLiftingHole/ which takes
a \coreC type parameter. It prints an appropriately decorated
/userhole/ and returns to its caller a string containing the name
of a speclib lifting function that works together with the said
decorations to lift values of the given type. The return string
contains the opening parenthesis of the lifting call. The intended
use is in a sequence like
\begin{verbatim}
  gegenstream << enter ;
  // ...
  char const *lf = gegenstream.EmitLiftingHole(type) ;
  // ...
  gegenstream << exit << ',' << lf << ... << ')' << exit ;
\end{verbatim}

\subsubsection{Output aspects}

\subsubsection{Implementation notes}

\subsection{gg-cascades: iterating over statically indexed arrays}

The submodule defines the /ForCascade/ class which other submodules
use for iterating through all elements of statically indexed arrays.
Constructing a single /ForCascade/ is used for iterating through
an arbitrary number of array levels, starting at zero---thus the
common case that an object is not a staticlly indexed array is
handled transparently.

The constructor for a /ForCascade/ takes three arguments:
\begin{itemize}
\item a \coreC type. The loop will iterate through as many levels of
      statically indexed arrays as it finds at the top of this type.
\item a reference to a /GegenEnv/
\item an integer specifying the number of spaces the loop must
      be indented in the output.
\end{itemize}
As the /ForCascade/ is constructed it outputs zero or more nested
loop statements to the output stream, and also outputs appropritate
indentation for the body statement of the loop.

Once the /ForCascade/ is created, the client can do these things
with it:
\begin{itemize}
\item inspect its /t2/ member which is the type after all the
      levels of statically indexed arrays has been stripped of.
      That is, the type of the actual array elements.
\item inspect of change its /cv/ member which is a /constvol/
      structure for holding type qualifiers. This member is
      initialized with the conjunction of the type qualifiers
      in the array types which was stripped off to produce /t2/.
\item output the /ForCascade/ to a /GegenStream/ using an
      overloaded /<</ operator. This causes a series of subscripts
      to be written to the /GegenStream/.
\item call the /addline()/ member function. This causes a comma or
      semicolon to be output to the output stream, followed by
      a newline and appropriate indentation for a second action in the
      loop body.
\end{itemize}

\subsubsection{Output aspects}

\subsubsection{Implementation notes}

\subsection{gg-struct: struct and union related issues}

This submodule defines one global function and several member
functions for /GegenEnv/.

The global function is /dotcmix(/\textit{type}/)/. It assumes the
type is dynamic and returns the string /".cmix"/ if the type is
a struct or union type and the empty string otherwise. This return
value can be used as a postfix operator to convert the ``object''
representation of a dynamic type and the ``expression'' representation.

The /GegenEnv/ member functions are:
\begin{description}
\item[/pgen_usertype_fwds()/]
	This function outputs forward declarations for the static
	versions of usertypes to \Pgen.
\item[/pgen_usertype_decls()/]
	This function outputs actual definitions for the static
	versiond of usertypes to \Pgen.
\item[/define_cmixMember()/]
	This function outputs a definition of an object called
	/cmixMember/ which is to hold the residual name nodes for
	members of residual usertypes. After this object has
	been properly initialized the expression /cmixMember.S.t[i]/
	evaluates to a /Code/ snippet containing the residual name
	of the /i/th of the residual members corresponding to the
	member /t/ of struct or union /S/. (In this example /t/'s
	type is supposed to be a one-level statically indexed array).
\item[/define_cmixPutNameX()/]
	This functions output definitions of functions that are
	used to intialize \Pgen object whose types map to
	``shadow structs''. The functions are called /cmixPutName/
	$X$ where $X$ is a positive integer.

	Every /cmixPutName/$X$ takes two arguments. The first is
	a reference to the \Pgen object to be initialized; the second
	is a /Code/ that should contain the residual name that has
	been created for the object.

	The speclib defines a /cmixPutName0/ which is used for types
	that map to plain /Code/ in \Pgen:
\begin{verbatim}
void cmixPutName0(Code &obj,Code name) {
  obj = name ;
}
\end{verbatim}
\item[/putnameseq(/\textit{\mdseries t}/)/]
	Returns the number $X$ in the /cmixPutName$X$/ that is
	appropriate for the given type.
\item[/init_struct()/]
	Outputs whatever usertype-related code should go in the
	/cmixGennit/ function. This code must initialize the
	/cmixMember/ structure and may pass on usertype definitions
	to the speclib.
\item[/exit_struct()/]
	Outputs whatever usertype-related code should go in the
	/cmixGenExit/ function. Apart from interacting with the
	speclib, this code may write text to the residual program
	using the /FILE/ pointer /fp/. This text will come before
	the residual program text that has been collected during
	specialization.
\end{description}

\subsubsection{Output aspects}

\subsubsection{Implementation notes}

\subsection{gg-memo: generating memoisation code}

This submodule defined the class /GegenMemo/ which is the type of
the /memo/ member of /GegenEnv/. /GegenMemo/ contains data that
are private to the task of generating memoization code.

Its publicly accessible member functions are
\begin{description}
\item[/follow_functions()/]
	Outputs to \Pgen any auxiliary functions that is needed for
	the memoisation code to work in addition to speclib. Currently
	this is functions that help the memoisation engine locate
	pointer members in usertype objects.

	Also this function makes global decisions about memoisation
	stategies, so it should be called before other member
	functions in /GegenMemo/.
\item[/emit_globalmemodata()/]
	Outputs to \Pgen any data declarations necessary for memoising
	global variables.
\item[/emit_globalmemocode()/]
	Outputs any code that must go into the body of /cmixGenInit/
	in order to initialize the global variable memoisation.
\item[/emit_localmemodata(/\textit{\mdseries{fun}}/,/%
	\textit{ec}/,/\textit{pp}/)/]
	Outputs those declarations that are necessary for supporting
	memoisation of local variables and parameters in the
	generating function corresponding to \coreC function
	\textit{fun}.

	\textit{\bfseries{ec}} and \textit{\bfseries{pp}} are booleans
	which specify whether it may be necessary to memoise,
	repectively, the function entry conditions, and program points
	within the function.

	/emit_localmemodata/ returns a number which the generator
	function must pass to the speclib function /cmixPopObjects/
	before returning.
\item[/emit_pendinsert(/\textit{\mdseries{fun}}/,/%
	\textit{\mdseries{bb}}/)/]
	Outputs a \Pgen expression which inserts the given program
	point in the pending list and evaluates to a reference to
	the label resulting from this.
\item[/emit_funmemo(/\textit{\mdseries{fun}}/)/]
	Outputs the memoization-related arguments in calls to
	the speclib function /cmixFunctionSeenB4/ or /cmixPushFun/
\end{description}

\subsubsection{Output aspects}

\subsubsection{Implementation notes}

\subsection{gg-decl: constructing residual declarations}

\subsubsection{Output aspects}

\subsubsection{Implementation notes}

\subsection{gg-code: constructing generating functions}

\subsubsection{Output aspects}

\subsubsection{Implementation notes}

\subsection{gegen itself: tying all of it together}

\subsubsection{Implementation notes}

\textbf{\Large NB! the rest of this chapter has not yet been updated to match
the submodule split of gegen, or even the latest changes to the
speclib interface definition. Trust the material below at your own
risk. Documentation updates will follow shortly (I hope).}

\subsection{Chapter I: Header inclusions}
\label{sec:gegen:HeaderInclusions}
This first chapter includes the header files needed for \Pgen.

First, the files specified in /header:/ specializer directives
appear. Each of the ``shadow header'' files we use for standard
headers include an appropriate /header:/ directive. Thus, when the
user specifies, \eg, ``/#include <math.h>/'' \Pgen will also contain
``/#include <math.h>/''.

After these user-specified headers, we include the header that defines
the specialization support library and check that its version is the
same as the one Gegen was written to work with:
\begin{verbatim}
#include <cmix/speclib.h>
static cmixNameManager cmixNameMan;
#if cmixSPECLIB_VERSION != 2002
#error Wrong speclib version
#endif
\end{verbatim}
The default file name for the speclib header is
\verb|<cmix/speclib.h>|; if the environment variable CMIX_SPECLIB_H is
set, its value is interpolated instead.

Why this chapter is the right place to define the name manager used
for constructing residual identifiers seems to have been lost in the
mists of history. The fact is that it happens here...

\subsection{Chapter II: Forward declarations for usertypes}
\label{sec:gegen:ForwardDeclarationsForUsertypes}
The usertypes used in the program can be mutually recursive through
pointers. This means that before we begin to define their contents we
need to declare their \emph{names} so that we can define members of
type ``pointer to struct S'' before the definition of struct S itself.

That is what happens in this phase. The only slight complications
are
\begin{itemize}
\item per the type mapping, an immortal union type maps to a
      \emph{struct} in \Pgen, not a union.
\item enum types cannot be forward declared; so we have a choice of
      defining them here or among the real definitions in the next
      chapter. We choose to define them early.
\end{itemize}

\subsection{Chapter III: Definitions for usertypes}
\label{sec:gegen:DefinitionsForUsertypes}
Here are the real definitions for the usertypes. This is completely
straightforward; the members of each usertype is declared according to
the type mapping.

The immortal usertypes always map to structs here, even if they are
really unions. Consider a source declaration
\begin{verbatim}
union U { int a, int b } u ;
\end{verbatim}
This will map to these definitions in \Pgen:
\begin{verbatim}
struct U : Code {
  Code a;
  Code b;
};
/* ... */
struct U u;
\end{verbatim}
At specialzation time, a residual name will be assigned to /u/---say,
/u42/---and the fields /u.a/ and /u.b/ will contain the residual
expressions ``/u42.a/'' and ``/u42.b/''.

\subsection{Chapter IV: Names for dynamic structutes}
\label{sec:gegen:NamesForDynamicStructures}
Recall (section \ref{sec:gegen:TheTypeMapping}) that the \emph{tags}
of dynamic structs and unions are fixed at Gegen time. We cannot do
the same thing for the members, because the number of members in each
structure type need not be known at Gegen time. Consider
\begin{verbatim}
#include <limits.h>
struct T { int a[UCHAR_MAX], b; };
\end{verbatim}
where the binding times of T.a resolve to a statically indexed array
of dynamic values. The residual program should contain a structure
like
\begin{verbatim}
struct T {
  int a, a0, a1, a2, a3, a4, a5, a6, ... a254, b;
}
\end{verbatim}

Clearly, we can only generate names for these fields when we know
how many of them we need. The shadow header for /<limits.h>/ declares
/UCHAR_MAX/ to be a `well-known constant' of type /unsigned/
/char/. Gegen cannot know its numeric value, but of course it will
be available when \Pgen{} is compiled. This means that \Pgen{} will
have to keep track of the member names itself.

This chapter declares variables in \Pgen{} to hold these member names.
If struct T from the above example was the only immortal struct in the
program, the chapter would look like
\begin{verbatim}
static struct {
  struct {
    const char *a[UCHAR_MAX];
    const char *b;
  } T;
} cmixMember;
\end{verbatim}

The /cmixMember/ struct is there purely to conserve namespace. It
contains the name data for all the immortal structs. (If there are
none, the entire chapter is omitted in \Pgen).

The types for the second-level members of /cmixMember/ are formed
by a process analogous to the \Pgen{} type mapping, replacing /Code/
with /char/ /const/ /*/.

The initialization function chapter (section
\ref{sec:gegen:InitializationFunction}) contains code to generate
names for the residual fields and store them into /cmixMember/.

\subsection{Pseudochapter: For cascades}
\label{sec:gegen:ForCascades}
Throughout \Pgen{} the need often arises to do something to a \Pgen{}
object unless it is an array in which case the something should be
done to each of the array's elements.

One main case of this is the handling of the /Code/ objects that make
up the parts of objects with immortal types. Recall (the type mapping,
section \ref{sec:gegen:TheTypeMapping}) that an immortal type is
mapped in \Pgen{} to the /Code/ type inside zero or more levels of
statically indexed arrays.

This pseudochapter defines an idiom for iterating through levels of
arrays. Say we have an object /x/ whose \Pgen{} type is array[42] of
array[117] of /short/ and we want to add 5 to each of the /short/s.
Then we could emit
\begin{verbatim}
  { unsigned cmixI0 ;
    for(cmixI0=0; cmixI0<42; cmixI0++ ) {
      unsigned cmixI1 ;
      for(cmixI1=0; cmixI1<117; cmixI1++ ) {
        x[cmixI0][cmixI1] = x[cmixI0][cmixI1] + 5 ;
  } } }
\end{verbatim}

This pattern will be known as a \emph{for cascade}. If the type of /x/
had just been /short/ the same for cascade would degenerate into
simply
\begin{verbatim}
  x = x + 5;
\end{verbatim}

In the descriptions that follow, the possible presence of a for
cascade will usually just be hinted at, and both the above examples
would be written as \texttt{x = x + 5}.

\subsubsection{Implemenation notes}
\begin{itemize}
\item The /for(...)/ lines of a for cascade is emitted by the function
      /EmitForCascade()/. This function is given a binding-time
      annotated type $T$ and itself calculates how many levels of
      arrays $[T]$ contains. A global variable records this number so
      that subsequent calls to /EmitIndexCascade()/ write out the
      right number of indexes.
\item /EmitForCascade()/ takes care of the indentation of the
      /for(...)/ lines \emph{and} the body line. The indentation
      will be wrong if the body statement spans multiple lines.
\item For convenience, /EmitForCascade()/ returns the binding-time
      annotated type stripped of the array levels. (Any dynamically
      indexed array levels are still present).
\item For cascades cannot be nested.
\item The same index variables /cmixI0/, /cmixI1/, /cmixI2/,$\ldots$
      are used in each for cascade.
\end{itemize}

\subsection{Chapter V: Initialization functions for dynamic usertypes}
\label{sec:gegen:InitializationFunctionsForDynamicUsertypes}
This chapter (which partially serves as a pseudochapter too) concerns
itself with name management for the residual program and the
initialization of those /Code/ objects in \Pgen{} that represent
objects in the residual program.

The principal role of the /Code/ type is to represent residual
\emph{expressions}. However each of the objects that will exist in the
residual program is represented as a /Code/ object in \Pgen{}. When
viewed as an expressions, these representative /Code/ objects are
lvalue expressions that refer to the residual objects.

The strategy for the management of residual variable names is that
when a name is generated it is put into the appropritate
representative /Code/ object. It finds its way into actual residual
expressions when the representative /Code/ object is used for
constructing them.

Finding ``the appropriate representative /Code/s'' and installing the
freshly generated name
there can be more or less complicated, according to the type of the
variable:
\begin{description}
\item[/short/$_d$:] This is easy; there is a single /Code/ to worry
about. We can simply construct an expression referring to the residual
variable (/Code/ has a built-in conversion from /char*/) and assign
it into the representative /Code/.
\item[pointer$_d$ to $T$:] Same as the /short/$_d$ case.
\item[array$_d$ of $T$:] Same as the /short/$_d$ case.
\item[array$_s$ of something:] This case never occurs; we generate
different residual name for each of the array elements and install
them separately.
\item[struct$_s$ S:] Here is where the trouble is. \Pgen's version of
struct S is derived from /Code/ and need to be initialized; but in
addition to this the members of the structure are also representative
/Code/ objects and need to be initialized with member selection
expressions. This task is left over to the special initialization
functions that gives this chapter its name.
\end{description}

\subsubsection{Initialization functions}
For each immortal usertype a custom initialization function is
generated. Say we have an fully dynamic struct originally defined as
\begin{verbatim}
struct S {
  int a ;
  struct T *b ;
  struct T c ;
}
\end{verbatim}
The initialization function that gets generated for this is
\begin{verbatim}
static void cmix_putName(S &cmixThis,const Code &cmixIt) {
  (Code&)cmixThis = cmixIt ;
  cmixThis.a=cmixStruct(cmixIt,cmixMember.S.a);
  cmixThis.b=cmixStruct(cmixIt,cmixMember.S.b);
  cmix_putName(cmixThis.c,cmixStruct(cmixIt,cmixMember.S.c));
}
static inline void cmix_putName(S &cmixThis,const char *cmixIt)
  { cmix_putName(cmixThis,Code(cmixIt)); }
\end{verbatim}

All initialization functions are overloaded onto the single
name /cmix_putName/. /cmixStruct/ is an expression constructor
defined in /speclib.h/; it applies a ``dot'' operator to an
expression. /cmixMember/ is discussed in section
\ref{sec:gegen:NamesForDynamicStructures}. Note the different way
of (pseudo)recursively initializing the /c/ field; this is because it
is not plainly a /Code/ but a \Pgen{} version of another immortal
struct.

For members that are statically indexed arrays for cascades are insert.

\subsubsection{Naming a dynamic object}
This chapter also serves as pseudochapter: its implementation contains
a function that emits \Pgen{} code to generate a name and install it
properly. This way all decisions related to the initialization of
representative /Code/ objects are collected in a single chapter.

Initializing a representative object consists of constructing a name
based on the variable's original name (if the variable is generated by
\cmix, a pseudo-original name is invented) and installing it according
to the object's type. A typical piece of code is:
\begin{verbatim}
  cmix_putName(x,cmixNameman.freshLocalName("x"));
\end{verbatim}
with the obvious variations if /x/ is not a struct. Well, nearly
obvious. Instead of
\begin{verbatim}
  x = cmixNameman.freshLocalName("x");
\end{verbatim}
the current Gegen emits
\begin{verbatim}
  x.putVarName(cmixNameman.freshLocalName("x"));
\end{verbatim}
where /putVarName/ is a member function of /Code/ that essentially just
converts its argument to /Code/ and assigns it to /*this/.

If /x/ were a statically indexed array the initialization code would
be wrapped in a for cascade.

Depending on the name management strategy required for the variable
type, /cmixNameman.freshLocalName/ can be
/cmixNameman.freshGlobalName/ or can be totally absent for variables that
\emph{must} keep their original names in the residual program.

\subsubsection{Implementation notes}
\begin{itemize}
\item The generation of /cmix_putName()/ functions should be
straightforward.
\item The function /NameDynamicObject()/ emits the initialization code
described above, as well as any needed for cascade. Its second
parameter is the function to call to generate a new name based on
the original name.
\item The generation of pseudo-original names take place in
/C_Decl::get_name()/ which is defined in /coreC.cc/.
\item As a convenience feature, and to share ``for cascades'',
/NameDynamicObject()/ can also emit code that emits the definition
of the residual object. When this is wanted the third parameter
to /NameDynamicObject()/ is the name of the function in speclib that
should be called to emit the definition. Its calling interface should
be similar to /cmixGlobal/. When the third parameter is /NULL/ the
caller takes the responsibility of defining the variable in the
residual program.
\end{itemize}

\subsection{Chapter VI: Memoization helper functions}
This chapter consists of functions that are called by the
specialization library when it needs to follow the
pointers to mortal data in a spectime object.

\subsubsection{Speclib-Pgen interface conventions}
Each object
has an associated \emph{pointer following function} declared
like this:
\begin{verbatim}
bool cmix_follow117(cmixPointerCallback &,void *,unsigned);
\end{verbatim}
The /void*/
parameter to the pointer following function points to the object
itself, and the /unsigned/ parameter is the number of elements in
the object if the object is an array.

The /cmixPointerCallback/ class is defined in /speclib.h/ and
has an overloaded function call operator with a /void*/ argument,
returning /bool/. The
pointer following function is supposed to call that operator once
for every pointer to mortal data in the object. If the
/cmixPointerCallback/ ever returns /false/ the pointer following
function must return /false/ immediately---otherwise it must
return /true/.

\subsubsection{Anatomy of pointer following functions}
If there are any struct types in the program containing pointers to
mortal data, one overloaded helper function
\begin{verbatim}
static inline bool cmix_followH
  (cmixPointerCallback &cmixCallback,T *cmixSrc)
\end{verbatim}
is defined for each of these. Its job is to follow pointers in a
single object of that struct or union type.

For each member /x/ of the struct, the helper function contains:
\begin{verbatim}
  if (!cmix_followH(cmixCallback,&cmixSrc->x)) return false;
\end{verbatim}
\noindent if /x/ is itself a struct containing pointers to mortal
data, or
\begin{verbatim}
  if (!cmix_followH(cmixCallback,(T*)&cmixSrc->x)) return false;
\end{verbatim}
\noindent if /x/ is a union containing pointers to motal data. The
BTA should only let this happen if all members of the union are
structs, and if all pointers to mortal data are duplicated in every
member of the struct. /T/ is then the type of the first member of
the union, or
\begin{verbatim}
  if (!cmixCallback(cmixSrc->x)) return false;
\end{verbatim}
\noindent if /x/ is a pointer to mortal data, or \\
\emph{nothing} if none of these cases hold. Each of these
constructs is called a \emph{memoizer statement}.

After the /cmix_followH/ functions the primary pointer following
functions follow. They all look like
\begin{verbatim}
static bool cmix_follow117
    (cmixPointerCallback &cmixCallback, void *cmixVoid, unsigned cmixI) {
  T* const cmixSrc = (T*)cmixVoid ;
  while ( cmixI-- )
    .. memoizer statement for cmixSrc[cmixI] .. ;
  return true;
}
\end{verbatim}

Several objects can share the same pointer following function if they
have the same (binding-time annotated) type. The connection between
pointer following functions and data objects is made in later
chapters.

The specialization library predeclares the function /cmix_follow1()/
which is used for objects that do not contain any pointers to mortal
data.

\subsubsection{Implementation notes}
\begin{itemize}
\item The function /AnyPointersToFollow()/ investigates if a given
	declaration contains pointers that should be followed. If
	it does, it also sets up a couple of global variables that
	makes the next call to /EmitMemoizer()/ emit a memoizer
	statement for the type specified by the declaration.
\item /AnyPointersToFollow()/ relies on the /SimpleMemoization/
	annotation on struct declarations. This annotation is
	set up while generating the /cmix_followH/ functions;
	a process which itself uses the /AnyPointersToFollow()/
	function. It is essential for correct results that the
	struct declarations are processed in ``topological order''
	(cv. section \ref{sec:StructureDefinitionSorting}).
\item /AlocMemodata()/ is called for each object in the program
	and does several things:
	\begin{itemize}
	\item decides if the object is mortal or immortal
	\item assigns a sequential ID number to each mortal object.
		Immortal objects get the pseudo-ID 0.
	\item for each mortal object, decides if a previously
		generated pointer following function can be used for
		it. Generates one if not.
	\item annotates the object with the number of the pointer
		following function that fits it.
	\end{itemize}
	The second argument to /AlocMemodata/ is a list that the
	function builds of the objects that have had specialized
	pointer following functions generated. It is used to share
	pointer following functions between objects.
\item /AlocMemodata(NULL,NULL,NULL)/ initializes the object
	sequence counter.
\item /AlocMemodata(NULL,NULL,streamp)/ emits a definition of
	/cmixMWCount/ which is an array of /unsigned/ with as many
	elements as there are object IDs. The specialization
	library uses this for some arcane purpose.
\end{itemize}


\subsection{Chapter VII: Global variables}
This chapter contains declaration for the program's global variables.

Most variables are simply declared as
\begin{verbatim}
static T x;
\end{verbatim}
according to the type mapping,
but external variables that have been annotated /dangerous spectime:/ and
variables with /visible spectime:/ annotations need to be able to link
together with external C code that is linked into \Pgen. They are
defined as
\begin{verbatim}
extern "C" T x;    // x is dangerous spectime
\end{verbatim}
or
are declared as
\begin{verbatim}
extern "C" {
  T x;             // x is visible spectime
}
\end{verbatim}

Mortal variables\footnote{with the exception of \texttt{dangerous
spectime} variables} must be registered with the specialization
library's memoization engine. Immediately after each definition
a /cmixDataObject/ is constructed to take care of that. This looks
like
\begin{verbatim}
static cmixDataObject cmixDO1235(1234,&x,sizeof x,cmix_follow29, 1, false);
\end{verbatim}
/1233/ is the object's ID number in the specialization library (which is
one less than the ID number used internally in gegen; that number is
0 for objects that should not be memoized, but the specialization
library wants the numbers to start at 0).

The number /1/ is the number of array elements that should be given to
the pointer following function. It is always 1 even if the object is
an array, this number is different only for heap allocations where the
number of array elements is determined at spectime.

The last argument /false/ to the constructor signals that the control
object should not register itself as a local variable with the
specialization library.


After all global variables from the subject have been declared, a
constant array /cmix_globals/ containing pointers to all the
/cmixDataObject/s is defined. This array is used for referring to
the set og global variables in calls to the specialization library.

\subsubsection{Implementation notes}
\begin{itemize}
\item The /DeclareDO()/ function that emits the declaration of
	/cmixDataObjects/ is also used by the generator functions
	to declare local /cmixDataObject/s.
\end{itemize}

\subsection{Pseudochapter: Generator function conventions}
\label{sec:gegen:GeneratorFunctionConventions}
This pseudochapter specifies the calling conventions among generating
functions.

A generating function corresponds to a function definition in the
subject program. It is called when a call to that function needs to
be specialized. The generating function is given the mortal arguments
to the call. It can then either decide to re-use a previously
generated specialized version, in which case it reproduces the
memoized static side effect, or to specialize the function anew.

The generating function's return value is a /Code/ which represents
the \emph{name} of the residual function that shold be used when
specializing the call. It is the caller's job to then construct
a function call expression.

The parameters to a generating function are those of the original
function's parameters whose types are mortal. If the original
function's return type is mortal and different from /void/ an
additional parameter is appended \emph{in front} of the others.
The additional parameter has type ``pointer to the return type'', and
locates the place where the caller wants the return value deposited.

Generating functions are always /static/ in the \Pgen{} source.

\subsubsection{Implementation notes:}
\begin{itemize}
\item The pseudochapter's primary contribution is an /EmitType/
	deriviate called /GeneratorProto/ whose job is to generate
	prototypes for generating functions. These are used in
	chapters VIII and IX and need to be identical in those
	two chapters. /GeneratorProto/ does not really use its
	/EmitType/ ancestry for anything serious but might
	change to do it when the calling convention changes to
	accomodate inlining at some time in the future.
\item Located in this pseudochapter is also the function
	/EmitThingsForCall/ which produces the \Pgen equivalent
	of a Core C call statement. The intention of having the
	function defined here was that is must be changed when
	the generator function interface changes, so is would
	be smart to have the two pieces of code close to each
	other. Yeah, right.
\end{itemize}

\subsection{Chapter VIII: Function prototypes}
This chapter contains prototypes for every external function declared
in the subject program. We may not need to call all of them at
spectime, but it does not hurt to mention all of them. And we have
no analysis to show us which ones we do need.

External functions whose name is mentioned in a /well-known:/
directive are explicitly excluded from the prototype list. This
feature is used by shadow header files where we want the appropriate
real header to be included in Chapter I instead of having our own
prototypes written out. If the real header file defines the function
as a macro it could be disasterous to try to declare it explicitly.

Also, since the generating functions may be mutually recursive, we
also declare prototypes for all of them.

\subsection{Chapter XI: Generator functions}
Here at last we reach the goal of all the previous preparations:
the functions that actually specialize code.

There is one such function for each function in the subject program;
their calling conventions have been discussed in an earlier section
(\ref{sec:gegen:GeneratorFunctionConventions}).

\subsubsection{Sharing strategy}
Before going into the details of the generating function we shall
discuss the strategy for sharing residual functions between multiple
residual calls. There are three possible strategies:
\begin{itemize}
\item \textbf{Non-shared:} the function is only used once at the
	call that prompted its initial generation. Even if a
	perfect match for the arguments and static state later
	arises, the function is not reused.
\item \textbf{Semi-shared:} The function
	can be shared, but only for calls that are specialized
	after it has been fully specialized---\ie, recursive
	and indirecly recursive calls cannot be shared.
\item \textbf{Fully shared:} Even recursive calls can be shared.
	For this to be possible, the function must have no
	static side effects---when a call is shared any side
	effects must be reproduced, and naturally that can't
	happen until we actually know that all side effects have
	been \emph{performed} in the first place.
\end{itemize}
\cmix{} selects sharing strategy on a per-function basis.

The only functions that are currently handled as non-shared are those
that have a static return value different from /void/. Such a return
value is equivalent to a static side effect, so they can never be
fully shared. However, all it would take to make them semi-shared
would be working code to memoize their return value.

We now turn to functions without static return values.
Naturally it is unsafe to specify ``fully shared'' for
a function that actually contains static nonlocal side
effects. On the other hand, there is a cost to using the
semi-shared stategy: an attepmt to create a recursive
residual program will loop.

If would appear that we need an analysis that tells which functions
may cause static side effects. It turns out, however, that there is
a clever way to use existing BTA information to select a sharing
strategy:

Assume the specialization process actually tries
to create a recursive residual program. The execution path
from this call to the recursive call either has a dynamic
branch in it, or it has not. If there is no dynamic branch,
then the subject program would also loop. It is commonly
accepted for partial evaluators to loop when faced with
an infinte loop under purely static control.

Hence we can assume that there \emph{is} a dynamic branch
somewhere in the recursion. But that means that the entire
recursion will be classified as ``under dynamic control'', which
again means that this function will not have been \emph{allowed}
by the BTA to have any static nonlocal side-effects.

The net result is that when the function is under dynamic
control it is safe to use the fully-shared strategy, and
when the function is not under dynamic control it is safe
to use a semi-shared strategy.

A function that was not under dynamic control, yet could
be shown to be free from static nonlocal side-effects, can
safely use either strategy. However, the dynamic control
information is readily available the only cost of letting
the function be fully shared is a slight waste of time while
detecting that there were no static side effects that need
to be reproduced.

Therefore, we choose to rely on the dynamic control information
for selecting sharing strategy.

\subsubsection{The generation function preamble}
Now we describes what happen in the generating function until
we reach the part that is directly derived from the subject
function's flow graph.

\begin{description}
\item[Local variables] \Pgen{} versions of all the function's
	local variables are declared, according to the type
	mapping. /cmixDataObject/s are constructed for the
	mortal ones. The /cmixDataObject/ constructor automatically
	registers itself with the specialization library.
\item[Parameters] The mortal parameters have already been declared
	in the function headings, but still /cmixDataObject/s
	need to be constructed for them. \Pgen{} versions of
	the immortal parameters are declared, again according
	to the type mapping.
\item[Locals array] An array called /cmix_locals/ that contains
	pointers to the /cmixDataObject/s just constructed for
	mortal local variables and parameters, is constructed.
\item[State object] A /cmix_state/ referring to the local and
	global objects is constructed. This will be used when
	memoizing program points within the function.
\item[Second state object] Another state object /cmix_funmemo/
	that includes the global variables and arguments, but
	not the local variables, is declared. If there are
	no mortal local variables it is an alias for /cmix_state/;
	else it is constructed by skipping past the first elements
	of /cmix_locals/.

	This state object is used when memoizing
	the function call itself---thus function calls will not
	be prevented from being shared just because they start
	with different random garbage in their local variables.
\item[Function memoization] The speclib function /cmixFunctionSeenB4/
	is called to determine whether a previously specialized
	function can be used (the sharing strategy is not used
	here: non-sharable functions are simply not found because
	they are never entered into the memoization pool).

	If a suitable function is found, the /cmixDataObject/s are
	de-registered (/cmixPopObjects/) and the generating
	function returns.
\item[Name managers] A name manager /cmixLabelNM/ is created
	to take care of goto labels in the residual functions.

	A name manager /cmixLocalNames/ is created to take care
	of the names of local variables and parameters in the
	residual function. This name manager works together
	with the master name manager to ensure that local names
	avoid names that have been used globally and, conversely,
	that later global namings avoid names that have been used
	at the local level.
\item[Residual function headning] The /cmixPushFun()/ is called
	to create a new residual functions and install that as
	the target of subsequent residual code emissions. The
	specialization library also gets told the sharing mode
	at this point.

	The specialization library only supports function return
	types that can be written \emph{before} the function name.
	That is, the residual return type cannot be pointer to array
	or pointer to function. \textbf{We do not check this!}---\Pgen{}
	simply contains a literal string with an abstract declaration
	as the ``type'', and un-compilable residual programs result
	when something goes wrong here.
\item[Residual parameters] The immortal parameters of the function
	map to parameters of the residual function. They get a name
	from the local name manager and are entered in the residual
	function's parameter list by /cmixParam()/. See 
	\ref{sec:gegen:InitializationFunctionsForDynamicUsertypes}
	for more on naming dynamic objects.
\item[Residual local variables] The immortal residual variables
	are initialized in the same way as the parameters.
\item[The pending loop] We create a fresh pending list with
	/cmixPushPend()/ and insert the function's first
	basic block into it. Since it is also the first block
	to be pulled out again, its code will be at the start
	of the residual function, too.

	The pending loop uses a GCC extension that allows goto
	labels in \Pgen{} to be stored in /void*/ pointers and
	jumped to later. A more portable solution would be
	to identify basic blocks numerically and have a /switch/
	statement, but that could be slow on C compilers that
	don't create jump tables for switches.

	When there are no more pending points for the function,
	we pop off the current function, the local memoization
	object, and the pending list, and return the name of
	the newly finished residual function.
\end{description}

\subsubsection{The actual generating code}
Now comes the code where residual production code is generated and
spectime actions done, one basic block at a time.

Each basic block consists of a series of transformed statements
and a transformed jump.

\begin{description}
\item[Assignment statements] are transformed straightforwardly
	according to the expression mapping. A residual assignment
	is generated iff the assigned value has a immortal type.
\item[Call statements] come in several flavors:
	\begin{description}
	\item[external pure calls] have been resolved by the
		btdebug phase to either external spectime calls
		or external residual calls.
	\item[external spectime calls] are made at spectime.
		The return value can be ignored or assigned to
		either an immortal or a mortal variable.
	\item[external residual calls] are simply constructed
		and emitted to the residual code. Again, the
		return value can be ignored or assigned
		to an immortal variable.
	\item[specializable calls] are more complex. They always
		have their own compound statement in \Pgen, so there
		can be local variables in their specialization.

		If the called function returns a mortal
		value it wants to write it to an actual mortal
		variable. If we have not a mortal variable at
		hand (because the call statement specifies
		it should be ignored, or lifted and assigned to an immortal
		variable), we need a temporary location to copy
		it into. If necessary, this location is declared
		with the name /cmixSR/.

		After possibly declaring a /cmixSR/ the generating
		function is called with the mortal of the arguments,
		and its return value (a /Code/ representing the
		function the residual program should call) is saved
		in a temporary names /cmixF/.

		Then a residual function call is constructed, and
		if the return value is genuinely immortal and not
		ignored the function call makes up the right-hand side
		of a residual assignment---else it is just a residual
		expression statement.

		Finally, if the return value should be lifted, a
		residual assignment is generated containing the
		lifted value of /cmixSR/.
	\end{description}
\item[Call through function pointers] are not handled yet and result
	in an error message if they ever make it to the Gegen phase.
\item[Alloc statements] depend on the binding time of the allocated pointer:
	\begin{description}
	\item[dynamic pointer] always points to immortal data. The
		call to {\catcode`\/=11 malloc/calloc} is emitted
		to the residual function.
	\item[static pointer to immortal data] This allocation
		makes a heap allocation in \Pgen{} but a global
		variable in the residual program. The \Pgen{}
		counterpart of the object is heap allocated,
		but is then named with a name from the global
		name manager and registered as a global residual
		variable.

		\textbf{Possible error:}In effect the allocation
		becomes a statically indexed dynamic array. Do
		all the separate dynamic variables get initialized
		correctly?
	\item[static pointer to mortal data] The allocation is
		performed by a special specialization library helper
		routine that allocates the memory and constructs and
		registers a /cmixDataObject/ for memoizing its
		contents.
	\end{description}
\item[Free statements] are ignored unless freeing a dynamic pointer.
	In that case a call to /free()/ is constructed and emitted
	to the residual program.
\item[Conditional jumps] If the condition is static the jump is
	performed at spectime, bypassing the pending list and jumping
	directly to the next basic block.

	If the condition is dynamic a residual /if/ is generated by
	a speclib function that also searches for the two targets in
	the pending list and enters them if they are not found.
	Then control resumes to the pending loop.
\item[Unconditional jumps] normally bypasses the pending list: an
	unconditional jump results in \Pgen.

	However the added potential for code sharing that can result
	from memoizing unconditional jumps is sometimes needed so much
	that it makes up for the very significant spectime slowdown
	that it causes. \cmix{} contains a switch that makes it
	generate \Pgen{}s that memoizes unconditional jumps when they
	are under dynamic control---it is not safe to memoize jumps
	that are not under dynamic control: that might violate the
	guarantees of execution sequence we make about /spectime:/
	annotated external calls.
\item[Return statements] generate a residual return statement. If
	the function's return value is static an assignment to
	the caller's static return value is generated before returning
	control to the pending loop.
\end{description}

\subsubsection{Implementation notes}
\begin{itemize}
\item The code for the translation of call statements is located in
	the pseudochapter ``Generator function conventions''.
\end{itemize}

\subsection{Chapter X: Initialization function}
This chapter contains the functions /cmixGenInit()/ which initializes
various spectime data before specialization can begin, and
/cmixGenExit()/ which writes the final residual program to a file
\subsubsection{cmixGenInit}
\begin{description}
\item[Taboo words] The global name manager is initialized with a
	list of words that must not be used for autogenerated residual
	identifiers. The list includes:
	\begin{itemize}
	\item words from explicit /taboo:/ directives
	\item tags of enums and dynamic structs and unions
	\item names of external and visible global residual variables
	\item names of all external functions
	\item C keywords
	\end{itemize}

	\textbf{Possible Error!} Shouldn't the names of enum constants
	appear in the taboo list too?

\item[Speclib version check] The initialization function calls the
	function /cmixSpeclibVersion/. This name is defined as a
	macro in /speclib.h/, and its expansion changes each time
	the \Pgen-speclib interface is modified. It has already been
	checked that the /speclib.h/ that is included agrees with
	the generated \Pgen; this call checks that the compiled
	specialization library that is linked in agrees with the
	header file. A version mismatch causes a linker error if
	one tries to link with an out-of-date speclib.

	The /cmixSpeclibVersion()/ function itself does nothing.
\item[Member names for dynamic structures] The /cmixMember/ structure
	described in section \ref{sec:gegen:NamesForDynamicStructures}
	is filled with freshly generated member names.

	A name manager is created for each usertype. \textbf{This is
	probably wrong}: member names should avoid at least some
	taboo words which may be defined as macros in header files.

\item[Global varibales] Initialization for global variables. This is
	different according to whether the variable is mortal or
	immortal:
	\begin{description}
	\item[Mortal variables] have their controlling
		/cmixDataObject/ (which was declared along with the
		variable itself).
	\item[Immortal variables] first have names assigned as
		descibed in section
		\ref{sec:gegen:InitializationFunctionsForDynamicUsertypes}.
		The names are either generated by the global name
		manager or (for external and visible variables) simply
		identical to the original names.

		After that, and unless the variable is /well-known:/
		annotated, a declaration is emitted to the residual
		program by calling the /cmixGlobal/ function.
		This is slightly messy because the specialization
		library does not have any independen concept of
		storage modes, so the residual variable storage modes
		must be included in the string that speclib thinks
		is the type.

		Also, the variable may have an intializer which is
		given as an optional third argument to /cmixGloba()/.
	\end{description}
\item[Start-up code] At the end of /cmixGenInit/ is a possibly empty
	series of fully spectime Core C statements that initialize
	global spectime variables. Initializers for global variables
	can result either in initializers in the variable's definition
	or in a statement here. The Core C generator tries to minimize
	the use of statements in this block to initialize global
	variables, because immortal variables cannot be initialized
	here. (If a statement with immortal parts occurs here, Gegen
	exits with an error message).
\end{description}

\subsection{cmixGenExit}
The /cmixGenExit()/ function is called after specialization is
complete to write out the residual program is complete.
/cmixGenExit()/ itself contains a \emph{preabmle} which it writes
to the specified files. Then it calls the specialization library
to write out the residual code that has been collected.

The preamble for the residual program contains:
\begin{description}
\item[Business card] A statement that the residual program was
	automatically generated by \cmix.
\item[Header inclusions] All headers which have been mentioned
	by /header:/ directives.
\item[Usertypes] Definitions of immortal structs and unions with
	the field names computed by /cmixGenInit()/. Definitions
	for every enum type.
\item[External functions] Prototypes for all external functions
	in the program, except those mentioned i /well-known:/
	directives.
\end{description}



\subsection{Implementation level (1998-03-02)}
\label{sec:GEGENImplementationLevel}

This chapter has become somewhat out of date after the speclib
interface has been reworked. The same general principles still
apply, though.

\DeleteShortVerb{/}

\end{docpart}

%%% Local Variables: 
%%% mode: latex
%%% TeX-master: /cmixII/
%%% End: 



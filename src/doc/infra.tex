% $Id: infra.tex,v 1.3 1999/04/21 13:12:10 jpsecher Exp $

\providecommand{\docpart}{\renewenvironment{docpart}{}{}
\end{docpart}
\documentclass[twoside]{cmixdoc}
%\bibliographystyle{apacite}

\makeatletter
\@ifundefined{@title}{\title{\cmix-documentation}}{}
\@ifundefined{@author}{\author{The \cmix{} Team}}%
{\expandafter\def\expandafter\@realauthor\expandafter{\@author}%
\author{The \cmix{} Team\\(\@realauthor)}}
\makeatother

\AtBeginDocument{%
\markboth{\hfill\today\quad\timenow\hfill\llap{\cmix\ documentation}}
{\hfill\today\quad\timenow\hfill}}

\renewcommand{\sectionmark}[1]{\markboth
{\hfill\today\quad\timenow\hfill\llap{\cmix\ documentation}}
{\rlap{\thesection. #1}\hfill\today\quad\timenow\hfill}}

%\newboolean{separate}
%\setboolean{separate}{true}

\renewenvironment{docpart}{\begin{document}}%
                          {\bibliography{cmixII}\printindex
                           \end{document}}
\begin{document}\shortindexingon
}
\begin{docpart}

\begin{center}
  \fbox{\huge\textsl{THIS SECTION IS OUT-OF-DATE}}    
\end{center}

\section{Boring infrastructure details}
\label{sec:Infrastructure}

This section describes a number of infrastructure details that work
behind the scenes and support the analyses.

\subsection{Command line and option processing}

\subsection{Specializer directive parser}

\subsection{Connection between C and Core C representations}

\subsection{Source code positions}

\subsection{Error and Warning messages}

\subsection{Lists and sets}

One of these days some better documentation than the following
will be written.

Traversing a list is very close to running through an array. If you
\eg want to traverse a structure \syntax{decls} of type
\syntax{list<C\_Decl*>*}, you can do it by using an \emph{iterator} in
a simple \syntax{for}-loop:
\begin{verbatim}
    for(list<C_Decl*>::iterator i = decl->begin(); i++; i!=decl->end())
    {
         (*i)->get_name() ...
    }
\end{verbatim}

A macro is defined in \verb|auxilary.h| which shorens the above to
\begin{verbatim}
    FOREACH(i, decl, C_Decl*)
    {
         (*i)->get_name() ...
    }
\end{verbatim}

Observe that an iterator works as a pointer to an element of an array.
There are several operations on lists, whereof a few are listed in
figure~\ref{fig:listops}.

\begin{figure}[htbp]
  \begin{frameit}
    \leavevmode
    A list is parameterized over a type $T$.

    \noindent{\small
    \begin{tabular}{ll}
    \syntax{iterator  begin();}
      & Iterator to the first element. \\
    \syntax{iterator  end();}
      & Iterator to the last element. \\
    \syntax{unsigned size();} & \\
    \syntax{bool      empty();} & \\
    \syntax{reference front();} 
      & Reference to the first element. \\
    \syntax{reference back();}
      & Reference to the last element. \\
    \syntax{void      push\_front($T$*);}
      & Insert as first element. \\
    \syntax{void      push\_back($T$*);}
      & Insert as last element. \\
    \syntax{iterator  insert(iterator p, $T$*);}
      & Insert an element before position p. \\
    \syntax{void      splice(iterator p, list<$T$>\& x);}
      & Insert list x before position p. \\
    \end{tabular}}
  \caption{Some members of the \syntax{Plist} class.}
  \label{fig:listops}
  \end{frameit}
\end{figure}

\subsection{Associative arrays}

\subsection{Inverse Sets}
Inverse sets (or complement sets) are implemented by a regular set and
a boolean value that tells whether the set is inversed or not. The
union, intersection and set difference can be calculated as follows:
\[
\begin{array}{rclrclrcl}
  & \bigcup & & & \bigcap & & & \setminus \\
  \complement X \cup \complement Y &=& \complement(X \cap Y)
  & \complement X \cap \complement Y &=& \complement(X \cup Y) 
  & \complement X \setminus \complement Y &=& Y \setminus X \\
  \complement X \cup Y &=& \complement(X \setminus Y)
  & \complement X \cap Y &=& Y \setminus X 
  & \complement X \setminus Y &=& \complement(X \cup Y) \\
  X \cup \complement Y &=& \complement(Y \setminus X)
  & X \cap \complement Y &=& X \setminus Y 
  & X \setminus \complement Y &=& X \cap Y \\
\end{array}
\]
Inverse sets are used in the dataflow analysis.

\end{docpart}
%%% Local Variables: 
%%% mode: latex
%%% TeX-master: "cmixII"
%%% End: 

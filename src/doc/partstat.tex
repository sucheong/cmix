% File: partstat.tex

\providecommand{\docpart}{\renewenvironment{docpart}{}{}
\end{docpart}
\documentclass[twoside]{cmixdoc}
%\bibliographystyle{apacite}

\makeatletter
\@ifundefined{@title}{\title{\cmix-documentation}}{}
\@ifundefined{@author}{\author{The \cmix{} Team}}%
{\expandafter\def\expandafter\@realauthor\expandafter{\@author}%
\author{The \cmix{} Team\\(\@realauthor)}}
\makeatother

\AtBeginDocument{%
\markboth{\hfill\today\quad\timenow\hfill\llap{\cmix\ documentation}}
{\hfill\today\quad\timenow\hfill}}

\renewcommand{\sectionmark}[1]{\markboth
{\hfill\today\quad\timenow\hfill\llap{\cmix\ documentation}}
{\rlap{\thesection. #1}\hfill\today\quad\timenow\hfill}}

%\newboolean{separate}
%\setboolean{separate}{true}

\renewenvironment{docpart}{\begin{document}}%
                          {\bibliography{cmixII}\printindex
                           \end{document}}
\begin{document}\shortindexingon
}
\title{Binding Time Dependency Information}
\author{Arne John Glenstrup}
\begin{docpart}
\maketitle

\MakeShortVerb{"}


\begin{center}
  \fbox{\huge\textsl{THIS SECTION IS OUT-OF-DATE}}    
\end{center}

\section{Partially Static Data}
\label{sec:PartiallyStaticData}
\index{partially static data}
\index{data!static!partially}
\index{struct!splitting}
\index{array!splitting}

The binding time analysis may result in some complex data structures like
structs or arrays  being classified as \emph{partially static}, e.g.~some
members of a struct are static and some dynamic. In the case of arrays, the
length might be known (static), but the contents is unknown (dynamic). In
these cases it is desirable to remove the static parts of the data
structures, thus in effect \emph{changing their type}: structs in the
residual program have fewer members, and arrays are split up into
individual variables.

This separation of static and dynamic parts of data structures can be done
as a preprocessing before the actual specialization phase,
and~\cite[section~3.3.1]{Andersen:1997:StaticMemoryManagementInCMix}
points out that it is a key transformation for keeping the time spent
during the specialization phase low. He gives an example of the splitting
transformation:\bigskip

\noindent
\begin{minipage}[c]{.45\textwidth}
\begin{alltt}
struct P \{ int x, y; \}; /*\rlap{\hspace{6em}*/} \(S\x{D}\)     
                      
 
struct P pair;          /*\rlap{\hspace{6em}*/} \(S\x{D}\)


struct P *p;            /*\rlap{\hspace{6em}*/} \(*(S\x{D})\)


int a[42];              /*\rlap{\hspace{6em}*/} \([42]D\) 

struct P b[42];         /*\rlap{\hspace{6em}*/} \([42](S\x{D})\)


pair.x;
pair.y;
p->x;
p->y;
a[3];
b[4].x;
b[4].y;
\end{alltt}
\end{minipage}
\hfil
$\displaystyle\mathop{==>}\limits\sp{\mathit{split}}$
\hfil
\begin{minipage}[c]{.40\textwidth}
\begin{alltt}
struct P\(\sb{s}\) \{ int x; \};
struct P\(\sb{d}\) \{ int y; \};

struct P\(\sb{s}\) pair\(\sb{s}\);
struct P\(\sb{d}\) pair\(\sb{d}\);

struct P\(\sb{s}\) *p\(\sb{s}\);
struct P\(\sb{d}\) *p\(\sb{d}\);

int a0, a1, ..., a42;

struct P\(\sb{s}\) b0\(\sb{s}\), b1\(\sb{s}\), ..., b41\(\sb{s}\);
struct P\(\sb{d}\) b0\(\sb{d}\), b1\(\sb{d}\), ..., b41\(\sb{d}\);

pair\(\sb{s}\).x;
pair\(\sb{d}\).y;
p\(\sb{s}\)->x;
p\(\sb{d}\)->y;
a3;
b4\(\sb{s}\).x;
b4\(\sb{d}\).y;
\end{alltt}
\end{minipage}\medskip

Although it looks tempting to split arrays of known length in
a preprocessing phase, it is hard to see that it would be sensible
in practise: expressions like "d = a[s]" (where "s" is statically known but
not a $\Ppann$ constant) would have to be translated into
\begin{center}
"switch (s) { 0 : d = a0; break;  1 : d = a1; break;  ... ;  41 : d = a41; break; }"
\end{center}
in $\Ppann$, and furthermore translated into \coreC's
"if"-"goto" constructions, resulting in myriads of basic blocks! 
Instead,
we will content ourselves with letting the gegen phase_{phase!gegen}
translate staic-length arrays with dynamic content into
\begin{alltt}
/* \Ppgen \textnormal{\emph{source code}} */
int a0, a1, ..., a41;
Code a[42] = \{ \emit{a0}, \emit{a1}, ..., \emit{a41} \};

struct P\(\sb{s}\) b\(\sb{s}\)[42];
struct P\(\sb{d}\) b0\(\sb{d}\), b1\(\sb{d}\), ..., b41\(\sb{d}\);
Code b\(\sb{d}\)[42] = \{ \emit{b0\(\sb{d}\)}, \emit{b1\(\sb{d}\)}, ..., \emit{b41\(\sb{d}\)} \};

d = a[s];
s = b\(\sb{s}\)[4].x;
d = b\(\sb{d}\)[4].y;
\end{alltt}
where $\emit{\cdot} : \DObject -> \TCode$ is some function returning the
$\TCode$ representation of its argument.  Doing array splitting this way,
static-length arrays still end up as individual variables in the residual
program, and $\Ppann$ has not exploded.

\paragraph{Structs in functions returns.} When passing a partially static 
struct as a return value from a function call, somehow both the static and
dynamic struct part must be returned, but in C, only one value can be
returned. We therefore \emph{extend} \coreC_{Core C!extension beyond C}
\emph{beyond} C
so that the "return" statement always takes a tuple containing 
the static and dynamic return value. If either or both of these are empty,
"null" is passed as argument: \bigskip

\noindent
\begin{minipage}[c]{.30\textwidth}
\begin{alltt}
struct P pair; /* \(S\x{D}\) */


struct P f(...) \{
  ...
  return pair;
\}    
\end{alltt}
\end{minipage}
\hfil
$\displaystyle\mathop{==>}\limits\sp{\mathit{split}}$
\hfil
\begin{minipage}[c]{.55\textwidth}
\begin{alltt}
struct P\(\sb{s}\) pair\(\sb{s}\);
struct P\(\sb{d}\) pair\(\sb{d}\);

struct P f(...) \{
  ...
  return (pair\(\sb{s}\), pair\(\sb{d}\));
\}
\end{alltt}
\end{minipage}
\bigskip

\noindent
The tuple is eliminated during the gegen phase_{phase!gegen}, so $\Ppgen$
will of course contain legal C code.  But as this needs special treatment,
\emph{the $\Ppann$ splitting does not change function return
  types_{type!function return}_{type!splitting}.} We have also chosen that
\emph{function declarations_{declaration!function} are never
  split_{splitting!function} into two declarations}, as this could
dramatically increase the overhead in administrating the static
store_{store!static} in the specialization phase_{phase!specialization}.


\subsection{Splitting transformation}
\label{sec:PSDSplittingTransformation}

The two principles in splitting is that
\begin{enumerate}
\item every object (e.g.~variable, struct, array) should be either 
  static_{data!static} or dynamic_{data!dynamic} data.
\item every pointer that can point to partially static data is transformed
  into a struct with two members, "s" and "d", that are static pointers to
  static and dynamic data.
\item Assignments that copy both static and dynamic data are split into two 
  assignments.
\end{enumerate}
When we split a struct assignment, we might have to split recursively into
nested structs_{struct!nested}. 
If nested struct types are viewed as trees, we must perform the split such that
all static leaf nodes end up in the static substruct and all the dynamic
leaf nodes in the dynamic substruct.  
Figure~\vref{fig:SplittingNested} shows how a struct containing an array of
known length (3) of partially static structs is represented in the
generating extension.
\begin{figure}[htb]
  \begin{center}\lineskip 1em
\leavevmode
\setlength{\unitlength}{0.0025\textwidth}%
\begin{tabular}[b]{c}
\begin{picture}(160,90)(-60,-90)
\thinlines
\put(  0,  0){\circle{4}}
  \put(  0,  0){\line(-4,-3){40}}
  \put(  0,  0){\line( 0,-1){30}}
  \put(  0,  0){\line( 1,-1){30}}

\put(-40,-30){\circle{4}}
  \put(-40,-30){\line(-1,-3){10}}
  \put(-40,-30){\line( 1,-3){10}}
\put(  0,-30){\circle*{4}}
  {\thicklines
  \put(  0,-30){\line(-1,-3){10}}
  \put(  0,-30){\line( 1,-3){10}}}
\put( 30,-30){\circle{4}}
  \put( 30,-30){\line( 0,-1){30}}

\put(-50,-60){\circle{4}}
  \put(-50,-60){\line(-1,-3){10}}
  \put(-50,-60){\line( 1,-3){10}}
\put(-30,-60){\circle{4}}
\put(-10,-60){\circle*{4}}
  {\thicklines
  \put(-10,-60){\line(-1,-3){10}}
  \put(-10,-60){\line( 1,-3){10}}}
\put( 10,-60){\circle*{4}}
\put( 30,-60){\circle{4}}
  \put( 30,-60){\line( 1, 0){30}}
  \put( 30,-60){\line(-1,-3){10}}
  \put( 30,-60){\line( 1,-3){10}}
\put( 60,-60){\circle{4}}
  \put( 60,-60){\line( 1, 0){30}}
  \put( 60,-60){\line(-1,-3){10}}
  \put( 60,-60){\line( 1,-3){10}}
\put( 90,-60){\circle{4}}
  \put( 90,-60){\line(-1,-3){10}}
  \put( 90,-60){\line( 1,-3){10}}

\put(-60,-90){\circle{4}}
\put(-40,-90){\circle{4}}
\put(-20,-90){\circle*{4}}
\put(  0,-90){\circle*{4}}
\put( 20,-90){\circle{4}}
\put( 40,-90){\circle*{4}}
\put( 50,-90){\circle{4}}
\put( 70,-90){\circle*{4}}
\put( 80,-90){\circle{4}}
\put(100,-90){\circle*{4}}
\end{picture}
\\
$\Ppann$ struct $S$
\end{tabular}
\begin{tabular}[b]{c}{\small
\begin{picture}(160,90)(-60,-90)
\thinlines
\put(  0,  0){\circle{4}}
  \put(  0,  0){\line(-4,-3){40}}
  \multiput(  0,  0)( 0,-3){10}{\makebox(0,0){.}}
  \put(  0,  0){\line( 1,-1){30}}

\put(-40,-30){\circle{4}}
  \put(-40,-30){\line(-1,-3){10}}
  \put(-40,-30){\line( 1,-3){10}}
%\put(  0,-30){\circle*{4}}
  \multiput(  0,-30)(-1,-3){11}{\makebox(0,0){.}}
  \multiput(  0,-30)( 1,-3){11}{\makebox(0,0){.}}
\put( 30,-30){\circle{4}}
  \put( 30,-30){\line( 0,-1){30}}

\put(-50,-60){\circle{4}}
  \put(-50,-60){\line(-1,-3){10}}
  \put(-50,-60){\line( 1,-3){10}}
\put(-30,-60){\circle{4}}
%\put(-10,-60){\circle*{4}}
  \multiput(-10,-60)(-1,-3){11}{\makebox(0,0){.}}
  \multiput(-10,-60)( 1,-3){11}{\makebox(0,0){.}}
%\put( 10,-60){\circle*{4}}
\put( 30,-60){\circle{4}}
  \put( 30,-60){\line( 1, 0){30}}
  \put( 30,-60){\line(-1,-3){10}}
  \multiput( 30,-60)( 1,-3){11}{\makebox(0,0){.}}
\put( 60,-60){\circle{4}}
  \put( 60,-60){\line( 1, 0){30}}
  \put( 60,-60){\line(-1,-3){10}}
  \multiput( 60,-60)( 1,-3){11}{\makebox(0,0){.}}
\put( 90,-60){\circle{4}}
  \put( 90,-60){\line(-1,-3){10}}
  \multiput( 90,-60)( 1,-3){11}{\makebox(0,0){.}}

\put(-60,-90){\circle{4}}
\put(-40,-90){\circle{4}}
%\put(-20,-90){\circle*{4}}
%\put(  0,-90){\circle*{4}}
\put( 20,-90){\circle{4}}
%\put( 40,-90){\circle*{4}}
\put( 50,-90){\circle{4}}
%\put( 70,-90){\circle*{4}}
\put( 80,-90){\circle{4}}
%\put(100,-90){\circle*{4}}
\end{picture}}
\\
static $\Ppgen$ substruct $S_s$
\end{tabular}
\begin{tabular}[b]{c}
\begin{picture}(160,90)(-60,-90)
\thinlines
\put(  0,  0){\circle{4}}
  \multiput(  0,  0)(-2.5,-1.9){16}{\makebox(0,0){.}}
  \put(  0,  0){\line( 0,-1){30}}
  \put(  0,  0){\line( 1,-1){30}}

%\put(-40,-30){\circle{4}}
  \multiput(-40,-30)(-1,-3){11}{\makebox(0,0){.}}
  \multiput(-40,-30)( 1,-3){11}{\makebox(0,0){.}}
\put(  0,-30){\circle*{4}}
  {\thicklines
  \put(  0,-30){\line(-1,-3){10}}
  \put(  0,-30){\line( 1,-3){10}}}
\put( 30,-30){\circle{4}}
  \put( 30,-30){\line( 0,-1){30}}

%\put(-50,-60){\circle{4}}
  \multiput(-50,-60)(-1,-3){11}{\makebox(0,0){.}}
  \multiput(-50,-60)( 1,-3){11}{\makebox(0,0){.}}
%\put(-30,-60){\circle{4}}
\put(-10,-60){\circle*{4}}
  {\thicklines
  \put(-10,-60){\line(-1,-3){10}}
  \put(-10,-60){\line( 1,-3){10}}}
\put( 10,-60){\circle*{4}}
\put( 30,-60){\circle{4}}
  \put( 30,-60){\line( 1, 0){30}}
  \multiput( 30,-60)(-1,-3){11}{\makebox(0,0){.}}
  {\thicklines
  \put( 30,-60){\line(1,-3){10}}}
\put( 60,-60){\circle{4}}
  \put( 60,-60){\line( 1, 0){30}}
  \multiput( 60,-60)(-1,-3){11}{\makebox(0,0){.}}
  {\thicklines
  \put( 60,-60){\line( 1,-3){10}}}
\put( 90,-60){\circle{4}}
  \multiput( 90,-60)(-1,-3){11}{\makebox(0,0){.}}
  {\thicklines
  \put( 90,-60){\line( 1,-3){10}}}

%\put(-60,-90){\circle{4}}
%\put(-40,-90){\circle{4}}
\put(-20,-90){\circle*{4}}
\put(  0,-90){\circle*{4}}
%\put( 20,-90){\circle{4}}
\put( 40,-90){\circle*{4}}
%\put( 50,-90){\circle{4}}
\put( 70,-90){\circle*{4}}
%\put( 80,-90){\circle{4}}
\put(100,-90){\circle*{4}}
\end{picture}
\\
dynamic $\Ppgen$ substruct $S_d$
\end{tabular}
\begin{tabular}[b]{c}
\begin{picture}(160,90)(-60,-90)
\thicklines
\put(  0,  0){\circle*{4}}
  \multiput(  0,  0)(-2.5,-1.9){16}{\makebox(0,0){.}}
  \put(  0,  0){\line( 0,-1){30}}
  \put(  0,  0){\line( 1,-2){30}}
  \put(  0,  0){\line( 1,-1){60}}
  \put(  0,  0){\line( 3,-2){90}}

%\put(-40,-30){\circle{4}}
  \multiput(-40,-30)(-1,-3){11}{\makebox(0,0){.}}
  \multiput(-40,-30)( 1,-3){11}{\makebox(0,0){.}}
\put(  0,-30){\circle*{4}}
  {\thicklines
  \put(  0,-30){\line(-1,-3){10}}
  \put(  0,-30){\line( 1,-3){10}}}
%\put( 30,-30){\circle{4}}
  \multiput( 30,-30)( 0,-3){10}{\makebox(0,0){.}}

%\put(-50,-60){\circle{4}}
  \multiput(-50,-60)(-1,-3){11}{\makebox(0,0){.}}
  \multiput(-50,-60)( 1,-3){11}{\makebox(0,0){.}}
%\put(-30,-60){\circle{4}}
\put(-10,-60){\circle*{4}}
  {\thicklines
  \put(-10,-60){\line(-1,-3){10}}
  \put(-10,-60){\line( 1,-3){10}}}
\put( 10,-60){\circle*{4}}
\put( 30,-60){\circle*{4}}
  \multiput( 30,-60)( 3, 0){10}{\makebox(0,0){.}}
  \thicklines
  \multiput( 30,-60)(-1,-3){11}{\makebox(0,0){.}}
  \put( 30,-60){\line(1,-3){10}}
\put( 60,-60){\circle*{4}}
  \multiput( 60,-60)( 3, 0){10}{\makebox(0,0){.}}
  \multiput( 60,-60)(-1,-3){11}{\makebox(0,0){.}}
  \put( 60,-60){\line( 1,-3){10}}
\put( 90,-60){\circle*{4}}
  \multiput( 90,-60)(-1,-3){11}{\makebox(0,0){.}}
  \put( 90,-60){\line( 1,-3){10}}

%\put(-60,-90){\circle{4}}
%\put(-40,-90){\circle{4}}
\put(-20,-90){\circle*{4}}
\put(  0,-90){\circle*{4}}
%\put( 20,-90){\circle{4}}
\put( 40,-90){\circle*{4}}
%\put( 50,-90){\circle{4}}
\put( 70,-90){\circle*{4}}
%\put( 80,-90){\circle{4}}
\put(100,-90){\circle*{4}}
\end{picture}
\\
residual $\Ppres$ substruct $S_{\mathit{res}}$
\end{tabular}
    \caption{Splitting nested structures into static and dynamic components}
    \label{fig:SplittingNested}
  \end{center}
\end{figure}
Notice how a layer of indirection has been specialized away in the residual 
program: the constant-length array has been replaced by direct references. 

As we must change the type of some pointers we assume given a predicate,
$\FpsdPtr : \DE -> \Dbool$, to determine whether a pointers points to
partially static data. In general, we say a type $t$ is
\emph{splitable}_{type!splitable} if it contains both static and dynamic
parts. For instance, in Figure~\ref{fig:SplittingNested} only the topmost
node and its rightmost child are splittable. Note that a pointer is
\emph{not} splitable: it is either static, pointing to static/partially
static or dynamic data, or dynamic, pointing to dynamic data.

The output of the split phase is a program where each assignment copies
only fully static or fully dynamic values, where every parameter is either
fully static or fully dynamic, and where the types are binding-time
annotated with subscripts $s$ or $d$ according to with the following
notation:
\begin{center}
\begin{tabular}{ll}\label{tab:PSDannotatedTypes}
$\CPointer_s$   & $\CPointer_s^s$ or $\CPointer_s^d$ \\
$\CPointer_s^s$ & static pointer to static data, $\CArray_s^s$ or $\CUser_s^s$
\\
$\CPointer_s^d$ & static pointer to dynamic data, $\CArray_s^d$ or $\CUser_s^d$ \\
$\CArray_s^s$ & array of known length of static elements \\
$\CArray_s^d$ & array of known length of dynamic elements \\
$\CUser_s^s$ & static struct with only static members \\
$\CUser_s^d$ & static struct with only dynamic members \\
\end{tabular}
\end{center}

In the following, we define a number of split functions_{split function}
for various syntactical constructs:
\[
\begin{array}{lcl}
  \Fsplitt &:& \DT -> (\DT \x \DT + \DT \x \DNil + \DNil \x \DT) \\
  \Fsplitd &:& \DD -> (\DD \x \DD + \DD \x \DNil + \DNil \x \DD) \\
  \Fsplite &:& \DE -> (\DE \x \DE + \DE \x \DNil + \DNil \x \DE) \\
  \Fsplit_{=} &:& \DAssignment -> \DAssignment \DList \\
  \Fsplits &:& \DS -> \DS \DList \\
  \Fsplitj &:& \DJ -> \DJ \\
  \Fsplitb &:& \DB -> \DB \\
  \Fsplitu &:& \DU -> \DU \DList \\
  \Fsplitp &:& \DP -> \DP, \\
\end{array}
\]
where $\DAssignment = \{\CAssign(\delta,e), \CPAssign(\delta,e),
\CStrtAssign(\delta,e,\varphi), \CPStrtAssign(\delta,e,\varphi)  \}$ and
$\DNil = \{\CNil\}$.

The functions for splitting types is given
in Figure~\ref{fig:PDSSplitTypes}.%
\begin{figure}[htbp]
  \begin{center}
    \leavevmode\hbox{\vbox{%
    \begin{pseudocode}
      $\Fsplitt(t) = \Kif\ \Fbt(t) = D\ \Kthen\ (\CNil, t)\ 
      \Kelse$ \+\\
        $\Kcase\ t\ \Kof$ \+\\
          $\CInt(q), \CFloat(q), ... : (t, \CNil)$ 
              \\[1ex]
          $\CPointer(q, t') : {}$ \+\\
            $\Kcase\ \Fsplitt(t')\ \Kof$ \+\\
              $(t_s,t_d)$ \ \=:\ \=$(\CUser_s^s(q, \Fsplit_{\sigma}
                                    ([], t, t_s, t_d)), \CNil)$\\
              $(t_s,\CNil) $ \>:\> $(\CPointer_s^s(q, t_s), \CNil)$ \\
              $(\CNil,t_d) $ \>:\> $(\CPointer_s^d(q, t_d), \CNil)$ 
              \-\-\\[1ex]
          $\CArray(q, t', \Voe) : {}$ \+\\
            $\Kcase\ \Fsplitt(t')\ \Kof$ \+\\
              $(t_s,t_d) $ \ \= : \= 
              $(\CArray_s^s(q, t_s, \Voe), 
                \CArray_s^d(q, t_d, \Voe))$ \\
              $(t_s,\CNil) $\>:\>$(\CArray_s^s(q, t_s, \Voe), \CNil)$\\
              $(\CNil,t_d) $\>:\>$(\CNil, \CArray_s^d(q, t_d, \Voe))$ 
              \-\-\\[1ex]
          $\CFunT(q, t', t_1 ... t_n) : 
             (\CFunT(q, \Fsplitt(t'), \FsplitArgst(t_1, ..., t_n)),
              \CNil)$ 
           \\[1ex]
          $\CUser(q, \sigma, d_1 ... d_n) : $ \+\\
            $\Kcase\ \FsplitMems(d_1, ..., d_n)\ \Kof$ \+\\
              $(\Vss,\Vdd)$ \= : \= 
                $(\CUser_s^s(q, \sigma_s, \Vss), 
                  \CUser_s^d(q, \sigma_d, \Vdd))$ \\
              $(\Vss,[])$ \>:\> $(\CUser_s^s(q, \sigma, \Vss), \CNil)$ \\
              $([],\Vdd)$ \>:\> $(\CNil, \CUser_s^d(q, \sigma, \Vdd))$ 
              \-\-\\[1ex]
          $\CAbstract(q, \Vid) : 
            \Kif\ \Fbt(t) = D\ \Kthen\ (\CNil, t)\ \Kelse\ (t, \CNil)$
          \-\-\\[1em]
      $\FsplitMems(d_1, ..., d_n) = {}$ \+\\
        $\Kfor\ i \in \{1, ..., n\}\ \Kdo\
          (\Vss_i, \Vdd_i) := \Fsplitd(d_i)$\\
        $\Kreturn\ (\Fconcat [\Vss_1, ..., \Vss_n],
                    \Fconcat [\Vdd_1, ..., \Vdd_n])$ 
          \-\\[1em]
      $\FsplitArgs_{\chi}(x_1, ..., x_n) = {}$ \+\\
        $\Vas := []$ \\
        $\Kfor\ i = 1, ..., n\ \Kdo$ \+\\
          $\Kcase\ \Fsplit_{\chi}(x_i)\ \Kof$ \+\\
            $(x_{s}, x_{d})$ \ \= : \=
              $\Vas := \Vas \append [x_{s},x_d]$ \\
            $(x_s, \CNil)$  \>:\> $\Vas := \Vas \append [x_s]$ \\
            $(\CNil, x_d)$  \>:\> $\Vas := \Vas \append [x_d]$ \-\-\\
        $\Kreturn\ \Vas$ 
        \-\\[1em]
      $\Fsplit_{\sigma}(\Vms, t, t_s, t_d) = {}$ \+\\
        $\Kcase\ t\ \Kof$ \+\\
        $\CUser(q, \sigma, d_1 ... d_n) : {} $ \+\\
\quad\=\quad\=\quad\=\quad\=\quad\=\quad\=\quad\=\quad\=\quad\=\quad\=\kill
           $\Kif\ \sigma_{\Vms} = \CNull\ \Kthen$ \+\\
            $\sigma_{\Vms} := \Ffresh();\
             \delta_s := \Ffresh();\ \delta_d := \Ffresh()$ \\
            $(\sigma_{\Vms}, \Void, \CStruct_s,
            \begin{array}[t]{@{}l@{}}
              \CStructMem(\delta_s, "s", \CPointer_s^s(q, t_s)), \\
              \CStructMem(\delta_d, "d", \CPointer_s^d(q, t_d)))
            \end{array} $ \-\\
          $\Kreturn\ \sigma_{\Vms}$ \-\\
        $\CArray(q, t, e) : 
          \Fsplit_{\sigma}([m_1, ..., m_k, e], t, t_s, t_d)$
    \end{pseudocode}}}
    \caption{Splitting transformation for \coreC types}
    \label{fig:PDSSplitTypes}
  \end{center}
\end{figure}
Note that each time a pointer to a (possibly multidimensional) array of
partially static structs is encountered, we must generate a new struct type
definition for a struct with two static pointers to arrays of the static and
dynamic struct parts, e.g.: 
\begin{center}\def\d{\(\sb{d}\)}\def\s{\(\sb{s}\)}%
\leavevmode
\hbox to .3\textwidth{\hss$p$\hss\hss}\hfil
\hbox to .68\textwidth{\hss$\Ppsplit$\hss\hss}\nopagebreak[4]\smallskip
\nopagebreak[4]

\small
\begin{minipage}[t]{.3\textwidth}
\begin{alltt}
struct T \verb"{" int s; int d; \verb"}";







struct T  *t0;
struct T (*t1)[42];
struct T (*t2)[17];
struct T (*t3)[17][42];
struct T (*t4)[42];
\end{alltt}
\end{minipage}\hfil
\begin{minipage}[t]{.68\textwidth}
\begin{alltt}
struct T\s \verb"{" int s; \verb"}";
struct T\d \verb"{" int d; \verb"}";

struct T0    \verb"{" struct T\s  *s;          struct T\d  *d; \verb"}";
struct T42   \verb"{" struct T\s (*s)[42];     struct T\d (*d)[42]; \verb"}";
struct T17   \verb"{" struct T\s (*s)[17];     struct T\d (*d)[17]; \verb"}";
struct T1742 \verb"{" struct T\s (*s)[17][42]; struct T\d (*d)[17][42]; \verb"}";

struct T0    t0;
struct T42   t1;
struct T17   t2;
struct T1742 t3;
struct T42   t4;
\end{alltt}
\end{minipage}
\end{center}
For this reason, we keep for each user type $\sigma$ a set of split pointer
types indexed on the dimension sizes: $\sigma_{\Vms}$ where $\Vms =
[m_1,...,m_k]$. The function $\Fsplit_{\sigma}$ makes sure to reuse any
previously defined split pointer type (like "t1" and "t4" in the example
above)---otherwise assignments like "t1 = t4" would become illegal after
the split.

The split functions for declarations and expressions are
shown in
Figures~\ref{fig:PDSSplitDeclarations}--\ref{fig:PDSSplitExpressionsII}.%
\begin{figure}[htbp]
  \begin{center}
    \leavevmode\hbox{\vbox{%
    \begin{pseudocode}
      $\Fsplitd(d) = {}$ \+\\
        $\Kcase\ d\ \Kof$ \+\\
          $\CVar(\delta, \Void, t, \Vocs) : {}$ \+\\
            $\Kcase\ \Fsplitt(t)\ \Kof$ \+\\
              $(t_s, t_d) : $ \+\\
                $(\Vocs_s, \Vocs_d) := \FsplitInits(t,\Vocs)$ \\
                $\delta_s := \Ffresh();\ \delta_d := \Ffresh()$ \\
                $\Kreturn\ (\CVar_s(\delta_s, \Void, t_s, \Vocs_s),
                            \CVar_d(\delta_d, \Void, t_d, \Vocs_d))$ \-\\
              $(t_s,\CNil) : 
              (\CVar_s(\delta, \Void, t_s, \Vocs), \CNil)$ \\
              $(\CNil,t_d) : 
              (\CNil, \CVar_d(\delta, \Void, t_d, \Vocs))$ 
              \-\-\\[1ex]
          $\CFun(\delta, \Void, t, d_1 ... d_n, d'_1 ... d'_m,
                 b_1 ... b_k,) : {}$ \+\\
            $\CFun(\delta, \Void, t, \FsplitArgsd(d_1, ..., d_n),
            \FsplitArgsd(d'_1, ..., d'_m), 
            \Fsplitb(b_1) ... \Fsplitb(b_k))$
            \-\\[1ex]
          $\CStructMem(\delta, \Void, t, \Voc) : $ \+\\
            $\Kcase\ \Fsplitt(t)\ \Kof$ \+\\
              $(t_s, t_d) : $ \+\\
                $\delta_s := \Ffresh();\ \delta_d := \Ffresh()$ \\
                $(\CStructMem_s(\delta_s, \Void_s, t_s, \CNoConst),
                  \CStructMem_d(\delta_d, \Void_d, t_d, \CNoConst))$
                  \-\\
              $(t_s, \CNil) : 
                (\CStructMem_s(\delta, \Void_s, t_s, \Voc),
                 \CNil)$ \\
              $(\CNil, t_d) :  
                (\CNil, \CStructMem_d(\delta, \Void_d, t_d, \Voc))$ 
              \-\-\\[1ex]
          $\CEnumMem(\delta, \Vid, t, \Voc) : {}$ \+\\
            $\Kif\ \Fbt(\delta) = S\ \Kthen\
               (\CEnumMem_s(\delta, \Vid, t, \Voc), \CNil)\
             \Kelse\ \Kerror$
    \end{pseudocode}}}
    \caption{Splitting transformation for \coreC declarations}
    \label{fig:PDSSplitDeclarations}
  \end{center}
\end{figure}%
\begin{figure}[htbp]
  \begin{center}
    \leavevmode\hbox{\vbox{%
    \begin{pseudocode}
      $\FsplitInits(t, \Vcs) =
         \Kcase\ \FsplitInits'(t, \Vcs, [], [])\ \Kof\ 
         (\_\!\_, \Vcs_s, \Vcs_d) : (\Vcs_s, \Vcs_d)$ \\[1em]
      $\FsplitInits'(t, c:\Vcs, \Vcs_s, \Vcs_d) = {}$ \+\\
         $\Kcase\ t\ \Kof$ \+\\
           $\CInt(q), \CFloat(q), ... : 
            \Kif\ \Fbt(t) = S\ \Kthen\ 
              (\Vcs, \Vcs_s \append [c], \Vcs_d)\
            \Kelse\
              (\Vcs, \Vcs_s, \Vcs_d \append [c])$ 
              \\[1ex]
           $\CPointer(q,t') : $ \+\\
             $\Kcase\ \Fsplitt(t)\ \Kof$ \+\\
               $(\CUser(...), \CNil) : {}$ \+\\
                 $\Kcase\ c\ \Kof$ \+\\
                   $\CBinary(e_1, \pm, e_2)$ \= : \= \kill
                   $\CRval(\delta)$ \> : \> 
                     $(\Vcs, \Vcs_s \append 
                     [\CRval(\delta)], \Vcs_d)$ \\
                   $\CLval(\delta)$ \> : \>
                     $(\Vcs, \Vcs_s \append 
                     [\CLval(\delta_s),
                      \CLval(\delta_d)], \Vcs_d)$ \\
                   $\CConst("NULL")$ \> : \>
                     $(\Vcs, \Vcs_s \append ["NULL", "NULL"],
                             \Vcs_d)$ \\
                   $\CUnary("&", \CRval(\delta))$ \> : \> 
                     $(\Vcs, \Vcs_s  \append 
                     [\CUnary("&", \CRval(\delta_s)),
                      \CUnary("&", \CRval(\delta_d))], \Vcs_d)$ \\
                   $\CUnary(\Vuop, e_1)$ \> : \> ? \\
                   $\CBinary(e_1, \pm, e_2)$ \> : \>
                     $(\Vcs, \Vcs_s \append 
                        [
                        \begin{array}[t]{@{}l@{}}
                          \CBinary(\CMember(e_1, "s"), \pm, e_2), \\
                          \CBinary(\CMember(e_1, "d"), \pm, e_2)]),
                          \Vcs_d)
                        \end{array}$ \\
                   $\CCast(t, e_1)$ \> : \> ? \\
                   $\CMember(e_1, \varphi)$ \> : \> ? \-\-\\
               $(\CPointer_s, \CNil) :
                  (\Vcs, \Vcs_s \append [c], \Vcs_d)$ \\
               $(\CNil, \CPointer_d) :
                  (\Vcs, \Vcs_s, \Vcs_d \append [c])$ \-\-
             \\[1ex]
           $\CArray(q,t',e) : {}$ \+\\
             $\Vcs := c:\Vcs$\\
             $\Kfor\ i = 1, ..., \Fvalue(e)\ \Kdo\
               (\Vcs, \Vcs_s, \Vcs_d) := 
                 \FsplitInits'(t', \Vcs, \Vcs_s, \Vcs_d)$ \\
             $\Kreturn\ (\Vcs, \Vcs_s, \Vcs_d)$ 
             \-\\[1ex]
%           $\CFunT(q, t, t_1 ... t_n) : 
%            \CFunT(q, t, \FsplitArgs(t_1, ..., t_n))$ 
%           \\[1ex]
           $\CUser(q, \sigma, d_1 ... d_n) : {}$ \+\\
             $\Kfor\ \varphi = 1, ..., n\ \Kdo\
               (\Vcs, \Vcs_s, \Vcs_d) := 
                 \FsplitInits'(t_{d_\varphi}, \Vcs, \Vcs_s, \Vcs_d)$ \\
             $\Kreturn\ (\Vcs, \Vcs_s, \Vcs_d)$ 
             \-\\[1ex]
           $\CAbstract(q, \Vid) : ???$
    \end{pseudocode}}}
    \caption{Splitting transformation for \coreC initializers}
    \label{fig:PDSSplitInitializers}
  \end{center}
\end{figure}%
\begin{figure}[htbp]
  \begin{center}\leavevmode\hbox{\vbox{%
    \begin{pseudocode}
      $\Fsplite(e) = \Kif\ \Fbt(e) = D\ \Kthen\ (\CNil, e)\ \Kelse$ \+\\
        $\Kcase\ e\ \Kof$ \+\\
          $\CRval(\delta) : $ \+\\
            $\Kcase\ \Fsplitt(t_\delta)\ \Kof$ \+\\
              $(t_s,t_d)$ \ \= : \= $(\CRval(\delta_s), \CRval(\delta_d))$ \\
              $(t_s,\CNil)$ \> : \> $(\CRval(\delta), \CNil))$ \\
              $(\CNil,t_d)$ \> : \> $(\CNil, \CRval(\delta))$
              \-\-\\[1ex]
\quad\=\quad\=\quad\=\quad\=\quad\=\quad\=\quad\=\quad\=\quad\=\quad\=\kill
          $\CLval(\delta) : $ \+\\
            $\Kcase\ \Fsplitt(t_\delta)\ \Kof$ \+\\
              $(t_s,t_d) : {}$ \+\\
                $\Vtemp := \Ffresh()$ \\
                $\VpreStmt := \VpreStmt \append
                [{\Vtemp}".s = "\CLval(\delta_s),\
                 {\Vtemp}".d = "\CLval(\delta_d)]$ \\
                $\Kreturn\ (\CRval(\delta_{\Vtemp}), \CNil)$ \-\\
              $(t_s,\CNil)$ \= : \= $(\CLval_s(\delta), \CNil)$ \\
              $(\CNil,t_d)$ \> : \> $(\CLval_s(\delta), \CNil)$
              \-\-\\[1ex]
\quad\=\quad\=\quad\=\quad\=\quad\=\quad\=\quad\=\quad\=\quad\=\quad\=\kill
          $\CConst(c) : {}$ \+\\
            $\Kif\ \FpsdPtr(e)\ \Kthen$ \+\\
               $(\CConst_s("NULL"_{\Vpsd}), \CNil)$ \-\\
             $\Kelse$ \+\\
               $\Kif\ \Fbt(e) = D\ \Kthen\ 
                  (\CNil, \CConst_d(c))\
                \Kelse\
                  (\CConst_s(c), \CNil)$ 
              \-\-\\[1ex]
          $\CUnary("*",e_1) : {}$ \+\\
            $\Kif\ \FpsdPtr(e_1)\ \Kthen$ \+\\
              $(\CUnary_s("*", \CMember(e_1, "s")), 
                \CUnary_s("*", \CMember(e_1, "d")))$ \-\\
            $\Kelse$ \+\\
              $\Kcase\ \Fsplite(e_1)\ \Kof$ \+\\
                $(e_s, \CNil)$ \=: \= $(\CUnary_s("*", e_s), \CNil)$ \\
                $(\CNil, e_d)$ \>: \> $(\CNil, \CUnary_d("*", e_d))$ 
              \-\-\-\\[1ex]
          $\CUnary("&",e_1) : $ \+\\
            $\Kcase\ \Fsplite(e_1)\ \Kof$ \+\\
              $(e_s,e_d) : {}$ \+\\
                $\Vtemp := \Ffresh()$ \\
                $\VpreStmt := \VpreStmt \append
                [{\Vtemp}".s = "\CUnary_s("&", e_s),\
                 {\Vtemp}".d = "\CUnary_s("&", e_d)]$ \\
                $\Kreturn\ (\CRval(\delta_{\Vtemp}), \CNil)$ \-\\
              $(e_s, \CNil)$ \=: \= $(\CUnary_s("&", e_s), \CNil)$ \\
              $(\CNil, e_d)$ \>: \> $(\CUnary_s("&", e_d), \CNil)$ 
              \-\-\\[1ex]
          $\CUnary(\Vuop,e_1) : $ \+\\
            $\Kcase\ \Fsplite(e_1)\ \Kof$ \+\\
              $(e_s,e_d)$ \ \= : \= 
                $(\CUnary_s(\Vuop, e_{s}), \CUnary_d(\Vuop, e_{d}))$ \\
              $(e_s, \CNil)$ \>:\> $(\CUnary_s(\Vuop, e_s), \CNil)$ \\
              $(\CNil, e_d)$ \>:\> $(\CNil, \CUnary_d(\Vuop, e_d))$ 
              \-\-\\[1ex]
    \end{pseudocode}}}
    \caption{Splitting transformation for \coreC expressions, part I}
    \label{fig:PDSSplitExpressionsI}
  \end{center}
\end{figure}%
\begin{figure}[htbp]
  \begin{center}\leavevmode\hbox{\vbox{%
    \begin{pseudocode}
          $\CBinary(e_1,\Vbop,e_2) : $ \+\\
            $\Kif\ \Vbop = {\pm} \ \land \ {\FpsdPtr}(e_1)\ \Kthen\ 
              \Kcase\ (\Fsplite(e_1), \Fsplite(e_2))\ \Kof\
              ((e'_1, \CNil), (e'_2, \CNil)) : {}$ \+\\
              $\Vtemp := \Ffresh()$ \\
              $\VpreStmt := \VpreStmt \append
              [\Vtemp'" = "e'_2,\
               {\Vtemp}".s = "e'_1".s + "{\Vtemp'},\
               {\Vtemp}".d = "e'_1".d + "{\Vtemp'}]$ \\
              $\Kreturn\ (\CRval_s(\delta_{\Vtemp}), \CNil)$ \-\\
            $\Kelse\ 
             \Kif\ \Vbop \in \{"-","==","!="\} \ \land \ 
             {\FpsdPtr}(e_1) \land {\FpsdPtr}(e_2)\ \Kthen\ $ \+\\
              $\Kcase\ (\Fsplite(e_1), \Fsplite(e_2))\ \Kof$ \+\\
                 $((e'_1, \CNil), (e'_2, \CNil)) : 
                  (\CBinary_s(\CMember(e'_1, "s"), \Vbop,
                               \CMember(e'_2, "s")), \CNil)$ \-\-\\
            $\Kelse$ \+\\
              $\Kcase\ (\Fsplite(e_1), \Fsplite(e_2))\ \Kof$ \+\\
                $((e_{1s}, \CNil), (e_{2s}, \CNil))$ \= : \=
                  $(\CBinary_s(e_{1s}, \Vbop, e_{2s}), \CNil)$ \\
                $((\CNil, e_{1d}), (\CNil, e_{2d}))$ \> : \> 
                  $(\CNil, \CBinary_d(e_{1d}, \Vbop, e_{2d}))$ \\
                $((\CNil, e_{1d}), (e_{2s}, \CNil))$ \> : \>
                  $(\CNil, \CBinary_d(e_{1d}, \Vbop, \Flift(e_{2s})))$ \\
                $((e_{1s}, \CNil), (\CNil, e_{2d}))$ \> : \> 
                  $(\CNil, \CBinary_d(\Flift(e_{1s}), \Vbop, e_{2d}))$ 
                \-\-\-\\[1ex]
          $\CCast(t,e_1) : $ \+\\
            $\Kcase\ (\Fsplitt(t), \Fsplite(e_1))\ \Kof$ \+\\
              $((t_s, t_d), (e_s, e_d))$ \quad \= : \= 
                 $(\CCast_s(t_s, e_s), \CCast_d(t_d, e_d))$ \\
              $((t_s, \CNil), (e_s, \CNil))$ \>:\> 
                $(\CCast_s(t_s,e_s), \CNil)$ \\
              $((\CNil, t_d), (\CNil, e_d))$ \>:\> 
                $(\CNil, \CCast_d(t_d,e_d))$
                \-\-\\[1ex]
\quad\=\quad\=\quad\=\quad\=\quad\=\quad\=\quad\=\quad\=\quad\=\quad\=\kill
          $\CMember(e_1,\varphi) : $ \+\\
            $\Kcase\ \Fsplite(e_1)\ \Kof$ \+\\
              $(e_s, e_d) : $ \+\\
                $\Kcase\ \Fsplitt(t_{(e_1.\varphi)})\ \Kof$ \+\\
                  $(t_s,t_d)$ \ \= : \= 
                    $(\CMember_s(e_s, \varphi), 
                      \CMember_d(e_d, \varphi))$ \\
                  $(t_s,\CNil)$ \> : \> 
                    $(\CMember_s(e_s, \varphi), \CNil)$ \\
                  $(\CNil,t_d)$ \> : \> 
                    $(\CNil, \CMember_d(e_d, \varphi))$ 
                    \-\-\\[1ex]
              $(e_s, \CNil) : $ \+\\
                $\Kcase\ \Fsplitt(t_{(e_1.\varphi)})\ \Kof$ \+\\
                  $(t_s,\CNil) : (\CMember_s(e_s, \varphi), \CNil)$
                    \-\-\\[1ex]
              $(\CNil, e_d) : $ \+\\
                $\Kcase\ \Fsplitt(t_{(e_1.\varphi)})\ \Kof$ \+\\
                  $(\CNil,t_d) : (\CNil, \CMember_d(e_d, \varphi))$
    \end{pseudocode}}}
    \caption{Splitting transformation for \coreC expressions, part II}
    \label{fig:PDSSplitExpressionsII}
  \end{center}
\end{figure}%

Function $\Fsplite$_{splite@$\Fsplite$} takes an expression $e$ and returns
either
\begin{enumerate}
\item a  pair of expressions $(e_s, e_d)$ if $e$ has a splitable type, or
\item a pair  $(e_s, \CNil)$ if $e$'s type is not splitable and
  is static, or
\item a pair $(\CNil, e_d)$ if $e$'s type is not splitable and
  is dynamic.
\end{enumerate}
The $e_s$ and $e_d$ components are the expressions by which the static and
dynamic parts of $e$ should be referenced. For instance,
\begin{center}
  $\Fsplite("t") = ("t"_s,"t"_d)$ \qquad 
  $\Fsplite("t.s") = ("t"_s".s", \CNil)$ \qquad
  $\Fsplite("t.d") = (\CNil,"t"_d".d")$,
\end{center}
assuming "t" is a struct with staic member "s" and dynamic member "d". 

Note that in the cases where an expression is a pointer to partially static
data (by the \& operator, by adding an integer to a pointer or by the
constant "NULL"), we must create a new temporary struct containing the two
versions of the pointer. This is done by adding some struct assignment
statements as a side effect to a list, $\VpreStmt$.


The
split functions for types and declarations work analogously.

The function $\Fsplit_{=}$_{split=@$\Fsplit_{=}$} shown in
Figure~\vref{fig:PDSSplittingAssignments} splits any of the four types of
assignment operations.%
\begin{figure}[htb]
  \begin{center}\leavevmode\hbox{\vbox{%
      \begin{pseudocode}
$\Fsplit_{=}[[e" = "e']] = {}$ \+\\
  $\Kcase\ \Fsplite(e')\ \Kof$ \+\\
    $(e'_s,e'_d)$\ \= : \= $[e\{\delta:=\delta_s\}" ="_s" "e'_s,
                             e\{\delta:=\delta_d\}" ="_d" "e'_d]$ \\
    $(e'_s,\CNil)$ \> : \> $[e\{\delta:=\delta_s\}" ="_s" "e'_s]$ \\
    $(\CNil,e'_d)$ \> : \> $[e\{\delta:=\delta_d\}" ="_d" "e'_d]$ \\
      \end{pseudocode}}}
    \caption{Algorithm for splitting assignments into static and dynamic
      parts. `$e\{\delta:=\delta'\}$' denotes substitution.} 
    \label{fig:PDSSplittingAssignments}
  \end{center}
\end{figure}

These functions can now be used to define splitting functions for
statements, control statements and basic blocks in
Figure~\ref{fig:PDSSplitStatementsBBs}.
\begin{figure}[htbp]
  \begin{center}
\[
    \begin{array}{lll}
      \textnormal{Statement } s       & \Fsplits(s) \\\hline
      \CAssign(\delta,e)              & \VpreStmt \append 
                                        \Fsplit_{=}[[\delta" = "e]] \\
      \CPAssign(\delta,e)             & \VpreStmt \append 
                                        \Fsplit_{=}[["*"\delta" = "e]] \\
      \CStrtAssign(\delta,e,\varphi)  
         & \VpreStmt \append 
           \Fsplit_{=}[[\delta"."\varphi" = "e]] \\
      \CPStrtAssign(\delta,e,\varphi) 
         & \VpreStmt \append 
           \Fsplit_{=}[["(*"\delta")."\varphi" = "e]] \\
      \CCall(\optdelta,\delta,e_1...e_n)
         & \VpreStmt \append 
           [\CCall(\optdelta,\delta, \FsplitArgse(e_1,...e_n))] \\
      \CPCall(\optdelta,\delta,e_1...e_n)
         & \VpreStmt \append 
           [\CPCall(\optdelta,\delta, \FsplitArgse(e_1,...,e_n))] \\
      \CMalloc(\delta,t)
         & [\CMalloc(\delta_s,t_s), \CMalloc(\delta_d,t_d)], 
         & \textnormal{if $t$ splitable} \\
      \CMalloc(\delta,t)
         & [\CMalloc(\delta,t)], 
         & \textnormal{else}\\
      \CCalloc(\delta,t,e)            
         & [\CCalloc(\delta_s,t_s,e), \CCalloc(\delta_d,t_d,e)],
         & \textnormal{if $t$ splitable} \\
      \CCalloc(\delta,t,e)            
         & [\CCalloc(\delta,t,e)],
         & \textnormal{else} \\
      \CFree(e)                       & ??? \\
      \CSequence                      & ??? 
      \\[1em]
      \textnormal{Control statement } j & \Fsplitj(j) \\\hline
      \CIf(e,l_1,l_2)   & \CIf(\tilde{e},l_1,l_2) \\
      \CGoto(l)         & \CGoto(l) \\
      \CReturn(\CNoExp) & \CReturn(\CNoExp, \CNoExp) \\
      \CReturn(e)  & \CReturn(\Fsplite(e)) \\[1ex]
      \mc{3}{l}{\textnormal{where } \tilde{e} = 
        \begin{array}[t]{@{}l@{}}
          \Kcase\ \Fsplite(e)\ \Kof \\
          \quad (e_s, e_d)   : \Kerror \\
          \quad (e_s, \CNil) : e_s \\
          \quad (\CNil, e_d) : e_d \\
        \end{array}}\\
      \\
      \textnormal{Basic block } b & \Fsplitb(b) \\\hline
      (l, \Void,s_1...s_n,j) &
      \mc{2}{l}{
        (l, \Void, \Fconcat(\Fsplits(s_1), ..., \Fsplits(s_n),
        \VpreStmt_{\mathcal{J}}),
        \Fsplitj(j))}
      \\[1em]
      \textnormal{User type definition } u & \Fsplitu(u) \\\hline
       (\sigma, \Void, \CEnum, d_1...d_n)  &
      [(\sigma, \Void, \CEnum, d_1...d_n)] \\
       (\sigma, \Void, \CUnion, d_1...d_n)  & 
      [(\sigma, \Void, \CUnion_s, d_1...d_n)]
      & \textnormal{if }\Fbt(d_1) = S \\
       (\sigma, \Void, \CUnion, d_1...d_n)  & 
      [(\sigma, \Void, \CUnion_d, d_1...d_n)]
      & \textnormal{if }\Fbt(d_1) = D \\
       (\sigma, \Void, \CStruct, d_1...d_n) &
      [(\sigma_s, \Void_s, \CStruct_s, \Vss),
       (\sigma_d, \Void_d, \CStruct_d, \Vdd)] 
      & \textnormal{if $u$ splitable} \\ 
       (\sigma, \Void, \CStruct, d_1...d_n) &
      [(\sigma, \Void, \CStruct_s, \FsplitArgsd(d_1, ..., d_n))] & 
      \textnormal{else if } \Fbt(d_1) = S \\ 
       (\sigma, \Void, \CStruct, d_1...d_n) &
      [(\sigma, \Void, \CStruct_d, \FsplitArgsd(d_1, ..., d_n))] & 
      \textnormal{else} \\[1ex]
      \mc{3}{l}{\textnormal{where } 
        (\Vss,\Vdd) = \FsplitMems(d_1, ..., d_n)}
      \\[1em]
      \textnormal{Program } p & \Fsplitp(p) \\\hline
      (
      \begin{array}[t]{@{}l@{}}
        u_1 ... u_n, d^v_1 ... d^v_m, \\
        d^f_1 ... d^f_l, d^{ve}_1 ... d^{ve}_k, \\
        d^{fe}_1 ... d^{fe}_j, b_1 ... b_h)
      \end{array}
      & \mc{2}{l}{
        \begin{array}[t]{@{}l@{}}
          \Klet\\
          \quad \Vus := \Fsplitu(u_1) \append \cdots 
                                      \append \Fsplitu(u_n) \\
          \quad \Vds^v := \FsplitArgsd(d^v_1, ..., d^v_m) \\
          \quad \Vds^f := \FsplitArgsd(d^f_1, ..., d^f_m) \\
          \quad \Vus':= [(\sigma_{\Vms}, \Void_{\sigma_{\Vms}},
                        \CStruct_s, d_s, d_d) \mid
            \sigma_{\Vms} \ne \CNull ] \\
          \Kin\\
          \quad (\Vus \append \Vus', \Vds^v, \Vds^f, 
          d^{ve}_1 ... d^{ve}_k, d^{fe}_1 ... d^{fe}_j, 
                 \Fsplitb(b_1) ...\Fsplitb(b_1))
        \end{array}}
      \end{array}
    \]
    \caption{Splitting transformation for \coreC statements, control
      statements,  basic blocks, userdefined types and programs}
    \label{fig:PDSSplitStatementsBBs}
  \end{center}
\end{figure}%
Function $\Fsplits$_{splits@$\Fsplits$} takes a statement and returns a
list of statements that perform the same operation, but split into totally
static_{assignment!totally static} and totally dynamic_{assignment!totally
  dynamic} operations.


\subsection{Implementation level (22.04.1998)}
\label{sec:PSDImplementationLevel}

The splitting algorithms in this chapter have not been implemented in
\cmix yet.

\end{docpart}

%%% Local Variables: 
%%% mode: latex
%%% TeX-master: "cmixII"
%%% End: 

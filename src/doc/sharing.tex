% File: sharing.tex
% Time-stamp: 

\providecommand{\docpart}{\renewenvironment{docpart}{}{}
\end{docpart}
\documentclass[twoside]{cmixdoc}
%\bibliographystyle{apacite}

\makeatletter
\@ifundefined{@title}{\title{\cmix-documentation}}{}
\@ifundefined{@author}{\author{The \cmix{} Team}}%
{\expandafter\def\expandafter\@realauthor\expandafter{\@author}%
\author{The \cmix{} Team\\(\@realauthor)}}
\makeatother

\AtBeginDocument{%
\markboth{\hfill\today\quad\timenow\hfill\llap{\cmix\ documentation}}
{\hfill\today\quad\timenow\hfill}}

\renewcommand{\sectionmark}[1]{\markboth
{\hfill\today\quad\timenow\hfill\llap{\cmix\ documentation}}
{\rlap{\thesection. #1}\hfill\today\quad\timenow\hfill}}

%\newboolean{separate}
%\setboolean{separate}{true}

\renewenvironment{docpart}{\begin{document}}%
                          {\bibliography{cmixII}\printindex
                           \end{document}}
\begin{document}\shortindexingon
}
\begin{docpart}

\section{Function sharing}
\label{sec:Sharing}\index{Sharing}

The code sharing analysis computes a set of functions that should be
inlined by the generating extention. This can have
various reasons, e.g. they contain impure spectime
actions or it is judged that their residuals would be
too trivial to warrant a separate residual function.

By default, functions are \emph{not} inlined.
The description below details the conditions why a function
may be marked for inlining.

\subsection{Impure static actions}

Any function that, perhaps indirectly, contain a call
with callmode \cons{CallOnceSpectime} or a static allocation
must be unsharable.

This is because these operations must be expected to yield different
result even when performed in identical (as far as \cmix can see)
static states. In the case of \cons{CallOnceSpectime} calls we
explicitly promise the user (in the user manual---this callmode
corresponds to the \texttt{spectime} user annotation on function
calls) that the generating extension will not duplicate or optimize
away the calls.

\subsection{Functions with static return values}

A function with a static, non-void, return value must be
unsharable. This is because gegen doesn't yet know how to memoise
function return values.

\subsection{Functions that are probably ``small'' residually}

We implement the following heuristics for identifying functions whose
residuals are probably small enough that we should inline them. A
function will be inlined if
\begin{itemize}
\item[1)] it contains no dynamic conditionals, and
\begin{itemize}
\item[2a)] it contains no dynamic assignments, or
\item[2b)] it it not recursive and contains no loops
\end{itemize}\end{itemize}
 In theory it should be ``safe'' (in the sense that it
 only introduces infinite specialization if there are
 statically-controlled infinite recursions without
 dynamic exits) to inline every function that has no
 dynamic conditionals. Rules (2a) and (2b) try to
 restrict inlining to functions that will be ``small''
 residually, in an attempt to reduce code blowup.

   Functions that match (2a) will be \emph{empty} in the
 residual programs, modulo other function calls, and
 are obvious candidates for inlining.

Functions that match (2b) can at least be trusted
not to grow without bounds.

\subsubsection{Implementation notes}

This is how we check whether the flow graph of a function
may be cyclic. The algorithm is derived from the description
of depth-first orderings in \citeA{AhoSethiUllman:1986:Compilers}.

   We do a depth-first traversal of the flow graph,
 maintaining along the way two sets of \emph{pending} and
 \emph{finished} blocks. Once a block gets marked as
 finished we know that no cycle can be reached from
 that block. The pending blocks are those that are
 ancestors to the one currently processed.

   At the scanning time each edge can lead to either
\begin{itemize}
\item an unprocessed node, which we descend to. If it
     finished, then all is fine
\item a finished node---then we know that the edge
     cannot be part of cycle.
\item a pending node, in which case we've spotted a
     cycle and abort the search.
\end{itemize}

\subsection{Implementation level 1999-03-16}

At the time of this writing the implementation of the function
sharing analysis is in sync with the above description.

\end{docpart}
%%% Local Variables: 
%%% mode: latex
%%% TeX-master: "cmixII"
%%% End: 

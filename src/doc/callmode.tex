% Edit Mode: -*- LaTeX -*-
% File: callmode.tex
% Time-stamp: <98/06/08 12:33:53 panic>
% $Id: callmode.tex,v 1.2 1999/05/13 10:58:38 skalberg Exp $

\providecommand{\docpart}{\renewenvironment{docpart}{}{}
\end{docpart}
\documentclass[twoside]{cmixdoc}
%\bibliographystyle{apacite}

\makeatletter
\@ifundefined{@title}{\title{\cmix-documentation}}{}
\@ifundefined{@author}{\author{The \cmix{} Team}}%
{\expandafter\def\expandafter\@realauthor\expandafter{\@author}%
\author{The \cmix{} Team\\(\@realauthor)}}
\makeatother

\AtBeginDocument{%
\markboth{\hfill\today\quad\timenow\hfill\llap{\cmix\ documentation}}
{\hfill\today\quad\timenow\hfill}}

\renewcommand{\sectionmark}[1]{\markboth
{\hfill\today\quad\timenow\hfill\llap{\cmix\ documentation}}
{\rlap{\thesection. #1}\hfill\today\quad\timenow\hfill}}

%\newboolean{separate}
%\setboolean{separate}{true}

\renewenvironment{docpart}{\begin{document}}%
                          {\bibliography{cmixII}\printindex
                           \end{document}}
\begin{document}\shortindexingon
}
\title{Annotation of External Functions}
\author{Jens Peter Secher}
\begin{docpart}
\maketitle
%\documentclass[a4paper]{cmixdoc}

\section{Call-modes}
\label{sec:callmodes}

To avoid making worst-case assumptions about the effects of calls to
\emph{external} functions, the user can give a description to \cmix on
how (a call to) a particular external function behaves by annotating
external functions and/or individual call-sites.

\subsection{Function description}
The following \emph{function} annotations exists:

\begin{quote}
\begin{description}
\item[\texttt{pure}] means that the function is a function in a
  mathematical sense.  Example: \texttt{strcmp()}.
\item[\texttt{stateless}] means that the function possibly performs
  side-effects on its arguments, but otherwise does not depend on any
  external objects. Example: \texttt{strcpy()}.
\item[\texttt{ROstate}] means that the value returned depend on the
  state of some external object, but that this external state is not
  changed by the function. The function may, however, perform
  side-effects on its arguments. Example: \texttt{feof()}.
\item[\texttt{RWtate}] means that the function might change the state
  of some externally defined object, or perform side-effects on its
  arguments.  Example: \texttt{scanf()}. This is the default if an
  external function is unannotated.
\end{description}
\end{quote}
These categories form a proper-subset chain:
\[ \mathtt{pure} \subset \mathtt{stateless} \subset \mathtt{ROstate}
   \subset \mathtt{RWstate}
\] 

These annotations are primary a vehicle to describe standard
functions: \cmix is shipped with a set of shadow header files that
conform with the ISO/ANSI standard. Functions declared in these header
files are pre-annotated with the above mentioned annotations.

The annotations can, however, also be used by users when
{\specialis}ation of non-monolithic programs --- see the User Manual
section on separate {\specialis}ation.

\subsection{Call-site description}
Each call-site inherits the description of 
The following \emph{call-site} annotations exists:

\begin{quote}
\begin{description}
\item[\texttt{spectime}] instructs \cmix to make the call at specialization
  time.
\item[\texttt{residual}] instructs \cmix to make the call at residual time.
\item[(no annotation)] makes \cmix choose the call-time as it pleases.
\end{description}
\end{quote}

Any call-site can thus be classified according the above categories,
which are then used by the various analysis. The following subsections
will describe how each analysis responds to calls to external
functions.

\subsection{PA}
The points-to analysis maintains two special pools of objects: the
spectime pool and the residual pool. Any object can be contained in at
most one of these pools, depending on \emph{when} the object is alive.
Calls to external functions are classified thus:

\begin{tabular}{l|c|c|c|c|} 
                  & \texttt{pure} & \texttt{stateless} &
                  \texttt{ROstate} & \texttt{RWtate} \\
\hline
\texttt{spectime} & & & 3 & 4 \\
\cline{1-1}\cline{4-5}
\texttt{residual} & 1 & 2 & & \\
\cline{1-1}
(\cmix chooses) & & & 
                  \raisebox{1.5ex}[0cm][0cm]{5} & \raisebox{1.5ex}[0cm][0cm]{6} \\
\hline
\end{tabular}

\begin{enumerate}
\item Any object (with the right type) reachable from the arguments
  can be returned.
\item Any object (with the right type) reachable from the arguments
  can be returned. Any pointer object reachable from the arguments can
  be written to and thus point to any object (with the right type)
  reachable from the arguments.  
\item Any pointer object reachable from the arguments can be written
  to and thus point to any object (with the right type) in the
  spectime pool. 
\item Any pointer object reachable from the arguments can be written
  to and thus point to any object (with the right type) in the
  spectime pool. Furthermore, any object reachable from the
  arguments is added to the spectime pool.
\item Like 3, but with ``spectime pool'' replaced by ``residual pool''.
\item Like 4, but with ``spectime pool'' replaced by ``residual pool''.
\end{enumerate}


\subsection{BTA}
The binding-time analysis uses the classification of calls to external
functions thus:

\begin{tabular}{l|c|c|c|c|} 
                  & \texttt{pure} & \texttt{stateless} &
                  \texttt{ROstate} & \texttt{RWtate} \\
\hline
\texttt{spectime} & \multicolumn{3}{c|}{1} & 2 \\
\hline
\texttt{residual} & \multicolumn{4}{c|}{}  \\
\cline{1-3}
(\cmix chooses) & \multicolumn{2}{c|}{4} & 
                  \multicolumn{2}{c|}{\raisebox{1.5ex}[0ex][0ex]{3\quad}} \\
\hline
\end{tabular}

\begin{enumerate}
\item The call is made at {\specialis}ation time and is allowed to be
  under dynamic control. All arguments must be completely spectime.
\item The call is made at {\specialis}ation time and is \emph{not}
  allowed to be under dynamic control. All arguments must be
  completely spectime.
\item The call is made at residual time. All arguments must be
  residual or liftable.
\item All arguments are forced to be either completely spectime or
  completely residual. The call is made accordingly.
\end{enumerate}


\subsection{Dataflow}
The dataflow analysis uses the classification of calls to external
functions thus:


\begin{tabular}{l|c|c|c|c|} 
                  & \texttt{pure} & \texttt{stateless} & \texttt{ROstate} & \texttt{RWtate} \\
\hline
\texttt{spectime} & & & 3 & 4 \\
\cline{1-1}\cline{4-5}
\texttt{residual} & 1 & 2 & & \\
\cline{1-1}
(\cmix chooses) &  & & \raisebox{1.5ex}[0cm][0cm]{5} & \raisebox{1.5ex}[0cm][0cm]{6} \\
\hline
\end{tabular}


\begin{enumerate}
\item The call might read from objects reachable from the arguments.
\item The call might write to or read from objects reachable from the
  arguments. 
\item All objects in the spectime pool might be read. All objects
  reachable from the arguments might be read or written.
\item All objects in the spectime pool and objects reachable from the
  arguments might be read or written.
\item Like 3, but with ``spectime pool'' replaced by ``residual pool''.
\item Like 4, but with ``spectime pool'' replaced by ``residual pool''.
\end{enumerate}

\end{docpart}

%%% Local Variables: 
%%% mode: latex
%%% TeX-master: "cmixII"
%%% End: 

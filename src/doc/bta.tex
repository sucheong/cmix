% File: bta.tex

\providecommand{\docpart}{\renewenvironment{docpart}{}{}
\end{docpart}
\documentclass[twoside]{cmixdoc}
%\bibliographystyle{apacite}

\makeatletter
\@ifundefined{@title}{\title{\cmix-documentation}}{}
\@ifundefined{@author}{\author{The \cmix{} Team}}%
{\expandafter\def\expandafter\@realauthor\expandafter{\@author}%
\author{The \cmix{} Team\\(\@realauthor)}}
\makeatother

\AtBeginDocument{%
\markboth{\hfill\today\quad\timenow\hfill\llap{\cmix\ documentation}}
{\hfill\today\quad\timenow\hfill}}

\renewcommand{\sectionmark}[1]{\markboth
{\hfill\today\quad\timenow\hfill\llap{\cmix\ documentation}}
{\rlap{\thesection. #1}\hfill\today\quad\timenow\hfill}}

%\newboolean{separate}
%\setboolean{separate}{true}

\renewenvironment{docpart}{\begin{document}}%
                          {\bibliography{cmixII}\printindex
                           \end{document}}
\begin{document}\shortindexingon
}
\title{Binding-time analysis in \cmix}
\author{Jens Peter Secher}
\begin{docpart}
\maketitle

\begin{center}
  \fbox{\huge\textsl{THIS SECTION IS OUT-OF-DATE}}    
\end{center}

\section{Binding-time analysis in \cmix}
This paper describes the binding-time analysis (BTA) on \coreC in \cmix.
The analysis is based on the one described in
\cite{Andersen:1994:ProgramAnalysisAndSpecialization} and extended with
ideas from \cite{Andersen:1997:PartiallyStaticBTA}. The analysis works on
annotated types, described in the next section. It consists of two phases:
constraint generation and constraint solving.  These are described in later
sections.

\subsection{Types and binding times}
The purpose of the BTA is to identify which parts of the program that
can be evaluated at compile time; these are the the \emph{static}
parts. All other parts must be suspended until the necessary data is
supplied; these are the \emph{dynamic} parts.  Figure~\ref{fig:btt}
defines binding times and annotated types.  Intuitively, $\beta +>
\beta'$ means that $\beta$ is at least as dynamic as $\beta'$.

\begin{figure*}%[htbp]
\begin{frameit}
\[\begin{array}{rcll}
  {\mathcal B} \ni \beta &::=& S & \mbox{Static} \\
                         & | & D & \mbox{Dynamic}\\
  \\
  +> &\subseteq& {\mathcal B} \times {\mathcal B} \\
  D & +> & D \\
  D & +> & S \\
  S & +> & S \\
  \\
    \tau_B &::=& \syntax{int}, \syntax{float}, \dots \\
  \\
  {\mathcal T} \ni T &::=& \btT{\tau_B}{\beta} & \mbox{Simple} \\
            & | & \btT{\mbox{struct }S}{\beta} & \mbox{Structure}  \\
            & | & \btT{\star}{\beta} T  & \mbox{Pointer to }T \\
            & | & \btT{[n]}{\beta} T  & \mbox{Array}[n]\mbox{ of }T \\
            & | & \btT{(T_1,...,T_n)}{\beta} T  & \mbox{Function
                         returning }T \\
\end{array}\]
\caption{binding times and types.}
\label{fig:btt}
\end{frameit}
\end{figure*}

\begin{Def}
  \label{def:wf-bttype}
  A binding-time type $T =
  \btT{\tau_1}{\beta_1}...\btT{\tau_n}{\beta_n}$ is \emph{well-formed}
  iff $\beta_n +> \dots +> \beta_1$ and for any function type $T =
  \btT{(\btT{\tau_1}{\beta_1},...,\btT{\tau_m}{\beta_m} )}{\beta} T_0$
  it must be the case that $\beta +> \beta_1 \wedge ... \wedge \beta
  +> \beta_m$.
\end{Def}

\noindent
The relation ${>>=} \subseteq \mathcal{T} \times \mathcal{T}$ is
defined below. Intuitively, $T >>= T'$ means that type $T$ is at least
as dynamic as $T'$.

\begin{Def}
  \label{def:bt-type-order}
  $T >>= T'$ iff $T = \btT{\tau_1}{\beta_1}...\btT{\tau_n}{\beta_n}$,
  $T' = \btT{\tau_1}{\beta_1'}...\btT{\tau_n}{\beta_n'}$ and $\beta_1
  +> \beta_1' \wedge ... \wedge \beta_n +> \beta_n'$. Also, for any
  pair of function types $T,T'$, we have $T >>= T'$ iff $T =
  \btT{(T_1,...,T_m)}{\beta}T_0, T' =
  \btT{(T_1',...,T_m')}{\beta}T_0'$, and $T_0 >>= T_0' \wedge T_1 >>=
  T_1' \wedge ... \wedge T_m >>= T_m'$.
\end{Def}

\begin{Def}
  \label{def:bt-of-type}
  The binding time of a type is defined thus: $\bt{T}=\beta$ iff $T =
  \btT{\tau}{\beta} \btT{\tau_1}{\beta_1} ...
  \btT{\tau_n}{\beta_n}$.
\end{Def}

\subsection{Constraint generation}
Constraints are generated to express dependencies between parts of the
program, \eg the binding time of an expression, which consists of an
operator and two operands, depends on the binding times of both
sub-expressions. 

There are three kinds of constraints: $T == T'$ means that the two
binding-time types must be identical; $T >>= T'$ is defined by
definition~\ref{def:bt-type-order}; $\beta +> \beta'$ is defined in
figure~\ref{fig:btt}; The short-hand notation $\bt{T} +> \bt{T'}$
means that if $\bt{T}=\beta$ and $\bt{T'}=\beta'$ then $\beta +>
\beta'$. The short-hand notation alldyn($e$) means that every part of
$e$ must be dynamic.

Initially, the types of the formal parameter declarations of the goal
function are assigned the initial binding-time division. All other
types are annotated with fresh binding-time variables. Since the
translation to \coreC has introduced a unique instance of every
structure declaration, each instance can have its own binding times.
Initialisers are static in Core C so they are not accounted for.
Parameter types of function pointers are made equal to parameters of
all functions they can point to, and so are the return types. Equality
constraints are denoted by $==$, which is realy a hort-hand notation
for two $>>=$ constraints.

\subsection{Expressions}
\label{sec:BTAExpressions}
The binding-time constraint generation for expressions is defined in
figure~\ref{fig:BTAExprConstraintGeneration}. During this phase we
implicitly assume that constraints are generated such that every type will
fulfil the well-formed criteria, see definition~\ref{def:wf-bttype}.

\bigskip
% -------- Expressions ---------

\begin{figure*}%[htbp]
\begin{frameit}
\noindent\fbox{Expressions}\hfill\fbox{$e : T,\mathcal{C}$}

\[\inference[const]
  {}
  {c : \btT{\tau_B}{\beta},\{\}}
\qquad
  \inference[rvar]
  {}
  {x : T, \{T == T_x\}}
\qquad
  \inference[lval]
  {\bt{T_x} = \beta_x}
  {\syntax{\&}x : \btT{\star}{\beta}T, \{T >>= T_x,\beta_x +> \beta\}}
\]


\[
  \inference[deref]
  {e : \btT{\star}{\beta} T', \mathcal{C} &
   \mbox{localglobal}(PT(e),\beta,\bt{T'}) = \mathcal{C}'}
  {\syntax{*}e : T, \mathcal{C}\mathcal{C}'\{T==T'\}}
\qquad
  \inference[addr]
  {e : T, \mathcal{C} & \bt{T} = \beta}
  {\syntax{\&}e : \btT{\star}{\beta'}T', \mathcal{C}
    \{T==T', \beta +> \beta'\}}
\]

\[\inference[unary]
  {e : T, \mathcal{C}}
  {uop\ e : T', \mathcal{C}\{\bt{T'} +> \bt{T}\}}
\qquad
  \inference[binary]
  {e_1 : T_1, \mathcal{C}_1 & e_2 : T_2, \mathcal{C}_2}
  {e_1\ bop\ e_2 : \btT{\tau}{\beta},
    \mathcal{C}_1\mathcal{C}_2\{\beta +> \bt{T_1}, \beta +> \bt{T_2}\}}
\]

\[
  \inference[struct]
  {e : T_S, \mathcal{C} & \mbox{refSet}(e.i) = \{ \alst{d} \}}
  {e.i : T, \mathcal{C}\{T >>= T_{d_1},...,T >>= T_{d_n}\}}
\qquad
  \inference[typesize]
  {}
  {\syntax{sizeof}(\tau) : \btT{\tau_B}{\beta},\{\beta +> D\}}
\]

\[
  \inference[expsize]
  {}
  {\syntax{sizeof}(e) : \btT{\tau_B}{\beta},\{\beta +> D, \mbox{alldyn}(e)\}}
\]

\[\inference[cast]
  {e : T, \mathcal{C}}
  {\syntax{(}T'\syntax{)} e : T', \mathcal{C} \cup \mbox{Cast}(T,T')}
\]

\[\begin{array}{lcl}
   \mbox{Cast}(\btT{\tau_B}{\beta},\btT{\tau_B'}{\beta'}) 
   & = & \{ \beta +> \beta' \} \\
  \mbox{Cast}(\btT{\star}{\beta}T,\btT{\star}{\beta'}T') 
   & = & \{ \beta=\beta' \} \cup \mbox{Cast}(T,T') \\
  \mbox{Cast}(\btT{\star}{\beta}T,\btT{[n]}{\beta'}T') 
   & = & \{ \beta=\beta' \} \cup \mbox{Cast}(T,T') \\
 \end{array}\]
 \caption{Constraint generation for expressions.}
 \label{fig:BTAExprConstraintGeneration}
\end{frameit}
\end{figure*}

Notice that if expression $e$ has a binding-time type $T$, then
$\bt{T}$ tells us the binding time of the \emph{result} of $e$ --- it
does not tell us when we can expect to evaluate various parts of
$e$. This information has to be extracted in a later phase,
see~\vref{sec:PartiallyStaticData}.

\subsection{Statements}
The binding-time type $T_f$ of a function $f$ has a binding time that
depends on all calls to $f$: If it is called under dynamic control,
the binding time is dynamic. The binding time $\alpha_f$ of a function
$f$ depends on the statements in that function: if it is static, no
dynamic statements are present in $f$ and thus any call to $f$ can be
eliminated.
  
The return type $T_{f_0}$ of a function $f$ tells us whether the
return value is static. It is thus possible to have a dynamic
function with a static return value, provided all return statements
are static and the control flow is totally static. Consider the
program

\begin{verbatim}
  int f(int d)
  {
    if (d) return 1; else return 2;
  }
\end{verbatim}

\noindent
where the return values are static, but under dynamic control since
\verb.d. is dynamic. If the return value of the function were to be
classified static, the function would have two different return
values. 

The constraint generation for statements is defined in
figure~\ref{fig:BTAStmtConstraintGeneration}. We implicitly assume that
each statement has a surrounding function $f$ with the type $T_f$ and
return type $T_{f_0}$.

\subsubsection{Non-local side-effects}
The binding-time analysis has to guarantee that static non-local
side-effects (NLS-effects) does not occur under dynamic control. To
ensure this, the BTA makes every NLS-effecting
statement\footnote{Recall that only statements can do side-effects in
  Core C.} dependent on the binding time of the preceding basic
block's control statement. Thus, if a basic block $A$ ends in a
conditional control statement of the form \syntax{if($e$) goto $B$
  else goto $C$}, then both blocks $B$ and $C$ will be dependent on
the expression $e$. To make this dependency transitive, every basic
block is dependend of all immediate preceding blocks; the first block
in a function is dependend on each call site's basic block.

Given a function $f$, a statement of the form \syntax{x=$e$} is
locally side-effecting (and thus harmless) if \syntax{x} is a local
variable in $f$. The same holds for statements of the form
\syntax{x.i=$e$}. When assignments are done through pointers, however,
we cannot rely on syntactical scope anymore: we need to decide whether
the objects reachable through a particular pointer are truly local
objects\footnote{A recursive function can have several sets of local
  variables at run-time, and thus a pointer can refer to a variable in
  any such set}.

Given the previously described points-to information (see
Section~\vref{sec:PointerAnalysis}), an approximation of the set of non-local
variables in a function $f$ can be calculated: the transitive closure
of the PA-info (denoted $PT^*$) is calculated for all global variables
and for each function $f$, such that $PT^*(f)$ is $PT^*$ of all formal
parameters, and $PT^*(\mbox{globals})$ is $PT^*$ of the globals. With
these sets in hand, a side-effecting statement of the form
\syntax{*p=$e$} is considered locally side-effecting if no object in
$PT(\syntax{p})$ is contained in $PT^*(\mbox{globals})$ or $PT^*(f)$.
We can thus define a function that returns a set of constraints when
given a declaration, the containing function and the binding-time
variable of the containing basic block.
\begin{eqnarray}
  \label{eqn:nonlocal}
  \mbox{nonlocal}(d,f,bb)
  &=& \mbox{if } d \in \mbox{locals}(f) \setminus (PT^*(\mbox{globals}) \cup PT^*(f)) \nonumber \\
  & & \mbox{then } \{ \} \nonumber  \\
  & & \mbox{else } \{ \bt{T_d} +> \bt{bb} \} \nonumber 
\end{eqnarray}

Each function should have a non-local side effect flag to remind the
generating extension that such a function needs a post-store
memoization.

\subsection{Static pointers to dynamic data}
Consider this program were \texttt{d} is dynamic.

\begin{verbatim}
int y,z;                    void swap(int* a, int *b)
                            {
int main(int d)               int tmp = *a;
{                             *a = *b, *b = tmp;
  int x = d+1;              }
  swap(&d,&x); /* 1 */  
  y = x;                
  z = d;                
  swap(&y,&z); /* 2 */  
  return d;             
}                       
\end{verbatim}

\noindent The first call to \texttt{swap} contains static pointers to
dynamic, non-local, non-global objects --- whereas the second contains
static pointers to dynamic, global objects. If this was allowed, the
residual program would wrongly be

\begin{verbatim}
int y,z;                    void swap_1()
                            {
int main(int d)               int tmp = d;     /* out of scope */
{                             d = x, x = tmp;
  int x = d+1;              }
  swap_1();                 
  y = x;                    void swap_2()
  z = d;                    {
  swap_2();                   int tmp = y;
  return d;                   y = z, z = tmp;
}                           }
\end{verbatim}

\noindent The problem is that dynamic, non-local, non-global objects
get out of scope during the first call. This can also happen with
global pointers to non-local objects. [So, we need to make constraints
such that if a dereference of static pointers results in a non-local,
non-global object, the pointer must be made dynamic.]

\begin{eqnarray}
  \label{eqn:localglobal}
  \mbox{localglobal}(\Delta,\beta,\beta_{*}) &=& \mbox{if }
  \exists\delta \in \Delta \mbox{ s.t. }
  \delta \not\in (\mbox{globals} \cup \mbox{locals}) \nonumber \\
  & & \mbox{then } \{ \beta_{*} +> \beta \} \nonumber  \\
  & & \mbox{else } \{ \} \nonumber
\end{eqnarray}

\bigskip
% -------- Statements ---------

\begin{figure*}%[htbp]
\begin{frameit}
\noindent\fbox{Statements}\hfill\fbox{$s : \mathcal{C}$}

\[\inference[if]
  {e : T, \mathcal{C}}
  {\syntax{if}(e)\ B\ B' : \mathcal{C}\{\bt{T_f} +> \bt{T},
                            \bt{B} +> \bt{T}, \bt{B'} +> \bt{e}\}}
\]

\[\inference[return]
  {e : T, \mathcal{C}}
  {\syntax{return}\ e : \mathcal{C}\{T_{f_0} >>= T,
     \bt{T_f} +> \bt{T}\}}
\qquad
  \inference[goto]
  {}
  {\syntax{goto}\ B : \{\}}
\]

\[\inference[assign]
  {e : T, \mathcal{C} & \mbox{nonlocal}(x,f,bb)=\mathcal{C}_x}
  {x\syntax{=}e : \mathcal{C}\mathcal{C}_x\{T_x >>= T, \bt{T_f} +> \bt{T}\}}
\]

\[\inference[passign]
  {e : T, \mathcal{C} & PT(x) = \{ \alst{d} \} & \mbox{nonlocal}(d_i,f,bb)=\mathcal{C}_i}
  {\syntax{*}x\syntax{=}e : \mathcal{C}\mathcal{C}_1 \cdots \mathcal{C}_n
    \{T_{d_1} >>= T,...T_{d_n} >>= T, \bt{T_f} +> \bt{T}\}}
\]

\[\inference[strassign]
  {e : T, \mathcal{C} & \mbox{nonlocal}(x,f,bb)=\mathcal{C}_x}
  {x\syntax{.}i\syntax{=}e : \mathcal{C}\mathcal{C}_x
    \{T_{x.i} >>= T, \bt{T_f} +> \bt{T}\}}
\]

\[\inference[pstrassign]
  {e : T, \mathcal{C} & PT(x) = \{ \alst{d} \}& \mbox{nonlocal}(d_i,f,bb)=\mathcal{C}_i}
  {\syntax{(*}x\syntax{).}i\syntax{=}e : \mathcal{C}\mathcal{C}_1 \cdots \mathcal{C}_n
    \{T_{d_1.i} >>= T,...T_{d_n.i} >>= T, \bt{T_f} +> \bt{T}\}}
\]

\[\inference[call]
  {e_i : T_i , \mathcal{C}_i &
    T_g = \btT{(T_1'\cdots T_n')}{\beta}T_0 & \mbox{nonlocal}(x,f,bb)=\mathcal{C}_x}
  {x\syntax{=}g(e_1,\dots,e_n) : \mathcal{C}_1 \cdots \mathcal{C}_n \mathcal{C}_x
    \{T_x >>= T_0, T_i' >>= T_i, 
      \bt{T_f} +> \bt{T_i}\}}
\]

\[\inference[pcall]
  {e_i : T_i , \mathcal{C}_i & PT(f\!p) = \{\alst[m]{d}\} &
   T_{f\!p} = \btT{\star}{\beta_{f\!p}}\btT{(T_1'\cdots
   T_n')}{\beta}T_0
   & \mbox{nonlocal}(x,f,bb)=\mathcal{C}_x}
  {x\syntax{=}f\!p(e_1,\dots,e_n) : \mathcal{C}_0 \cdots \mathcal{C}_n
    \mathcal{C}_x
    \{T_{d_j} == T_{f\!p}, T_x >>= T_0, T_i' >>= T_i, 
      \bt{T_f} +> \bt{T_i}\}}
\]

\[\inference[free]
  {e : \btT{\star}{\beta}T, \mathcal{C}}
  {\syntax{free}(e) : \mathcal{C}\{\bt{T_f} +> D, \beta +> D\}}
\]

\[\inference[calloc]
  {e : T_e, \mathcal{C} & T_x = \btT{\star}{\beta}T'
    & \mbox{nonlocal}(x,f,bb)=\mathcal{C}_x}
  {x\syntax{=calloc}(e,T,d) : \mathcal{C}\mathcal{C}_x
    \{ T==T',  T==T_d, \bt{T_f} +> \beta, \beta +> \bt{T_e} \}}
\]
 \caption{Constraint generation for statements.}
 \label{fig:BTAStmtConstraintGeneration}
\end{frameit}
\end{figure*}

\subsection{Assigning binding times}
Binding times can be induced directly from the types.

[Work-lists in \cite{KanamoriWeise:1994:WorklistManagement}]

[Each declaration $d$ that has pointer type: Make all the types of
objects $\in PT(d)$ depend on each other.]


\begin{tabular}{|c|c|c|l|}  \hline
  Allocation & Contents & Freed & Action \\ \hline
  S          & S        & +     & Ignore free \\
  S          & S        & -     & OK \\
  S          & D        & +     & $->$ DD \\
  S          & D        & -     & OK \\
  D          & D        & +/-   & Residualize \\ \hline
\end{tabular}

\end{docpart}
%%% Local Variables: 
%%% mode: latex
%%% TeX-master: "cmixII"
%%% hst: "Herrens Allermest Ultimative Naade"
%%% End: 


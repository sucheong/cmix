% Edit Mode: -*- LaTeX -*-
% File: c2core.tex

\providecommand{\docpart}{\renewenvironment{docpart}{}{}
\end{docpart}
\documentclass[twoside]{cmixdoc}
%\bibliographystyle{apacite}

\makeatletter
\@ifundefined{@title}{\title{\cmix-documentation}}{}
\@ifundefined{@author}{\author{The \cmix{} Team}}%
{\expandafter\def\expandafter\@realauthor\expandafter{\@author}%
\author{The \cmix{} Team\\(\@realauthor)}}
\makeatother

\AtBeginDocument{%
\markboth{\hfill\today\quad\timenow\hfill\llap{\cmix\ documentation}}
{\hfill\today\quad\timenow\hfill}}

\renewcommand{\sectionmark}[1]{\markboth
{\hfill\today\quad\timenow\hfill\llap{\cmix\ documentation}}
{\rlap{\thesection. #1}\hfill\today\quad\timenow\hfill}}

%\newboolean{separate}
%\setboolean{separate}{true}

\renewenvironment{docpart}{\begin{document}}%
                          {\bibliography{cmixII}\printindex
                           \end{document}}
\begin{document}\shortindexingon
}

\title{From \ansiC to \coreC}
\author{Henning Makholm}
\begin{docpart}
\maketitle

\newcommand{\ctoc}{\texttt{c2core}\xspace}
\newcommand{\installleft}[2]{{#1}\curvearrowright{#2}}
\newcommand{\atomconnect}[2]{{#1}\rightsquigarrow{#2}}

\section{Transformation from C to Core C}
This section describes the \ctoc phase of \cmix. This phase
translates a type-checked \ansiC program into the \coreC
representation defined in the previous section.

\subsection{Types}
\label{sec:c2coreTypes}
The \ansiC program that is input to \ctoc has already been type
checked and each relevant \ansiC contruct annotated with type
information. However, this type information has to be translated
to the \coreC representation, and the rules for structural
sharing in the \coreC representation (section
\vref{sec:corecTypeSharing})
must be enforced.

The types in the \ansiC program are represented as a classical
object class hierarchy---\eg, ``a pointer type'' is a subclass
of ``some type'', and so forth. Each type object knows how to
translate itself into a fresh, fully owned, \coreC representation
of itself. This translation is used when a declaration or expression must
own its entire type, and is the source of most of the \coreC types.

For some expressions with pointer type,
the type sharing rules specify that the expression must own
the ``outermost'' pointer and borrow the rest of the type from
its subexpression. In these cases the pointer type is simply
generated and the \ansiC type representation is not consulted
at all.

\subsubsection{Translation of recursive types}

Locally, the type translation is a trivial recursive operation.
However, types
in C can be recursive, and we do not want to try to translate
them into infinite \coreC representations. We need some sort
of control over the recursion.

Any type recursion that can be expressed in \ansiC involves
a \syntax{struct} or \syntax{union} type, because
those are the only types that can be expressed before their successors
in the type graph have been specified\footnote
        {An \syntax{enum} type can be named without having been fully
        defined, too, but the translation of an \syntax{enum} type to
        \coreC does not involve translating other types, so it cannot
        be part of an infinite recursion.

        Note that this fact depends on the representation of \syntax{enum}
        types as primitive types in \coreC. Other imaginable
        representations of C's type system might need to treat
        recursion via \syntax{enum}s in pathological cases such as
        \texttt{enum foo \{ bar = sizeof(enum foo*) \};}}.
Therefore, we only check for type recursion while translating
these types, known in most of the \cmix documentation as
\emph{usertypes}.

The rule is that if a translation of a usertype is needed directly or
indirectly while tranlating the members of that usertype, the two
instances of the usertype are made the same by forming a loop.
Thus, given the definitions
\begin{verbatim}
  struct vertex ;
  struct neighbours {
     struct neighbours *next;
     struct vertex *neighbour;
  } ;
  struct vertex {
     struct neighbours *preds;
     struct neighbours *succs;
  } ;
  struct Vertex *foo ;
  struct Edgelist *bar ;
\end{verbatim}
the translation of \syntax{foo}'s type (shown on figure
\vref{fig:c2coreVertexExample}) contains a single copy
of \syntax{struct vertex} and two separate copies of
\syntax{struct neighbours}. The translation of \syntax{bar}'s type,
however, contains only a single copy of each \syntax{struct}.

\begin{figure}[htbp]
  \begin{frameit}
  \begin{center}
        This figure has not been drawn yet.
  %  \leavevmode
  %  \epsfig{file=struct1.eps,height=0.7\textheight,width=0.7\textwidth}
  \end{center}
  \end{frameit}
  \caption{Type recursion example}
  \label{fig:c2coreVertexExample}
\end{figure}

This strategy is implemented by each \ansiC usertype node\footnote
        {The implementation of this is a little more complicated,
        since there are actually two nodes involved in the \ansiC
        representation: one representing the \emph{definition} of
        the struct/union, and one representing the \emph{reference}
        to the definition. The definition holds the member types,
        so that is the place where the recursion prevention most
        naturally takes place. However, the reference node holds the
        type qualifiers that are to be put into the final type node,
        and we don't want to identify two instances with different
        qualifiers. Thus, it is the reference node that ``owns'' the
        current translation pointer; and it passes it as a reference
        parameter to the definition when translating itself.}
having a
pointer to a ``current'' \coreC translation, and translating it
with the following algorithm:
\begin{itemize}
\item[1.] If there is a current \coreC translation, return that.
\item[2.] Create a \cons{StructUnion} node with an empty member
          type list. Install it as the current \coreC translation.
\item[3.] Recursivly translate the types of the members into the
          member type list of the newly created node.
\item[4.] Set the current \coreC translation to ``none'', and
          return the node created in step 2.
\end{itemize}

Observe that the recursion resolution strategy directly effects
the precision of the binding-time analysis. Figure
\vref{fig:c2coreLinkedListBTs} shows two possible \coreC
representations of a linked list type. The one on the left
is able to represent a list where every other element is dynamic
and the rest are static. The \ctoc translation, however,
produces the one on the right, forcing all elements to have
the same binding time.

\begin{figure}[htbp]
  \begin{frameit}
  \begin{center}
        This figure has not been drawn yet.
  %  \leavevmode
  %  \epsfig{file=struct1.eps,height=0.7\textheight,width=0.7\textwidth}
  \end{center}
  \end{frameit}
  \caption{The translation strategy for recursive types affects the BTA}
  \label{fig:c2coreLinkedListBTs}
\end{figure}

The two depicted representations would be equally bad if the
application happened to need every \emph{third} element to
be dynamic. It is not at all trivial to select the best (finite)
number of iterations to unfold in the translation\footnote
        {It might be possible to implement an analysis that
        made good decisions about this. However, it would
        probably be complex and would \emph{need} to employ
        pointer analysis data---and in \cmix the pointer analysis
        comes \emph{after} the \ctoc translation},
so we simply chose to use the strategy that is easiest to implement.

\subsubsection{Typedef types}
\syntax{typedef} type synonyms are expanded in the syntax analysis phase.

\subsection{Global definitions and declarations}
\label{sec:c2coreGlobally}
It is tricky to find a safe overall strategy for the translation
of global definitions and declarations. \syntax{struct},
\syntax{union} and \syntax{enum} are easy:
\begin{itemize}
\item Apart from their role in the type translation (section
      \ref{sec:c2coreTypes}), \syntax{struct} and \syntax{union}
      definitions translate to \syntax{C\_UserDef}s that hold
      the type tag and the member names. These ``master definitions''
      must be present before we can translate any types. On the other
      hand, there are no preconditions for building them.
\item Likewise, \syntax{enum} definitions should be translated
      to \syntax{C\_EnumDef}s before any type translation begins.
      Because expression translation may involve type translation,
      this means that explicit value expressions for the enumeration
      constants cannot take place while building the
      \syntax{C\_EnumDef}.
\end{itemize}
Consequenty, the very first step in the \ctoc translation is to
traverse the \ansiC program\footnote
        {In the \ansiC representation usertype definitions can
        appear in the declaration part of compound statements
        as well as globally. In \coreC they are all moved to
        the global level.}
and translate any \syntax{struct}, \syntax{union}, and \syntax{enum}
definitions found there. The \ansiC definitions are annotated with
references to the \coreC definitions to aid the type translation.
Additionally, a list of original \syntax{enum} definitions is
collected so that as the very \emph{last} step of \ctoc, the
value expressions can be found and translated.

Global variable and function definitions are more complicated. A function
body usually contains many expressions, and variable definitions can contain
expressions in initializers. The translation of expressions may
contain \cons{Var} nodes that refer to---and must borrow the \coreC type
of---other functions or global variables.

This means that to fully translate a function or global variable, we
may need to already have created \coreC declarations for any other
global declaration, and to have translated their type.

This fact suggest that we should translate the program in two passes:
\begin{itemize}
\item[I] Make a \coreC declaration for each global declaration, and
        translate its type, but do not translate its function body,
        repectively its initializer.
\item[II] Make a second pass translating bodies and initializers.
\end{itemize}

Our problems are not totally solved by this, however, because even
the \emph{types} of declarations may contain expressions that need
translating, in the form of array size expressions. Among several
possible strategies, all complicated, we choose this:

\emph{When an array type is translated before pass I is finished, its
size is \emph{not} translated straight away but put into a list of
deferred array sizes.} The deferred array size expressions are then
translated before pass II begins.

It would be nice if the type translation could just keep deferring
all array sizes, even while the loop that tries to empty the deferred
list runs. Unfortunately it is possible to construct pathological
examples such as
\begin{verbatim}
    struct S { int array[sizeof (struct S*)]; } s ;
\end{verbatim}
where the deferred list would never shrink to size 0. When the type
translations translates the size expression right away, this
example will be caught by the recursive type strategy, and we will
end up with a type recursion that involves expressions.

\subsection{Variables}
\label{sec:c2coreVariables}
The same construct is used in the \ansiC representation to model
global variables, local variables, formal parameters, and usertype
members. Usertype members are not made directly into \coreC
declarations, but the former three cases are treated similarly
by the \ctoc translation.

The first step in the translation is to create a \syntax{C\_Decl}
object and supply it with a type. This step is handled differently
in the three cases:
\begin{description}
\item[Global variables] have this done in pass I of the translation,
        viz.\ section \ref{sec:c2coreGlobally}. The type is a fresh
        translation of the \ansiC type.
\item[Local varlaibles] are created by the translation of the
        compound statement that contains their definition. The type
        is a fresh translation of the \ansiC type.
\item[Formal parameters] are created by the translation of the
        function. Their type is borrowed from the function's type's
        parameter list.
\end{description}
As part of the first step, subsidiary declarations (see section
\vref{sec:corecVariables}) are generated for the variable if its
type demands it. The constructor for the \syntax{C\_Decl} class takes
care of this, recursively.

Then the variable's name and varmode is transferred to the \coreC
declaration.

If the variable has an initializer, that is translated and
installed in the \coreC declaration. As part of the type check,
the distinction between \cons{FullyBraced} and \cons{SloppyBraced}
initializer lists has already been resolved, so the translation
of initializers mostly consists of translating the initializer
expressions. Each initializer expression should translate to a
\cons{Value} (see section \vref{sec:c2coreTransExpr})
without pre- or post-sideeffects. If it does not, an error message
is emitted\footnote
        {The type checker does not check for constness of expressions,
        since a good approximation to the required check is
        automatically provided here.

        The approximation contains an error, however: a constant
        expression that uses the \syntax{?:} operator cannot be
        translated into a single \coreC expression, so it is
        erroneously flagged as non-constant.}.

\subsubsection{Well-known constants}
\label{sec:c2coreWellKnownConst}
The ``\syntax{well-known constant:}'' annotation can be applied to
a global variable to declare its name as a ``magic constant'' that
evaluates to an rvalue of the given type when it appears in an
expression.

This feature is used in shadow headers. \emph{E.g.},
\syntax{<stdio.h>} contains
\begin{verbatim}
extern FILE *const stderr, *const stdin, *const stdout ;
#pragma cmix well-known constant: stderr stdin stdout
\end{verbatim}
which causes \syntax{stderr} \emph{et al.}\ to be not
variables but primitive expressions of type ``pointer to
\syntax{FILE}''. The difference from a plain \syntax{const} variable
is that the binding times of different occurences of \syntax{stderr}
need not be equal. Thus, the program can contain calls of
\syntax{fprintf\discretionary{}{}{}(stderr,...)} that are performed
at spectime and other such calls that are residualized.

The implication for \ctoc is that the \ansiC variables that represent
these ``magic constants'' should not be translated into \coreC
variables. Leaf expressions referring to these variables
should be contained into \cons{Const} expressions instead of
\cons{Var}.

The annotation is implemented as a special varmode \cons{VarConstant}
which only appears in the \ansiC representation, not in \coreC.
In both translation passes, the global variables' varmodes are
inspected, and \cons{VarConstant} variables are simply ignored.
Similarly, the varmode is inspected at identifier leaf expressions
and decides how to translate the expression.

\subsection{Functions}
In pass I function definitions are treated little differently from
global variables. In pass II, however, the ``real'' translation to
\coreC takes place:

From pass I we have a \coreC type for the function. The parameter
types are located and used to generate \coreC declarations for
the formal parameters. These are finished off as described in
section \ref{sec:c2coreVariables}.

The body of the function is translated into a list of basic blocks as
described below and entered in the \coreC definition. As a by-product
of this translation the local variables in the body are translated
into the function's list of local variables, or (if they are
\syntax{static}) the list of global variables.

The translation of function bodies is akin to the intermediate
code generator in a compiler. \coreC has other caracteristics
than most compilers' intermediate code representations, so there
are also differences.
Most notably, \ctoc's attitude towards introducing new temporary
variables is much more cautious than the average intermediate code
generator's attitude to pseudo-registers.

The temporaries have proved
to be somewhat confusing to readers (human or machine) of the
residual program.
Additionally, while splitting \ansiC expressions into multiple
\coreC statements is often needed to fulfill the ``only one side
effect per statement'' principle, it also reduces the compiler's
freedom to choose the evaluation order of operands to fit its
optimization, so it should be avoided when possible.

\subsubsection{Proto-flowgraphs}
\label{sec:c2coreAtoms}

The body of a function is initially translated into a fine-grained
proto-flowgraph where each \coreC statement is a unit. Only when the
translation has been completed are the statements collected into
\coreC basic blocks with an ordinary graph traversal. This
collection process performs transition compression and dead
code elimination, so the number of basic
blocks in the final representation is minimized.

The elements that make up the proto-flowgraph are called
``atoms''. They come in several flavors, as depicted in
figure \vref{fig:c2coreAtoms}.
Every atom maintains a count $n$ of predecessors. When the basic-block
collection reach an atom with more than one predecessor, an
unconditional \cons{Goto} jump is generated and a new block
is started off.

\begin{figure}\begin{center}\small\begin{tabular}{|r@{~}c@{~}l|p{8cm}|}
\hline
\multicolumn{3}{|l|}{Grammar} & comments \\
\hline
\hline
$\alpha$ &::=& $\omega$ &
        \cons{StmtAtom}s and \cons{GotoAtom}s are \emph{simple} atoms.
        When they are created they have an empty slot where a
        reference to a successor atom must be plugged in
        before the proto-flowgraph is collected into basic blocks.\\
\cline{4-4}
&$|$& $\cons{IfAtom}(n,e,\coreXref{\alpha_{\mathrm{then}}},
                        \coreXref{\alpha_{\mathrm{else}}})$ &
        This atom models the \coreC \cons{If} jump. It does not
        contain the jump itself, because the pointers to the
        target basic blocks do not exist until the final collection.
        Rather, it contains the condition expression (translated
        to \coreC) and pointers to target atoms. \\
\cline{4-4}
&$|$& $\cons{ReturnAtom}(n,j)$ &
        This atom contains a \coreC \cons{Return} jump. \\
\cline{4-4}
&$|$& $\cons{StopperAtom}(n,t)$ &
        A single \cons{StopperAtom} is generated in each
        function translation; it represents returning without
        a value. \syntax{return;} statements jump to this, and
        a jump to it is inserted at the end of the function
        body (so it will be reached if control falls over the
        end of the function).

        This is different from the regular \cons{ReturnAtom}, in that
        it does not contain a \coreC \cons{Return} jump. The
        jump is only generated if the atom is actually reached
        in the collection phase. Then a suitable return
        expression must be invented. That relieves the
        following analyses from worrying about missing
        return expressions.

        The generated return expression depends on the function's
        return type as defined in figure \vref{fig:c2coreReturnExp}.
        As it is seen, the generation of a return expression sometimes
        involves generating a new variable. This is the reason why
        it is only generated if the \cons{StopperAtom} is reachable, and
        why the entire function shares a single \cons{StopperAtom}. \\
\hline
$\omega$ & $::=$ & $\cons{StmtAtom}(n,\coreXref \alpha ?,s)$ &
        This atom contains a \coreC statement.

        As a special rule, a \cons{Assign} and \cons{Alloc} statements
        may have an empty left side as long as it belongs to a
        \cons{StmtAtom}. However, if the empty left side has not been filled
        when the basic block generation begins, the statement is
        thrown away instead of being inserted into the basic block.
        This facilty is used in the ``Ignore III'' expression
        translation coercion (figure \vref{fig:c2coreCoerce}, page
        \pageref{fig:c2coreCoerce}). \\
\cline{4-4}
&$|$& $\cons{GotoAtom}(n,\coreXref \alpha ?)$ &
        A \cons{GotoAtom} is \emph{not} a representation of the \coreC
        \cons{Goto} jump. It is rather a ``no-op'' atom that is
        used as a jump target when the code that is jumped to
        has not yet been translated.

        When collecting atoms into basic blocks, a jump to a
        GotoAtom is treated just like a jump to its successor. \\
\hline
\end{tabular}\end{center}
\caption{Proto-flowgraph atom types}
\label{fig:c2coreAtoms}
\end{figure}

\begin{figure}\begin{center}\begin{tabular}{|l|l|}
\hline
Return type & Generated default return expression \\
\hline \hline
$\cons{Abstract}(\syntax{void},q)$ & none (in this case only)\\
$t = \cons{Liftable}(id,q)$ & $\cons{Const}(t,\syntax{0})$ \\
$t = \cons{Pointer}(id,q)$ & $\cons{Null}(t)$ \\
Any other $t$ & An anonymous global variable \\
& $\quad d = \cons{Var}(-,t,\cons{VarIntAuto},-,\ldots)$ \\
& is generated, and its value is returned \\
\hline
\end{tabular}\end{center}
\caption{Generation of default return expressions}
\label{fig:c2coreReturnExp}
\end{figure}

When the successor slot of a \cons{GotoAtom} is filled, it exports its
predecessor count to the successor, and any later predecessors
are also counted at the successor.

The exception is if an empty infinite loop is produced. The front
end of the last edge in the loop detects that it is trying to
propagate its predecessor count through itself, and sets a flag
that makes it act like a \cons{StmtAtom} with a discardable statement but
at least 2 predecessors. Thus in the collection phase an empty
basic block with an unconditional jump that points to itself is
generated.

\medskip
\noindent
When the basic blocks for a function have been collected, all the
generated atoms are freed.

\subsubsection{Proto-flowgraph intervals}
\label{sec:c2coreIntervals}
In texts about flow graph analysis, an ``interval'' is usually
a connected subgraph with a single entry node. However, as we
are here concerned with flow graph \emph{synthesis} it turns
out to be convenient to use the word for a less strict concept.

In the \ctoc context, an ``interval'' can be thought of simply
as a subset of the final flow graph with designated entry and
exit atoms. The intuition is that an interval models the
proto-flowgraph translation of an \ansiC statment. In
\begin{verbatim}
    a = 1; goto foo;
    while ( 1 ) {
      b = 42; goto bar;
    foo:
      c = a+117;
    }
  bar:
    d = b*c ;
\end{verbatim}
the body of the loop naturally translates to a piece of the
proto-flowgraph that is neither connected nor single-entry.

In practise an interval (represented by the C++ class
\syntax{Interval}) is just a shorthand for a pair of
references to an ``entry atom'' and an ``exit atom''.

An interval can have several different forms, shown on figure
\vref{fig:c2coreIntervals}

\begin{figure}\begin{center}\small\begin{tabular}{|r@{~}c@{~}l|p{10cm}|}
\hline
\multicolumn{3}{|l|}{Grammar} & comments \\
\hline
\hline
$I$ & $::=$ & $ [\coreXref\alpha:\coreXref\omega]$ &
        This is the most general form of an interval. The
        successor field $\alpha_\omega$ of $\omega$ is
        is yet empty. \\
\cline{4-4}
&$|$& $ [\coreXref\alpha,-]$ &
        This is equivalent for $[\alpha:\omega_0]$ where
        $\omega_0$ is a fresh \cons{GotoAtom} that does
        not have and will never get any predecessors---only
        this $\omega_0$ does not really exist. \\
\cline{4-4}
&$|$& $ [-] $ &
        The \emph{trivial interval}, equivalent to
        $[\alpha_0:\alpha_0]$ where $\alpha_0$ is
        a fresh \cons{GotoAtom}. \\
\hline
& & $ s $ &
        In the defintions of \coreC generation, a
        \coreC statement can appear where an interval is
        expected. That is a shorthand for $[\alpha:\alpha]$,
        where $\alpha=\cons{StmtAtom}(0,-,s)$ is a new atom. \\
\hline
\end{tabular}\end{center}
\caption{Interval types}
\label{fig:c2coreIntervals}
\end{figure}

A concatenation operator $+$ for intervals is defined by
$[\alpha_1:\omega_1] + [\alpha_2:\omega_2] = [\alpha_1:\omega_2]$
and has the side effect of connecting $\omega_1$ to $\alpha_2$.

\subsubsection{Translating \ansiC statements}
\label{sec:c2coreStatements}
Using the interval concept, \ansiC statements can translate
themselves to \coreC in a natural, recursive way. The translation is
parameterised on
\begin{itemize}
\item [$\coreXref{\alpha_R}$:] the atom to jump to if a
        \syntax{return;} without an expression is encountered.
\item [$\coreXref{\alpha_B}$:] the atom to jump to from a
        \syntax{break;} statement.
\item [$\coreXref{\alpha_C}$:] the atom to jump to from a
        \syntax{continue;} statement.
\item [$e_S$:] the expression that is tested in the innermost
        enclosing \syntax{switch} statement.
\item [$\coreXref{\coreXref{a_D}}$:] when translating the
        \syntax{default:} label, this reference is set to
        point to the atom it translates to. Later, the
        \syntax{switch} translation generates a jump to
        it if control falls through the test chain.
\item [$\coreXref{I_S}$:] references the interval in which
        the test chain of the innermost enclosing
        \syntax{switch} statement is being built. The
        translation of a \syntax{case} label adds a
        test to the end of the chain.
\item [$\coreXref{\vec d_L}$:] the list of declarations
        local \syntax{auto} or \syntax{register} variables
        and temporary variables should be added to when
        translated.
\item [$\coreXref{\vec d_G}$:] the list of declarations
        local \syntax{static} variables should be added to
        when translated.
\end{itemize}
The result of the the translation of a statement is an
interval, together with possible changes to the $a_D$, $I_S$,
$d_L$, and $d_G$ parameters.

The rest of this section summarizes the non-trivial features
of the translation:
\begin{description}
\item[\syntax{goto} and labeled statements:] For each label,
        a single \cons{GotoAtom} is generated to function
        as the target for \syntax{goto} jumps. This atom
        is generated when the label itself is translated,
        or when a \syntax{goto} statement referring to it
        is translated, whichever comes first.
\item[\syntax{switch}, \syntax{case}, and \syntax{default:}]
        The controlling expression in a \syntax{switch} is
        translated into a $\cons{Value}(I,e,-)$ (see section
        \vref{sec:c2coreExprRtn} for a description of different
        ways to translate expressions).

        Then the body is translated with $I_S$ initiallised
        to $I$ and $a_D$ initially empty. The translation of
        the \syntax{case:} labels then incrementally add
        tests to $I_S$ to form the test chain.

        If a \syntax{default:} label has not been met in the
        body, one is generated and appended to the translated
        body $I_B$. Then $\omega_{I_S}$ is connected to the
        \syntax{default} label and the translation of the
        whole \syntax{switch} statement is now $[\alpha_{I_S}:\omega_{I_B}]$.

\begin{figure}
  \begin{frameit}
  \begin{center}
        This figure has not been drawn yet.
  %  \leavevmode
  %  \epsfig{file=struct1.eps,height=0.7\textheight,width=0.7\textwidth}
  \end{center}
  \end{frameit}
  \caption{Translation of \syntax{switch} statements}
  \label{fig:c2coreSwitch}
\end{figure}

\item[\syntax{if} statements:] The expression is translated
        to a \cons{BoolRtn}; its ``then'' and ``else'' branches
        are connected to the translated then and else
        branches\footnote
                {The \ansiC parser generates empty
                else brances in else-less ifs}.
        A \cons{GotoAtom} is generated to serve as the exit
        of the translated statement; both branches are connected
        to it.
\item[loop statements:] The translation is obvious from the
        semantics of \ansiC loops, although there are many
        details to keep track of in the implementation.
\item[\syntax{return} statements] If the \syntax{return} statement
        has an expression, it is translated into
        $\cons{Value}(I,e,-)$. A \cons{ReturnAtom} is generated
        and appended to $I$.

        \syntax{return;} without an expression is translated to
        $[\alpha_R:-]$.
\end{description}

\subsection{Expression translation}
The following description of expression translations depend
on the ``atom'' and ``interval'' concepts defined in sections
\ref{sec:c2coreAtoms} and \ref{sec:c2coreIntervals}. In the
expression translation context an interval is always a
genuine single entry, single exit subgraph.

\subsubsection{Generating \coreC expressions}
\cons{PtrArith}, \cons{PtrCmp}, and \cons{Binary} expressions
are generated with the same constructor in C++. This constructor
can distinguish between the three cases by inspecting the
operands and the operator. In the following, applications
of this constructor are simply notated $\cons{Binary}(e_1,\circ,e_2)$.

The rules for structural sharing of types (section
\ref{sec:corecTypeSharing}) are enforced by the constructors for
\cons{Var}, \cons{PtrArith}, \cons{Member}, \cons{Array}, and
\cons{DeRef} expressions. These constructors do not take a type
parameter, and when one of those expressions are created in
this description, the type parameter is not shown.

Similarly, the constructors for \cons{Unary}, \cons{PtrCmp},
\cons{Binary}, \cons{SizeofType}, and \cons{SizeofExpr} knows
how to construct their types themselves. They are shown with
the type parameter omitted in the following, too.

\cons{Const}, \cons{Null}, and \cons{Cast} expressions are
always constructed with an explicit \coreC type.

\bigskip

\noindent\begin{tabular}{|l|l|}
\hline
\multicolumn{2}{|c|}{Pseudocode notation for building flow graph} \\
\hline
\hline
$x <- \mbox{blah blah blah}$
        & Define $x$ for the rest of the translation step \\
$\installleft{e}{\vec s}$
        & Set the left side of each $s$ to a fresh copy of $e$ \\
$\atomconnect{\omega}{\alpha}$
        & Establish atom $\alpha$ as the successor of $\omega$ \\
$==> \ldots$
        & The result of the translation is ``$\ldots$''. \\
\hline
\end{tabular}

\subsubsection{Translated expression representations}
\label{sec:c2coreExprRtn}
In order to minimize the need for introducing temporary
variables or allocating parts of the \coreC program that
later turn out to be unused, we do not fix a single canonical
shape for the result of translating an expression. Rather,
depending on the shape of an expression it can be translated
into one of the first four forms shown on figure \vref{fig:c2coreExprRtn}.

\begin{figure}\begin{center}\small\begin{tabular}{|r@{~}c@{~}l|p{10cm}|}
\hline
\multicolumn{3}{|l|}{Grammar} & comments \\
\hline
\hline
$\mathcal R$ & $::=$ & $\cons{Value}(I,e,\vec s)$ &
        $e$ is a \coreC expression that evaluates to
        the same value as the original expression,
        provided that $I$ has been executed first.
        $\vec s$ is a list of \coreC statements that
        perform side effects to take place \emph{after}
        $e$ has been evaluated. \\
\cline{4-4}
&$|$& $\cons{Place}(I,e,\vec s)$ &
        This is the same as \cons{Value}, except that
        $e$ evaluates to a pointer to the object that
        the original (lvalue) expression referred to,
        rather than its value. \\
\cline{4-4}
&$|$& $\cons{Statements}(I,\vec{\coreXref s})$ &
        Each possible execution path of $I$ contains
        exactly one of the statements $\vec s$, and the
        value assigned by that statement is the value
        of the original expression. The statements
        have yet empty left-hand sides which are filled
        according to the expression's context. \\
\cline{4-4}
&$|$& $\cons{BoolBranch}(\coreXref \alpha,
                \coreXref{\omega_1},\coreXref{\omega_0})$ &
        The original expression is a boolean value. When
        execution starts at $\alpha$ it comes out at
        $\omega_1$ or $\omega_0$ according to the expressions,
        value. \\
\cline{4-4}
&$|$& $\cons{Ignored}(I)$ &
        The expression's value is ignored.
        The interval $I$ performes its side effects. \\
\hline
\end{tabular}\end{center}
\caption{The different results of expression translation}
\label{fig:c2coreExprRtn}
\end{figure}

The \cons{Ignored} form is never spontaneously generated
by an expression but is required in certain contexts.

\subsubsection{Coercions}
If the context of an expressions requires another form than
the one it actually translates to, it is coerced into the
correct form by one or more of the primitive coercions in
figure \vref{fig:c2coreCoerce}. The sequence of primitive
coercions is determied uniquely by the rules
\begin{itemize}
\item[a)] An $\mathcal R$ is never coerced \emph{away from} a form
        that the context can accept without further coercion.
\item[b)] If generation is performed only when \cons{BoolBranch}
        is the only form accepted by the context.
\item[c)] Assignment generation is not performed where if generation
        applies.
\end{itemize}

\begin{figure}\begin{displaymath}\begin{array}{|l|l|l|}
\hline
\mbox{Name} & \mbox{input} & \mbox{action} \\
\hline
\hline
\mbox{Dereference} & \cons{Place}(I,e,\vec s) &
   ==> \cons{Value}(I,\cons{DeRef}(e),\vec s) \\
\hline
\mbox{Tempvar} & \cons{Statements}(I,\vec s) &
   \mbox{Create temporary variable $d$ of appropriate type} \\
&& \installleft{\cons{Var}(d)}{\vec s} \\
&& ==> \cons{Place}(I,\cons{Var}(d),-) \\
\hline
\mbox{Zero/one} & \cons{BoolBranch}(\alpha,\omega_1,\omega_0) &
   I_1 <- \cons{Assign}(-,\cons{Const}(\cons{Primitive}(-,\syntax{int}),
                        \syntax{1})) \\
&& I_0 <- \cons{Assign}(-,\cons{Const}(\cons{Primitive}(-,\syntax{int}),
                        \syntax{0})) \\
&& \omega <- \mbox{fresh \cons{GotoAtom}} \\
&& \forall i \in \{1,0\}: \atomconnect{\omega_i}{\alpha_{I_i}} \\
&& \forall i \in \{1,0\}: \atomconnect{\omega_{I_i}}{\omega} \\
&& ==> \cons{Statements}([\alpha:\omega],\{s_1,s_0\}) \\
\hline
\mbox{Assignment generation} & \cons{Value}(I,e,\vec s) &
   s' <- \cons{Assign}(-,e) \\
&& ==> \cons{Statements}(I+s'+\vec s,s') \\
\hline
\mbox{If generation} & \cons{Value}(I,e,\vec s) &
   I_{\neq 0} <- \mbox{new interval containg fresh copies of }\vec s \\
&& I_0 <- \mbox{new interval containing the statements }\vec s \\
&& \alpha <- \cons{IfAtom}(e,\alpha_{I_{\neq 0}},\alpha_{I_0}) \\
&& \atomconnect{\omega_I}{\alpha} \\
&& ==> \cons{BoolBranch}(\alpha_I,\omega_{I_{\neq 0}},\omega_{I_0}) \\
\hline
\mbox{Ignore I} & \cons{Value}(I,e,\vec s) &
   ==> \cons{Ignored}(I+\vec s) \\
\mbox{Ignore II} & \cons{Place}(I,e,\vec s) &
   ==> \cons{Ignored}(I+\vec s) \\
\mbox{Ignore III} & \cons{Statements}(I,\vec s) &
   \installleft{-}{\vec s} \\
&& ==> \cons{Ignored}(I) \\
\mbox{Ignore IV} & \cons{BoolBranch}(\alpha,\omega_1,\omega_0) &
   \omega <- \mbox{fresh \cons{GotoAtom}} \\
&& \forall i \in \{1,0\}: \atomconnect{\omega_i}{\omega} \\
&& ==> \cons{Ignored}([\alpha:\omega]) \\
\hline
\end{array}\end{displaymath}
\caption{Coercions for translated expressions}
\label{fig:c2coreCoerce}
\end{figure}

In some cases the
context wants a $\cons{Value}(I,e,-)$, \ie, with an empty
post-statement list. A
post-statement list is emptied by performing assignment generation,
temporary introduction, and dereferencing in that order.

\subsubsection{Translating \ansiC expressions}
\label{sec:c2coreTransExpr}
The translation of the various forms of \ansiC expressions are as
follows:

\begin{figure}\begin{displaymath}\begin{array}{|l|l|}
\hline
\mbox{Primitive} & \mbox{action} \\
\hline
\hline
\mbox{Variable identifier} &
  d <- \mbox{the \coreC declaration for the variable} \\
& ==> \cons{Place}([-],\cons{Var}(d),-) \\
\mbox{Internal function name} &
  d <- \mbox{the \coreC declaration for the function} \\
& ==> \cons{Value}([-],\cons{Var}(d),-) \\
\mbox{External function name} &
  t <- \mbox{fresh copy of \syntax{foo}'s type} \\
& ==> \cons{Value}([-],\cons{Const}(\cons{Pointer}(-,t),\syntax{foo}),-) \\
\mbox{Enumeration constant} &
  \epsilon <- \mbox{(look up the constant's definiton)} \\
& ==> \cons{Value}([-],\cons{EnumConst}(\epsilon),-) \\
\mbox{Well-known constant} &
  t <- \mbox{fresh copy of the constant's type} \\
& ==> \cons{Value}([-],\cons{Const}(t,\syntax{baz}),-) \\
\mbox{Numeric constant} &
  ==> \cons{Value}([-],\cons{Const}(\cons{Primitive}(-,...),\syntax{42}),-) \\
\mbox{String literal} &
  \mbox{Create initialized global array of \syntax{char} $d$} \\
& ==> \cons{Place}([-],\cons{Var}(d),-) \\
\hline
\end{array}\end{displaymath}
\caption{Translating primary expressions}
\label{fig:c2corePrimary}
\end{figure}

\begin{figure}\begin{displaymath}\begin{array}{|l|l|l|}
\hline
\mbox{Operator} & \mbox{\syntax{e} translates to} & \mbox{actions} \\
\hline
\hline
\syntax{e.id} & \cons{Place}(I,e,\vec s) &
   m <- \mbox{(look up the index of the member)} \\
&& ==> \cons{Place}(I,\cons{Member}(e,m),\vec s) \\
\syntax{e->id} & \cons{Value}(I,e,\vec s) &
   m <- \mbox{(look up the index of the member)} \\
&& ==> \cons{Place}(I,\cons{Member}(e,m),\vec s) \\
\hline
\syntax{e}\!\circ\!\circ & \cons{Place}(I,e,\vec s) &
   e' <- \mbox{fresh copy of }e \\
&& e'' <- \mbox{fresh copy of }e \\
&& s' <- \cons{Assign}(e',\cons{Binary}(\cons{DeRef}(e''),\circ,\syntax{1})) \\
&& ==> \cons{Place}(I,e,s' + \vec s) \\
\hline
\circ\!\circ\!\syntax{e} & \cons{Place}(I,e,\vec s) &
   e' <- \mbox{fresh copy of }e \\
&& e'' <- \mbox{fresh copy of }e \\
&& s' <- \cons{Assign}(e',\cons{Binary}(\cons{DeRef}(e''),\circ,\syntax{1})) \\
&& ==> \cons{Place}(I+e',e,\vec s) \\
\hline
\syntax{\&e} & \cons{Place}(I,e,\vec s) &
   ==> \cons{Value}(I,e,\vec s) \\
\syntax{*e} & \cons{Value}(I,e,\vec s) &
   ==> \cons{Place}(I,e,\vec s) \\
\syntax{+e} & {\mathcal R} &
   ==> {\mathcal R} \\
\syntax{-e} & \cons{Value}(I,e,\vec s) &
   ==> \cons{Value}(I,\cons{Binary}(\syntax{-},e),\vec s) \\
\syntax{\char`\~e} & \cons{Value}(I,e,\vec s) &
   ==> \cons{Value}(I,\cons{Binary}(\syntax{\char`\~},e),\vec s) \\
\syntax{!e} & \cons{Value}(I,e,\vec s) &
   ==> \cons{Value}(I,\cons{Binary}(\syntax{!},e),\vec s) \\
& \cons{BoolBranch}(\alpha,\omega_T,\omega_F) &
   ==> \cons{BoolBranch}(\alpha,\omega_F,\omega_T) \\
\hline
\syntax{sizeof e} & \cons{Value}(I,e,\vec s) &
   ==> \cons{Value}([-],\cons{SizeofExpr}(e),-) \\
\syntax{sizeof (T)} & - &
   t <- \mbox{translation of \syntax{T}} \\
&& ==> \cons{Value}([-],\cons{SizeofType}(t),-) \\
\hline
\syntax{(T) e} & \cons{Value}(I,e,\vec s) &
   t <- \mbox{translation of \syntax{T}} \\
&& ==> \cons{Value}(I,\cons{Cast}(t,e),\vec s) \\
\hline
\end{array}\end{displaymath}
\caption{Translating unary expressions}
\label{fig:c2coreUnary}
\end{figure}

\begin{figure}\begin{displaymath}\begin{array}{|l|ll|l|}
\hline
\mbox{Operator} & \multicolumn{2}{l|}{\mbox{operand translations}}
   & \mbox{actions} \\
\hline
\hline
\syntax{=} & \cons{Place}(I,e,\vec s) & \cons{Statements}(I',\vec {s'} &
    \installleft{e}{\vec{s'}} \\
&&& ==>\cons{Place}(I+I',e,\vec s) \\
\hline
\circ\syntax{=} & \cons{Place}(I,e,\vec s) & \cons{Value}(I',e',\vec{s'} &
    e_0 <- \mbox{fresh copy of }e \\
&&& e_1 <- \mbox{fresh copy of }e \\
&&& s_0 = \cons{Assign}(e,\cons{Binary}(\cons{DeRef}(e_0),\circ,e')) \\
&&& ==> \cons{Place}(I+I'+s_0,e_1,\vec s + \vec{s'} \\
\hline
\verb.&&. & \cons{Value}(I,e,\vec s) & \cons{Value}([-],e',-) &
    ==> \cons{Value}(I,\cons{Binary}(e,\syntax{\&\&},e'),\vec s) \\
& \cons{BoolBranch}(\alpha,\omega_T,\omega_F)
& \cons{BoolBranch}(\alpha',\omega'_T,\omega'_F) &
    \atomconnect{\omega_T}{\alpha'} \\
&&& \omega_0 <- \mbox{new \cons{GotoAtom}} \\
&&& \atomconnect{\omega_F}{\omega_0} \\
&&& \atomconnect{\omega'_F}{\omega_0} \\
&&& ==> \cons{BoolBranch}(\alpha,\omega'_T,\omega_0) \\
\verb.||. & & & \mbox{mutatis mutandi} \\
\hline
\syntax{,} & \cons{Ignored}(I_0) & \cons{Value}(I,e,\vec s) &
    ==> \cons{Value}(I_0+I,e,\vec s) \\
& \cons{Ignored}(I_0) & \cons{Place}(I,e,\vec s) &
    ==> \cons{Value}(I_0+I,e,\vec s) \\
& \cons{Ignored}(I_0) & \cons{Statements}(I,\vec s) &
    ==> \cons{Statements}(I_0 + I,\vec s) \\
& \cons{Ignored}(I_0) & \cons{BoolBranch}(\alpha,\omega_1,\omega_0) &
    \atomconnect{\omega_{I_0}}{\alpha} \\
&&& ==> \cons{BoolBranch}(\alpha_I,\omega_1,\omega_0) \\
\hline
\mbox{other } \circ & \cons{Value}(I,e,\vec s) & \cons{Value}(I',e',\vec{s'}) &
    ==> \cons{Value}(I+I',\cons{Binary}(e,\circ,e'),\vec s+\vec{s'}) \\
\hline
\syntax{[]} & \cons{Value}(I,e,\vec s) & \cons{Value}(I',e',\vec{s'}) &
    ==> \cons{Place}(I+I',\cons{Binary}(e,\syntax{+},e'),\vec s+\vec{s'}) \\
\hline
\end{array}\end{displaymath}
\caption{Translating binary expressions}
\label{fig:c2coreBinary}
\end{figure}

\begin{description}
\item[Primary expressions.] The translation is shown on figure
        \vref{fig:c2corePrimary}.
\item[Postfix expressions.] Except for array subscripting (which is
        shown on figure \vref{fig:c2coreBinary}) and function
        call, the translation is shown on figure
        \vref{fig:c2coreUnary}.
\item[Function calls.]
        \begin{displaymath}\begin{array}{|l|l|}
        \hline
        \syntax{e}_0\syntax{(e}_1\ldots\syntax{e}_n\syntax{)} &
          \forall i \leq n: \cons{Value}(I_i,e_i,-) <-
                \mbox{translation of \syntax{e}}_i \\
        & s <- \cons{Call}(-,e_0,e_1\ldots e_n) \\
        & ==> \cons{Statements}(I_0 + \cdots + I_n + s,s) \\
        \hline
        \end{array}\end{displaymath}
\item[Unary operators and casts.] The translation is shown on figure
        \vref{fig:c2coreUnary}.
\item[Binary operators.] The translation is shown on figure
        \vref{fig:c2coreBinary}.
\item[The conditional operator.]
        \begin{displaymath}\begin{array}{|l|l|}
        \hline
        \syntax{e}_0\syntax{?e}_1\syntax{:e}_2 &
          \cons{BoolBranch}(\alpha,\omega_2,\omega_3) <-
                \mbox{translation of \syntax{e}}_0 \\
        & \forall i \in \{1,2\}: \cons{Statements}(I_i,\vec{s_i}) <-
                \mbox{translation of \syntax{e}}_i \\
        & \forall i \in \{1,2\}: \atomconnect{\omega_i}{\alpha_{I_i}} \\
        & \omega <- \mbox{new \cons{GotoAtom}} \\
        & \forall i \in \{1,2\}: \atomconnect{\omega_{I_i}}{\omega} \\
        & ==> \cons{Statements}([\alpha,\omega],\vec{s_2}\cup\vec{s_3}) \\
        \hline
        \end{array}\end{displaymath}
\end{description}

\subsection{Implementation level (1998-11-16)}
Not very much of the \ctoc translation yet works as described
here. The old translation that is still used emits many temporary
variables that the translation described here avoids.

A full implementation of the described translation will have to
wait until the relevant parts of \coreC, especially the orthogonal
left-hand sides of statements, have been implemented.

\begin{itemize}
\item Sections \ref{sec:c2coreAtoms} and \ref{sec:c2coreIntervals}
have already been implemented, except for the \cons{StopperAtom},
and except that intervals are called \syntax{FlowGraph} in the code.
\item The basic principles of the statement translation in
section \ref{sec:c2coreStatements} have been implemented. Instead
of $I_S$ only an $\omega_S$ is part of the parameterization.
$\vec d_L$ and $\vec d_G$ are global variables, not translation
parameters.
\end{itemize}

\end{docpart}

%%% Local Variables: 
%%% mode: latex
%%% TeX-master: "cmixII"
%%% End: 

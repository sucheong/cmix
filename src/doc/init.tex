% File: init.tex
% Time-stamp: 
% $Id: init.tex,v 1.1 1999/03/02 06:09:10 makholm Exp $

\providecommand{\docpart}{\renewenvironment{docpart}{}{}
\end{docpart}
\documentclass[twoside]{cmixdoc}
%\bibliographystyle{apacite}

\makeatletter
\@ifundefined{@title}{\title{\cmix-documentation}}{}
\@ifundefined{@author}{\author{The \cmix{} Team}}%
{\expandafter\def\expandafter\@realauthor\expandafter{\@author}%
\author{The \cmix{} Team\\(\@realauthor)}}
\makeatother

\AtBeginDocument{%
\markboth{\hfill\today\quad\timenow\hfill\llap{\cmix\ documentation}}
{\hfill\today\quad\timenow\hfill}}

\renewcommand{\sectionmark}[1]{\markboth
{\hfill\today\quad\timenow\hfill\llap{\cmix\ documentation}}
{\rlap{\thesection. #1}\hfill\today\quad\timenow\hfill}}

%\newboolean{separate}
%\setboolean{separate}{true}

\renewenvironment{docpart}{\begin{document}}%
                          {\bibliography{cmixII}\printindex
                           \end{document}}
\begin{document}\shortindexingon
}
\begin{docpart}

\section{Initializer conversion}
\label{sec:Initializer conversion}

In \ansiC, an initializer in a variable declaration must be regarded
as executable code. \textit{E.g.}, the following program must output
\texttt{abab} because the initializer is executed each time the
loop body is entered:
\begin{verbatim}
int main(void) {
   int i ;
   for(i=0; i<2; i++) {
     char j = 'a' ;
     printf("%c.",j);
     j = 'b' ;
     printf("%c.",j);
   }
}
\end{verbatim}

Not so in \coreC. First, in \coreC variables are can only be declared
at the beginning of a function. If, in the above program, we were to
just move the declaration and the associated initializer of \texttt{j}
to the beginning of the function, the meaning of the program would
silently change and it would output \texttt{abbb} instead.

Moreover, it is not even safe for variables that appear at the
beginning of a function to keep their initializers. In the above
program, the loop body could just as well be a separate function that
happened to be inlined during specialization---which would again
move the declaration of j to the top of the \emph{residual} function.

This section describes how \cmix avoids that sort of problems.

\subsection{Simple initialization}
When the initializer can be expressed as a single expression---\eg
when the initialized variable has arithmetic or pointer type---the
matter is handled in the c2core translation phase. It simply directly
converts the initializer to a regular assignment statement at the
right point in the flow graph and removes the initializer from the
\coreC declaration of the variable.

\subsection{Complex initialiation}
However, not all types that can be initialized in C do have a
syntax for constant \emph{expressions}\footnote{What a silly, filthy
unorthogonal language!}. For these types it is not simple to generate
an assignment statement.

One option would be to convert the initializer into multiple
assignment statements. That approach, however, does not apply
if the elements of the initializer cannot be resolved at analysis
time (which may happen if an array index cannot be interpreted by
the analyser). Thus we reject it.

The remaining option is to move the original initializer to a new
auxiliary variable of the same type as the original one, and copy
the values between the two initializers at the point where
initialization was originally supposed to take place.

This happens to be safe because \ansiC requires that the
expressions in a complex initializer does not depend on the
values of any object, so they will evaluate to the same value
at the top of the function as they would at the point of
initialization.

On the other hand it would not be safe to let the auxiliary
variable be global, because the expressions in the initializer may
need to refer to the \emph{addresses} of other local variables, as in
\begin{verbatim}
void foo(void) {
  int i,j ;
  struct { int *a, int j } bar = { &i, sizeof j };
  /* ... */
}
\end{verbatim}

\subsubsection{Selecting candidates for initializer movement}
The generation of auxiliary variables and assignments was felt
to be possibly impeding on the efficiency of the generated program,
and certainly confusing to its human readers. Therefore we want
to only do it when it is absolutely necessary.

We believe that in practise many variables with complex initializers
are not actually changed after initialization. They could have been
\texttt{const} but programmers do not always declare them thus.
These variables do not need initializer motion: because of the
requirement that the initializing expressions are constant the
value of the initializer cannot be different each time it is
``executed''.

This suggests that the initializer motion should be performed
after the pointer analysis, so that we can decide to keep
the initializers for variables that are never the target of
any side effects in the program.

\subsubsection{Special consideration for array initializers}
A special complication arises in the not uncommon case that the
initialized variable is an array. Because C has no array assignment\footnote
{Which in turn is because C has no array \emph{expressions} that
could go on the right-hand side of such an assignment} one cannot
simply generate an assignment statement between the original and
the auxiliary array.

The remedy is to wrap the array in a struct type, since C does have
struct assignment. That is, before initializer movement on
\begin{verbatim}
   T blarf[42] = { blah blah };
\end{verbatim}
we convert it to
\begin{verbatim}
struct blarf { T a[42] } ;
/*...*/
   struct blarf blarf = { { blah blah } };
\end{verbatim}
(notice the extra set of braces in the initializer).

The point where this gets complex is that we then need to replace
any $\cons{Var}(\texttt{blarf})$ expression with
$\cons{Member}(\cons{Var}(\texttt{blarf}),1)$. Fortunately the scope
rules guarantee that such expressions are only found within the same
function as \texttt{foo} itself.

\subsection{Implementation level, 1999-03-02}

The initializer conversions has been implemented as described.
The code for the complex initializer movement is in \texttt{init.cc}

\end{docpart}
%%% Local Variables: 
%%% mode: latex
%%% TeX-master: "cmixII"
%%% End: 


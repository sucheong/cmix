% File: overview.tex

\providecommand{\docpart}{\renewenvironment{docpart}{}{}
\end{docpart}
\documentclass[twoside]{cmixdoc}
%\bibliographystyle{apacite}

\makeatletter
\@ifundefined{@title}{\title{\cmix-documentation}}{}
\@ifundefined{@author}{\author{The \cmix{} Team}}%
{\expandafter\def\expandafter\@realauthor\expandafter{\@author}%
\author{The \cmix{} Team\\(\@realauthor)}}
\makeatother

\AtBeginDocument{%
\markboth{\hfill\today\quad\timenow\hfill\llap{\cmix\ documentation}}
{\hfill\today\quad\timenow\hfill}}

\renewcommand{\sectionmark}[1]{\markboth
{\hfill\today\quad\timenow\hfill\llap{\cmix\ documentation}}
{\rlap{\thesection. #1}\hfill\today\quad\timenow\hfill}}

%\newboolean{separate}
%\setboolean{separate}{true}

\renewenvironment{docpart}{\begin{document}}%
                          {\bibliography{cmixII}\printindex
                           \end{document}}
\begin{document}\shortindexingon
}
\title{Introduction to \cmix, a Partial Evaluator for C}
\author{Jens Peter Secher}
\begin{docpart}
\maketitle


\section{Introduction}
\label{sec:Introduction}

This is part of the main documentation accompanying the \cmix system, the
other part being the \emph{\cmix user manual}. As the present document is
intended for implementation maintainance and people who would like to
experiment with extending \cmix, a fairly good knowledge of partial
evaluation is assumed. 

\subsection{\cmix Overview}

\cmix consists of two separate parts. The first part analyses the
subject program and produces another program, a \emph{generating
  extension}, based on the initial division of the input data. The
generating extension reads the static data and produces a
\emph{residual} program. This process utilises the second part of
\cmix, the specialisation library (speclib, for short), which takes
care of memory management and memorisation during specialisation.

\subsubsection{Phases}
\cmix has two internal representations of the subject program and
produces a new program in yet another language together with an
annotated program. This is depicted in figure~\ref{fig:phases}. Each
closed box depicts a transformation/analysis phase --- from the
subject program to the generating extension. After the subject program
is transformed into Core C (cf.{ }
chapter~\vref{sec:TheCoreCLanguage}), both the original C
representation and the Core C representation of the subject program is
kept in memory, and the two representations share the annotations made
by later phases. This enables us to produce an annotated program in
two flavours: a short one that resembles the original subject program;
and an elaborate one that shows the representation that the different
analyses work on.

Each phase can be instructed to produce the annotated subject program
as it appears after that particular phase.  Such annotated subject
program is represented in an abstract format (see
section~\ref{sec:output}). A file in this format can be processed by a
filter (see section~\ref{sec:filter}) and inspected by a
\emph{viewer}, \eg a HTML browser.

\begin{figure}[htbp]
  \begin{center}
\[
\entrymodifiers={=<15ex,8ex>[F]}
\xymatrix{
  *=<15ex,8ex>[o][F]{pgm\mathtt{.c}} \ar[d] \\
  {\txt{C Parser}} \ar[r] & {\txt{User\\directived}} \ar[r] & {\txt{Type\\check}}
   \ar[r] & {\txt{C to \coreC\\translation}} \ar[d] \\
  {\txt{Binding-time\\analysis}} \ar[d] &
  {\txt{Truly-locals\\analysis}} \ar[l] & {\txt{Call\\analysis}}
  \ar[l] & {\txt{Pointer\\analysis}} 
  \ar[l] \\
  {\txt{Structure\\seperation}} \ar[r] & {\txt{Sharing\\analysis}}
  \ar[r] & {\txt{Binding-time\\Sanity\\check}}\ar[r] & {\txt{In-use\\analysis}} \ar[d] \\
   *{} & *{} & {\txt{Generating-\\extension\\generator}} \ar[d] &
   {\txt{Topologically\\sort\\structures}}
  \ar[l] 
  \save "3,1"."5,4"*+[F--]\frm{}\restore
  \save "2,1"."5,4"*++[F.]\frm{} \ar"6,1" \restore \\
  *=<15ex,8ex>[o][F]{pgm\mathtt{.ann}} & *{} & *=<15ex,8ex>[o][F]{pgm\mathtt{-gen.cc}} \\
  }
\]
    \caption{\cmix analyser phases. The phases inside the dashed box
      manipulate the \coreC representation pf the program.}
    \label{fig:phases}
  \end{center}
\end{figure}


\subsection{Capabilities and restrictions}
\label{sec:CapabilitiesAndRestricitions}

Ideally, \cmix should be able to treat all strictly conforming \ansiC
programs. At the time of writing, this is not the case, but if the
subject program basically is well-typed, \cmix should not choke on it.
In particular, you should not use \verb|union|s to typecast data, only
to save memory. \cmix have the following characteristics:

\begin{itemize}
\item Polyvariant specialisation of functions: An unbounded number of
  specialised instances of a particular function can be generated.
\item Functions are allowed to contain dynamic actions yet static return value.
\item Polyvariant specialisation of basic blocks: An unbounded number of
  instances of a particular basic block can be generated.
\end{itemize}

\noindent \cmix has the following limitations:
\begin{itemize}
\item Static union restriction, cf.\ Constraint~\vref{cns:SLFUnion}
\item Bit-fields in structs are accepted, but may produce unexpected
  behaviour
\item Monovariant binding times of variables
\item Monovariant function end-configuration
\item Non-local static side effects under dynamic control are
  suspended by a very conservative approach.
\item Programs that use the \texttt{<setjump.h>} or \texttt{<stdarg.h>}
  headers cannot be specialized: the collide with the basic control-flow
  assumptions.
\end{itemize}



\subsection{Files}
\label{sec:Files}

The \cmix system consists of three parts: The analyser, the
specialization library, and the annotation viewer.

\subsubsection{Analyser}
The analyser is written in C{++} and consists of the files described
in table~\ref{tab:AnalyserFilesI}. These files are found in
\textfnam{src/analyzer}.

\begin{table}
   \begin{center}
     \begin{tabular}{ll}
\hline
Filename                    & Explanation \\ \hline
\textfnam{ALoc.\{cc,h\}}    & Abstract location sets (sets of declarations)\\
\textfnam{GNUmakefile.in}   & File for making the \cmix system \\
\textfnam{Plist.h}          & Abstract data structure for manipulating
lists of pointers\\
\textfnam{Pset.h}           & Abstract data structure for manipulating
sets of pointers \\
\textfnam{analyses.h}       & Interface file for all analyses\\
\textfnam{array.\{cc,h\}}    & Abstract data structure \\
\textfnam{auxilary.\{cc,h\}}  & Small auxiliary functions \\
\textfnam{bta.\{cc,h\}},textfnam{bt\{vars,solve\}.cc}
			   & Binding-time analysis (BTA) \\
\textfnam{btsanity.cc}      & Binding-time sanity checker \\
\textfnam{c2core.\{cc,h\}}  & Translation from C to \coreC \\
\textfnam{check.cc}         & Type check of C program \\
\textfnam{closures.cc}      & Compute sets of ``sure locals'' for each
     function \\
\textfnam{cmix.cc}          & Main \cmix function \\
\textfnam{commonout.\{cc,h\}} & Common declarations for annotation generators \\
\textfnam{corec.\{cc,h\}}   & Abstract \coreC syntax objects \\
\textfnam{cpgm.\{cc,h\}}    & Abstract C syntax objects \\
\textfnam{dataflow.cc}      & In-use analysis (IUA) (dummy at present time) \\
\textfnam{diagnostic.\{cc,h\}} & Error and warning message engine \\
\textfnam{direc.\{l,y\}}    & Parser for user directives \\
\textfnam{directives.\{cc,h\}} & User directives interpreter and repository \\
\textfnam{fileops.\{cc,h\}} & Basic file operations with error handling \\
\textfnam{fixiter.\{cc,h\}} & Generic Least-Recently-Fired fixpoint
iteration \\
\textfnam{gegen.h}          & Local definitions for the gegen phase \\
\textfnam{gg-cascades.cc}   & Gegen helper for statically split arrays \\
\textfnam{gg-code.cc}       & Gegen functions for functions,
     statements and control flow \\
\textfnam{gg-decl.cc}       & Generate code to generate residual declarations \\
\textfnam{gg-expr.cc}       & Generate code for expressions and types \\
\textfnam{gg-memo.cc}       & Generate memoisation code \\
\textfnam{gg-struct.cc}     & Handles struct issues at specialization
     and residual time \\
\textfnam{gegen.cc}         & Generating-extension generator main function \\
\textfnam{generator.cc}     & Translating generator directives to \coreC \\
\textfnam{getopt.\{c,h\}}   & Command-line options parser \\
\textfnam{lex.l},\textfnam{parser.h}
\textfnam{gram.\{cc,y\}}     & C parser \\
\textfnam{init.cc}          & Conversion of initializers to assignments \\
\textfnam{liststack.h}      & Stack of lists (used by C parser for scopes) \\
\textfnam{options.org},
\textfnam{options.perl}     & Automatic command-line options generator \\
\textfnam{outanno.h}        & Wrapper for all output generators \\
\textfnam{outcore.cc}       & Core C output generator \\
\textfnam{outcpgm.cc}       & Core C output generator \\
\textfnam{out\{pa,bta,misc\}.h} & annotation generators \\
\textfnam{output.\{cc,h\}}  & Abstract output definitions \\
\textfnam{pa.\{cc,h\}},\textfnam{paprune.cc}  & Points-to analysis (PA) \\
\textfnam{release.cc}       & The release number (mirrored from
    \textfnam{release-stamp} in the top directiry) \\
\textfnam{renamer.\{cc,h\}} & Name management \\
\textfnam{separate.cc}      & Structure splitting \\
\textfnam{share.cc}         & Function-sharability analysis (FSA) \\
\textfnam{strings.\{cc,h\}} & Common string constants \\
\textfnam{structsort.cc}    & Topologocal sorting of structures \\
\textfnam{symboltable.h}    & Generic hash+backet symboltable \\
\textfnam{syntax\{cc,h\}}   & Intermediate abstract syntax (for the parser) \\
\textfnam{taboos.\{org,perl\}}  & Reserved names (for gegen) \\
\textfnam{tags.\{cc,h\}}    & Common tags for C and \coreC \\
\textfnam{traverse.cc}      & A code template for traversing a \coreC program \\
\textfnam{varstrings.cc}    & Configured locations of CPP and shadow headers \\
\textfnam{bindist.cc}       & Generic replacement for \textfnam{bindist.cc}
     used for binary distributions \\
\textfnam{ygtree.\{cc,h\}}  & Generic balanced (yellow-green) trees
(used by lists, sets, etc.) \\
     \end{tabular}
     \caption{Analyser source files}
     \label{tab:AnalyserFilesI}
   \end{center}
 \end{table}%

\subsubsection{Specialisation library}
The specialisation library is written in C and consists of the files
described in table~\ref{tab:SpeclibFiles}. These files are found in
\textfnam{src/speclib}.

\begin{table}[htb]
   \begin{center}
     \begin{tabular}{ll}
\hline
Filename                 & Explanation \\ \hline
\textfnam{aux.c}         & Auxiliary functions \\
\textfnam{cmix.h}        & General header file \\
\textfnam{code.h}        & Intermediate format for residual code \\
\textfnam{code.c}        & Collect residual code as it is generated \\
\textfnam{floats.c}      & Lift and print floating values \\
\textfnam{mem.c}         & Memory management \\
\textfnam{pending.c}     & Management of the pending list \\
\textfnam{rusage.c}, \\
\textfnam{rusage.h}      & Resource usage statistics \\
\textfnam{unparse.c}     & Printing of a residual program \\
\hline
     \end{tabular}
     \caption{Specilisation-library source files}
     \label{tab:SpeclibFiles}
   \end{center}
 \end{table}


\subsubsection{Annotation viewer}
The annotation viewer is written in C and consists of the files
described in table~\ref{tab:CmixshowFiles}. These files are found in
\textfnam{src/cmixshow}.

\begin{table}[htb]
   \begin{center}
     \begin{tabular}{ll}
\hline
Filename                 & Explanation \\ \hline
\textfnam{GNUmakefile.in}    & Makefile \\
\textfnam{annocut.c}         & Annotation-class selection dialog \\
\textfnam{annofront.h}       & General data structures and function
                                 prototypes \\
\textfnam{annohash.c}        & Primitives for the output tree data
                                structures \\
\textfnam{annolink.c}        & Outputting annotated programs as HTML \\
\textfnam{annomisc.c}        & Auxiliary functions \\
\textfnam{annotext.c}        & Outputting annotated programs as text \\
\textfnam{frames.c}          & Dispatcher for the on-line framed browser \\
\textfnam{http.\{c,h\}}      & Generic HTTP protocol stuff \\
\textfnam{latch.\{c,h\}}     & Buffer for storing long lines of output \\
\textfnam{main.c}            & Main program, option parser, etc. \\
\textfnam{outgram.y},
\textfnam{outlex.l}          & Abstract output-format parser \\
\textfnam{server.\{c,h\}}    & Generic server loop \\
%\textfnam{service.c}         & Not distributed \\ 
\textfnam{webfront.c}        & Fork browser client, start server \\
\hline
     \end{tabular}
     \caption{Annotation viewer source files}
     \label{tab:CmixshowFiles}
   \end{center}
 \end{table}

\end{docpart}
%%% Local Variables:
%%% mode: latex
%%% TeX-master: "cmixII"
%%% End:




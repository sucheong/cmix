% File: dataflow.tex
% Time-stamp: 
% $Id: dataflow.tex,v 1.6 1999/07/19 12:01:00 makholm Exp $

\providecommand{\docpart}{\renewenvironment{docpart}{}{}
\end{docpart}
\documentclass[twoside]{cmixdoc}
%\bibliographystyle{apacite}

\makeatletter
\@ifundefined{@title}{\title{\cmix-documentation}}{}
\@ifundefined{@author}{\author{The \cmix{} Team}}%
{\expandafter\def\expandafter\@realauthor\expandafter{\@author}%
\author{The \cmix{} Team\\(\@realauthor)}}
\makeatother

\AtBeginDocument{%
\markboth{\hfill\today\quad\timenow\hfill\llap{\cmix\ documentation}}
{\hfill\today\quad\timenow\hfill}}

\renewcommand{\sectionmark}[1]{\markboth
{\hfill\today\quad\timenow\hfill\llap{\cmix\ documentation}}
{\rlap{\thesection. #1}\hfill\today\quad\timenow\hfill}}

%\newboolean{separate}
%\setboolean{separate}{true}

\renewenvironment{docpart}{\begin{document}}%
                          {\bibliography{cmixII}\printindex
                           \end{document}}
\begin{document}\shortindexingon
}
\begin{docpart}

\newcommand{\Wdef}{\ensuremath{W_{\!\mathrm{def}}}}
\newcommand{\Wmay}{\ensuremath{W_{\!\mathrm{may}}}}
\newcommand{\subdecl}[1]{\ensuremath{\widehat{#1}}}
\newcommand{\exactlyone}{\mbox{one}}
\newcommand{\keyword}[1]{\textbf{#1}}
\newcommand{\Use}{\ensuremath{U}}
\newcommand{\trulylocal}{\mbox{truelocals}}

\section{Dataflow Analysis}
The purpose of the dataflow analysis is to make a safe approximation
of how and where objects are used. The result of the analysis is used
during {\specialis}ation to avoid unnecessary {\memois}ation of
spectime data.

[Example: why in-use info makes a difference] [Example: why
do-not-alter info makes a difference]

\subsection{Write sets}
An object is \emph{written} when a statement writes a value to it. The
purpose of the write analysis is to calculate which objects are written
in each program point. The calculation of write sets are based on the
result of the pointer analysis (section~\ref{sec:PointerAnalysis})
which only gives imperfect knowledge about reference patterns. We are
therefore not able to calculate precise write information.

[Example: If $\OPT(p) |-> \{ x, y \}$, we are not able to say which
object \emph{definitely} gets written by the assignment \texttt{*p =
  1}; we can, however, say that the assigment \emph{may} write
\texttt{x} or \texttt{y}.]

We will calculate two functions that map each program point in the
program to two possibly different write sets: a definite-write set and
a may-write set, denoted $\Wdef$ and $\Wmay$, respectively. The
definite-write function is used to calculate which objects are in use
in each function.

The intuitive interpretation of $\Wdef$ and $\Wmay$ is that they
describe what may happen until the current function returns.

The may-write function is used to be sure to
{\memois}e spectime objects that may be written to by a function.
Even it the objects are not read by the function, {\memois}ation is
necessary in order to correctly replictate the side effects when the
function is later shared.

Let $D$ be the set of all objects in the program and define the
function $\subdecl{\bullet} : \powset(D) -> \powset(D)$ that
(recursively) calculates all subdeclarations for each declaration in
the argument set, \ie it calculates which objects a declaration
encompasses.

[Example: the objects in a struct are all the members and all members
of these members, and so forth. Consider the definition \texttt{struct
  S \{ int x, y; \} s;}. $\subdecl{\{s\}} = \{s, s.x, s.y\}$. ]



Define the function $\exactlyone : \powset(D) -> \powset(D)$ that
``rejects'' imprecise information:
\[
\begin{array}{lcl}
\exactlyone(\{d\}) &=& 
  \begin{array}[t]{ll}
    \keyword{if} & d \mbox{ is contained in an array, or} \\
                 & d \mbox{ is a heap-allocated object, or} \\
                 & d \mbox{ is a local variable but not truly-local} \\                 
    \keyword{then} & \emptyset \\
    \keyword{else} & \{ d \} \\                 
  \end{array} \\
  \exactlyone(\_\!\_) &=& \emptyset \\
\end{array}
\]

The write-functions can then be defined by the following recursive equations:
\[
\begin{array}{lll}
  \hfil \bullet & \hfil \Wdef(\bullet) & \hfil \Wmay(\bullet) \\
  \hrulefill & \hrulefill & \hrulefill \\
  \cons{If}(e,\coreXref{b_{\mathrm{then}}},
  \coreXref{b_{\mathrm{else}}}) & \Wdef(b_{\mathrm{then}}) \cap
  \Wdef(b_{\mathrm{else}}) & \Wmay(b_{\mathrm{then}}) \cup
  \Wmay(b_{\mathrm{else}}) \\
  \cons{Goto}(\coreXref b) & \Wdef(b) &
  \Wmay(b) \\
  \cons{Return}(e?) & \emptyset & \emptyset \\
  \cons{Assign}(e,e') & \subdecl{\exactlyone(\OPT(e))} &
  \subdecl{\OPT(e)} \\
  \cons{Call}(e?,\gamma,e_f,\vec e\,)\footnotemark
  & \subdecl{\exactlyone(\OPT(e))}
  \cup {\displaystyle \bigcap_{d_i \in \OPT(e_f)}} \Wdef(d_i) &
  \subdecl{\OPT(e)} \cup {\displaystyle \bigcup_{d_i \in \OPT(e_f)}} \Wmay(d_i) \\
  \cons{Alloc}(e,d,e') & \subdecl{\exactlyone(\OPT(e))} & \subdecl{\OPT(e)} \\
  \cons{Free}(,e') & \emptyset & \emptyset \\
  b = (\vec s,j) & \Wdef(j) \cup {\displaystyle \bigcup_{s_i \in \vec s}}
  \Wdef(s_i) & \Wmay(j) \cup {\displaystyle \bigcup_{s_i \in \vec s}}
  \Wmay(s_i) \\
  \cons{Fun}(id,t,\vec d_p,\vec d_v,\vec b\,) & \Wdef(b_1) \setminus
  \trulylocal(\bullet) &
  \Wmay(b_1) \setminus  \trulylocal(\bullet) \\
\end{array}
\]\footnotetext{If the return value is thrown away, the parts refering to
  $\OPT(e)$ are replaced by $\emptyset$. For an externally defined
  function, $\Wdef$ and $\Wmay$ is $\emptyset$ and the $\OPT$-closure
  of variables that are passed to external functions, respectively.}
The \trulylocal function defines the subset of the local variables
that are known to not be aliased between recursive calls to the function.
See section~\ref{sec:TrulyLocals}. 

\subsection{In-use sets}
The in-use information is used to avoid {\specialis}ing a function
or a program point with respect to spectime data that it does not use.

We will now define an in-use function $\Use$ that maps every \coreC
construct to a set of objects. For expressions, the in-use function
can be calculated directly from the points-to information:
\[
\begin{array}{ll}
  \hfil \bullet & \hfil \Use(\bullet) \\
  \hrulefill & \hrulefill \\
\cons{Var}(t,\coreXref d) & \emptyset \\
\cons{EnumConst}(t,\coreXref\epsilon) & \emptyset \\
\cons{Const}(t,c) & \emptyset \\
\cons{Null}(t) & \emptyset \\
\cons{Unary}(t,\diamond,e) & \Use(e) \\
\cons{PtrArith}(t,e_1,\circ,e_2) &  \Use(e_1) \cup \Use(e_2) \\
\cons{PtrCmp}(t,e_1,\circ,e_2) &  \Use(e_1) \cup \Use(e_2) \\
\cons{Binary}(t,e_1,\circ,e_2) &  \Use(e_1) \cup \Use(e_2) \\
\cons{Member}(t,e,m) &  \Use(e) \\
\cons{Array}(t,e) &  \Use(e) \\
\cons{DeRef}(t,e) &  \Use(e) \cup \subdecl{\OPT(e)} \\
\cons{Cast}(t,e) &  \Use(e) \\
\cons{SizeofType}(t,t') & \emptyset \\
\cons{SizeofExpr}(t,e) & \Use(e) \\
\end{array}
\]

For program points, the in-use function is defined by a set of
recursive equations. For each statement $s$, let $\Use(s)$ be the set
objects that are in use immediately before $s$ is executed, and let
$\overline{\Use}(s)$ be the set objects that are in use immediately
after $s$ is executed. [Example: Consider a statement $s$ and its
immediate successor $s'$; then $\overline{\Use}(s) = \Use(s')$.
\[
\begin{array}{ll}
  \hfil \bullet & \hfil \Use(\bullet)  \\
  \hrulefill & \hrulefill \\
  \cons{If}(e,\coreXref{b_{\mathrm{then}}},
  \coreXref{b_{\mathrm{else}}}) & \Use(e) \cup \Use(b_{\mathrm{then}}) \cup
  \Use(b_{\mathrm{else}}) \\
  \cons{Goto}(\coreXref b) & \Use(b) \\
  \cons{Return}(e?) & \Use(e) \\
  \cons{Assign}(e,e') & \Use(e) \cup \Use(e') \cup
  (\overline{\Use}(\bullet) \setminus \Wdef(\bullet)) \\
  \cons{Call}(e?,\gamma,e_f,\vec e\,) & \Use(e) \cup \Use(e_f) \cup
  {\displaystyle \bigcup_{e_i \in \vec e}} \Use(e_i) \cup
  {\displaystyle \bigcup_{d_i \in \OPT(e_f)}} \Use(d_i) \cup
  (\overline{\Use}(\bullet) \setminus
  \Wdef(\bullet)) \\  
  \cons{Alloc}(e,d,e') & \Use(e) \cup \Use(e') \cup
  (\overline{\Use}(\bullet) \setminus \Wdef(\bullet)) \\
  \cons{Free}(,e') & \Use(e') \cup (\overline{\Use}(\bullet) \setminus
  \Wdef(\bullet)) \\ 
  b = (\vec s,j) & \Use(s_1) \qquad (\mbox{or } \Use(j) \mbox{ if }
  \vec s = []) \\
  \cons{Fun}(id,t,\vec d_p,\vec d_v,\vec b\,) &
  \Use(b_1) \setminus \trulylocal(\bullet) \\
\end{array}
\]

\subsection{Constraint Formulation}
The recursive equations above can be stated as a constraint system
that can be solved by fixpoint iteration. The constraints employed are
\[
\begin{array}{ccccc}
X = \bigcup Y_i &
X = \bigcap Y_i &
X = Y \setminus Z &
X \subseteq Y &
X \supseteq Y
\end{array}
\]
where $X,Y,Z \in \powset(D)$. 

For the definitely-write function, we want to find the greatest
fixpoint\footnote
	{because otherwise a loop in the function would block
	the propagation of definitely-write information},
and we therefore start out with the assumption that all
objects are killed. The constraints generated are thus
\[
\begin{array}{lll}
  \hfil \bullet & \hfil \mbox{Initial value} & \hfil \mbox{Constraints} \\
  \hrulefill & \hrulefill & \hrulefill \\
  \cons{If}(e,\coreXref{b_{\mathrm{then}}},
  \coreXref{b_{\mathrm{else}}}) & \Wdef(\bullet) = D & \Wdef(\bullet) \subseteq
  \Wdef(b_{\mathrm{then}}) ,\hfil \Wdef(\bullet) \subseteq
  \Wdef(b_{\mathrm{else}}) \\
  \cons{Goto}(\coreXref b) & \Wdef(\bullet) = D & \Wdef(\bullet) \subseteq \Wdef(b) \\
  \cons{Return}(e?) & \Wdef(\bullet) = \emptyset &  \\
  \cons{Assign}(e,e') & \Wdef(\bullet) = \subdecl{\exactlyone(\OPT(e))} & \\
  \cons{Call}(e?,\gamma,e_f,\vec e\,)
  & \Wdef(\bullet) = D
  & \Wdef(\bullet) = {\displaystyle \bigcup} \{
  \subdecl{\exactlyone(\OPT(e))}, X_{\mathrm{tmp}} \} \\
  & X_{\mathrm{tmp}} = D & X_{\mathrm{tmp}} \subseteq \Wdef(d_i)
  \qquad \forall d_i \in \OPT(e_f)  \\
  \cons{Alloc}(e,d,e') & \Wdef(\bullet) = \subdecl{\exactlyone(\OPT(e))} \\ 
  \cons{Free}(,e') & \Wdef(\bullet) = \emptyset \\
  b = (\vec s,j) & \Wdef(\bullet) = D & \Wdef(\bullet) =
  {\displaystyle \bigcup} \{ \Wdef(j) \} \cup \{ \Wdef(s_i) |
  s_i \in \vec s \}  \\
  \cons{Fun}(id,t,\vec d_p,\vec d_v,\vec b\,)
  & \Wdef(\bullet) = D & \Wdef(\bullet) = \Wdef(b_1) \setminus
  \trulylocal(\bullet) \\
\end{array}
\]

For the may-write function, we want to find the least fixpoint, and we
therefore start out with the assumption that no objects are killed.
The constraints generated are thus
\[
\begin{array}{lll}
  \hfil \bullet & \hfil \mbox{Initial value} & \hfil \mbox{Constraints} \\
  \hrulefill & \hrulefill & \hrulefill \\
  \cons{If}(e,\coreXref{b_{\mathrm{then}}},
  \coreXref{b_{\mathrm{else}}}) & \Wmay(\bullet) = \emptyset & \Wmay(\bullet) \supseteq
  \Wmay(b_{\mathrm{then}}) ,\hfil \Wmay(\bullet) \supseteq
  \Wmay(b_{\mathrm{else}}) \\
  \cons{Goto}(\coreXref b) & \Wmay(\bullet) = \emptyset & \Wmay(\bullet) \supseteq \Wmay(b) \\
  \cons{Return}(e?) & \Wmay(\bullet) = \emptyset &  \\
  \cons{Assign}(e,e') & \Wmay(\bullet) = \subdecl{\OPT(e)} & \\
  \cons{Call}(e?,\gamma,e_f,\vec e\,)
  & \Wmay(\bullet) = \subdecl{\OPT(e)}
  & \Wmay(\bullet) \supseteq \Wmay(d_i) \qquad \forall d_i \in
  \OPT(e_f)  \\ 
  \cons{Alloc}(e,d,e') & \Wmay(\bullet) = \subdecl{\OPT(e)} \\ 
  \cons{Free}(,e') & \Wmay(\bullet) = \emptyset \\
  b = (\vec s,j) & \Wmay(\bullet) = \emptyset & \Wmay(\bullet) \supseteq
  \Wmay(j) , \Wmay(\bullet) \supseteq \Wmay(s_i) \quad
  \forall s_i \in \vec s  \\
  \cons{Fun}(id,t,\vec d_p,\vec d_v,\vec b\,)
  & \Wmay(\bullet) = \emptyset & \Wmay(\bullet) = \Wmay(b_1) \setminus
  \trulylocal(\bullet) \\
\end{array}
\]

When the definitely-write function has been calculated, the in-use
function can be calculated. We want to find the least fixpoint, so the
constraints generated are thus
\[
\begin{array}{lll}
  \hfil \bullet & \hfil \mbox{Initial value} & \hfil \mbox{Constraints} \\
  \hrulefill & \hrulefill & \hrulefill \\
  \cons{If}(e,\coreXref{b_{\mathrm{then}}},
  \coreXref{b_{\mathrm{else}}}) & \Use(\bullet) = \Use(e) & \Use(\bullet) \supseteq
  \Use(b_{\mathrm{then}}) ,\hfil \Use(\bullet) \supseteq
  \Use(b_{\mathrm{else}}) \\
  \cons{Goto}(\coreXref b) & \Use(\bullet) = \emptyset & \Use(\bullet) \supseteq \Use(b) \\
  \cons{Return}(e?) & \Use(\bullet) = \Use(e)   \\
  \cons{Assign}(e,e') & \Use(\bullet) = \Use(e) \cup \Use(e') & \Use(\bullet)
  \supseteq X_{\mathrm{tmp}} \\
  & X_{\mathrm{tmp}} = \emptyset & X_{\mathrm{tmp}} =
  \overline{U}(\bullet) \setminus \Wdef(\bullet) \\
  \cons{Call}(e?,\gamma,e_f,\vec e\,)
  & \Use(\bullet) = \Use(e) \cup \Use(e_f) \cup \Use(e_i) 
  & \Use(\bullet) \supseteq X_{\mathrm{tmp}}, \Use(\bullet) \supseteq \Use(d_i) \qquad \forall d_i \in
  \OPT(e_f)\\
  & X_{\mathrm{tmp}} = \emptyset 
  & X_{\mathrm{tmp}} = \overline{U}(\bullet) \setminus \Wdef(\bullet) \\
  \cons{Alloc}(e,d,e') & \Use(\bullet) = \Use(e) \cup \Use(e') & \Use(\bullet)
  \supseteq X_{\mathrm{tmp}} \\ 
  & X_{\mathrm{tmp}} = \emptyset 
  & X_{\mathrm{tmp}} = \overline{U}(\bullet) \setminus \Wdef(\bullet) \\
  \cons{Free}(,e') & \Use(\bullet) = \Use(e') & \Use(\bullet)
  \supseteq X_{\mathrm{tmp}} \\ 
  & X_{\mathrm{tmp}} = \emptyset 
  & X_{\mathrm{tmp}} = \overline{U}(\bullet) \setminus \Wdef(\bullet) \\
  b = (\vec s,j) & \Use(\bullet) = \emptyset & \Use(\bullet) \supseteq
  \Use(s_1) \qquad (\mbox{or } \Use(j)) \\
  \cons{Fun}(id,t,\vec d_p,\vec d_v,\vec b\,)
  & \Use(\bullet) = \emptyset & \Use(\bullet) = \Use(b_1) \setminus
  \trulylocal(\bullet) \\
\end{array}
\]

\subsection{Implementation level (1999-07-19)}
\label{sec:DataflowImplementationLevel}
\index{implementation level!dataflow analysis}

The dataflow analyses have been implemented according to the above
description.

\end{docpart}
%%% Local Variables: 
%%% mode: latex
%%% TeX-master: "cmixII"
%%% End: 


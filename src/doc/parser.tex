% File: parser.tex
% Time-stamp: 

\providecommand{\docpart}{\renewenvironment{docpart}{}{}
\end{docpart}
\documentclass[twoside]{cmixdoc}
%\bibliographystyle{apacite}

\makeatletter
\@ifundefined{@title}{\title{\cmix-documentation}}{}
\@ifundefined{@author}{\author{The \cmix{} Team}}%
{\expandafter\def\expandafter\@realauthor\expandafter{\@author}%
\author{The \cmix{} Team\\(\@realauthor)}}
\makeatother

\AtBeginDocument{%
\markboth{\hfill\today\quad\timenow\hfill\llap{\cmix\ documentation}}
{\hfill\today\quad\timenow\hfill}}

\renewcommand{\sectionmark}[1]{\markboth
{\hfill\today\quad\timenow\hfill\llap{\cmix\ documentation}}
{\rlap{\thesection. #1}\hfill\today\quad\timenow\hfill}}

%\newboolean{separate}
%\setboolean{separate}{true}

\renewenvironment{docpart}{\begin{document}}%
                          {\bibliography{cmixII}\printindex
                           \end{document}}
\begin{document}\shortindexingon
}
\begin{docpart}

\begin{center}
  \fbox{\huge\textsl{THIS SECTION IS OUT-OF-DATE}}    
\end{center}

\section{The C Parser}
\label{sec:Parser}\index{parser}

The input to \cmix is a program fragment in form of a set of C source
files. The purpose of the parser is to transform the set of C source
files into an internal representation of the program fragment such
that all dependencies between the files are resolved. In the rest of
this section we will denote such a program fragment as \emph{the
 subject program}.\index{subject program}

\subsection{Lexing and preprocessing}
Each source is first put through the standard C preprocessor available
on the system (defaults to \texttt{cpp}) and then divided into tokens
by GNU \texttt{flex}. The file \texttt{lex.l} specifies how the input
files are divided into tokens. This is done in a standard way, except
that identifiers are divided into \texttt{typedef}ed names and regular
names. The lexer also picks up \texttt{\#pragma}s that contain user
annotations.

\subsection{The parser}
We use the Roskind Grammar~\cite{Roskind:1996:CGrammar}, a freely
available grammar for the full \ansiC. This grammar works with
standard YACC tools (we use GNU bison).

The parser (\texttt{gram.\{y,cc\}}) parses the tokens (produced by by
the lexer) and creates an internal representation of the subject program.
During parsing, a stack of symbol tables
(\texttt{liststack.h},\texttt{symboltable.h}) keep track of the
variable and function defintions such that the stack represents the
static scope of names. 

For parsing declarations, a
set of ``switchboard'' containers (\texttt{syntax.\{h,cc\}}) are
created and filled with information during parsing. From these
switchboards, the internal representation of the subject program's types is
created. For instance, consider the declaration \texttt{char *const
  (*(*g)()) []}, \ie variable \texttt{g} is a pointer to function
returning pointer to array of const pointer to char.  The type of
\texttt{g} is created by merging the switch boards for the prefix and
the postfix (which themselves are merged from other switchboards).

\subsection{Linkage}
\cmix can read in multiple source files and ``link'' them together
to produce a single module for specialization.

Earlier this was implemented by sharing the declarations
between the \ansiC representations of the source files. This created
problems with structs, however: C uses structural equivalence for
struct types between source files, so it is legal for the members of a
struct to have different names in different files. Sharing the
declarations implied that the types would also be identical in several
files, so there was no way to tell the type checker that, say,
\texttt{f().x} is the first member of the struct that \texttt{f}
returns, when mentioned in one file, but the second member of the
struct when mentioned in another file.

Now we use the following strategy:

An identifier with external linkage has a declaration structure for
each of the source files it is declared in. Each of these declarations
is typed with local types.

The parser also maintains, however, a union--find structure on
declarations that tie all the declarations for each externally linked
identifier together. This union--find structure is maintained such
that the definition, if one has been met, is the representative
element of the declaration set.

When a new declaration is constructed, a type unification routine is
run to make sure its type structurally matches the types of any
previous declarations for it. During this unification process it might
be needed to unify usertype declarations; a parallel union--find
structure for usertype declarations is used to build an equivalence
relation that will later be used by the c2core translation of types.

\subsection{Initializers}
Inializers are not evaluated during parsing.

\subsection{Internal C representation}

\subsection{Parsing types}

\subsubsection{basic types}
The \ansiC specification gives tremendous freedom with regard to type
specifications which is a problem when building a parser. The
following observations are important when creating a faithful \ansiC
parser:

\begin{enumerate}
  \item There are two type qualifiers, \texttt{const} and
    \texttt{volatile}. Each of these can be left unspecified or they can
    be specified a number of times, \eg \texttt{const volatile const
      volatile a}, which should give a warning.
  \item There are five basic types which are \texttt{int},
    \texttt{char}, \texttt{float}, \texttt{double}, \texttt{void}. The
    base type can be left unspecified, in which case it will be
    considered an \texttt{int}.
  \item There are two size specifiers, \texttt{short} and
    \texttt{long}. If none is specified, the size is ``normal''.
  \item There are two sign qualifiers, \texttt{signed} and
    \texttt{unsigned}. If non is specified, the type will be
    considered \texttt{signed}.
\end{enumerate}

Testing that a declaration/definition is in order can be done by
having a ``switchboard'' structure (\texttt{Parse\_Type}) with a field
for each of the above groups. During parsing, such a structure can be
filled in and checks can be made whether two such structures can be
combined; this is done by the \texttt{combine} function.
\texttt{combine} is appropriate for productions ``below''
\texttt{type\_qualifier\_list} and \texttt{base\_type\_specifier} in
the grammar.  The result is a new \texttt{BaseType}.

\subsubsection{User-defined types}
We use the term user-defined types about \texttt{struct}ures,
\texttt{union}s and \texttt{enum}erations.  Appropriate for
productions \texttt{sue\_type\_specifier} in the grammar. User defined
types are put in the global symboltable \texttt{userdecl}.

\subsubsection{Type synonyms}
Type synonyms are types defined by \texttt{typedef}s.
\texttt{typedef}ined names have to be registered as such as soon as
possible [insert example that shows this].

All type synonyms are expanded during parsing, so the internal
representation of the subject program does not contain type synonyms.

\subsection{Storage modifiers}
There are five storage modifiers, \texttt{static}, \texttt{extern},
\texttt{typedef}, \texttt{auto}, \texttt{register}. If non is
specified, \texttt{auto} is assumed.  These modifiers are also checked
by \texttt{combine}, because only one of them is allowed at any time.

\subsection{Declarators}
Declarators are variable identifiers and possibly a bit-field size.

\subsection{Internal representation of C programs}
All data structures are annotated by a line number. This makes it
possible to give meaningful user feed-back.

\subsection{Declarations and Definitions}
A declaration that allocates storage is called a
\emph{definition}. Consider the program
  \begin{verbatim}
     struct S2;
     struct S1 { struct S2 *s2p; };
     struct S2 { struct S1 *s1p; };
  \end{verbatim}
The first line is a declaration and last two are definitions. Consider
the legal program
  \begin{verbatim}
     struct S1 { struct S3 *p; };
     struct S2 { struct S3 *q; };
     struct S3 { struct S2 *r; };
  \end{verbatim}
When \texttt{struct S3} is met for the first time (in the definiation
of \texttt{struct S1}), it is not necessary to know the size of
\texttt{struct S3}, since we just need a pointer to such a user type.
This is also the case in the second line, but here we must ensure that
\texttt{p} and \texttt{q} have the same type. It is thus necessary to
introduce a declaration that serves as a place holder as soon as a new
names is met. When the definition appears, the missing information is
filled into the placeholder. A definition must have a \emph{complete}
type.


User types can be declared in an inner scope. During parsing we need
to maintain several levels of scope (\texttt{usertypes}), \eg in
  \begin{verbatim}
     struct S;
     enum { A } E;
     void f()
     {
        enum { A } E;
        struct S;
        ...
     }
  \end{verbatim}
But it seems that most compilers complains about the initializer not
being of the right type in the following (but not a redeclaration).
  \begin{verbatim}
    struct S { int a; };

    int f1 (struct S b)
    {
       struct S;
       struct S { char d; };
       struct S c = b;
       return 42;
    }
  \end{verbatim}
so we can just make a general lookup in case there is no definition of
the structure. If it is a definition, we make a new structure.

\begin{itemize}
\item In declarations it is the \emph{types} that flow upwards in the
  parse tree. If a declaration is a user type, the actual declaration
  (\ie members, etc) is put in the \texttt{usertypedecls} table as a
  side-effect.
\item Shouldn't \emph{varargs} be a type (instead of an attribute for
  functions)? No, I think we need both a tag in a function decl and a
  special type.
\item Types of variables and funtions are lists of
  \texttt{Type}s.
\item \grule{abstract\_declarator} (using
  \grule{postfixing\_abstract\_declarator},
  \grule{array\_abstract\_declarator},
  \grule{unary\_abstract\_declarator},
  \grule{postfix\_abstract\_declarator}) are concerned with pointers,
  arrays and functions in declarations (\ie the cast \texttt{(char
    \underline{**[2][3]});}). They should result in a list of types
  (\ie array 2 of array 3 of pointer to pointer to char).
  $\Rightarrow$ build a temporary list during parsing and use this
  list later on to construct a real type list. For array sizes, make a
  reverse list.
\item All declarations are put in scope as side-effect --- it is the
  types that propagate up to through the grammar rules. When a scope
  ends, the declarations (objecs,structures) are placed in the right
  spot (\eg a function definition, a statement block, etc.). Wait, no!
  The problem is that the identifier in a declaration appears in the
  middle of the type (\eg \texttt{int *\underline{a}[]}). Hmm, in case
  of \grule{indentifier\_declarator} we could propagate the both the
  identifier and the list of postfixes. $\Rightarrow$
  \texttt{Parse\_IdPostfix}.
\item We do not want to make a full type (in the sense of the user
  class \texttt{Type}) or a full declaration
  (\textsl{xxx}\texttt{Decl}) before it can be fully instantiated (but
  we have to, anyway).  Different temporary forms exist
  (\texttt{Parse\_}\textsl{xxx} in \texttt{syntax.h}) as fragments of
  a real type/declaration.
\item When the base type (\ie ``the end point'' of a type chain) is
  known, only pointers, array, and functions postfix modifiers with
  qualifiers (\texttt{const}/\texttt{volatile}) can occur later on.
  Thus, at this point a real type (\texttt{Type}) can be instantiated
  and parsed on. The above mentioned postfixes can then be converted
  to a list of types, effectively pushing the base type in front of
  them.
\item is it wise to represent ellipsis as a type (at the end of a type
  list) as it is now, or should it be an attribute?
\item Some of the namespaces are only used as place-holders for a
  particular name. The actual declaration is placed under another
  declaration, to which it belongs.
\end{itemize}

\subsection{New Design}
\begin{itemize}
\item Typedefs, variables, prototypes, and enum constants can all be
  put in the class \texttt{VarDecl}. Special cases: Prototypes are
  not allowed to have initializers (and ellipsis is a type); Typedefs
  can only have storage specifier \texttt{Typedef} and cannot have
  initializers; Enum constants cannot have storage specifier.
  Function \emph{definitions} are in the same namespace, which means
  we need to look in both scopes to check for name clashes.
\item What about function type declarations? Where are they handled
  (or: are \emph{all} the ``redundant parenthesis'' really redundant)?
\item The grammar would like us to put the (type of) the identifier in
  scope as soon as possible. In the case of \eg \texttt{int
    a=sizeof(a);} we need to be able to look up the type of \texttt{a}
  when the expression is parsed.
\item Lists of declarations (\texttt{list<Vardecl*>}) are parsed
  around between the rules, until such a list can be put in the right
  place, \eg under a function definition. To create a list from \eg
  ``\texttt{int a,*b,c}'', we also need to carry around the \emph{base
    type} of such a declaration; \texttt{int} in this case. An
  alternative to this is to associate a stack of lists with each
  namespace and put declaration there. This would be a good choice for
  user types, as they ``pop up'' inside regular declarations.
\item Struct/enum/typedef types should point to the actual declaration
  so we don't have to search for anything after the program has been
  parsed. This also goes for function calls; they should point to
  function definitions.
\item Now I got it! We hold the user types in a stack of lists that
  follows the normal scope and is re-enterable (\texttt{liststack.h}).
  This enables us to deal with the user types as mere side-effects. At
  the end of a block the full list of user types can be put where it
  belongs. 
\item Old-style function parameters are inserted into current scope as
  declarations with no type. When the proper declaration turns up the
  declaration is updated. The global variable \texttt{old\_function}
  signifies that we are parsing an old-style function definition,
  which enables the appropriate checks and update.
\item Typedef types point to the declaring \texttt{VarDecl}. This way
  we can give meaningful messages and it should be possible to keep
  typedefs as real typedefs (\ie no weird expansion). This, however,
  demands that great care is taken to follow typedefs when comparisons
  are made: both operands in a comparison could potentially be
  typedefs. We a \texttt{normalize} function on types.
\item Local enum declarations in old-style function [this is realy
  bizarrre!] need to temporarily lower the \texttt{old\_function} flag
  because the enum constants are in the same namespace as the fromal
  parameters.
\item A global variable \texttt{reenter\_scope\_at\_next\_block} is needed
  to tell \texttt{enter\_scope()} to continue with the parameter
  declarations when the statements of a funtion definition is parsed.
  Wait, no, the statement can just start yet another scope (as they
  normally do). Now objects, like userdecls can be put in a liststack
  --- the symboltable only holds the names and pointers into this list.
  No, this makes it \emph{realy} hard to enforce the rules about
  variables only being declared once because the parameters of a
  function \emph{is} in the same scope as the rest of the variables
  declared in the body of the function. We still have to separate
  these, though. This can be done by using the liststack method and
  ``emptying'' the list of declarations when the parameter list is
  finish (but keeping the \emph{names} in scope).
\item Initializers are trouble. It is possible to use \texttt{sizeof}
  constructs in constant expressions in a program (which then makes it
  implementation dependend, of course). This means that we have to
  make an array like \texttt{char a[sizeof(int)]} dynamic because the
  size is unknown. This brings us another problem: if such a structure
  has an initializer, we would have to keep this initializer attached
  to the structure in the CoreC representation (\ie not transform all
  initializer to assignments), because it would be an error if the
  initializer is too big, and if it too small the structure should be
  padded with zeros. Solution: we provide an ``evaluate-sizeof''
  switch which enables the user to force sizeof expression to be
  static. If this switch is not turned on and initializers are used
  for a structure that thus becomes dynamic, we abort specialization.
\item We have to backpatch in some cases. An array with no explicit
  size but initializers have to be backpatched, and unspecified types
  in old-style functoin declarations have to be converted to
  \texttt{int}s. It is also necessary to ``normalize'' initializer
  when they are converted to assignments.
\item It is a good idea to keep the string representation of
  constants, in case they must be lifted or as an aid when giving
  feed-back.
\item Characters are kept as ints (we do not support wide charaters).
  In general there is a problem with literal constants. If we don't
  want to be implementation dependend, we cannot do must partial
  evaluation. This is because constants take the ``least'' type they
  fit into.
\item The lexer/parser can be configured to serve as a general purpose
  ANSI C lexer/parser by setting \texttt{only\_ANSI} in \texttt{lex.l}
  to 1. Otherwise, the lexer/parser will recognize special \cmix words
  (\texttt{pure}, \texttt{residual}, \texttt{spectime} and
  \texttt{cmix\_}\dots) which will give rise to special attributes in
  the internal representation\ref{internal-rep}. In the former case
  these attributes can safely be ignored.
\item Labels have function scope $\Rightarrow$ only one symboltable
  for labels. To resolve forward references from \texttt{goto}
  statements to labels, the following is done. The label symboltable
  contains \emph{backpatch fields}; Backpatch fields
  (\texttt{Indirection}) contain pointers to label statements; A label
  statement contains the label name and the actual statement; the goto
  statements contain pointers to backpatch fields. The indirection
  introduced by the backpatch field makes it easy to do backpathing.
\item Shouldn't prototype result in \texttt{FunDef}initions? If so, a
  truly external function could be recognized as one that has a null
  body.
\item Initializer are added after the declaration. This means we need
  the symboltable to cash the last added item. (Returning the last
  added declaration is not a good idea since it then has to be wrapped
  into something in the parser in order to keep it in scope.)
\item \texttt{return} statements have to know what type the embedding
  function must return to be able to make casts/checks.
  
\end{itemize}

\subsection{Sanity check and internal linking}
\begin{itemize}
\item \cmix reads in all files. 
    
\end{itemize}

\subsection{Run-through}
\subsubsection{Types}
When all files are parsed we run through the program. Every type has
to be complete (except \texttt{typedef}s) and every construction need
to abey the ANSI standard. To avoid processing a construction several
times we need a check flag in every construction.

\subsubsection{Types}
Every internal structure should have a \texttt{check} functions that
either approves of the internal state and returns a type, or complains
about it. This is done recursively. We could, at this point, make
every implicit cast explicit --- when all subexpressions have been
checked, we could call an \texttt{explicit} function that either
returns the subexpression as is (if the types match) or inserts an
explicit cast. The latter is also done when an expression statement
discards its return value.

To do this we need a \texttt{composite} function that returns the
composite type of the subexpressions, \eg in \texttt{1 + 2.0} the type
of composite type of the the two subexpressions would be
\texttt{float} and this could be parsed together with each
subexpression to the explicit function. This way the above expression
would become \texttt{(float) 1 + 2.0}. As a side-effect the expression
would get this type.

What about structures? They do not need to have the same name, only be
structurally equal. And this is a problem when we have (mutually)
recursive structs: To avoid infinite comparison we need a ``mark''
that tells us that we already are in the process of comparing these
two structs (or unions). This mark has to be an integer that we are
sure that a depth-first search is not tricked: the two marks have to
have the exact same value. The mark could be a class variable for
\texttt{StructDecl}s and get counted up every time two unmarked
structs are seen.

There are two kinds of type ``equality'' in C: 1) a \emph{full}
equlity where the types have to be \emph{exactly} the same, \ie same
qualifiers and main type (recursively). 2) a \emph{cast} equality
where one type can be more or equally qualified, but not less
(recursively).  The first kind is needed when \eg structures are
compared. The second one is needed in \eg an assignment \texttt{a =
  b}, where \texttt{a} is allowed to be more qualified than
\texttt{b}. Therefore we need a parameterized equality function for
each type to do both sorts of comparison.

[Maybe all types (and declarations) should have a ``isComplete'',
``isScalar'', \etc flag. This way it can be set/resolved once only.]

In function call expressions, the left-hand side has to be a pointer
to a function (!). This is a problem, since a function
\emph{definition} is considered a function \emph{designator} and not a
pointer to a function. May we should make an explicit conversion.

When temporary variables are generated from arrays, they need to be of
type pointer.

There is a problem with ``normalizing'' all types during translation
to Core C: \syntax{typedef}s annotated ``include'' (\ie they are to be
omitted from the residual, but defined in some included file instead)
should not be ``normalized''. Maybe normalizing should only be done to
disambiguate (\eg ``is this an array type?''). What happens if we try
to normalize an incomplete type?

[In general: we need two error and warning functions: one with line
number and one without. And a debug output functions that work with
streams Or a special debug stream; we want free form expressiveness
--- actually we also want that for warnings. What about errors? here
we want to output something (preferably in free form) and then stop
execution. Could this be done by operators on streams alone? \eg.

\begin{verbatim}
    global.fatal << fatalMsg << e->pos()
                 << e->name << `` is not ...'' << die();
\end{verbatim}
Maybe it could be done with macros?
]

We need to copy types in the \texttt{Cpgm} representation.

\texttt{PostExpr} and \texttt{PreExpr} can be the same type.

Expressions have to be tagged whether or not they are rvalues or
lvalues.

\subsection{Output}
\begin{itemize}
\item Potentially, every declaration needs an anchor. To avoid putting
  excessively many anchors in the output, each declaration annotation
  must have an indicator that controls whether this declaration should
  have an anchor. Anchors are taken from a global pool to make them
  unique. [This causes problems when using multiple files.] Also,
  every object that is connected with a declaration needs a back-link
  to the declaration to facilitate output, \ie annotations need the
  name of a declaration (and its anchor) to produce readable output.
\end{itemize}

\subsection{Implementation level (01.03.1999)}
\label{sec:ParserImplementationLevel}
\index{implementation level!parser}

The parser is implemented as described above, except
\begin{enumerate}
\item If several declarations of the same union exist, we require that
  the members are declared in the same order. (According to \ansiC, all
  possible permutations should be tried out.)
\item 
\end{enumerate}

Files: \texttt{lex.l}, \texttt{gram.y}, \texttt{gram.cc},
\texttt{liststack.h}, \texttt{syntax.\{h,cc\}},
\texttt{cpgm.\{h,cc\}}, \texttt{tags.\{h,cc\}},
\texttt{symboltable.h}, \texttt{}, \texttt{}. 


\end{docpart}
%%% Local Variables: 
%%% mode: latex
%%% TeX-master: "cmixII"
%%% End: 


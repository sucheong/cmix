% Edit Mode: -*- LaTeX -*-
% File: pa.tex
% Time-stamp: <98/06/08 12:33:53 panic>
% $Id: pa.tex,v 1.3 1999/06/03 17:24:32 jpsecher Exp $

\providecommand{\docpart}{\renewenvironment{docpart}{}{}
\end{docpart}
\documentclass[twoside]{cmixdoc}
%\bibliographystyle{apacite}

\makeatletter
\@ifundefined{@title}{\title{\cmix-documentation}}{}
\@ifundefined{@author}{\author{The \cmix{} Team}}%
{\expandafter\def\expandafter\@realauthor\expandafter{\@author}%
\author{The \cmix{} Team\\(\@realauthor)}}
\makeatother

\AtBeginDocument{%
\markboth{\hfill\today\quad\timenow\hfill\llap{\cmix\ documentation}}
{\hfill\today\quad\timenow\hfill}}

\renewcommand{\sectionmark}[1]{\markboth
{\hfill\today\quad\timenow\hfill\llap{\cmix\ documentation}}
{\rlap{\thesection. #1}\hfill\today\quad\timenow\hfill}}

%\newboolean{separate}
%\setboolean{separate}{true}

\renewenvironment{docpart}{\begin{document}}%
                          {\bibliography{cmixII}\printindex
                           \end{document}}
\begin{document}\shortindexingon
}
\title{Pointer Analysis for Core C}
\author{Jens Peter Secher}
\begin{docpart}
\maketitle

\mathligsoff
\begin{inferencesymbols}
\renewcommand{\predicate}[1]{$ #1 $}

\begin{center}
  \fbox{\huge\textsl{THIS SECTION IS OUT-OF-DATE}}    
\end{center}

\section{Pointer analysis}
\label{sec:PointerAnalysis}

\subsection{Preliminaries}
\label{sec:PApreliminaries}

Core C is the internal representation of C programs in \cmix (cf.\ 
section~\vref{sec:TheCoreCLanguage}). The purpose of the pointer
analysis is to make a \emph{safe} approximation of which memory
locations pointers \emph{may} point to during program
execution~\cite{Andersen:1994:ProgramAnalysisAndSpecialization}. A
piece of memory will from here on be denoted an \emph{object} or
\emph{abstract location}.  Consider this program where \texttt{d} is
dynamic:

\begin{verbatim}
  int f(int d)
  {
    int x = 1;
    int *p = &x;
    *p = d;
    ...
  }
\end{verbatim}

\noindent
If no points-to information was present, the binding-time analysis
would wrongly assume that \texttt{x} was static, which is not the
case since a dynamic value is assigned to \texttt{x} through
the pointer \texttt{p}.

\subsection{The Analysis}
\label{sec:PAResult}
Given the domains of objects $\delta \in \DAloc$, definitions $d \in
\DD$, basic blocks $b \in \DB$, statements $s \in \DS$, control
statements (jumps) $j \in \DJ$ and expressions $e \in \DE$ described
in sections~\vref{sec:TheCoreCLanguage}, the point-to analysis also
uses the domains: sets of objects $\Delta \in \powset(\DAloc)$.

We will define a function $\OPT : \DD ->\Delta$ for determining
which objects may be pointed to by objects and expressions: $\delta'
\in \OPT(\delta)$ means that $\delta$ may point to $\delta'$; Likewise
for $\OPT(e)$.

For each object $\delta$, we will associate a \emph{points-to set}
$\Delta_{\delta}$. Informally $\Delta_p$ means the set of objects
pointed to by variable \texttt{p}. Likewise, for each expression $e$,
we will associate a points-to set $\Delta_{e}$.

The analysis proceeds in two phases: \emph{constraint generation} by
inference on the subject program, and \emph{constraint solving} by
fix-point iteration. Constraints have one of these forms:

\begin{description}
\item[$\Delta_p \supseteq \{x\}$] which means that variable $p$ may
  point to object $x$, as in \texttt{p = \&x;}
\item[$\Delta_p \supseteq \star\Delta_q$] which means that variable
  $p$ may point to any object that the set of objects $\Delta_q$ can
  point to, where $\Delta_q$ is set of the objects that variable $q$
  can point to, as in \texttt{ int **q; int *p; p = *q;}
\item[$\star\Delta_{f\!p} \supseteq \Delta_q\leftarrow
  (\Delta_{p_1},\dots,\Delta_{p_n})$] which means that all functions
  pointed to by $f\!p$ may be called with any tuple of objects in
  $\Delta_{p_1} \times \dots \times \Delta_{p_n}$. Furthermore,
  variable $q$ may point to any object that functions $f\!p$ may
  return a pointer to (via \syntax{return} statements), as in
  \texttt{q = fp(p1,...,pn);}
\end{description}

\noindent
Solving these constraints then involves updating sets of objects until
a least fix-point is reached.

In Core C, an object is (an approximation of) a set of memory
locations used during program execution, \ie a declaration \syntax{int
  x} in a possible recursive function $f$ represents all instances of
$x$.  Furthermore, we assume that each dynamic allocation point
originating from a \syntax{calloc} and string constants have been
given a unique placeholder declaration. Likewise, the transformation
from C to \coreC associates a special declaration with each array
object so that it can represent set of objects the array can contain.

\subsubsection{Constraint generation}
The inference rules create a set of constraints $\mathcal{C}$, written
$[c_1,c_2,\dots,c_n]$.  Constraint sets are unioned by juxtaposition:
$\mathcal{C}\mathcal{C}'$ means $\mathcal{C}$ extended with
$\mathcal{C}'$. Types are

\bigskip
% -------- Expressions ---------

\begin{figure}[htb]
  \begin{center}
    \[
    \begin{array}{ll}
      \textnormal{Expression } e & \Delta_e \\\hline
      \CRval(\delta)             & \Delta_e \supseteq \{\delta\} \\
      \CLval(\delta)            & \alpha = \mbox{new},
                                  \Delta_\alpha \supseteq \{\delta\},
                                  \Delta_e \supseteq \{\alpha\} \\
      \CConst(c)                & \{\}       \\
      \CUnary(\Vuop,e_1)        & \OU(e_1) \cup \OL(e) \\
      \CBinary(e_1,\Vbop,e_2)   & \OU(e_1) \cup \OU(e_2) \\
      \CCast(t,e_1)             & \OU(e_1)   \\
      \CMember(e_1,\varphi)     & \OU(e_1) \cup \OL(e) \\
    \end{array}
    \]
    \caption{Function $\OU$ for determining which objects may
      possibly be used in an expression}
    \label{fig:PAsomething}
  \end{center}
\end{figure}

\[\inference[const]
  {}
  {c : \Delta,[\Delta\supseteq\{\}]}
\qquad
  \inference[lvar]
  {}
  {x : \Delta, [\Delta\supseteq\{\&\},\Delta_\&\supseteq\{x\}]}
\]

\[\inference[rvar]
  {}
  {x : \Delta, [\Delta\supseteq\{x\}]}
\qquad
  \inference[struct]
  {e : \Delta' , \mathcal{C}}
  {e.i : \Delta, \mathcal{C}[\Delta\supseteq \Delta'\cdot i]}
\]

\[
  \inference[deref]
  {e : \Delta',\mathcal{C}}
  {\syntax{*}e : \Delta,\mathcal{C}[\Delta\supseteq\star\Delta']}
\qquad
  \inference[addr]
  {e : \Delta' ,\mathcal{C}}
  {\syntax{\&}^\alpha e : \Delta, \mathcal{C}
    [\Delta\supseteq\{\alpha\},\Delta_\alpha\supseteq\Delta']}
\]

\[\inference[unary]
  {e : \Delta,\mathcal{C}}
  {o_{uop}\ e : \Delta,\mathcal{C}}
\qquad
  \inference[binary]
  {e_1 : \Delta_1,\mathcal{C}_1 & e_2 : \Delta_2,\mathcal{C}_2}
  {e_1\ o_{bop}\ e_2 : \Delta,\mathcal{C}_1\mathcal{C}_2
    [\Delta\supseteq\Delta_1,\Delta\supseteq\Delta_2]}
\]

\[\inference[typesize]
  {}
  {\syntax{sizeof}(\tau) : \Delta,[\Delta\supseteq\{\}]}
\qquad
  \inference[expsize]
  {}
  {\syntax{sizeof}(e) : \Delta,[\Delta\supseteq\{\}]}
\]

\[\inference[cast]
  {e : \Delta',\mathcal{C} & (\mathcal{C}',\Delta)=\mbox{Cast}(\tau,\mbox{typeOf}(e),\Delta')}
  {\syntax{(}\tau\syntax{)} e : \Delta,\mathcal{C}\mathcal{C}'}
\]

\noindent
where the Cast function is defined below
% -------- Cast ---------

\[\begin{array}{l}
  \mbox{Cast}(to,from,\Delta) = \mbox{case}\ (to,from)\ \mbox{of}\\
  \begin{array}{lll}
  (\langle*\tau\rangle,\langle*\langle\mbox{struct S}\rangle\rangle) 
   & => \left\{\begin{array}{ll}
        ([\Delta'\supseteq \Delta\cdot 1],\Delta')
          & \mbox{if}\ \mbox{typeOf}(S.1)=\tau \\
        (\epsilon,\Delta)
          & \mbox{if}\ \mbox{typeOf}(S.1)\not=\tau \\
        \end{array}\right.\\
  (\langle*\tau\rangle,\langle[n]\tau\rangle)
   & => (\epsilon,\Delta) \\
  (\_\,,\_) & => (\epsilon,\Delta) \\
  \end{array}
  \end{array}
\]


% -------- Statements ---------

\noindent\fbox{Statements}\hfill\fbox{$s : \mathcal{C}$}

\[\inference[branch]
  {}
  {\syntax{if(}e\syntax{)}\ bb_1\ bb_2 : []}
\qquad
  \inference[jump]
  {}
  {\syntax{goto}\ bb : []}
\]


\[\inference[assign1]
  {e : \Delta,\mathcal{C}}
  {x\ \syntax{=}\ e : \mathcal{C}[\Delta_x\supseteq\star\Delta]}
\qquad
  \inference[assign2]
  {e : \Delta , \mathcal{C}}
  {\syntax{*}x\ \syntax{=}\ e : \mathcal{C}[\star\Delta_x\supseteq\star\Delta]}
\]

\[\inference[assign3]
  {e : \Delta' , \mathcal{C}}
  {x.i\ \syntax{=}\ e : \mathcal{C}[\Delta\supseteq\{x\},\Delta\cdot i\supseteq\star\Delta']}
\]

\[\inference[assign4]
  {e : \Delta , \mathcal{C}}
  {\syntax{(*}x\syntax{)}.i\ \syntax{=}\ e :
    \mathcal{C}[\Delta_x\cdot i\supseteq\star\Delta]}
\]

\[\inference[call]
  {f\!p : \Delta_{f\!p},\mathcal{C}  & e_i : \Delta_i,\mathcal{C}'_i}
  {x\ \syntax{=}\ f\!p \syntax{(}e_1,\dots,e_n\syntax{)} :
    \mathcal{C}\mathcal{C}'_i[\Delta_{f\!p}\supseteq\Delta_x
           \leftarrow (\Delta_1,\dots,\Delta_n)]}
\]

\[\inference[alloc]
  {}
  {x\syntax{ = calloc}^\alpha(\tau,e) :
    [\Delta_x\supseteq\{\alpha\},\Delta_\alpha\supseteq\{\}]}
\qquad
  \inference[free]
  {}
  {\syntax{free}\ e : []}
\]

\[\inference[return]
  {e : \Delta,\mathcal{C} & f\mbox{ is surrounding function}}
  {\syntax{return}\ e : \mathcal{C}[\Delta_{f_0}\supseteq\star\Delta]}
\]

\subsubsection{Constraint solving}
[Describe pre-normalization]
[The constraints make a term rewrite system]

\bigskip
% -------- Solving ---------

\begin{tabbing}
foreach $\Delta \in \nabla$ do touched$(\Delta):=$true  \\
repeat \= \\
       \> fixpoint $:=$ true \\
       \> foreach $\Delta \in \nabla$ do \= changed$(\Delta) := $ touched$(\Delta)$  \\
       \>                                \> touched$(\Delta) := $ false\\
       \> foreach $c \in \mathcal{C}$  \\
       \> do \= \\
       \>    \> case $c$ of \\
       \>    \> $\begin{array}{rclcl}
                \Delta&\supseteq&\{x\} &=>& \mbox{update}(\Delta,\{x\}) \\
                \Delta&\supseteq&\Delta' &=>& \mbox{update}(\Delta,\Delta') \\
                \Delta&\supseteq&\star\Delta' &=>& \mbox{update}(\Delta,\mbox{indr}(\Delta')) \\
                \Delta&\supseteq&\Delta'\cdot i &=>& \mbox{foreach}\
                s\in\Delta' \mbox{do update}(\Delta,\mbox{struct}(s,i)) \\
                \star\Delta&\supseteq&\star\Delta' &=>&
                                                     \mbox{if changed}(\Delta)\mbox{ or changed}(\Delta')\ \mbox{then} \\ 
                           &         &             &  & \mbox{foreach}\ x\in\Delta
                                                     \mbox{ do update}(\Delta_x,\mbox{indr}(\Delta')) \\
                \Delta\cdot i&\supseteq&\star\Delta' &=>&
                                      \mbox{foreach}\ s\in\Delta:
                                      \mbox{do update}(\mbox{struct}(s,i),\mbox{indr}(\Delta')) \\
                \Delta&\supseteq&\Delta_x \leftarrow
                                (\Delta_1\dots\Delta_n)&=>& \mbox{if changed}(\Delta)\ \mbox{then}\\
                           &&&& \mbox{foreach}\ (\Delta_1'\dots\Delta_n')->\Delta_{f_0}\in\mbox{params}(\Delta):\\
                           &&&& \mathcal{C} := \mathcal{C} \cup [\Delta_1'\supseteq\star\Delta_1,\dots,
                                \Delta_n'\supseteq\star\Delta_n,\Delta_x\supseteq\Delta_{f_0}] \\
                           &&&& \mbox{fixpoint:=false}\\
                \end{array}$\\
while fixpoint $=$ false \\
\\
indr$(\Delta) = \bigcup_{x\in\Delta}{\Delta_x}$ \\
\\
update$(\Delta,\Delta') = \mbox{foreach}\ x\in\Delta'\ $
                          do if $(x\not\in\Delta)$\ 
                          then\ \=$\Delta:=\Delta\cup\{x\}$ \\
                                       \>touched$(\Delta):=$ true\\
                                       \>fixpoint$:=$ false\\
\\
params$(\Delta) =\ $\=$\sigma:=\{\} $\\
          \> $\mbox{foreach}\ f\in\Delta\ \mbox{do}$\
             \=$\sigma:=\sigma\cup (\Delta_1\dots\Delta_n)->\Delta_{f_0}$ \\
          \> \>$\mbox{where}\ \Delta_i\ \mbox{represents parameter}\ d_i\ 
                             \mbox{in}\ f$ \\
          \> return $\sigma$ \\
\\
struct$(s,i) = \mbox{the}\ i\mbox{'th member of struct}\ s$ \\

\end{tabbing}

\subsection{Effects on other stuff}
\begin{itemize}
\item After the pointer analysis, struct types have to flow together
  because the type of a struct identifies what instance it is.
\item Free expressions can now be traversed to mark alloc expressions
  to be either freed or not freed.
\end{itemize}

\end{inferencesymbols}
\mathligson

\subsection{Implementation level (01.01.1998)}
\label{sec:PAImplementationLevel}
\index{implementation level!points-to analysis}

The in-use analysis has been implemented according to this chapter with the
following changes:

In the implementation the set $\Delta_p$ is explicitly connected with
the declaration of $p$; the formal definition assumes this implicitly.
This means that we can regard $\Delta_x$ as a unique representative of
the object $x$ and thus only work on $\Delta$-sets that contain
\emph{references} to other $\Delta$-sets.  When such a system
stabilizes (\ie the constraint solving reaches the least fix-point),
the result of the analysis is extracted from the relevant sets; \eg if
variable $p$ is connected with the set $\Delta_p =
\{\Delta_x,\Delta_y\}$, it means that $p$ may point to the objects
associated with $\Delta_x$ and $\Delta_y$, \ie the variables $x$ and
$y$.

\epsfig{file=extfunSE.eps}


\end{docpart}

%%% Local Variables: 
%%% mode: latex
%%% TeX-master: "cmixII"
%%% End: 

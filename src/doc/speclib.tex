% File: speclib.tex

\providecommand{\docpart}{\renewenvironment{docpart}{}{}
\end{docpart}
\documentclass[twoside]{cmixdoc}
%\bibliographystyle{apacite}

\makeatletter
\@ifundefined{@title}{\title{\cmix-documentation}}{}
\@ifundefined{@author}{\author{The \cmix{} Team}}%
{\expandafter\def\expandafter\@realauthor\expandafter{\@author}%
\author{The \cmix{} Team\\(\@realauthor)}}
\makeatother

\AtBeginDocument{%
\markboth{\hfill\today\quad\timenow\hfill\llap{\cmix\ documentation}}
{\hfill\today\quad\timenow\hfill}}

\renewcommand{\sectionmark}[1]{\markboth
{\hfill\today\quad\timenow\hfill\llap{\cmix\ documentation}}
{\rlap{\thesection. #1}\hfill\today\quad\timenow\hfill}}

%\newboolean{separate}
%\setboolean{separate}{true}

\renewenvironment{docpart}{\begin{document}}%
                          {\bibliography{cmixII}\printindex
                           \end{document}}
\begin{document}\shortindexingon
}
\title{Specialization library functions for \cmix}
\author{Arne John Glenstrup}
\begin{docpart}
\maketitle

\MakeShortVerb{"}

\begin{center}
  \fbox{\huge\textsl{THIS SECTION IS VERY OUT-OF-DATE}}    
\end{center}


\section{Specialization Library Functions}
\emph{NB:
Restructuring code has been added to the library, and is turned on as
default.  To switch off restructuring, set the integer "cmixRestruct"
to 0 (zero).  The restruturing algorithms used are slightly tweaked versions
of the algorithms presented by Cifuentes in her
thesis~\cite{cifuentes:rev-comp-tech}
}
\label{sec:SpecializationLibraryFunctions}
\index{speclib}\index{function!library!specialization}

This section describes the auxillary functions that are used by the
generating extension. They are grouped into three main classes:
\begin{description}
\item[Memory management functions] that take care of saving, restoring and
  comparing the static store,
\item[Memoization functions] that handle the sets of specializations points 
  that are waiting to be specialized, and checks whether a specialization
  point with the same static store has been seen before
\item[Code generation functions] that simply create and gather up the
  residual code, including lifting functions and functions for generating
  fresh identifiers.
\item[Miscellaneous helper functions] including functions for handling
  static return values.
\end{description}
Section~\ref{sec:SLFMemoryManagement} is based on an excerpt from a report
by~\cite{Andersen:1997:StaticMemoryManagementInCMix}.

\subsection{Memory management}
\label{sec:SLFMemoryManagement}
\index{management!memory}\index{store!static}

The present description does not yet discuss the management of
heap-allocated objects.

\paragraph{Objects.} On the one hand, it is desirable to have a
fine-grained representation of the active store during specialization
to enable memoization with high precision (e.g., the memoization
algorithm may want to treat the members of a struct as separate
objects, so that it can memoize only the members that are actually
in use at a given program point).

On the other hand, each object must be registered somewhere, so it is
desirable to have as coarse-grained representation as possible to
minimize the storage overhead.

We believe that this stage of the implementation calls for simplicity,
so we choose a relatively coarse-grained representation: The storage
that is allocated by the ``execution'' of a (static) declaration is
perceived as one object. This implies, for instance, that all elements of
an array are treated together.

At specialization time a static object is thus allocated by the
``execution'' of a declaration. Global declarations are executed at startup
and local declarations are executed when the function they appear in is
entered. Due to recursive calls one local declaration may thus give rise to
several  objects that are allocated at different locations on the
stack at a given time.

At specialization time we need to keep some book-keeping information
for each object. We shall refer to this as \emph{the object
description} and as the C++ class $\TDataObject$_{DataObject@$\TDataObject$}.

\paragraph{Copies of Objects.} Since the operations needed
for objects in the active store are different from those needed for copies,
we distinguish between the two kinds of objects. The book-keeping
information for a copied object is referred to as \emph{the description of
  a copied object} or as the C++ class
$\TDataObjectCopy$_{DataObjectCopy@$\TDataObjectCopy$}.


\subsubsection{Memory management functions and data structures}
In general, we need to make three operations available on objects:
\begin{center}\def\arraystretch{1.2}
{\setbox0=\hbox{$\Tint\ \Fcmp(\TDataObject\ *\Vobj,\ \TDataObjectCopy
  *\Vcpy)$}
\newdimen\commentwidth\commentwidth=-\wd0
\advance\commentwidth by .9\textwidth
\ifdim\commentwidth<5em\commentwidth=5em\fi
\begin{tabular}{lp{\commentwidth}}
$\TDataObjectCopy\ {*}\Fcopy(\TDataObject\ {*}\Vobj)$
&
copies the object described by $\Vobj$ and
returns a description of the copied object.
\\
$\Tvoid\ \Frestore(\TDataObjectCopy\ {*}\Vcpy)$
&
takes a description
of a copy and restores it to its original location.
This requires that the original
location of the object must be part of the description of the copied
object\footnotemark
\\
$\Tbool\ \Fcmp(\TDataObject\ *\Vobj,\ \TDataObjectCopy *\Vcpy)$
&
compares an object in the active
store ($\Vobj$) with a copied object ($\Vcpy$),  returning true
if they match, false otherwise.
\end{tabular}\footnotetext{Alternatively, the restore function could take the
  description of the original object as parameter, thus reducing the
  memory overhead in the object description, but this makes it more
  complicated to use the $\Frestore$ function}}
\end{center}

The C++ data structures that describe an object and a copy of an object 
must contain the following fields:
\begin{center}\def\arraystretch{1.1}
\begin{tabular}{|l|l|}\hline
\hfil $\TDataObject$        & \hfil $\TDataObjectCopy$ \\\hline
A pointer to the object           & A pointer to the original 
                                    location of the copy \\
                                  & A pointer to the copy \\
The size of the object            & The size of the copy \\
A pointer to its copy function    & \\
A pointer to its compare function & \\\hline
\end{tabular}
\end{center}
The first two fields are specific to the object, whereas the rest are
specific to the declaration (type) of the object.  
%The visit flag is used to mark already processed objects, thus avoiding
%infinite looping on cyclic data-structures when doing a recursive copy.

The size can be hard-coded (generated) into the copy and compare
functions, and the two functions can be joined into a general function
which takes an extra argument specifying whether it should copy or
compare, but we leave this for future optimizations.

Note that the original location is only needed for restoring
and not for comparing.

The data structure used when copying an object must in fact contain
copies of several objects, because if the object contains a pointer,
we must also copy whatever object it points to. This data structure must
support restoring and comparing. Restoring of the state does not
require the copied objects to be structured in any way, since it
amounts to restoring all copied objects. Comparing is a bit more
difficult, since we must make sure that the right objects are compared
to the right copies.

We thus arrange the copied objects in a list where the copy of an
object comes immediately before the copies of the objects it
references. This list structure incurs an overhead of two pointers per
copied object. Alternatively, we could use an array of references to
objects, which would reduce the overhead to one pointer per object,
but at the same time it complicates the implementation, because the
number of elements is unknown when the copying begins. Another
possibility is to store the copied objects contiguously in memory,
which does not involve any overhead, but it also complicates the
implementation and makes it difficult to share identical copies. For
simplicity we have chosen to use a list, and leave trimming of the
memory usage to the future.

\subsubsection{The Standard and Pointers} Four rules from the Standard
are important to note when dealing with pointers:

\begin{enumerate}
\item \label{item-array} Strictly conforming \ansiC allows pointer
arithmetic to move a pointer around inside the same array, and to move
it one past the last element of the array.
\item It is guaranteed that a pointer to an object may be converted to
a pointer to an object whose type requires less or equally strict
storage alignment and back again without change.
\item\label{rule-first-member} If a pointer to a structure is cast
to the type of a pointer to its first member, the result refers to the
first member.
\item A cast of a pointer to an integral type\footnote{Types {\tt
char}, and {\tt int} of all sizes, each with or without sign, and also
enumerations types, are collectively called \emph{integral}
types.~\cite{Kernighan:1988:CProgrammingLanguage}.} or vice versa is implementation
defined and therefore not strictly conforming to the Standard.
\end{enumerate}
The first rule implies that it may be necessary to copy an
entire array when a pointer refers to it. The second rule means that
the type of the pointer may not be sufficient to determine how many
bytes to copy (due to casts). The third rule implies that it may be
necessary to copy the entire struct when copying a pointer to the
first element of the struct. Finally, the fourth rule drastically
limits the number of objects a pointer may point to. Without the forth
rule we would have to copy the entire store in the worst case.

\cmix allows a pointer to an object to be cast to another pointer to an
object, as long as the level of indirection remains unchanged, whereas
other casts from or to pointers are residualized, cf.\ 
Section~\vref{sec:BTAExpressions}. This implies that a cast of a pointer to
an integral type or vice versa is made dynamic by \cmix.


\subsubsection{Type specific memory management functions}
\label{sec:TypeSpecificMemoryFunctions}

We now consider each C type in turn to see what is
required to implement memory management support for objects of that
particular type.

We divide the types into \emph{arithmetic} (char, int, float, double,
enumerations, and derived forms; i.e., signed, unsigned, short, long),
\emph{function pointers}, \emph{pointers to objects}, \emph{structs},
\emph{unions}, and \emph{arrays}.


\paragraph{Arithmetic Types}\label{sec:SLFArithmeticTypes}
 are easy:~given a variable $x$ of type $T$,
we simply copy $\Fsizeof(T)$ bytes starting at $\&x$. 
Comparing is equally simple. 

Thus, the specialization library must contain two functions for
copying and comparing simple objects:
\[
\begin{array}{rl}
\TDataObjectCopy & \FcopySimple(\TDataObject\ *\Vobj); \\
\Tbool & \FcmpSimple(\TDataObject\ *\Vobj, \TDataObjectCopy\ *\Vcpy); 
\end{array}
\]

\paragraph{Function
  Pointers}\label{sec:SLFFunctionPointers}\index{pointer!function}
can be treated in exactly the same manner as arithmetic types. The
referenced function cannot change, and therefore there is no need to copy
it.

\paragraph{Pointers to Objects}
\label{sec:PointersToObjects}
\index{pointer!object}

In this section we describe how we at specialization time can
determine exactly which object a given pointer refers to.

Consider a static pointer \texttt{p}. If \texttt{p} points to dynamic
data, we simply copy the pointer. If, however, \texttt{p} is a pointer
to static data, we must also copy what it points to. In general it is
not sufficient just to copy the integer \texttt{*p}, though, since
other addresses may be referenced as well, either via pointer
arithmetic or via casts.

If \texttt{p} refers to an element of an array, then we must copy the
entire array. This implies that the copy function must know the size
of the array.

When we later during the specialization compare the memoized array
with a new array referenced by \texttt{p}, the new array may be of
another size, so before comparing the two their sizes must be found to
match.  However, this is still not sufficient:~it could be the case
that the pointers match and the referenced arrays match, but the
offset of the pointers into the arrays differ, as in this example with a
4-element array:
\begin{center}
  \begin{tabular}{l|c|r}\cline{2-2}
    "0x4020": & 42 \\\cline{2-2}
    "0x4016": & 17 \\\cline{2-2}
    "0x4012": & 42 & $<--$"p" \\\cline{2-2}
    "0x4008": & 17 \\\cline{2-2}
    \mc{1}{l}{"0x4004":} & \mc{1}{c}{--} \\
    \mc{1}{l}{"0x4000":} & \mc{1}{c}{--} \\
  \end{tabular}
\hfil
  \begin{tabular}{l|c|r}
    \mc{1}{l}{"0x4020":} & \mc{1}{c}{--} \\
    \mc{1}{l}{"0x4016":} & \mc{1}{c}{--} \\\cline{2-2}
    "0x4012": & 42 & $<--$"p" \\\cline{2-2}
    "0x4008": & 17 \\\cline{2-2}
    "0x4004": & 42 \\\cline{2-2}
    "0x4000": & 17 \\\cline{2-2}
  \end{tabular}
\end{center}
In this case, the two states should not compare equal; hence the offset
of of a pointer into an array must be memoized as well.
Similar considerations apply to pointers to structs: we must copy the
entire struct and the pointer's offset relative to the start address
of the struct.
This way, what we later compare are not the absolute
locations_{location!absolute}, but the \emph{structure} of the
data_{data!structure of}.

When copying a pointer $p$ at specialization time we therefore examine
each object in turn to determine whether or not $p$ may refer to it.
One could take an offline approach, and memoize all objects that a
pointer may point to as reported by the pointer analysis. However, we
fear that this would be overly conservative. For example, the pointer
analysis does not distinguish between different objects allocated by
the same declaration, and it is non-trivial to extend it to do so.

If a pointer points to an array, then it may point one past the last
element of the array (cf.\ Rule~\vref{item-array}). This implies that
we are not able to determine whether a pointer refers to an array or
the object allocated right after the array. A safe approximation would
be to copy both objects, but this may give redundant specialization,
it may waste storage, and it complicates the comparison algorithm, so
to avoid ambiguities we \emph{increase the size of all static arrays
  by one prior to specialization_{array!length!increasing}.} Since the
original program is strictly conforming to the Standard then we know
that pointer arithmetic will only move pointers around inside the
slightly larger arrays and never outside any. The only exceptions are
expressions like "sizeof(a)" where "a" is an array; in this case we
transform it into "(sizeof(a) - sizeof(a[0]))".

Now it is the case that a (non-function) object $x$ is referenced by
pointer $p$ if $\&x =< p < \&x + \Fsize(x)$.
%The following condition determines whether $p$ may refer
%to the object:
%\begin{center}
%\begin{tabular}{ll}
% \emph{Object $x$ of type} & \emph{is referenced by pointer $p$ if} \\[.8ex]
% arithmetic & $\&x = p$ \\
% pointer    & $\&x = p$ \\
% struct     & $\&x =< p < \&x + \Fsize(x)$ \\
% union      & $\&x =< p < \&x + \Fsize(x)$ \\
% array      & $\&x =< p < \&x + \Fsize(x)$
%\end{tabular}
%\end{center}
As mentioned above, pointers to functions do not require any object to be
copied.

Thus, the specialization library must contain two functions for copying and
comparing a referenced object and the offset of the reference (i.e.\ $p -
\&x$)%
_{copyWithOffset@$\FcopyWithOffset$}%
_{cmpWithOffset@$\FcmpWithOffset$}:
\[
\begin{array}{rl}
\TDataObjectCopy & \FcopyWithOffset(\TDataObject\ {*}\Vobj,\ \Tvoid\
{*}\Vref); \\
\Tbool & \FcmpWithOffset(\TDataObject\ {*}\Vobj,\ \TDataObjectCopy\
{*}\Vcpy,\ \Tvoid\ {*}\Vref); \\
\end{array}
\]

\paragraph{Pointers to Strings.} The Standard specifies that the
behavior of a program that attempts to alter a string literal is
undefined. This implies that string literals need not be copied.
 
\paragraph{Extern Pointers.} Since the binding-time analysis of C-Mix
residualizes all external pointers, the object referred to by a static
pointer will always stem from a static declaration in the program.


\paragraph{Structs}
\label{sec:SLFStructs}\index{struct!static!memoizing}
are copied like arithmetic types by copying $\Fsize(x)$
bytes starting at $\&x$, but additionally, if the struct contains
pointers then we must also recursively copy objects referenced by
pointers in the struct.

Since the static data is represented directly in the generating extension,
all we have is a void pointer to a collection of bytes inside which one or
more pointers are ``hidden''. The simplest way to extract the pointers is
to cast the void pointer to a pointer of the appropriate struct type. To
this end we generate \emph{type-specific copy and compare
  functions}_{function!copy!type specific}_{function!copy!struct}%
_{function!compare!type specific}_{function!compare!struct} for each
user defined type in use by the static objects~\cite[Section
3.10.4]{Andersen:1994:ProgramAnalysisAndSpecialization}.  These functions
are generated in the gegen phase, prior to specialization, and added to the
generating extension.

\begin{example}[Copying structs]
\index{struct!static!memoizing}
\label{exm:StructCopy}
Suppose the subject program contains a static variable of  type
"struct U":
\begin{verbatim}
struct T { int a; int *p; };
struct U { int *q; struct T t[42]; };
\end{verbatim}
When copying an object of type "struct U" with object
description "obj", the function first copies the struct itself,
then it casts the void pointer denoting the location of the object
"obj->loc" to a "struct U" pointer, and extracts the 42 pointers
from it:
\begin{verbatim}
DataObjectCopy *copy_struct_U(DataObject *obj, void *ref) {
  if (obj->visit) return NULL;

  DataObjectCopy *copy = copyWithOffset(obj, ref);   /* Copy struct U itself */

  struct U *pu = (struct U *) (obj->loc);            /* Copy any referenced objects */
  copy = append(copy, copyReferencedObject(pu->q));
  for (i = 0; i < 42; i++) {
    struct T *pt = &(pu->t[i]);
    copy = append(copy, copyReferencedObject(pt->p));
  }
  return copy;
}
\end{verbatim}
The compare function will simply compare the copied objects in the
same order:
\begin{verbatim}
bool cmp_struct_U(DataObject *obj, DataObjectCopy *copy, void *ref) {
  if (obj->visit) return true;

  if (!cmpWithOffset(obj, copy, ref)) return false; /* Compare struct U itself */

  struct U *pu = (struct U *) (obj->loc);           /* Compare any referenced objects */
  if (!cmpReferencedObject(pu->q, copy)) return false;
  for (i = 0; i < 42; i++) {
    struct T *pt = &(pu->t[i]);
    if (!cmpReferencedObject(pt->p, copy)) return false;
  }
  return true;
}
\end{verbatim}
Note that no copy or compare functions are generated for "struct T"
unless a static variable is declared with this type.
\end{example}


The copy and compare functions are generated by inserting a list of
transformations $\gegen{u_{i_1}}_{\Vcopy}$; \ldots
; $\gegen{u_{i_g}}_{\Vcopy}$; \ldots;
$\gegen{u_{i_1}}_{\Vcmp}$; \ldots
; $\gegen{u_{i_g}}_{\Vcmp}$ after transforming the user defined types
in Figure~\vref{fig:GEGENPrograms}. This is only done for those user types
$u_{i_1}, ..., u_{i_g}$ for which variable declarations exist. These
transformations_{<.>copy@$\gegen{\cdot}_{\Vcopy}$}%
_{<.>cmp@$\gegen{\cdot}_{\Vcmp}$}_{<.>@$\gegen{\cdot}$} are defined in
Figures~\ref{fig:SLFUserTypesCopyFun}
and~\ref{fig:SLFUserTypesCompareFun}%
_{copys@$\Fcopy_{\sigma}$}_{cmps@$\Fcmp_{\sigma}$}.
\begin{figure}[htbp]
  \begin{center}
    \small
    \[\def\arraystretch{1.2}
    \begin{array}{@{}lcl@{}}
      \gegen{(\sigma, \Void, \CStruct_s, d_1... d_n)}_{\Vcopy} &
      =& \TDataObjectCopy\ \widehat{\Fcopy_{\sigma}}
           (\TDataObject\ {*}\Vobj, \Tvoid\ {*}\Vref)\ \{ \\
      && \quad \Kif\ \Fvisited(\Vobj)\ \Kthen\ \Kreturn\ [] \\
      && \quad \Vcopy
               := \FcopyWithOffset(\Vobj, \Vref) \\
      && \quad \Tstruct\ \Void_{\sigma}\ *\widehat{x} 
               := (\Tstruct\ \Void_{\sigma}*)\Fobject(\Vobj) \\
      && \quad \gegen*{(*x), d_1}_{\Vcopy}; ...; 
               \gegen*{(*x), d_n}_{\Vcopy}\\
      && \quad \Kreturn\ \Vcopy \\
      && \} \\
      \gegen{(\sigma, \Void, \CStruct_d, d_1... d_n)}_{\Vcopy} &
      =& \mbox{---}
      \\
      \gegen{(\sigma, \Void, \CUnion_s, d_1... d_n)}_{\Vcopy} &
      =& \TDataObjectCopy\ \widehat{\Fcopy_{\sigma}}
           (\TDataObject\ {*}\Vobj, \Tvoid\ {*}\Vref)\ \{ \\
      && \quad \Kif\ \Fvisited(\Vobj)\ \Kthen\ \Kreturn\ [] \\
      && \quad \Vcopy
               := \FcopyWithOffset(\Vobj, \Vref) \\
      && \quad \Tstruct\ \Void_{\sigma}\ *\widehat{x} 
               := (\Tstruct\ \Void_{\sigma}*)\Fobject(\Vobj) \\
      && \quad \gegen*{(*x), d_1}_{\Vcopy} \\
      && \quad \Kreturn\ \Vcopy \\
      && \} \\
      \gegen{(\sigma, \Void, \CUnion_d, d_1... d_n)}_{\Vcopy} &
      =& \mbox{---}\\[1em]
      \gegen*{x, \CStructMem(\delta, \Void, t_s^s, \Voc)}_{\Vcopy} &
      =& \mbox{---}
      \\
      \gegen*{x, \CStructMem(\delta, \Void, 
        \CPointer_s^s(q, t), \Voc)}_{\Vcopy} &
      =& \Vcopy := \Vcopy \append \FcopyReferencedObject(x.\Void)
      \\
      \gegen*{x, \CStructMem(\delta, \Void, 
        \CPointer_s^d(q, t), \Voc)}_{\Vcopy} &
      =& \mbox{---}
      \\
      \gegen*{x, \CStructMem(\delta, \Void, 
        \CArray_s^s(q, t, e), \Voc)}_{\Vcopy} &
      =& \gegen*{x.{\Void}, [e], t}_{\Vcopy}
      \\
%      \gegen*{ 
%        \CStructMem(\delta, \Void, 
%                    \CFunT(q, t, t_1 ... t_n), \Voc)}_{\Vcopy} &
%      =& \Kerror
%      \\
      \gegen*{x, \CStructMem(
        \begin{array}[t]{@{}l@{}}
          \delta, \Void, \\
          \CUser_s^s(q, \sigma, \CStruct, d_1 ... d_n),
          \Voc)}_{\Vcopy} 
      \end{array} &
      =& \gegen*{x.\Void_{d_1}, d_1}_{\Vcopy}; ...; 
         \gegen*{x.\Void_{d_n}, d_n}_{\Vcopy}
      \\
      \gegen*{x, \CStructMem(
        \begin{array}[t]{@{}l@{}}
          \delta, \Void, \\
          \CUser_s^s(q, \sigma, \CUnion, d_1 ... d_n), \Voc)}_{\Vcopy}
      \end{array} &
      =& \gegen*{x.\Void_{d_1}, d_1}_{\Vcopy}
      \\
      \gegen*{x, \CStructMem(\delta, \Void, 
        \CAbstract(q, \Vid), \Voc)}_{\Vcopy} &
      =& ???
      \\[1ex]
      \gegen*{x, \Vms,  t_s^s}_{\Vcopy} &
      =& \mbox{---}
      \\
      \gegen*{x, \Vms, \CPointer_s^s(q, t_d)}_{\Vcopy} &
      =& \Kfor\ (i_1, ..., i_k) \in 
           \{0, ..., m_1\} \x \cdots \x \{0, ..., m_k\}\ \Kdo\\
      && \quad \Vcopy := \Vcopy \append 
           \FcopyReferencedObject(x[i_1]\cdots[i_k])
      \\
      \gegen*{x, \Vms, \CPointer_s^d(q, t)}_{\Vcopy} &
      =& \mbox{---}
      \\
      \gegen*{x, \Vms, \CArray_s^s(q, t, e)}_{\Vcopy} &
      =& \gegen*{x, [m_1, ..., m_k,e], t}_{\Vcopy}
      \\
%      \gegen*{ 
%        \Vms, \delta, \Void, \CFunT(q, t, t_1 ... t_n), \Voc}_{\Vcopy} &
%      =& \Kerror
%      \\
      \gegen*{x, \Vms, 
          \CUser_s^s(q, \sigma, \CStruct, d_1 ... d_n)}_{\Vcopy} &
      =& \Kfor\ (i_1, ..., i_k) \in 
           \{0, ..., m_1\} \x \cdots \x \{0, ..., m_k\}\ \Kdo\\
      && \quad \Tstruct\ \Void_{\sigma}\ *\widehat{x'} 
         := \&(x[i_1]\cdots[i_k]) \\
      && \quad \gegen*{x'->\Void_{d_1}, d_1}_{\Vcopy}; ...; 
               \gegen*{x'->\Void_{d_n}, d_n}_{\Vcopy}
      \\
      \gegen*{x, \Vms, 
          \CUser_s^s(q, \sigma, \CUnion, d_1 ... d_n)}_{\Vcopy} &
      =& \Kfor\ (i_1, ..., i_k) \in 
           \{0, ..., m_1\} \x \cdots \x \{0, ..., m_k\}\ \Kdo\\
      && \quad \Tstruct\ \Void_{\sigma}\ *\widehat{x'} 
         := \&(x[i_1]\cdots[i_k]) \\
      && \quad \gegen*{x'->\Void_{d_1}, d_1}_{\Vcopy}
      \\
      \gegen*{x, \Vms, \CAbstract(q, \Vid)}_{\Vcopy} &
      =&
    \end{array}
    \]
    \caption{Adding copy functions for user defined type to the gegen
      transformation}
    \label{fig:SLFUserTypesCopyFun}
  \end{center}
\end{figure}

\begin{figure}[htbp]
  \begin{center}
    \small
    \[\def\arraystretch{1.2}
    \begin{array}{@{}lcl@{}}
      \gegen{(\sigma, \Void, \CStruct_s, d_1... d_n)}_{\Vcmp} &
      =& \Tbool\ \widehat{\Fcmp_{\sigma}}
          (\begin{array}[t]{@{}l@{}}
             \TDataObject\ {*}\Vobj, \\
             \TDataObjectCopy\ {*}\Vcpy, \Tvoid\ {*}\Vref)\ \{ 
           \end{array}\\
      && \quad \Kif\ \Fvisited(\Vobj)\ \Kthen\ \Kreturn\ \Ctrue \\
      && \quad \Kif\ \Fnot(\FcmpWithOffset(\Vobj, \Vcpy, \Vref))\
               \Kthen\\
      && \qquad \Kreturn\ \Cfalse \\
      && \quad \Tstruct\ \Void_{\sigma}\ *\widehat{x} 
               := (\Tstruct\ \Void_{\sigma}*)\Fobject(\Vobj) \\
      && \quad \gegen*{(*x), d_1}_{\Vcmp}; ...; 
               \gegen*{(*x), d_n}_{\Vcmp}\\
      && \quad \Kreturn\ \Ctrue \\
      && \} \\
      \gegen{(\sigma, \Void, \CStruct_d, d_1... d_n)}_{\Vcmp} &
      =& \mbox{---}
      \\
      \gegen{(\sigma, \Void, \CUnion_s, d_1... d_n)}_{\Vcmp} &
      =& \Tbool\ \widehat{\Fcmp_{\sigma}}
          (\begin{array}[t]{@{}l@{}}
             \TDataObject\ {*}\Vobj, \\
             \TDataObjectCopy\ {*}\Vcpy, \Tvoid\ {*}\Vref)\ \{ 
           \end{array}\\
      && \quad \Kif\ \Fvisited(\Vobj)\ \Kthen\ \Kreturn\ \Ctrue \\
      && \quad \Kif\ \Fnot(\FcmpWithOffset(\Vobj, \Vcpy, \Vref))\
               \Kthen\\
      && \qquad \Kreturn\ \Cfalse \\
      && \quad \Tstruct\ \Void_{\sigma}\ *\widehat{x} 
               := (\Tstruct\ \Void_{\sigma}*)\Fobject(\Vobj) \\
      && \quad \gegen*{(*x), d_1}_{\Vcmp}\\
      && \quad \Kreturn\ \Ctrue \\
      && \} \\
      \gegen{(\sigma, \Void, \CUnion_d, d_1... d_n)}_{\Vcmp} &
      =& \mbox{---}\\[1em]
      \gegen*{x, \CStructMem(\delta, \Void, t_s^s, \Voc)}_{\Vcmp} &
      =& \mbox{---}
      \\
      \gegen*{x, \CStructMem(\delta, \Void, 
        \CPointer_s^s(q, t), \Voc)}_{\Vcmp} &
      =& \Kif\ \Fnot(\FcmpReferencedObject(x.\Void, \Vcpy))\ \Kthen\\
      && \quad \Kreturn\ \Cfalse
      \\
      \gegen*{x, \CStructMem(\delta, \Void, 
        \CPointer_s^d(q, t), \Voc)}_{\Vcmp} &
      =& \mbox{---}
      \\
      \gegen*{x, \CStructMem(\delta, \Void, 
        \CArray_s^s(q, t, e), \Voc)}_{\Vcmp} &
      =& \gegen*{x.{\Void}, [e], t}_{\Vcmp}
      \\
%      \gegen*{ 
%        \CStructMem(\delta, \Void, 
%                    \CFunT(q, t, t_1 ... t_n), \Voc)}_{\Vcmp} &
%      =& \Kerror
%      \\
      \gegen*{x, \CStructMem(
        \begin{array}[t]{@{}l@{}}
          \delta, \Void, \\
          \CUser_s^s(q, \sigma, \CStruct, d_1 ... d_n), \Voc)}_{\Vcmp}
      \end{array} &
      =& \gegen*{x.\Void_{d_1}, d_1}_{\Vcmp}; ...;
         \gegen*{x.\Void_{d_n}, d_n}_{\Vcmp}
      \\
      \gegen*{x, \CStructMem(
        \begin{array}[t]{@{}l@{}}
          \delta, \Void, \\
          \CUser_s^s(q, \sigma, \CUnion, d_1 ... d_n), \Voc)}_{\Vcmp}
      \end{array} &
      =& \gegen*{x.\Void_{d_1}, d_1}_{\Vcmp}
      \\
      \gegen*{x, \CStructMem(\delta, \Void, 
        \CAbstract(q, \Vid), \Voc)}_{\Vcmp} &
      =& ???
      \\[1ex]
      \gegen*{x, \Vms, t_s^s}_{\Vcmp} &
      =& \mbox{---}
      \\
      \gegen*{x, \Vms, \CPointer_s^s(q, t_d)}_{\Vcmp} &
      =& \Kfor\ (i_1, ..., i_k) \in 
           \{0, ..., m_1\} \x \cdots \x \{0, ..., m_k\}\ \Kdo\\
      && \quad \Kif\ \Fnot
          (\FcmpReferencedObject(x[i_1]\cdots[i_k], \Vcpy))\
         \Kthen\\
      && \qquad \Kreturn\ \Cfalse
      \\
      \gegen*{x, \Vms, \CPointer_s^d(q, t)}_{\Vcmp} &
      =& \mbox{---}
      \\
      \gegen*{x, \Vms, \CArray_s^s(q, t, e)}_{\Vcmp} &
      =& \gegen*{x, [m_1, ..., m_k,e], t}_{\Vcmp}
      \\
%      \gegen*{ 
%        \Vms, \delta, \Void, \CFunT(q, t, t_1 ... t_n), \Voc}_{\Vcmp} &
%      =& \Kerror
%      \\
      \gegen*{x, \Vms, 
          \CUser_s^s(q, \sigma, \CStruct, d_1 ... d_n)}_{\Vcmp} &
      =& \Kfor\ (i_1, ..., i_k) \in 
           \{0, ..., m_1\} \x \cdots \x \{0, ..., m_k\}\ \Kdo\\
      && \quad \Tstruct\ \Void_{\sigma}\ *\widehat{x'} 
         := \&(x[i_1]\cdots[i_k]) \\
      && \quad \gegen*{x'->\Void_{d_1}, d_1}_{\Vcmp}; ...; 
               \gegen*{x'->\Void_{d_n}, d_n}_{\Vcmp}
      \\
      \gegen*{x, \Vms, 
          \CUser_s^s(q, \sigma, \CUnion, d_1 ... d_n)}_{\Vcmp} &
      =& \Kfor\ (i_1, ..., i_k) \in 
           \{0, ..., m_1\} \x \cdots \x \{0, ..., m_k\}\ \Kdo\\
      && \quad \Tstruct\ \Void_{\sigma}\ *\widehat{x'} 
         := \&(x[i_1]\cdots[i_k]) \\
      && \quad \gegen*{x'->\Void_{d_1}, d_1}_{\Vcmp}
      \\
      \gegen*{x, \Vms, \CAbstract(q, \Vid)}_{\Vcmp} &
      =& ???
    \end{array}
    \]
    \caption{Adding compare functions for user defined type to the gegen
      transformation}
    \label{fig:SLFUserTypesCompareFun}
  \end{center}
\end{figure}


\paragraph{Unions}\index{union!static!memoizing}
are a bit more tricky, since a member of pointer type might
share storage with other members. Consider  the following
union:
\begin{verbatim}
union U { int a; double x; int *p; } u;
\end{verbatim}
In general the Standard does not allow a member of a union
to be inspected unless the value of the union has been assigned using
that same member. The exception from the rule is that ``if a union
contains several structures that share a common initial sequence, and
if the union currently contains one of these structures, it is
permitted to refer to the common initial part of any of the contained
structures'' (an example can be found in
\cite[p.~214]{Kernighan:1988:CProgrammingLanguage}). Thus, when we
want to perform the memoizing, we can copy "u" itself, but we do not
know how to interpret the data: as an int, double or pointer?

Two possible ways of handling unions are:

\begin{enumerate}
\item[1.] keeping track of which members are the currently valid, or
\item[2.] splitting the union, so that members of pointer type are
allocated at private memory locations (keeping in mind the exception
mentioned above).
\end{enumerate}

\noindent The first approach should be avoided since it conflicts with
the idea directly representing static objects in the generating extension.
The second approach can be implemented with some storage overhead, but to
keep things simple, we will instead make a restriction for the present
version of \cmix:
\begin{constraint}[Unions]\index{union!static!constraint on}%
\index{constraint!union!static}%
\label{cns:SLFUnion}
In static unions, pointers to static data are only allowed if the
union exclusively contains structures, and then only in the initial
part that is common for \emph{all} the structures\footnote{Recall
  from Seciton~\ref{sec:BTAUnions} that the binding times of the
  members that are in an intial common part for two or more struct
  members must be identical.}.
\end{constraint}
This implies that pointers "u.t1.pt1" and "u.t2.pt2", but not pointer
"u.t1.qt" in this example can become pointers to static data:
\begin{verbatim}
struct T1 { int tag1; struct T *pt1; struct T *qt; };
struct T2 { int tag2; struct T *pt2; };
union U { struct T1 t1; struct T2 t2; } u;
\end{verbatim}
Note also that pointers "u.p" and "u.q" in this example can \emph{not}
become pointers to static data because they are not contained in
structs:
\begin{verbatim}
union U { int *p; int *q; } u;
\end{verbatim}
Most programs that do not satisfy this restriction can be rewritten to 
do so.

\paragraph{Arrays}\index{array!static!memoizing}
\label{sec-arrays}
are handled similarly to structs. If an array contains
pointers to static data then the referenced objects must be copied as
well as the array itself.

The type-specific copy and compare functions for arrays
containing pointers to static data are also placed in the generating
extension by gegen transformations shown in
Figure~\ref{fig:SLFArrayCopyCmpFun}.
\begin{figure}[htb]
  \begin{center}
    \small
    \[\def\arraystretch{1.2}
    \begin{array}{@{}lcl@{}}
      \gegen{t' == \CArray_s^s(q, t, e)}_{\Vcopy} &
      =& \TDataObjectCopy\ \widehat{\Fcopy_{t'}}
        (\TDataObject\ {*}\Vobj, \Tvoid\ {*}\Vref)\ \{ \\
      && \quad \Kif\ \Fvisited(\Vobj)\ \Kthen\ \Kreturn\ [] \\
      && \quad \Vcopy := \FcopyWithOffset(\Vobj, \Vref) \\
      && \quad t'\ {*}\widehat{x} := (t'*)\Fobject(\Vobj) \\
      && \quad \gegen*{x,[e],t}_{\Vcopy} \\
      && \}
      \\[1em]
      \gegen{t' == \CArray_s^s(q, t, e)}_{\Vcmp} &
      =& \Tbool\ \widehat{\Fcmp_{t'}}
        (\TDataObject\ {*}\Vobj, \TDataObjectCopy\ {*}\Vcpy,
         \Tvoid\ {*}\Vref)\ \{ \\
      && \quad \Kif\ \Fvisited(\Vobj)\ \Kthen\ \Kreturn\ \Ctrue \\
      && \quad \Kif\ \Fnot(\FcmpWithOffset(\Vobj, \Vcpy, \Vref)\
                \Kthen\ \Kreturn\ \Cfalse\\
      && \quad t'\ {*}\widehat{x} := (t'*)\Fobject(\Vobj) \\
      && \quad \gegen*{x,[e],t}_{\Vcmp} \\
      && \quad \Kreturn\ \Ctrue \\
      && \}
    \end{array}
    \]
    \caption{Adding copy and compare functions for array types to the
      gegen transformation}
    \label{fig:SLFArrayCopyCmpFun}
  \end{center}
\end{figure}


\subsection{Memoization functions}
\label{sec:SLFMemoizationFunctions}
\index{set!pending}\index{memoization}

The generating extension as described in
Section~\ref{sec:GeneratingTheGeneratingExtension} makes use of several
functions for maintaining a set of pending specialization points:
\begin{description}
\item[{$[\cdot] : \DLabel -> \DResidualLabel$}]_{[.]@$[\cdot]$} 
  Calls $\FpendInsert$ with the current state.
\item[$\FpendInsert:\DLabel \x \DState ->
  \DResidualLabel$.]_{pendInsert@$\FpendInsert$} When a program
  point_{point!program} $p$ represented by a label in the generating
  extension is to be memoized with respect to the current state $S$,
  $\FpendInsert$ generates a fresh label $l$ for the residual program and
  inserts a tuple $(p,S,l)$ called a \emph{specialization
    point}_{point!specialization} into the pending set, along with a copy,
  $\sigma$, of the part of the active store that is reachable from $S$.
\item[$\Fpending:\Dvoid -> \Dbool$.]_{pending@$\Fpending$}
  Checks whether there are any specialization
  points_{point!specialization!pending} waiting to be specialized.
\item[$\FpopSP : \Dvoid -> (\DProgramPoint \x \DState \x
  \DResidualLabel)$.]_{popSP@$\FpopSP$} Removes a pending specialization
  point from the pending set and returns it.
\item[$\Frestore:\DState -> \Dvoid$.]_{restore@$\Frestore$} Given a state,
  $S$, $\Frestore(S)$ copies the static data $\sigma$ associated with $S$
  to the active store.
\item[$\Fcopy:\DState -> \DState$.]_{copy@$\Fcopy$} Given the current
  state, $S$, $\Fcopy(S)$ returns a copy of it, along with the static
  data $\sigma$ in the active store associated with $S$.
\item[$\Fequal : \DState \x \DState$.]_{equal$\Fequal$} Given two states
  $S_1, S_2$, $\Fequal(S_1, S_2)$ checks whether they are equal and whether the
  static data reachable from them is equal.
\end{description}
During specialization of a function, \emph{two} sets of specialization
points are maintained: the set of \emph{pending}
 points_{set!specialization point!pending} and the set of
\emph{processed} points_{set!specialization point!processed}. Whenever a
specialization point is to be added to the pending set, we check whether it
has already been processed, and if so, simply return the residual label
associated with that specialization point_{memoization}. This way, residual 
code sharing_{sharing!code} is obtained.

There are some  opportunities for ``peephole''
optimizations_{optimizations!peephole}: 
\begin{itemize}
\item The state consists of local (+parameter) variables and global
  variables. In functions that perform no non-local side 
  effects_{side effects}, we do not need to compare the global
  variables. This can 
  be implemented by passing an extra $\VsideEffects$ parameter to
  $\FpendInsert$.
\item Consider specializing a dynamic "if" statement, cf.\ 
  Figure~\vref{fig:GEGENControlStatements}. If the current state has not
  been seen before, the following happens:
  \begin{enumerate}
  \item The current state + label $l_1$ and the current state + label $l_2$ 
    is added to the pending set, involving two copyings of the store
  \item In the pending loop, a state is removed from the pending set,
    involving a restoring of the store
  \end{enumerate}
  By observing that the static store is unchanged between these two
  operations, we can skip the restoring in those cases where we remove a
  specialization point we have just added. As store copying is one of the
  most time-consuming operations during specialization, this can
  potentially lead to a visible reduction of specialization time.
  
  This can be implemented by keeping a set of recently added specialization
  points.
\end{itemize}
The function for inserting specialization points into the pending set
is shown in Figure~\ref{fig:SLFpendInsert}.%
\begin{figure}[htb]
\begin{center}\leavevmode\hbox{\vbox{%
\begin{pseudocode}
  $\FpendInsert(\VpgenLabel, \Vstate, \VsideEffects) = {}$ \+\\
    $\VpendSet := \Ftop(\VpendSetStack)$ \\
    $\Kif\ \VsideEffects\ \Kthen$ \+\\
      $\Kfor\ (p,S,l) \in \VpendSet\ \Kdo$ 
      \> \hskip 15em $/*$ check the pending specialization points $*/$ \+\\
        $\Kif\ p = \VpgenLabel \land \Fequal(S, \Vstate)\ \Kthen\
          \Kreturn\ l$ \-\\
      $\Kfor\ (p,S,l) \in \VprocessedSet\ \Kdo$ 
      \> \hskip 15em $/*$ check the processed specialization points $*/$ \+\\
        $\Kif\ p = \VpgenLabel \land \Fequal(S, \Vstate)\ \Kthen\
          \Kreturn\ l$ \-\-\\
    $\Kelse$ \+\\
      $\Kfor\ (p,S,l) \in \VpendSet\ \Kdo$ 
      \> \hskip 15em $/*$ check the pending specialization points $*/$ \+\\
        $\Kif\ p = \VpgenLabel \land \FlocalEqual(S, \Vstate)\ \Kthen\
          \Kreturn\ l$ \-\\
      $\Kfor\ (p,S,l) \in \VprocessedSet\ \Kdo$ 
      \> \hskip 15em $/*$ check the processed specialization points $*/$ \+\\
        $\Kif\ p = \VpgenLabel \land \FlocalEqual(S, \Vstate)\ \Kthen\
          \Kreturn\ l$ \-\-\\[1ex]
    $\VresLabel := \FfreshResidLabel()$ 
    \>\> \hskip 15em $/*$ the specialization point has not been seen before $*/$ \\
    $\Kif\ \VsideEffects\ \Kthen$ \+\\
      $\Vsp := (\VpgenLabel, \Fcopy(\Vstate), \VresLabel)$ \-\\
    $\Kelse$ \+\\
      $\Vsp := (\VpgenLabel, \FlocalCopy(\Vstate), \VresLabel)$ \-\\
    $\Fadd(\VpendSet, \Vsp)$ \\
    $\Fadd(\VrecentSet, \Vsp)$ \\
    $\Kreturn\ \VresLabel$
\end{pseudocode}}}
    \caption{Specialization library function for inserting specialization points into the pending set }
    \label{fig:SLFpendInsert}
  \end{center}
\end{figure}

Functions $\Fpending$ and $\FpopSP$ are in fact always used together (in
the pending loop_{loop!pending} of each generating
function_{function!generating}, cf.\
Figure~\vref{fig:GEGENFunctionsDynamic}), so
they can be merged into one, that also
takes care of restoring the state and emitting the residual
label_{label!residual}, cf.\ Figure~\ref{fig:SLFpendingSP}.
\begin{figure}[htb]
\begin{center}\leavevmode\hbox{\vbox{%
\begin{pseudocode}
  $\FpendingSP() = {}$ \+\\
    $\Kif\ \Ftop(\VpendSetStack) = \{\}\ \Kthen\ \Kreturn\ \CNull$ \\
    $\Kif\ \VrecentSet \ne \{\}\ \Kthen$ \+\\
      $\Vsp := \Fremove(\VrecentSet)$ \\                          
      $(p,S,l) := \Fremove(\Vsp, \Ftop(\VpendSetStack))$ \-\\
    $\Kelse$ \+\\
      $(p,S,l) := \Fremove(\Ftop(\VpendSetStack))$ \\
      $\Frestore(S)$ \-\\
    $\VrecentSet := \{\}$ \\
    $\Fadd(\VprocessedSet, (p,S,l))$ \\
    $\emit{\bqt{l}":"}$ \\
    $\Kreturn\ p$
\end{pseudocode}}}
    \caption{Specialization library function for finding a
      specialization point in the pending set}
    \label{fig:SLFpendingSP}
  \end{center}
\end{figure}


\subsubsection{Sharing residual functions}

Another form of code sharing in the residual program is obtained by
sharing residual functions_{sharing!function} across several call
sites.

Due to non-local side effects_{side effect!non-local} during
specialization, sharing of
residual functions becomes complicated. Suppose we have
specialized a function $f$ to the input state $s$ and the end state
after $f$ is $e$. If we later during specialization encounter a call
to $f$ with the same input store $s$, then we must generate a call to
the 
specialized function and restore the end 
configuration_{end configuration!function} $e$ to mimic
the effect of the static non-local side-effects in $f$.

\paragraph{Functions Allocating Heap Storage.}_{allocation!heap}
Recall that restoring a
copied store means copying the data back to their original location in
the store. This makes sharing of a function that allocates heap
storage almost impossible, because the end configuration may refer to
some memory allocations that are unavailable when a new call to the
function is encountered.

However, if the code subsequent to the allocation does not depend on
the {\em location} of the allocated store, then restoring of a store
is allowed to copy the heap allocated objects to a new location, so
functions that allocate heap storage can be shared. Whether the code
may depend on the location of an object can be determined by a special
in-use analysis.

Since we are not implementing such an analysis at this stage we will
mark functions that allocate heap storage as unsharable.

\paragraph{Functions Deallocating Heap Storage.}_{deallocation!heap} A
function that
deallocates heap storage can be shared provided that the memoization
algorithm also keeps a record of which objects are deallocated.

Since calls to functions that may refer to a heap-allocated object can
only be shared if the object is allocated at the same location, it is
unlikely that two calls may be shared if the object has been
deallocated and allocated in between the calls.

Since functions that deallocate heap storage are almost always
unsharable, practically nothing is lost by marking all functions that
deallocate heap storage as unsharable, so we have chosen to do this.

\paragraph{Recursive Functions.}_{function!recursive} A function with
a non-local side
effect can be shared even if it is called recursively, as long as we
only allow sharing {\em after} it has been specialized, since the end
configuration is unknown until specialization of the function has been
completed.

Allowing functions to call themselves recursively without memoizing
the calls involves a risk of non-termination, but rather than
automatically suspending all non-local side effects in recursive
functions, we make them static unless the user specifies otherwise.

Another possibility is to compute the end configuration as a fix point
of the recursion, but this complicates the specialization process
considerably~\cite{Bulyonkov:1996:PracticalAspects}.


\paragraph{Implementation.} The changes required to implement function 
sharing amounts to adding a library function call to each generating
function, i.e.\ by augmenting Figure~\ref{fig:GEGENFunctionsDynamic}
as shown in
Figure~\vref{fig:SLFFunctionSharingGegen},
and defining  library functions
$\FfunctionSeenBefore$_{functionSeenBefore@$\FfunctionSeenBefore$},
$\FpushFun$_{pushFun@$\FpushFun$}
and $\FpopFun$_{popFun@$\FpopFun$} to handle function memoization, cf.\
Figure~\vref{fig:SLFFunctionSharingLibrary}. The function
$\FnonLocalCopy$_{nonLocalCopy@$\FnonLocalCopy$}_{copy!non-local}
copies exactly those non-local objects that are reachable
(via pointers etc.) from the global \emph{and} local part of the
state_{state!local}_{state!global}.

\subsection{Utilizing in-use information}
\index{in-use analysis!using results from}
\label{sec:SLFUsingInUseInformation}

Consider the following program fragment where "x", "y" and "n" are
static, "z" and "total" dynamic:
\begin{verbatim}
total = (x + y) * (z - x);           /* basic block 1 */
if (total >= 0)
  printf("%d: profit!  x=%d", n, x); /* basic block 2 */
else
  printf("%d: deficit!", n);         /* basic block 3 */

return;                              /* basic block 4 */
\end{verbatim}
If, during specialization, "x" and "y" can take many different static
values, basic block 2 will result in many identical specialized versions
(one for each value of "y"), and basic block 3 will result in an identical
specialized version for each $("x","y")$, increasing both specialization
time and residual program size.  The problem is that we compare \emph{all}
the static values, even though their values cannot influence the following
specialization.

We reduce this problem by utilizing the in-use information computed in
Section~\ref{sec:In-UseAnalysis},  only comparing values for variables
that are actually in use.

After the split phase described in Section~\ref{sec:PartiallyStaticData}
has completed, we can give each static variable and parameter declaration
in the program a unique identification number, its $\VID$.
Whenever a call to $\FpendInsert$ is made, an extra parameter is passed,
containing (a representation of the $\VID$'s of) those variables that are
actually in use at the present program point. This parameter is then passed 
on when comparing or copying the static store.

We can represent the set of variables in use as a bit string $s$,
where $s[\VID] = $`1' if variable $v_{\VID}$ is in use, $s[\VID] = $`0'
otherwise. This is probably not a space-efficient
representation\footnote{If it is a problem in practice, the string can be
  compressed by a factor 4 (or more) by encoding it as hexadecimal digits},
but it is a simple way of hardcoding the in-use information into the
generating extension as string constants.

Assuming $\VID_{\mathtt{x}} = 0$, $\VID_{\mathtt{y}} = 1$ and
$\VID_{\mathtt{n}} = 2$, the above example would now translate into the
following:
\begin{center}\def\d{\(\sb{d}\)}\def\s{\(\sb{s}\)}%
\leavevmode\small
\hbox to .44\textwidth{\hss$p$\hss\hss}\hfil
\hbox to .55\textwidth{\hss$\Ppgen$\hss\hss}\nopagebreak[4]\smallskip
\nopagebreak[4]

\begin{minipage}[t]{.44\textwidth}\normalsize
\begin{alltt}
total = (x + y) * (z - x);
if (total >= 0)
  printf("%d: profit!  x=%d", n, x);
else 
  printf("%d: deficit!", n); 




return;
\end{alltt}  
\end{minipage}\hfil
\begin{minipage}[t]{.55\textwidth}\def\arraystretch{1.08}
\begin{array}[t]{rl}
l_1: & \emit{\bqt{\Vtotal}" = "\bqt{\Flift(x + y)}" * ("
       \bqt{z}" - "\bqt{\Flift(x)}");"} \\
     & \emitstmtbegin"if ("\bqt{\Vtotal}" >= 0)" \\
     & \quad "goto "\bqt{[l_2,\mbox{``101''}]}";" \\
     & \ "else" \\
     & \quad "goto "\bqt{[l_3,\mbox{``001''}]}";"\emitstmtend;\
       \Kgoto\ \LpendLoop \\
l_2: & \emitstmtbegin\verb+printf("%d: profit! x=%d", +
       \bqt{\Flift(n)}", "\bqt{\Flift(x)}");"\emitstmtend;\\
     & \emitstmtbegin"goto "\bqt{[l_4,
       \mbox{``000''}]}";"\emitstmtend;\
       \Kgoto\ \LpendLoop\\
l_3: & \emitstmtbegin\verb+printf("%d: deficit!", +
       \bqt{\Flift(n)}");"\emitstmtend;\\
     & \emitstmtbegin"goto "\bqt{[l_4,
       \mbox{``000''}]}";"\emitstmtend;\
       \Kgoto\ \LpendLoop\\
l_4: & ...
\end{array}
\end{minipage}\hfil
\end{center}



\paragraph{The memory management functions and data structures}
must be augmented to accommodate and check the in-use parameter before copying as shown in
Figure~\ref{fig:SLFMemoryManagementAdditions}:%
\begin{figure}[htbp]
\begin{center}\leavevmode\hbox{\vbox{%
\begin{pseudocode}
  $\TDataObjectCopy\ {*}
    \FcopySimple(\TDataObject\ {*}\Vobj,\ \TIUInfo\ \VinUse)$ \+\\
    $\Kif\ \Fnot(\FisInUse(\FID(\Vobj), \VinUse))\ \Kthen\ \Kreturn\ \CNull$ \\
      $...$ \-
  \\[1em]
  $\Tbool\ \FcmpSimple(\TDataObject\ *\Vobj,\
                       \TDataObjectCopy *\Vcpy,\ \TIUInfo\ \VinUse)$ \+\\
    $\Kif\ \Fnot(\FisInUse(\FID(\Vobj), \VinUse))\ \Kthen\ \Kreturn\ \Ctrue$ \\
    $...$ \-
  \\[1em]
  $\TDataObjectCopy\ {*}
    \FcopyWithOffset(\TDataObject\ {*}\Vobj,\ \Tvoid\ {*}\Vref,\
                     \TIUInfo\ \VinUse)$ \+\\
    $\Kif\ \Fnot(\FisInUse(\FID(\Vobj), \VinUse))\ \Kthen\ \Kreturn\ \CNull$ \\
      $...$ \-
  \\[1em]
  $\Tbool\ \FcmpWithOffset(\TDataObject\ *\Vobj,\
                       \TDataObjectCopy *\Vcpy, \
                       \Tvoid\ {*}\Vref,\ \TIUInfo\ \VinUse)$ \+\\
    $\Kif\ \Fnot(\FisInUse(\FID(\Vobj), \VinUse))\ \Kthen\ \Kreturn\ \Ctrue$ \\
    $...$ \-
\end{pseudocode}}}
    \caption{Additions to memory management functions to make use of in-use 
      information}
    \label{fig:SLFMemoryManagementAdditions}
  \end{center}
\end{figure}
just before copying or comparing an object, we check whether the
object is in use by using the object's ID to index into the in-use
string: type $\TIUInfo$ is a synonym for $\Tchar*$ and $\FisInUse :
\Tint \x \TIUInfo -> \Tbool$_{isInUse@$\FisInUse$} is defined by
$\FisInUse(\Tint\ \VID, \Tchar\ *\VinUse) = (\VinUse[\VID] =
{}$`1'$)$. If the object turns out not to be in use, we return
immediately; any other objects it might point to need \emph{not} be
copied due to this object. If some other object in use points to these
objects, \emph{they} will take care of the copying: If some object is
in use, it must either be a local or global variable which is in use,
or be reachable via local or global pointers that are in use.

Quite  similar additions apply to the copy and compare
functions_{function!copy!type specific}%
_{function!compare!type specific} for structs and arrays generated by
gegen in
Figures~\ref{fig:SLFUserTypesCopyFun}--\ref{fig:SLFArrayCopyCmpFun}.

The library functions for handling residual function sharing must be
augmented so that  in-use information is passed along when comparing and
copying the state, cf.\ Figure~\ref{fig:SLFFunctionSharingLibrary}.%
\begin{figure}[hbtp]
\begin{center}\leavevmode\hbox{\vbox{%
\begin{pseudocode}
  $\FfunctionSeenBefore(\Vid, \Vstate, \VinUse, \VtmpRetVal) = {}$ \+\\
    $\Kfor\ (\Vid_f, \Vstate_f, \VendState_f, \VtmpRetVal_f, f_{\Vresid}) 
             \in \VseenBefore\ \Kdo$ \+\\
      $\Kif\ \Vid_f = \Vid \mathrel{\land} 
             \Fequal(\Vstate_f, \Vstate, \VinUse)\ \Kthen$ \+\\
        $\Frestore(\VendState_f)$ \\
        $\VretVal := \VtmpRetVal_f$ \\
        $\Kreturn\ f_{\Vresid}$ \-\-\\
    $\Kreturn\ \CNoName$ \-
  \\[1em]
  $\FpushFun(\Vid, \Vstate, \VinUse) = {}$ \+\\
    $f_{\Vresid} := \FfreshResidFun(\Vid)$ \\
    $\Fpush(\VfunStack, (\Vid, f_{\Vresid}, \VemptyBody))$ \\
    $\Kif\ \Fsharable(\VcurrentFun)\ \Kthen$ \+\\
      $\VcurrentFun := (\Vid, \Fcopy(\Vstate, \VinUse), 
                        \CNull, \CNull, f_{\Vresid})$ \\
      $\Fadd(\VseenBefore, \VcurrentFun)$ \-\\
    $\Kelse$ \+\\
      $\VcurrentFun := (\Vid, \CNull, 
                        \CNull, \CNull, f_{\Vresid})$ \-\\
    $\Kreturn\ f_{\Vresid}$ \-
  \\[1em]
  $\FpopFun(\Vstate, \VtmpRetVal) = {}$ \+\\
    $(\Vid_f, \Vstate_f, \VendState_f, \VtmpRetVal_f, f_{\Vresid})
     == \VcurrentFun$ \\
    $\Kif\ \Fsharable(\VcurrentFun)\ \Kthen$ \+\\
      $\VendState_f := \FnonLocalCopy(\Vstate)$ \\
      $\VtmpRetVal_f := \VtmpRetVal$ \-\\
    $\Kelse$ \+\\
      $\Ffree(\VtmpRetVal)$ \-\\
    $\Kreturn\ f_{\Vresid}$
\end{pseudocode}}}
    \caption{Specialization library functions supporting
      residual function sharing and in-use information} 
    \label{fig:SLFFunctionSharingLibrary}
  \end{center}
\end{figure}
Note that in-use information is \emph{not} needed at
function return, because we do not know at gegen time what objects are 
in use at the call site that has invoked the function; all we can do
is to reduce the copying to copy only non-local objects.

\paragraph{The gegen transformations}
for dynamic "if" and "goto"  in
Figure~\ref{fig:GEGENControlStatements} must be augmented thus:_{IU@$\OIU$}
\[
\begin{array}{lcl}
      \gegen{\CIf_d(e,l_1,l_2)} & 
      =& \emit{"if ("\bqt{\gegen{e}}") goto "
         \bqt{[l_1, \widetilde{\textsl{iuCst}_{l_1}}]}
        "; else goto "\bqt{[l_2, \widetilde{\textsl{iuCst}_{l_2}}]}";"};\ 
        \hfill\Kgoto\ \LpendLoop \mathpunct{}
      \\[1.2ex]
      \gegen{\CGoto_d(l)} &
      =& \emit{"goto "\bqt{[l, \widetilde{\textsl{iuCst}_l}]}";"};\ 
        \hfill\Kgoto\ \LpendLoop\mathpunct{\rlap{,}}
  \\[1em]
  \mc{3}{l}{\textnormal{where $\widetilde{\textsl{iuCst}_l} =
      \FtoStr(\OIU(b_l))$ is a constant computed at gegen time}}
\end{array}
\]
and the gegen transformation for dynamic functions must be augmented as
shown in Figure~\ref{fig:SLFFunctionSharingGegen}_{<.>@$\gegen{\cdot}$}.
\begin{figure}[hbt]
\begin{center}\leavevmode\hbox{\vbox{%
\begin{pseudocode}
  \>$\quad\vdots$ \+\\
  $\Kif\ \VcallType = \CMem\ \Kthen$ \+\\
    $f_{\Vresid} := 
     \FfunctionSeenBefore(\Void, \VactiveState,
     \widetilde{\textsl{iuCst}}, \VtmpRetVal)$ \\
    $\Kif\ f_{\Vresid} \ne \CNoName\ \Kthen$ \+\\
      $\FassignRetVal(\VretVal_s, \VtmpRetVal, \Fsizeof(t_s))$ \\
      $\FpopObjects(q)$ \\
      $\Kreturn\ f_{\Vresid}$ \-\\
    $\Kelse$ \+\\
      $f_{\Vresid} := 
      \FpushFun(\Void, \VactiveState, \widetilde{\textsl{iuCst}})$ \\
      $\emitstmtbegin\bqt{t_d\ f_{\Vresid}}"("
           \bqt{\gegen{d_1}_{\Vpini}}", " ... ", "
           \bqt{\gegen{d_n}_{\Vpini}}
           "){"\emitstmtend$ \-\-\\
  $\Kelse$ \+\\
    $\vdots$ \-\-\\[1em]
  where $\widetilde{\textsl{iuCst}} = \FtoStr(\OIU(b_1))$ is a
  constant computed
  at gegen time 
\end{pseudocode}}}
    \caption{Additions to gegen for supporting residual function
      sharing and in-use information}
    \label{fig:SLFFunctionSharingGegen}
  \end{center}
\end{figure}

\subsection{Code generation}
\label{sec:SLFCodeGeneration}
\index{[.]@$\emit{\cdot}$|(} 

The overloaded code generating function, $\emit{\cdot}$, used in the
generating extension is simply implemented by various "cmix"$x$ library
functions: "cmixIf"_{cmixIf@"cmixIf"}, "cmixGoto"_{cmixGoto@"cmixGoto"},
"cmixAssign"_{cmixAssign@"cmixAssign"}, "cmixExpr"_{cmixExpr@"cmixExpr"},
\ldots

\index{[.]@$\emit{\cdot}$|)}

\subsubsection{Lift functions}
\label{sec:SLFLiftFunctions}
\index{lifting}

During specialization, constant values need to be exported to the residual
program---this is called \emph{lifting} static values into $\Ppres$.  The
(type overloaded) library function for lifting simple types is called
"cmixLift"_{cmixLift@"cmixLift"}. For function pointers, we need a special
``dereference-and-lift'' function: given a \emph{pointer} we want the
residual representation (name) of a \emph{function}. This is done by the
library function $\FliftFP$, which is passed a list of pointers to all the
functions in the generating extension, so that it can compare a pointer
that must be dereferenced-and-lifted:
\begin{center}\leavevmode\hbox{\vbox{%
\begin{pseudocode}
  $\TCode\ \FliftFP(\Tstruct\ \{\ \Tvoid\ {*}f;\ \TCode\ \Vname\ \}\
   {*}\Vfcns,\ \Tvoid\ {*}f)\ \{$ \+\\
    $\quad \Kreturn\ \Flookup(\Vfcns,\ f)$ \-\\
  $\}$
\end{pseudocode}}}
\end{center}


\subsubsection{Namespace Management}
\label{sec:SLFNamespaceManagement}
\index{namespace!$\Ppres$}

$\Ppgen$'s namespace manager for $\Ppres$ identifiers is quite similar to
gegen's (cf.\ Section~\vref{sec:GEGENNamespaceManagement}). 


\subsection{Miscellaneous helper functions}
\label{sec:SLFMiscellaneous}

The functions for handling the static part of return
values_{value!return!static} (cf.\ Figures~\ref{fig:GEGENControlStatements}
and~\ref{fig:GEGENFunctionsDynamic}) are defined
by_{setRetVal@$\FsetRetVal$}_{assignRetVal@$\FassignRetVal$}:
\begin{center}\leavevmode\hbox{\vbox{%
  \begin{pseudocode}
    $\FsetRetVal(\Tvoid\ {*}\Vval,\ \Tvoid\ {*}\VtmpRetVal,\ 
                 \Tint\ \Vsize) = {}$
    \+\\
      $\Kif\ \VtmpRetVal \ne \CNull\ \Kthen$ \+\\
        $\Kif\ \Fmemcmp(\VtmpRetVal,\ \Vval,\ \Vsize) \ne 0\ \Kthen$ \+\\
           $\Kerror($``Two different function 
                       end configurations!''$)$ \-\-\\
      $\Kelse$ \+\\
        $\VtmpRetVal := \Fmalloc(\Vsize)$ \\
        $\Fmemcpy(\VtmpRetVal,\ \Vval,\ \Vsize)$ \-\-
    \\[1em]
    $\FassignRetVal(\Tvoid\ {*}\VretVal,\ \Tvoid\ {*}\VtmpRetVal,\ 
                 \Tint\ \Vsize) = {}$
    \+\\
      $\Kif\ \VtmpRetVal \ne \CNull\ \Kthen$ \+\\
        $\Fmemcpy(\VtmpRetVal,\ \Vval,\ \Vsize)$ \\
  \end{pseudocode}}}
\end{center}



\subsection{Implementation level (1999-03-16)}
\label{sec:SLFImplementationLevel}

This chapter has not been updated in a year and is wildly incorrect.

\end{docpart}

%%% Local Variables: 
%%% mode: latex
%%% TeX-master: "cmixII"
%%% End: 
